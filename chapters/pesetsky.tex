\documentclass[output=paper]{langscibook}
\ChapterDOI{10.5281/zenodo.5524292}

\author{David Pesetsky\orcid{0000-0003-1530-9230}\affiliation{Massachusetts Institute of Technology}}
\title{Tales of an unambitious reverse engineer}
\abstract{This paper suggests a non-standard explanation for the limited range of semantics available to constructions in which certain elements of a normal finite TP are phonologically absent. These include English \textsc{aux}-drop questions (\citealt{fitzpatrick2006deletion}) and infinitival clauses (\citealt{Wurmbrand:2014aa}), where the proposal suggests an answer to some particularly vexing questions arising from the derivational (“exfoliation”) theory of infinitivization that I have advanced elsewhere (\citealt{Pesetsky:2019aa}). The core idea attributes apparent restrictions on the constructions themselves to restrictions on a hearer’s creativity in positing possible identities for material deleted in the speaker’s derivation (with “hearer” understood as an abstract concept, including self-monitoring by the speaker). Specifically, the hearer may consider only the minimally semantic contentful possibilities compatible with the morphosyntactic environment, in obedience to a \emph{principle of unambitious reverse engineering} (PURE).}

\begin{document}
\SetupAffiliations{mark style=none}
\maketitle

\section{Introduction}

Every time I hear a talk by Susi Wurmbrand, discuss syntax with her, or read one of her papers, my view of the world changes. Not only do I learn about new discoveries and novel approaches to complex problems, I myself am inspired to think new thoughts and explore new topics I never thought of exploring. She is one of the great linguists of our time, and it is our privilege to be her contemporaries. She has the important gift of spotting the inner simplicity in ridiculously tangled puzzles, and the equally important gift of discovering new puzzles for all of us, thus making our intellectual lives simultaneously easier and harder (just as a great colleague should). It is my personal good fortune that the two of us share several research interests -- most notably \textit{finiteness} -- which means that I have been especially able to benefit from these gifts of hers.

By an extraordinary coincidence, the invitation to contribute to this volume arrived shortly after I had prepared several of Wurmbrand's recent papers on non-finite complementation for a graduate seminar that I was co-teaching with my colleague Athulya Aravind, and was contemplating writing a paper in direct response to one of these papers. This portion of the seminar was of particular importance to me as the author of recent work (\citealt{Pesetsky:2019aa}) that argued for a crucially \textit{derivational} theory of finiteness. In this work, I revived the idea originated by \citet{Lees1963} and Rosenbaum \citeyearpar{Rosenbaum:1965aa,Rosenbaum1967}, that non-finite clauses are derived from full and finite clauses in the course of the syntactic derivation, as a response to cross-clausal processes such as raising and control. Could this proposal of mine be reconciled (I wondered) with Wurmbrand's compelling findings about such clauses, which presupposed the opposite view: that non-finite clauses are non-finite from the outset, and have properties (most notably, tenselessness) that distinguish them throughout the derivation from their finite counterparts? 

A centerpiece of that class was \citet{Wurmbrand:2014aa}, a brilliant paper that explores the full range of English non-finite complement clauses formed with \textit{to}, and makes a strong case for their deep \textit{tenselessness}. As I prepared that work for class, however, it seemed to me that there might be a way to retain all the logical threads that Wurmbrand traced from construction to construction in support of her central claim, while reaching a very different conclusion.\footnote{As noted below, it turned out that this was not the first time this realization had dawned on me, at least for one class of constructions discussed in Wurmbrand's paper.} The origins of the paper that you are reading now can be traced directly to that class and the fruitful discussions it engendered.

In her paper, Wurmbrand argued for the tenselessness of English infinitival complements by showing that no matter what element one might propose as the finite counterpart of infinitival tense, the infinitival version differs in its behavior in multiple respects, in ways explainable as consequences of tenselessness. I will suggest that a different conclusion is at least equally plausible: that the semantics of tense in the English infinitives studied by Wurmbrand fails to correspond to any single finite counterpart because it actually ranges in principle across the full gamut of possible finite counterparts (as expected if infinitives are derived from finite clauses), behaving in some cases like a present or past modal, in other cases like non-modal present or past tense, and in still other cases (closely following Wurmbrand's discussion) copying its tense value from the embedding verb.\thispagestyle{empty} 

Of course, it is also well-known that the semantics of English infinitival clauses is much more restricted in its possibilities than is the semantics of finite clauses. So while it might be the case that the semantics of infinitival tense does range \textit{in principle} across the full gamut of finite possibilities, in reality, the actual possibilities are tightly constrained. While Wurmbrand argues that these constraints reflect deep tenselessness, I will argue that they are actually extra-grammatical, reflecting strong limitations on the ability of a hearer to “reverse engineer” the derivation behind a speaker's utterance, precisely when obligatory elements such as tense have been generated but not phonologically interpreted. This explanation for the limitations on the interpretation of infinitival complements, I will suggest, dovetails with an explanation for the properties of a seemingly different puzzle studied by \citet{fitzpatrick2006deletion}, for which he proposed an ingenious but ultimately self-contradictory solution that the approach suggested here resolves. I will also briefly suggest that this “reverse engineering” mode of explanation for the phenomena charted by Wurmbrand suggests new approaches to other phenomena as well.

The data discussed in this paper are almost entirely drawn from \citet{fitzpatrick2006deletion}, \citet{Pesetsky:2019aa}, and \citet{Wurmbrand:2014aa}. This is thus a “new perspectives” paper, and not a “new empirical discoveries” paper. It is the dovetailing with Fitzpatrick's puzzles and the proposals that I argued for in \citet{Pesetsky:2019aa} that may argue for my perspective over Wurmbrand's, not (at least for now) new empirical predictions of the new approach. As I noted at the outset, Wurmbrand's work does not only teach, it also inspires. I am delighted to offer this paper as a modest but characteristic example.

\section{The factative effect in English  \textsc{aux-}drop}

\subsection{Fitzpatrick's discoveries}

\citet{fitzpatrick2006deletion} studied a type of yes/no question, common in spoken Standard English, in which the auxiliary verb that is moved to C is omitted. He called this the \textit{\textsc{aux}-drop} construction: 

\needspace{5\baselineskip}
\pex<auxdrop>
\ptxt{\textsc{aux}-drop (English)}
\a Anybody want a hot dog? (= \emph{Does anybody want a hot dog?}) 
\a Anyone seen John today? (= \emph{Has anyone seen John today?})
\a Anybody going to the game? (= \emph{Is anybody going to the game?})
\trailingcitation{(\citealt[400, ex. 1]{fitzpatrick2006deletion})}
\a Anybody accused of a crime today? (= \emph{Was anybody accused of a crime today?)}
\xe

\noindent Fitzpatrick provides several arguments that an \textsc{aux}-drop clause is generated as a full interrogative CP in which an auxiliary verb moves from T to C, just as it does in the more formal non-\textsc{aux-}drop counterparts given in parentheses above. Fitzpatrick shows first that \textsc{aux}-drop questions are at least as large as TP, by noting that they may contain negation (\textit{Anybody not like John?}) and higher adverbs (\textit{Everyone probably coming tomorrow?}). A subject pronoun in an \textsc{aux}-drop question must be nominative (\textit{He here yet?}\slash\relax*\textit{Him here yet}), as expected if the pronoun is a normal subject in a finite T (despite the absence of any overt finite auxiliary verb or exponent of T). 

Fitzpatrick then proceeds to show that an \textsc{aux}-drop question is even larger than TP, and in particular, that it is generated as a full CP, in which an auxiliary verb in T has moved to C. As he notes first, if the interpretation of examples like (\getref{auxdrop}a--d) as yes/no questions relies on semantics crucially provided by an interrogative C (as in most accounts of such questions), an \textsc{aux-}drop clause must have been generated as a CP. An additional argument rests on the observation that licensing a negative polarity item like \textit{anyone} and \textit{anybody} in examples (\getref{auxdrop}a--d) correlates with movement of \textsc{aux} to C in less controversial matrix yes/no questions. For example, \textit{how come} differs from its near-synonym \textit{why} in not triggering \textsc{aux}-to-C movement, and also differs in not licensing NPIs:

\pex<howcome> 
\ptxt{\textit{Why} vs. \textit{how come} as NPI licenser}
\a Why did you ever give anyone that?
\a \ljudge{*}How come you ever gave anyone that? 
\trailingcitation{(\citealt[409, adapted from ex. 21]{fitzpatrick2006deletion})}
\xe

\noindent Likewise, though \textit{You gave him that?} with a rising intonation but no movement of \textsc{aux} to C may be understood as a question requiring a \textit{yes} or \textit{no} answer (perhaps reflecting a variety of interrogative C that fails to trigger \textsc{aux-}to-C movement), here too an NPI is not licensed:

\pex<auxinsitu>
\ptxt{\textsc{aux}-to-C vs. its absence correlates with NPI licensing} 
\a Did you ever give anyone that?
\a \ljudge{*}You ever gave anyone that? 
\trailingcitation{(\citealt[409, adapted from ex. 20d--e]{fitzpatrick2006deletion})}
\xe

\noindent In addition, both \textsc{aux}-drop and non-\textsc{aux}-drop yes/no questions show the effects of what is arguably an adjacency requirement preventing an adverb from intervening between the subject and the auxiliary verb that has moved to C, as (\getref{nowadjacency}\getref{nowadjacency.auxadjacency}--\getref{nowadjacency.auxdropadjacency}) show. This requirement (perhaps a case filter effect) is not found when \textsc{aux-}to-C movement has not taken place, as the embedded question in (\getfullref{nowadjacency.noadjacency}) shows:\footnote{Fitzpatrick does not ground either the correlation between NPI licensing and \textsc{aux}-to-C movement or the \textsc{aux-}subject adjacency condition in deeper principles, with the result that these arguments might weaken as we learn more about these effects. Note that an embedded yes/no question introduced by \textit{whether} or \textit{if} does license a subject NPI, despite the absence of \textsc{aux}-to-C, but at the same time permits an adverb to intervene between C and the subject (\textit{I wonder whether/if now everyone is aware of the problem}), a difference between the two diagnostics that will need an explanation.}

\pex<nowadjacency>
\ptxt{\Aux{}-subject adjacency condition with and without \Aux{}-drop}
\a<auxadjacency> \ljudge{*}Is now everyone aware of the problem?
\a<auxdropadjacency> \ljudge{*}Now everyone aware of the problem?
\a<noadjacency> I wonder whether (now) everyone (now) is aware of the problem.
\trailingcitation{(\citealt[408, adapted from ex. 18--19]{fitzpatrick2006deletion})}
\xe

\noindent Constructions that arguably require semantic parallelism with a full CP provide additional arguments (not from Fitzpatrick). For example, an \textsc{aux-}drop question may be disjoined with a negative alternative in which \textsc{aux} has moved to C, as (\getref{disjunction}) shows, and may provide an antecedent for null complement anaphora where the non-null version would be a full-CP \textit{whether} question:

\pex<disjunction>
\ptxt{Disjunction with full CP alternative}
You give your talk or didn't you?
\xe
\pex[aboveexskip=0pt]<NCA>
\ptxt{Antecedent for null complement anaphora of a \textit{whether} question}
\a You giving your talk tomorrow, or don't you know yet Δ?
\a A: Anyone want coffee? \\B: I'll find out Δ.
\xe

\noindent If these arguments are correct, \textsc{aux}-drop questions are generated as normal interrogative CPs in which the auxiliary verb moves to C, but some property of the grammar permits the auxiliary verb in C to remain unpronounced. 

Fitzpatrick makes several further crucial observations. First, he observes that an auxiliary may be silenced by \textsc{aux}-drop only if it has raised to C. A declarative clause like \textit{It has given me a headache}, for example, cannot lose its auxiliary: *\textit{It given me a headache}. The same is true of an embedded yes/no question in which the auxiliary remains in situ: *\textit{I wonder whether Mary written a letter}. Furthermore, \textsc{aux}-drop is limited to root clauses. The examples in (\getref{rootonly}) below also show that movement of \textsc{aux} to C in the antecedent of a counterfactual conditional, an embedded environment, does not have a variant with \textsc{aux}-drop. The additional examples in (\getref{rootonly2}) show that when \textsc{aux} moves to C in an embedded yes/no question, which is possible in informal spoken English (McCloskey \citeyear{McCloskey:2006}), \textsc{aux}-drop is also impossible.\footnote{The binding of \textit{she} by \textit{each actress} in this pair helps ensure that the yes/no question is normally embedded, and is not a quotation (cf. \citealt{fitzpatrick2006deletion} p. 420 fn. 24).} 

\pex<rootonly>
\ptxt{\Aux{}-drop only at the root: counterfactual inversion}
\a Had you written a better speech, Sue would probably have won the election.
\a *You written a better speech,  Sue would probably have won the election.
\trailingcitation{(\citealt[409, adapted from ex. 20d--e]{fitzpatrick2006deletion})}
\xe
\pex[aboveexskip=0pt]<rootonly2>
\ptxt{\Aux{}-drop only at the root: T-to-C movement in embedded questions}
\a Each actress wanted to know had she been chosen for the part or not.
\a \ljudge{*}Each actress wanted to know she been chosen for the part or not.
\xe

Fitzpatrick concludes that what yields the \textsc{aux}-drop construction is the \textit{optional non-interpretation by the phonology of the top layer of a fully built syntactic structure}. Though information about the entire structure dominated by the highest TP gets sent to the phonology during the course of the derivation, the contents of the root CP are not. This option makes sense, as Fitzpatrick notes, in the model of phase-by-phase interpretation proposed by \citet{Chomsky2001}, according to which the merging of each phrase head triggers phonological interpretation of the phase head's \textit{complement}, as well as semantic interpretation of the complement, to which I return below.\largerpage

\begin{sloppypar}
The crucial case for the \textsc{aux}-drop construction concerns the merger of C, which triggers interpretation of its complement TP. Chomsky's \citeyearpar{Chomsky2001} regime for phase-by-phase interpretation of syntactic structure works hand in hand with the additional hypothesis that constituents interpreted by this process are impenetrable for the remainder of the syntactic derivation. This impenetrability hypothesis both permits and requires apparent instances of extraction from phasal constituents to proceed through the phase edge, merging as a specifier of each phasal head on the path to its final destination. 
\end{sloppypar}

As Fitzpatrick notes, however, this proposal comes at a cost: the theory must posit a special clean-up rule to interpret the final CP layer at the root. This fact lies at the heart of Fitzpatrick's account of \textsc{aux-}drop: namely, “that this extra operation need not apply in all cases, and that \textsc{aux}-drop is one case where it fails to apply.” This proposal, however, raises serious questions concerning semantic interpretation. These questions embroil Fitzpatrick's account in a contradiction for which this paper suggests a new resolution. It is this resolution, in turn, that will serve as a model for a reinterpretation of Wurmbrand's \citeyearpar{Wurmbrand:2014aa} findings concerning the tense properties of English infinitives. 

Fitzpatrick discovered a remarkable semantic property of \textsc{aux}-drop that makes immediate sense if phonological and semantic interpretation apply together as a unit (Chomsky's \citeyear{Chomsky2001a} rule of ``\textsc{transfer}''). In particular, \textsc{aux}-drop sentences show a phenomenon made famous by Déchaine \citeyearpar{Dechaine1991}, which she called the \textit{factative effect} (adapting terminology from Africanist linguistics credited to Welmers \citeyear{Welmers:1973aa}, 346). In some languages, a clause may lack overt tense marking but nonetheless be interpreted as tensed. Crucially, how tense is understood in such clauses depends on whether its main predicate is eventive (episodic) or non-eventive (e.g. stative). If the predicate is non-eventive, the tense is understood as \textsc{present}, but if it is eventive, the tense may or must be understood as \textsc{past}. The contrast in (\getref{HaitianFactative}) demonstrates the factative effect in Haitian declarative clauses. Fitzpatrick cites additional examples from Fɔ̀ngbè, Yoruba, and Igbo. Example (\getref{AuxDropFactative}) illustrates Fitzpatrick's key discovery: a similar effect at work in the English \textsc{aux}-drop construction:

\NumTabs{4}
\pex<HaitianFactative>
\ptxt{Factative effect: Haitian Kreyòl}

\a
\begingl
\gla Pyè vann bèf yo.  {\nogloss{\tabto{2.2in} \textit{eventive→past}}}//
\glb Pyè sell cattle \Det{}//
\glft `Pyè sold the cattle.'//
\endgl

\a
 \begingl
\gla Sisi renmen chat mwen. {\nogloss{\tabto{2.2in} \textit{non--eventive→present}}}//
\glb Sisi like cat my//
\glft `Sisi likes my cat.'//
\endgl
\xe

\pex[aboveexskip=0pt]
<AuxDropFactative>
\ptxt{Factative effect: English \Aux{}-drop}
\a
\begingl
\gla You sell that cattle? {\nogloss{\tabto{2.2in} \textit{eventive→past}}}//
\glft `Did you sell that cattle?'//
\endgl
\a
\begingl
\gla You like my cat? {\nogloss{\tabto{2.2in} \textit{non--eventive→present}}}//
\glft `Do you like my cat?'\trailingcitation{(\citealt[414, ex. 27a--d]{fitzpatrick2006deletion})}//
\endgl
\xe


\noindent Following Déchaine \citeyearpar{Dechaine1991}, Fitzpatrick suggested that the factative effect arises when no tense features are available to semantic interpretation. For languages like Haitian that show the factative effect in simple declarative clauses, Fitzpatrick posited a semantically underspecified but syntactically present T in sentences like (\getref{HaitianFactative}a--b). By contrast, the English \textsc{aux-}drop construction involves a fully-specified T that moves to C as part of the auxiliary verb, by the normal process that builds matrix non-\textsc{aux}-drop questions. Fitzpatrick also proposed the following: just as the phonological properties of the tensed auxiliary in C are not submitted to phonological interpretation in the \textsc{aux-}drop constructions, its semantic properties are also not submitted to semantic interpretation. As far the semantics is concerned, therefore, the tense specification of T might as well never have been generated in the first place (even though it was). Because the contents of C are not interpreted, an \textsc{aux}-drop sentence is understood as if it entirely lacked tense, yielding the same factative effect found in Haitian when T actually does entirely lack a tense specification.


\subsection{Reverse-engineering an \textsc{aux-}drop derivation}\label{subsec:Reverse-engineering-an-aux-drop}\largerpage

As mentioned above, Fitzpatrick's proposal ends up enmeshed in an apparent contradiction concerning the semantic interpretation of the silent material in the \textsc{aux-}drop construction. The remarkable cross-linguistic correlation highlighted by Fitzpatrick between the factative effect and the absence of overt tense does indeed suggest that the phonological suppression of a tensed auxiliary has semantic repercussions. For English \textsc{aux}-drop, however, the claim that these repercussions arise from total non-interpretation of the root CP contradicts some of the very evidence that argued that an entire CP had been built in the first place. That evidence had a crucial semantic component: normal yes/no question semantics correlating with T-to-C movement, correlating in turn with NPI licensing. Fitzpatrick noted this problem (p. 422), but left it unresolved. He also noted an additional related puzzle (p. 419): though the failure to submit the root CP layer to semantic interpretation might entail the lack of access to tense information that had been lodged in C thanks to T-to-C movement -- given the copy theory of movement -- that information should still be present on T in its original position (rendering the factative effect especially mysterious, as Carlo Geraci, personal communication, notes). Since reconstruction phenomena in other domains teach us that unpronounced earlier positions of moved elements may be semantically interpreted, it is puzzling that T-to-C movement followed by \textsc{aux-}drop should be an exception.\footnote{\citet{fitzpatrick2006deletion} offered a tentative conjecture that the components of meaning relevant to question interpretation and NPI licensing might actually be contributed by a phasal head lower than C but higher than T, a head that is submitted to semantic interpretation as the (actual) complement to C. No additional evidence was offered for the existence of such a head with the properties attribute to it, and various problems are raised by this conjecture, briefly discussed in a footnote (p. 422, fn. 26).}

\begin{sloppypar}
I believe that a different approach to Fitzpatrick's discovery can resolve these issues in a new way. This approach will turn out to have implications for the proper treatment of other clauses with “missing” content, such as nonfinite clauses.
\end{sloppypar}

Let us accept as convincing the data (mostly) from Fitzpatrick with which we began this section, which seem to show that though the contents of the CP layer in the \textsc{aux}-drop construction are not subject to phonological interpretation, they \textit{are} submitted to semantic interpretation. The interpretation of \textsc{aux}-drop clauses as yes/no questions may thus rely on the semantic properties of interrogative C, and the availability of NPI licensing in such clauses will follow from whatever factors turn out to be relevant in non-\textsc{aux-}drop yes/no questions.

What should we then make of the correlation between the phonological absence of the tensed auxiliary verb and the factative effect, which limits tense interpretation to \textsc{present} with non-eventive predicates, and \textsc{past} with eventive predicates? I propose that this correlation does not reflect the \textit{grammar} of non-pronunciation and semantic interpretation at all, but reflects an entirely different consequence of non-pronunciation. When a speaker produces an utterance based on a derivation in which elements that normally receive a phonological interpretation are unpronounced, the language system of the hearer must \textit{reverse-engineer} the speaker's derivation, supplying its own guesses concerning the identity of the unpronounced elements. If the speaker produces an \textsc{aux-}drop question, for example, missing its tensed auxiliary verb, the language system of the hearer must rely on its own resources to supply the missing auxiliary verb and tense. 

But what are those resources? I propose that they are limited, as a property of the human language faculty, and that it is the tightly limited capacity of a hearer for reverse-engineering the speaker's syntactic derivation, not the architecture of that derivation itself, that lies behind the factative effect in the English \textsc{aux}-drop construction.\largerpage

Let us begin by noting, with Fitzpatrick himself (p. 14), that \textsc{aux}-drop is impossible when the auxiliary is “semantically contentful”. The auxiliary verbs whose pronunciation can be suppressed (supportive \textit{do}, perfect \textit{have}, progressive and passive \textit{be}) are those that arguably lack semantics of their own, and are inserted to satisfy independent requirements of their morphosyntactic environment (cf. \citealt{GrOnn:2021aa}, Section 3 on perfect \textit{have}). By contrast, as Fitzpatrick points out, modals that do have semantics of their own cannot be omitted as part of the \textsc{aux-}drop construction:

\pex<NoModal>
\ptxt{No modal \Aux{}-drop}
\a<a> Anyone pick up John at the airport?\\
Impossible with the meaning `Can anyone pick up John at the airport?'
\a<b> Anyone play the piano at the party tomorrow?\\
Impossible with the meaning `Will anyone play the piano at the party tomorrow?'
\trailingcitation{(\citealt[412, ex. 25a--b]{fitzpatrick2006deletion})}
\xe

\begin{sloppypar}
\noindent I propose that this fact itself should be understood as a consequence of a general limitation on the reverse-engineering of phonologically suppressed material. This limitation dictates that the reverse-engineering process must be semantically \textit{unambitious}, positing as little semantics as possible that is not unambiguously reconstructable from the input provided by the speaker. I will call this limitation the \textit{principle of unambitious reverse engineering} (PURE). In essence, PURE is a version of the much-debated principle of “recoverability of deletion” (\citealt[41]{Chomsky1964c} and \citealt[79ff]{Katz:1964}), now viewed as an extra-grammatical property of the hearer's language system attempting to parse input from a speaker. 
\end{sloppypar}

\needspace{4\baselineskip}
\pex<Recoverability>
\ptxt{Principle of unambitious reverse engineering (PURE)}

When determining the identity of unpronounced material in the course of reverse-engineering a speaker's syntactic derivation, the language system of the hearer considers only the \textit{minimally semantically contentful possibilities} compatible with the morphosyntactic environment. 
\xe

I use the phrase “language system of the hearer” to emphasize that the discussion does not concern conscious decisions of the hearer, but rather the automatic behavior of the cognitive systems that parse linguistic input, determine its meaning, and communicate this information to other cognitive systems. In what follows, I will sometimes refer directly to the hearer's language system with the word “hearer” to keep the prose simple, but it is the hearer's language system that I intend throughout. I also assume that a speaker's language system self-monitors in the process of speech production, functioning as hearer as well as speaker, so that the planning of an utterance takes into account the restricted range of interpretations a hearer is permitted to entertain by the principle proposed below. So the term \textit{hearer} in this paper stands for an emphatically abstract concept. 

Our intuitions that certain derivations do not permit \textsc{aux}-drop, on this view, are hearer-side intuitions concerning what derivations can and cannot be reverse-engineered (in response to a signal in which the root CP is unpronounced), not limitations on what the grammar can generate in the first place. The speaker's grammar thus overgenerates, and the effects of PURE have the logical status of a filter.\footnote{Carlo Geraci (personal communication) notes a similarity between these considerations and aspects of the “perceptual loop” theory of self-monitoring advanced by Levelt (\citeyear[96]{Levelt:1983aa}, \citeyear[chapter 12]{Levelt:1989aa}), as developed and debated in subsequent work. Levelt proposes a monitoring process that is “normally. opaque to the speaker, {[}which{]} should, rather, be regarded as based on the parsing of inner or overt speech”. “The great advantage of {[}such{]} a perceptual theory,” he continues, “is that controlling one’s own speech is like attending to somebody else’s talk. This makes it natural for the speaker to apply the same parsing procedures and sources of knowledge to his own speech as to other people’s speech” (\citealt[96--97]{Levelt:1983aa}). PURE and its consequences as discussed in this paper thus have obvious psycholinguistic implications, which I will not explore here, for lack of expertise, but hope may be clarified in future work.}

Let us consider how PURE interacts with examples like (\getref{NoModal}a--b), where the hearer must posit an auxiliary verb in C. Here a semantically contentless supportive \textit{do} is minimally semantically contentful and compatible with the morphosyntactic environment (since the highest audible verb is in the morphological bare form that cooccurs with auxiliary \textit{do}). As a consequence, PURE guides the hearer of an utterance like (\getfullref{NoModal.a}) or (\getfullref{NoModal.b}) to posit a suppressed form of \textit{do} and prevents the positing of a suppressed modal such as \textit{can} or \textit{will}. Likewise, because the morphosyntactic environment of an \textsc{aux}-drop shows T movement to C, the least semantically contentful possibility for reverse-engineering the contents of C features an interrogative complementizer. We might also attribute to PURE the fact that the hearer is not free to assume that the CP of an \textsc{aux-}drop construction contains any contentful \textit{wh-}phrase other than the yes/no operator, conceivably the least contentful \textit{wh}-form (as it is invokes two fixed focus alternatives and is phonologically null in main clauses independent of \textsc{aux-}drop), but I will leave the details of this aspect of the puzzle for later work.\footnote{Carlo Geraci (personal communication) notes an unsolved problem for this approach: the fact that silencing of a \textit{wh}-phrase other than \textit{whether} is blocked even when the selectional properties of an obligatorily transitive verb might render this compatible with PURE. Thus *\textit{You wear?} is not acceptable, for example with the reading \textit{What did you wear?}, despite the transitivity of \textit{wear}. I leave this issue open.} 

I return now to the factative effect in \textsc{aux-}drop, which I suggest is just another consequence of PURE. When it is necessary to reverse-engineer a derivation in which a tensed but unpronounced auxiliary verb has raised to C, PURE requires the hearer to posit a semantically minimal specification for the unpronounced T.\largerpage

But why should \textsc{past} qualify as a minimally contentful tense for an eventive predicate, while only \textsc{present} counts as minimally contentful for a non-eventive predicate? If \textsc{present} is a tense bound to the utterance time, then this relation may count as ubiquitous component of the “morphosyntactic environment” of any utterance, licensing the hearer to posit \textsc{present} as the tense specification of a silenced T, in keeping with PURE. A \textsc{past} specification for T, however, would add\textit{ anteriority} to the meaning of \textsc{present}, and thus qualify as less minimally semantically contentful. PURE might therefore prevent the hearer's parser from positing \textsc{past} with a non-eventive predicate, all things being equal. This derives the obligatorily \textsc{present} interpretation of an \textsc{aux-}drop clause with a non-eventive predicate.

Why then should an eventive predicate license the positing of \textsc{past} by the hearer as the tense of the speaker's derivation that is being reverse-engineered? Note that eventive predicates are incompatible with the simple\textsc{ present}, unless coerced into a habitual or generic use (a fact that will be important in our discussion of infinitival clauses below):\largerpage[1]

\pex<eventivepresent>
\ptxt{Present tense incompatible with eventive predicates (unless coerced)}
\a<a> *Mary sings in the shower now. / *Alice reads a book now. / *Bob sells that car now.\\	~~[unless habitual]
\a<b> Sue owns a car now. / John likes my cat now. / Bill knows German now. etc.
\xe
\noindent I propose that it is precisely because of the incompatibility of the English \textsc{present} with an eventive predicate that PURE permits the hearer to posit an underlying \textsc{past} in an \textsc{aux-}drop construction where the unpronounced auxiliary in C is \textit{do} and the main predicate is eventive. \textsc{past} is the least semantically contentful option compatible with the morphosyntactic environment. I will leave it as an open question whether this suggestion for English \textsc{aux}-drop illuminates the roots of the factative effect in other languages such as Haitian Kreyòl.{\interfootnotelinepenalty=10000\footnote{\citet{Dechaine:1995} offers a more detailed proposal concerning the tense interpretation of eventive predicates in these constructions, which I believe could be incorporated into the present discussion. Her account also correctly predicts the fact that Haitian Kreyòl favors a non-past generic interpretation for eventive predicates with a bare indefinite direct object, a fact also found in the \textsc{aux}-drop construction when the direct object is a mass singular or bare plural (an observation also made by Michelle Sheehan, personal communication):

\pexcnn<bareplurals>
\a
\begingl
\gla Pyè vann bèf. {\nogloss{\tabto{1.8in} \textit{eventive/indefinite object →present}}}//
\glb Pyè sell cattle//
\glft `Pyè sells cattle.'\trailingcitation{\citet[74, ex. 37a]{Dechaine:1995}}//
\endgl
\a
\begingl
\gla You sell cattle/cars? //
\glft `Do you sell cattle/cars?'//
\endgl
\xe

\noindent I am grateful to Athulya Aravind (personal communication) for bringing \citet{Dechaine:1995} to my attention.}}

We may now adopt Fitzpatrick's proposal that \textsc{aux}-drop arises from the more general possibility of leaving the highest layer of the root clause phonologically uninterpreted, without the contradictions that arose from extending this possibility to semantic interpretation as well. If the proposal advanced here is correct, there is no comparable optionality for semantic interpretation. The syntactic derivation is subject to semantic interpretation up to the root. The factative effect is a by-product of failing to phonologically interpret the CP layer of the main clause, just as Fitzpatrick proposed. But it is not a direct result of the grammatical derivation per se, but instead reflects the strictures imposed by PURE on the hearer forced to reverse-engineer the speaker's derivation. In in the absence of evidence concerning the value of T that was included in the speaker's syntactic derivation, the hearer must assume a maximally unmarked value compatible with the morphosyntactic environment.\footnote{Our discussion leaves several important unresolved questions unanswered. We must ensure, for example, that a hearer's disambiguation of a syncretic form ambiguous between \textsc{past} and \textsc{present} such as \textit{put} or \textit{hit} is not subject to PURE. Ignorance concerning the precise identity of an item that has been phonologically interpreted (albeit confusingly) is evidently not the same problem for the hearer as determining the identity of an item that has avoided phonological expression entirely. Ellipsis is another, much larger elephant in the room of this analysis. There it is tempting to view the “surface anaphora” property of ellipsis (the need for a linguistic antecedent) as a sign of the strictures of PURE at work, but I leave the possible development of this idea for future work as well.}

\section{Exfoliation and the tense interpretation of infinitives}

\subsection{The derivational theory of infinitivization}

We are now in a position to take up the main topic of this paper: a second environment in which I have recently argued that tense and other material ends up unpronounced due to a property of the grammar that absolves this material from phonological interpretation, though for reasons quite different from those relevant to \textsc{aux}-drop. Here juxtaposition of these arguments with the semantic findings reported by \citet{Wurmbrand:2014aa} raises questions similar to the contradictions that I have attempted to resolve concerning Fitzpatrick's theory of English \textsc{aux}-drop. Once again, I will suggest a reverse-engineering reinterpretation of Wurmbrand's discoveries resolves these contradictions. Rather than reflecting semantic consequences of the speaker's syntactic derivation, as Wurmbrand proposes, I will suggest that they actually reflect the restrictions placed by PURE on the hearer's ability to reverse-engineer that derivation.

In work reported in \citet{Pesetsky:2019aa}, I have argued for a \textit{derivational theory of infinitival complementation}. On this view, all embedded clauses are generated by the syntax as full and finite CPs. Infinitival clauses are the result of a rule of \textit{exfoliation}, which strips away the outer layers of a finite CP, leaving behind an infinitival clause, under very specific circumstances: namely, when a probe external to CP finds a goal internal to that CP that does not occupy its edge. Exfoliation eliminates as many clausal layers as is necessary to place that goal at the edge of what remains, so it can interact with that goal (see \figref{fig:pesetsky:1}).\footnote{Crucial to the proposal as an account of English infinitives is the existence of a \textit{to} projection lower than T, and a principle (argued to have more general applicability) that leaves \textit{to} unpronounced when exfoliation does not strip away the projections higher than \textit{to}. See \citet{Pesetsky:2019aa} for discussion and argumentation.}

\begin{figure}
\caption{Exfoliation\label{fig:pesetsky:1}}
\begin{forest}for tree={s sep=15mm, inner sep=0, l=0}
[ [V\\\textit{\textbf{φ-probe}},name=probe] 
[CP,name=cp
[\ldots,name=speccp] [C$'$,name=cbar [C,name=comp] [TP,name=tp  
[T,name=tense,tikz={\node [rectangle,draw=black, fill=black, fill opacity=0.05,text opacity=1,anchor=right,fit=()(!u)(!uuu)(speccp)] {};}]
[\textit{to}P,name=exfoliate [\textit{\textbf{subject}},name=lowsubject] [\textit{to}$'$ [\textit{to}] [\textit{vP}]]]]]
{\draw (.east) node[right=1cm, align=left] {\small{← \textit{\begin{minipage}{\widthof{Exfoliation removes this portion}}Exfoliation removes this portion of the embedded clause\end{minipage}}}};}
]]
\draw[overlay, -{Triangle[]}, dashed] (probe) to[out=south,in=west] (lowsubject);
\end{forest}
\end{figure}

Arguments from several directions have been advanced to support this proposal.\footnote{In a Festschrift honoring Susi Wurmbrand, it is especially important to note that this proposal does not necessarily include Restructuring infinitives in its purview. Restructuring clauses of the German type studied by Wurmbrand are not in any obvious sense a response to cross-clausal probing, and might represent small constituents generated as such. I leave the integration of exfoliation with the phenomenon of Restructuring for future work.} 

{\def\lingexbreakpenalty{10}
First, I argued that \textit{paradigms of acceptability for infinitival complementation} do indeed correlate with probe-goal relations across the clause boundary. Whenever a probe capable of triggering Raising successfully contacts an element in the specifier of \textit{to}P across a CP boundary, that CP is reduced to an infinitive, but not otherwise. The presence of a Raising probe on the higher V in (\getref{nominalsubjects}a), and on the higher \textit{v} in (\getref{nominalsubjects}b) accounts for the infinitivization of the embedded clause, while the absence of any comparable probe in (\getref{nominalsubjects}c--f) accounts for the impossibility of infinitivization in these examples.

\pex<nominalsubjects> 
\NumTabs{8}
\ptxt{Nominal subjects of an infinitival clause} 
\a Sue considers Mary to have solved the problem. \\\hbox{}\hfill\hbox{\emph{Raising to Object (spec,VP)}}
\a Mary seems to speak French well.\hbox{}\hfill\hbox{\emph{Raising to Subject (spec,\textit{v}P)}}
\a<seemsMary> \ljudge*It seems Mary to have solved the problem. \hfill\emph{unaccusative V}
\a \ljudge*It was believed Mary to speak French well.\hfill\emph{passive V}
\a \ljudge*Mary is aware Bill to be the best candidate.\hfill\emph{A}
\a<beliefit> \ljudge*Mary's belief it to have been raining \hfill\emph{N}
\xe

\noindent The standard competitor to this proposal is the traditional claim that infinitives are not derived from finite clauses but are generated nonfinite, with Case Theory accounting for contrasts like those in (\getref{nominalsubjects}), on the assumption that the subject of a nonfinite clause can only pass the case filter if some external element such as the higher verb in (\getref{nominalsubjects}a) or the higher T in (\getref{nominalsubjects}b) case-licenses it. The fact that non-nominals that otherwise do not need to pass the case filter, such as CP subjects and fronted adjectival predicates, show exactly the same paradigm, however, argues against this standard competitor:\largerpage

\pex<CPsubjects> \ptxt{Clausal subjects of an infinitival clause} 
\a Sue considers [that the world is round] to be a tragedy.
\a {[That the world is round]} seems to be a tragedy. 
\a \ljudge*It seems [that the world is round] to be a tragedy.
\a \ljudge*It was believed [that the world is round] to be a tragedy.
\a \ljudge*Mary is aware [that the world is round] to be a tragedy.
\a  \ljudge*Mary's belief [that the world is round] to be a tragedy.
\xe
\pex[aboveexskip=0pt]<predicatefronting> \ptxt{Predicate fronting in an infinitival clause}
\a<considerPred> Sue considers [even more important than linguistics] to be the fate of the planet.
\a {[Even more important than linguistics]} seems to be the fate of the planet. 
\a \ljudge*It seems [even more important than linguistics] to be the fate of the planet. 
\a \ljudge*It was believed [even more important than linguistics] to be the fate of the planet. 
\a \ljudge*Mary is aware [even more important than linguistics] to be the fate of the planet.
\a \ljudge*Mary's belief [even more important than linguistics] to be the fate of the planet. 
\xe

\noindent Other more complex arguments reinforce the claim that the distribution of infinitival complements reflects conditions on exfoliation rather than factors such as subject case licensing traditionally claimed to be at work in these paradigms. The reader is referred to \citet{Pesetsky:2019aa} for these arguments, as well as for answers to certain obvious questions raised by this proposal that I will not attempt to answer here, such as the analysis of English infinitives introduced by \textit{for}. To keep the discussion simple, let us also imagine that Control infinitives, like their counterparts created by Raising, also involve a probe-goal interaction between the embedded subject occupying the specifier of \textit{to}P and an some element in the higher clause, as in the movement theory of control (Bowers \citeyear[675 ff.]{Bowers1973}, \citeyear{bowers}, Wehrli \citeyear[115--131]{Wehrli1980}, \citeyear{Wehrli1981}, \citealt{Hornstein1999b}), though \citet{Pesetsky:2019aa} presents an alternative possibility that I will not address here.}

A second type of argument advanced for this proposal is the fact that it generalizes to configurations in which a probe finds a goal occupying a position higher than the specifier of \textit{to}P. When this happens, the embedded clause is once again reduced by exfoliation, but now to something larger than an infinitive. This provides an account of the well-known \textit{complementizer-trace effect} (Perlmutter \citeyear{Perlmutter1968,Perlmutter:1971}; see \citealt{pesetsky2015complementizer} for a survey of subsequent discussion), in which an otherwise possible overt complementizer is obligatorily absent when a subject or subject-like phrase is extracted, leaving behind a clause that lacks its complementizer but remains finite:\largerpage

\noindent \pex<Complementizer-trace-Effect> 
\ptxt{Complementizer-trace effect}
\a<ObjectExtraction> Who do you think (that) Sue met \gap.
\a<SubjectExtraction> Who do you think (*that) \gap met Sue.

\smallskip
\a Exactly how much more important than linguistics did she say (that) the fate of the planet was~\gap? 
\a Exactly how much more important than linguistics did she say (*that) \gap was the fate of the planet? 
\xe

\noindent If the overall proposal is correct, the explanation for complementizer-trace effects falls together with an explanation for why nonfinite clauses should exist in the first place, uniting two phenomena previously viewed as quite distinct.

Finally and most significantly in the present context, \citet{Pesetsky:2019aa} presents \textit{derivational opacity} arguments for the proposal that infinitival clauses are born full and finite, and become infinitives during the course of the syntactic derivation. The core of one such argument can already be seen in the predicate fronting examples (\getref{predicatefronting}a--b). A traditional account of infinitival clauses that attributes to the case filter the unacceptability of the starred examples in (\getref{nominalsubjects}) not only struggles to explain the parallel effect in (\getref{CPsubjects}) and (\getref{predicatefronting}) but also struggles to explain how the postverbal nominal passes the case filter at all, since there is neither an accusative-assigning verb in its immediate vicinity, nor an available instance of finite T. The much-discussed Icelandic phenomenon exemplified by (\getref{DATNOMR2}) below presents the same problem in a stronger form. Here an infinitival from which Raising has taken place has a quirky case-marked subject. Not only is the postverbal object of the embedded clause acceptable, but it bears \Nom{} case morphology, with no visible instance of finite T that could have entered an agreement relation with the it that results in \textsc{nom} case: 

\pex<DATNOMR2>
\begingl
\glpreamble Quirky subject + \Nom{} object in an infinitival complement (Icelandic)//
\gla Læknirinn\ix{\textnormal{i}} telur barninu (í barnaskap sínum\ix{\textnormal{i}}) hafa batnað veikin.//
\glb the.doctor.\Nom{} believes the.child.\Dat{} (in foolishness his) have.\Inf{} recovered.from the.disease.\textbf{\Nom{}}//
\glft `The doctor\ix{\textnormal{i}} believes the child  (in his\ix{\textnormal{i}} foolishness)  to have recovered from the disease.'\trailingcitation{(\citet[242]{Yip1987a}, adapted}) //
\endgl
\xe

\noindent This is of course one of the phenomena that inspired \citet{Yip1987a} and \citet{Marantz1991} to abandon the proposal that \Nom{} depends on agreement with finite T in the first place (and with it, the proposal that nominals need to be licensed at all), in favor of a theory in which \Nom{} morphology is a default assigned to a nominal in an appropriate position when other case rules fail to apply. 

On an exfoliation account, however, licensing and \Nom{} morphology in examples like (\getref{DATNOMR2}) pose no problem for theories that posit a connection between \textsc{nom} case assignment and finite T.\footnote{One might reject this connection for other reasons, of course, but it does appear to be cross-linguistically robust in environments without the special characteristics of (\getref{predicatefronting}a--b) and (\getref{DATNOMR2}) and others for which an exfoliation derivation might be plausible.} Since the embedded infinitival clause started its life full and finite, the postverbal nominal could enter an agreement relation with finite T within the embedded clause, just as it would in a clause that remained finite throughout the derivation. The interaction between the quirky subject and a Raising probe in the higher clause triggers exfoliation, which left the embedded clause infinitival, but this operation came later than the licensing of the postverbal nominal and the assignment of \textsc{nom} case to it. On this view, the presence of \Nom{} morphology on the postverbal subject is a relic of its earlier life as a nominal in an agreement relation with finite T, an instance of derivational opacity, since the T that played the crucial role in these events has been eliminated in the course of the derivation. \citet{Pesetsky:2019aa} provides independent evidence for this proposal from the observation that the anaphor-agreement effect blocks a reflexive as the postverbal object in these constructions, despite the absence of any visible agreement.

The derivational approach to the existence of nonfinite clauses faces an important problem, however, concerning semantic interpretation. All things being equal, we might expect to find straightforward derivational opacity arguments in this domain as well. Just as \Nom{} morphology is preserved on the postverbal subject in (\getref{DATNOMR2}) even after the T with which it agreed and from which it (arguably) received \Nom{} has been eliminated, so we might expect the various tenses and modals available to finite clauses to continue to show semantic signs of their former presence. In fact, however, tense interpretation in infinitival clauses is severely restricted, in a manner illuminated and clarified by \citet{Wurmbrand:2014aa}. Why do infinitival clauses not show the full range of semantic possibilities available to finite clauses? If they did, it would furnish a semantic derivational opacity argument analogous to the morphosyntactic arguments that support the exfoliation theory of infinitivization.

One response might be to reject the derivational view of infinitivization (in favor of a more standard approach according to which nonfinite clauses are generated as such, and problems like those raised above are solved in some other way). Another response might propose that some aspects of semantic interpretation apply late in the derivation, after exfoliation has taken place. This is a logical possibility mentioned in \citealt{Pesetsky:2019aa}, but entails that semantic interpretation does not apply entirely cyclically during the course of the syntactic derivation, contradicting results such as those reported by \citet[66--73]{Fox1999} and others that argue that semantic interpretation is strongly cyclic, fully interspersed with the syntactic derivation. 

A variant of this second response might acknowledge that semantic interpretation is interspersed with the syntactic derivation, but permit the semantics of a constituent targeted by exfoliation to be \textit{revised}, deleting or altering those components of meaning that owed their existence to material deleted by the exfoliation operation.\footnote{I am grateful to Carlo Geraci and Michelle Sheehan for helping me clarify the reasoning in this paragraph.} Phonological interpretation might work this way as well, if it too is fully interspersed with the syntactic derivation. If, for example, a fully built CP undergoes phonological interpretation, only to lose its outer layers to exfoliation later in the derivation, we must entertain a theory according to which cyclic phonological interpretation is subject to later revision, and it would not be surprising to learn that semantic interpretation follows a similar pattern. As I noted in the introduction to this paper, \citet{Wurmbrand:2014aa} argues that English nonfinite clauses are deeply \textit{tenseless}, a proposal that might seem to fit this variant response quite neatly. Semantic tenselessness is a natural outcome if the elimination of TP by exfoliation triggers elimination of the semantics that TP introduced.\footnote{Michelle Sheehan (personal communication) makes the interesting observation that under Chomsky's \citeyearpar[13, ex. 9]{Chomsky2001} proposal concerning the timing of phonological and semantic interpretation, one might be able to adopt this variant without any notion of revision. According to this proposal, a phase is spelled out (its contents transferred to PF and to LF) and rendered impermeable to processes such as movement only when the \textit{next} phase head is merged. On this view, a clausal complement to V will not be subject to spell-out and rendered impermeable until the higher \textit{v}P has been completed. By this time, exfoliation of that clausal complement will already have taken place, since the relevant triggers are all contained in that\textit{ v}P phase. The entire raison d'être of exfoliation as developed in \citet{Pesetsky:2019aa}, however, rests on the impermeability of non-edge positions within the embedded clause to movement across the clause boundary. Exfoliation takes place precisely so as to leave the goal for the \textit{v}P-internal probe at the edge of what remains of the embedded clause, rendering it accessible for movement triggered by that probe. Though one can imagine reformulations that might render versions of the two proposals compatible, they are clearly at odds with respect to the status of the pre-exfoliation embedded clause.}

Nonetheless, I will argue for an entirely different solution to this puzzle here. I will suggest that semantic interpretation is \textit{not} revised in the wake of exfoliation, and thus that the interpretation of nonfinite clauses is always an interpretation inherited from derivational period when it was full and finite. On this view, the semantic effects charted by Wurmbrand are not indications of tenselessness, and in fact, are not restrictions on the semantics of infinitival complements at all. They are actually PURE effects: limitations on a hearer's ability to ascribe semantic properties to phonologically suppressed material, when reverse-engineering the derivation behind a speaker's utterance. I believe this alternative is more attractive because the semantics of nonfinite clauses (in English at least) do not actually point in the direction of tenselessness. The mapping among the semantic possibilities available to nonfinite and finite clauses is indeed complex (as we shall see). Nonetheless, the set of temporal and modal interpretations available to nonfinite clauses appears to be a \textit{proper subset} of the set of interpretations available to tensed finite clauses, its tense (and modal) semantics always corresponding to that of some type of tensed clause, with no sui generis possibilities that might indicate total tenselessness. I take this to be an important observation that may favor the approach developed below over the approach developed by \citet{Wurmbrand:2014aa}.

On the speaker's side of the story, I therefore suggest that in principle, any tense or modal in T may be eliminated by exfoliation in the process of generating an infinitival clause. Crucially, the semantics provided by this tense or modal remains intact and unrevised through the end of the derivation. It is the \textit{hearer}'s side of the story that imposes the restrictions documented by Wurmbrand and discussed below. Though in theory any tense or modal can be exfoliated away in the course of the speaker's derivation, in practice a hearer can posit only those tenses and modals to the embedded clause that are semantically \textit{minimal} and compatible with their environment, in the cases at hand, compatible with the selectional properties of the higher predicate and the ubiquitous availability of the utterance time as an anchor for tense. This is the source of our sense that infinitival clauses are inherently restricted in the tense and modal specifications that they can express. Not every meaning producible by a speaker's derivation can be reverse-engineered and attributed to it by the hearer. 

\begin{sloppypar}
Though the proposal advocated here is essentially the opposite of Wurmbrand's (interpretation as tensed vs. deep tenselessness), my presentation will be entirely derivative of the findings reported in \citet{Wurmbrand:2014aa}, including her classification of the phenomena she discovered. Following Wurmbrand, I first consider future infinitives (complements to verbs like \textit{want} and \textit{decide}) and then propositional infinitives (complements to verbs like \textit{claim} and believe), followed by a brief discussion of infinitival clauses understood as simultaneous in tense with the clause in which they are embedded (complements to verbs like \textit{manage} and \textit{seem}). We are able to reach such different conclusions from the same set of findings because we pursue different strategies of argumentation concerning these findings. These can be summarized as follows:\largerpage
\end{sloppypar}

\pex
\ptxt{Strategies of argumentation}
\a \citet{Wurmbrand:2014aa}: The behavior of future, propositional, and simultaneous infinitives cannot be exclusively identified with any single value that tense may bear in a corresponding finite clause. These complements do display behavior consistent with tenselesness.  Therefore they are deeply tenseless.
\a This paper: The behavior of future, propositional, and simultaneous infinitives may be identified with the \textit{union of behaviors} expected from all the semantically minimal values for tense that a hearer can posit when unambitiously reverse-engineering the pre-exfoliation portion of the speaker's derivation (as required by PURE).  Therefore they are not deeply tenseless.
\xe

Crucially, if the alternative advocated in this paper is correct, we do have a derivational opacity argument for tense semantics after all, since the tense interpretation of an infinitive does reflect the pre-exfoliation tense properties of a T that is later deleted, a fact obscured by the severe restrictions imposed on the hearer by PURE. This will leave us with one apparent discrepancy between the outcome of PURE for \textsc{aux}-drop and its outcome for infinitivization, but this discrepancy follows from the difference between (1) non-pronunciation of syntactically present structure (\textsc{aux}-drop, following Fitzpatrick), and (2) actual deletion of syntactic structure by exfoliation.

\subsection{PURE and future infinitives}

Following Wurmbrand, I consider first the class of infinitival complements with future (or irrealis) semantics, like the Raising (ECM) complement (\getfullref{futureinf.a}) and the Control complement in (\getfullref{futureinf.b}):

\pex<futureinf>
\ptxt{Future infinitives}
\a<a> Yesterday, Mary wanted/needed John to leave tomorrow.
\a<b> Yesterday, Mary decided/wanted/planned to leave tomorrow.
\trailingcitation{(\citealt[408, adapted from ex. 6]{Wurmbrand:2014aa})}
\xe

\noindent Future infinitives have often been described as “tensed” in the literature since Stowell (\citeyear{Stowell:1981}, 40ff.; \citeyear{Stowell1982b}). Such theories entail that these infinitives contain in some fashion a silent variant of English \textit{will} or \textit{would}. Wurmbrand sought to dispel this idea, by demonstrating that the properties of future infinitives are not identical to those of either English \textit{will} or \textit{would}, which she analyzes (following Abusch \citeyear{Abusch:1985tm}, \citeyear{Abusch1988}) as bimorphemic auxiliary verbs consisting of an abstract modal \textit{woll} plus \textsc{present} tense (\textit{will}) or \textsc{past} tense (\textit{would}). She argues at length that the properties of future infinitives favor a theory according to which such infinitives are \textit{deeply tenseless}. Specifically, they contain \textit{woll} but no specification for \textsc{past} or \textsc{present} whatsoever. If her conclusions are correct, future infinitives present the exact opposite of a derivational opacity argument for the syntactic derivation of nonfinite clauses by exfoliation. They present a derivational conundrum for an exfoliation theory. If a future infinitive was indeed tensed in its derivational youth, as the exfoliation proposal claims, the theory must somehow ensure that no residue of its tensed beginnings survives in the semantics of its final infinitival form. Below, I survey these arguments and suggest an alternative.

Wurmbrand first contrasts the behavior of future infinitives with the behavior of present-tense \textit{will}. \textit{Will} places a situation in the absolute future with respect to the utterance time, while a future infinitive may pick out a time that merely follows the time of the higher clause: 

\pex<absolutefuture>
\ptxt{Future infinitive → relative future vs. \textit{will} → absolute future}
\a<a> Leo decided a week ago [that he will go to the party (*yesterday)].
\a<b> Leo decided a week ago [to go to the party yesterday].
\trailingcitation{(\citealt[414, ex. 22]{Wurmbrand:2014aa})}
\xe

\noindent Sequence of tense (SOT) effects also reveal ways in which future infinitives do not behave as though they contain \textit{will}. Following \citet{Ogihara1996}, Wurmbrand assumes that sequence of tense effects are the result of a rule that deletes a tense at LF, if it is in the immediate scope of another tense with the same value, and binds the situation time of the lower clause to that of the higher clause. For this reason, as she notes, the embedded clause in \textit{We found out that Mary was happy} does not require the time of the embedded clause to precede the time of finding out, but permits the time of the embedded clause to overlap that time, as a consequence of the higher occurrence of \textsc{past} deleting the embedded occurrence. 

As she also notes, citing Ogihara, the sequence of tense rule applies in the same way to \textsc{present} in a sentence like \textit{John will see the unicorn that is walking}, yielding a possible interpretation under which the unicorn's walking takes place at the seeing time, not utterance time. Crucially, it is the \textsc{present} component of \textit{will} that triggers the deletion at LF of embedded \textsc{present} (resulting of the binding of the lower situation time by the higher). 

Wurmbrand now considers the three-clause structure in (\getfullref{SOTwill.a}) in which \textsc{past} in the highest clause is separated from \textsc{past} in the lowest clause by an intervening clause containing \textit{will}, which as we have seen contains \textsc{present}. As predicted, \textsc{past} in the lowest clause cannot be deleted, since the intermediate clause contains \textsc{present}, and the closest higher instance of \textsc{past} is in the highest clause. Crucially, however, replacing \textit{will} in the middle clause with a future infinitive in (\getfullref{SOTwill.b}) yields a different result, the possibility of an SOT interpretation of the embedded clause, which Wurmbrand interprets as directly triggered by \textsc{past} in the highest clause. \textsc{past} in the highest clause can trigger SOT deletion of \textsc{past} in the lowest clause, Wurmbrand suggests, because the intermediate clause is truly tenseless, and in particular does not contain a null counterpart to the \textsc{present}-tense \textit{will} in (\getfullref{SOTwill.a}). 

\pex<SOTwill>
\ptxt{\textit{Will} blocks SOT deletion of \textsc{past}, but future infinitive does not}
\a<a>
{[}\textsubscript{\textsc{past}} John promised me yesterday {[}\textsubscript{\textit{will}} that he \dotuline{\textnormal{will}} tell his mother tomorrow {[}\textsubscript{\textsc{past}} that they were having their last meal together{]]]}.\smallbreak  
*\textit{telling time = meal time}
\a<b> {[}\textsubscript{\textsc{past}} John promised me yesterday {[}\textsubscript{\textsc{fut infin}} \dotuline{to} tell his mother tomorrow {[}\textsubscript{\textsc{past}} that they were having their last meal together{]]]}.\smallbreak 
✓\textit{telling time = meal time}
\trailingcitation{(\citealt[415, ex. 24, 25a]{Wurmbrand:2014aa}), building on \citealt{Abusch1988})}
\xe

Wurmbrand next contrasts the behavior of future infinitives with the behavior of past-tense \textit{would}. As she notes, an idiosyncrasy of \textit{would} is the fact that (except in the consequent of a conditional) it is permitted only in an SOT environment where its \textsc{past} feature can be deleted by \textsc{past} in the immediately containing clause. It is therefore blocked in a main clause (except as the consequent of a conditional missing its antecedent, e.g. *\textit{Yesterday, I would be king}), and blocked in an embedded clause if the immediately containing clause is not \textit{\textsc{past}}, as illustrated in (\getfullref{wouldrestriction.a}), where the higher clause contains \textsc{present}-tense \textit{will}. Crucially, a future infinitive is possible in the same environment where \textit{would} is blocked, as (\getfullref{wouldrestriction.b}) shows: 

\pex<wouldrestriction>
\ptxt{\textit{Would} is excluded in non-\textsc{past} SOT environment, but future infinitive is not}
\a<a> \ljudge{*}{[}\textsubscript{\textnormal{will}} John will promise me tonight {[}\textsubscript{\textit{would}} that he \dotuline{\textnormal{would}} tell his mother tomorrow \ldots {]]}
\a<b> {[}\textsubscript{\textit{will}} John will promise me tonight {[}\textsubscript{\textsc{fut infin}} \dotuline{to} tell his mother tomorrow {[}\textsubscript{\textsc{past}} that they were having their last meal together{]]]}.
\trailingcitation{(\citealt[415, ex. 29a, 30a]{Wurmbrand:2014aa})}
\xe

\noindent Furthermore, as Wurmbrand also notes, the most embedded clause in (\getfullref{wouldrestriction.b}) lacks any SOT reading that could permit the meal-eating time to be identical with the telling time, as we would expect if the future infinitive could be understood as a silent version of \textsc{past}-tense \textit{would} (perhaps immune for some reason to the restriction to \textsc{past} SOT environments). It is therefore clear that the future infinitive cannot be uniformly identified as a silent version of \textit{would} any more than it can be uniformly identified as a silent version of \textit{will}. Once again, Wurmbrand concludes that future infinitives are simply tenseless, containing an untensed \textit{woll}. 

In fact, however, another interpretation of these findings is possible, mentioned by Wurmbrand herself, who attributes the observation to “David Pesetsky and a reviewer” (p. 440).\footnote{I have no memory of making this observation.} Although the future infinitive does not behave \textit{uniformly} like either \textit{will} or like \textit{would}, wherever it fails to behave like \textit{will} it behaves like \textit{would}, and wherever it fails to behave like \textit{would}, it behaves like \textit{will}. 

Consider first the availability of SOT deletion of \textsc{past} in the lowest clauses of (\getref{SOTwill}a--b) , impossible if the middle clause contains \textit{will}, but possible if the middle clause contains a future infinitive. Wurmbrand took these data to show that the middle clause is untensed, but they could equally well show that the middle clause contains a silenced \textsc{past}-tense \textit{would}:

\pex<SOTwould>
\ptxt{Substituting \textit{would} for \textit{will} in (\getfullref{SOTwill.a}) permits the missing reading}

{[}\textsubscript{\textsc{past}} John promised me yesterday {[}\textsubscript{\textit{would}} that he \dotuline{would} tell his mother tomorrow {[}\textsubscript{\textsc{past}} that they were having their last meal together{]]]}. \smallbreak
✓\textit{telling time = meal time}
\xe

On this view, it is \textsc{past} in the middle clause, not \textsc{past} in the highest cause, that deletes \textsc{past} in the lowest clause, yielding the SOT interpretation under which the telling time and the meal-eating time are identical. Note that the \textsc{past} feature of this silence \textit{would} will itself be deleted under the influence of \textsc{past} in the highest clause, but that is exactly what overt \textit{would} requires. Assuming that SOT applies cyclically, we have an instance of LF derivational opacity, since the tense responsible for deleting \textsc{past} in the lowest clause is not present in the final LF representation. 

Now consider the availability of the future infinitive in (\getfullref{wouldrestriction.b}) in an environment where \textit{would} is blocked. Once again, though this possibility is compatible with Wurmbrand's view that the middle clause is untensed, it could equally well show that here the future infinitive contains a silenced \textsc{present-}tense \textit{will}, which is not blocked in this environment. And indeed, (\getfullref{wouldrestriction.b}) can be paraphrased with overt \textit{will} in the middle clause:\largerpage[1.75]

\pex[belowexskip=0pt,aboveexskip=.5\baselineskip]<SOTwould>
\ptxt{Substituting \textit{will} for \textit{would} in (\getfullref{wouldrestriction.a}) eliminates the star}

{[}\textsubscript{\textit{will}} John will promise me tonight {[}\textsubscript{\textit{will}} that he \dotuline{will} tell his mother \\tomorrow \ldots{]]}
\xe

The view that a future infinitive may be understood as containing either a silenced \textit{will} or a silenced \textit{would} is exactly what we expect under the exfoliation hypothesis for nonfinite clauses, according to which they are generated by Merge as full and finite CPs, with exfoliation responsible for stripping them of their CP and TP layers in the course of the derivation. On this view, all things being equal, the source of a future infinitive must be a finite clause with future content, but that content may in principle be either \textit{will} or \textit{would}. From this vantage point, the discovery that both possibilities are in fact instantiated comes as no surprise. Example (\getfullref{SOTwill.b}) is acceptable on an SOT reading because there is a derivation in which its middle clause was generated with \textit{would}, while (\getfullref{wouldrestriction.b}) is acceptable because there is a derivation in which its middle clause was generated with \textit{will}.

Now note that because \textit{would} (except as the consequent of a conditional) is idiosyncratically restricted to SOT environments, the two kinds of future modals that may be generated in a complement clause bear either \textsc{present} tense at LF (\textit{will}) or no tense whatsoever at LF, due to the tense-deleting action of SOT (\textit{would}). If these modals disappear in the course of the derivation as a consequence of exfoliation, yielding an infinitive, the hearer of such a clause faces a reverse engineering task not unlike that posed by an English \textsc{aux}-drop clause. In particular, the hearer's parser must assign content to the finite T of the derivational ancestor of the infinitival clause. If PURE is correct as stated in (\getref{Recoverability}), the hearer's options are tightly restricted, limited to “least semantically contentful possibilities compatible with the morphosyntactic environment”. 

Is the distribution and range of possible interpretations for a future infinitives compatible with PURE? If \textit{semantic selection} and \textit{binding} count as elements of the morphosyntactic environment relevant to PURE, the answer is yes. Assuming with Wurmbrand that \textit{will} and \textit{would} are the \textsc{present} and \textsc{past} tense forms, respectively, of an abstract morpheme \textit{woll}, we need to ask (1) whether PURE permits the positing of \textit{woll} in infinitival complement clauses where no form of the modal is visible, and (2) whether PURE permits positing both \textsc{present} and \textsc{past} in free alternation as the tense of this modal. I believe the answer is plausibly yes. 

If selection is a component of the morphosyntactic environment relevant to PURE, then the positing of an “ancestral” \textit{woll} in the complement to a verb like \textit{promise} can be justified by the semantic selectional properties of \textit{promise} and any other predicate that describes an attitude towards a future situation. \textit{Woll} adds no semantics to what is required by the morphosyntactic environment, and therefore should count as “minimal” in the sense relevant to PURE. 
 
What about \textsc{present}, the non-modal component of \textit{will}? Building on the proposals advanced in Section \ref{subsec:Reverse-engineering-an-aux-drop}, if \textsc{present} is a tense bound to the utterance time, this relation alone should license positing \textsc{present} as the tense specification of T in a future infinitive, without violating PURE. 

Finally, what about \textsc{past}, the non-modal component of \textit{would}? Continuing to build on the proposals advanced in Section \ref{subsec:Reverse-engineering-an-aux-drop}, a \textsc{past} specification for T that survived until LF should count as non-minimal, since it adds\textit{ anteriority} to the meaning of \textsc{present}. PURE should therefore prevent the hearer from positing ancestral \textsc{past} as part of the derivation of a future infinitive, all things being equal, with one important qualification. If an instance of \textsc{past} makes no semantic contribution at all because it is deleted by the SOT rule, positing such an instance of \textsc{past} will be perfectly compatible with the strictures of PURE. As Wurmbrand noted and as discussed above, \textit{would} is in fact restricted to SOT environments. It thus follows that the hearer's parser should be free to posit ancestral \textit{would} as an auxiliary verb of a future infinitive, just as suggested above.

Summarizing the crucial properties of the speaker and the unambitious
re\-verse-engineering hearer in this domain:

\pex
\ptxt{Speaker and hearer summary: Future infinitives}
\textit{Speaker}: Free to posit any content whatsoever for T of the embedded clause\\
\textit{Hearer (restricted by \textnormal{PURE})}:
\a Hearer posits \textit{woll} because it is selected by the higher verb. No other modal is possible.
\a Hearer may posit \textsc{present} as the pre-exfoliation tense of the future modal because it is semantically minimal (as we saw in discussing \textsc{aux}-drop), yielding \textit{will}.
\a Hearer may posit \textsc{past} as the pre-exfoliation tense of the future modal so long as it is semantically inert due to SOT (as is always the case with \textit{would}).
\xe

\subsection{PURE and propositional infinitives}\largerpage

I turn now to non-future infinitival clauses with propositional semantics, such Raising/ECM complements to verbs like \textit{believe} (e.g. \textit{She believes Mary to be the winner}) and control complements to verbs like \textit{claim} (\textit{She claimed to be the winner}). As Wurmbrand notes (cf. \citealt{Pesetsky1991}), these complements have aspectual properties strongly reminiscent of the English \textsc{present}, resisting eventive interpretation of simple VPs, as briefly discussed in Section \ref{subsec:Reverse-engineering-an-aux-drop} above:\footnote{Wurmbrand uses the term “episodic”, where I use “eventive” for consistency with other discussion. If there are crucial differences between these notions that might compromise the overall argument, I leave that issue for future research.}

\pex<eventive>
\ptxt{Eventive interpretation: propositional infinitives that pattern with English \textsc{present} tense}
\a<d> Bill knows German well. \tabto{3in}\textit{✓non-eventive}
\a<e> They believe Bill to know German well. 
\a<f> They claim to know German well.
\vspace{.5\baselineskip}
\a[label=d]<a> \ljudge*Mary sings in the shower right now.\tabto{3in}\textit{*eventive}
\a[label=e]<b> \ljudge*They believe Mary to sing in the shower right now.
\a[label=f]<c> \ljudge*They claim to sing in the shower right now.
\trailingcitation{(\citealt[431, adapted from ex. 55--56]{Wurmbrand:2014aa})}
\xe

\noindent The English \textsc{past} does license eventive interpretation, but infinitival complements to verbs like \textit{believe} and \textit{claim} cannot be understood as bearing \textsc{past} tense semantics (without the addition of \textsc{have}+-\textit{en}, discussed below), regardless of eventivity:

\pex<eventive2>
\ptxt{Propositional infinitives that may not be understood as \textsc{past} tense}
\a They knew German well when they were young.
\a \ljudge*They believe(d) Bill to know German when they were young.
\a \ljudge*They claim(ed) [to know German well when they were young].
\vspace{.5\baselineskip}
\a Mary sang in the shower yesterday at 8:00.
\a<b> \ljudge*They believe(d) Mary to sing in the shower yesterday at 8:00.
\a<c> \ljudge*They claim(ed) to sing in the shower yesterday at 8:00.
\xe

\begin{sloppypar}
Let us first consider these observations from an exfoliation perspective. If infinitival clauses like those in (\getref{eventive}\getref{eventive.e}--\getref{eventive.f}) are derived from full finite clauses, once again the hearer of such a complement must reverse-engineer the speaker's derivation, and posit a tense value for T in that clause. If PURE permits the hearer to posit ancestral \textsc{present} but not \textsc{past}, for the reasons just discussed, the contrasts in (\getref{eventive}) and (\getref{eventive2}) are immediately predicted. If the hearer posits ancestral \textsc{present}, it is no surprise that eventive interpretation is restricted just as it is in \textsc{present} tense clauses that have not been reduced to infinitives by exfoliation. Positing \textsc{past} is ruled out by PURE, since \textsc{past} is not semantically minimal as \textsc{present} is. 
\end{sloppypar}

Wurmbrand, however, presents an SOT environment in which infinitival complements like these behave differently from \textsc{present} tense finite clauses. The argument once again involves SOT in a three-clause structure in which the infinitival clause is the middle clause:

\pex<pregnancy1>
\ptxt{Propositional infinitives that appear not to block SOT}
\a {[}\textsubscript{\textsc{past}} A year ago, they believed Mary {[}\textsubscript{\textsc{prop infin}} to know {[}\textsubscript{\textsc{past}} that she was pregnant{]]]}.
\a {[}\textsubscript{\textsc{past}} A year ago, Mary claimed {[}\textsubscript{\textsc{prop infin}} to know {[}\textsubscript{\textsc{past}} that she was pregnant{]]]}.
\trailingcitation{(\citealt[433, ex. 59b, 59c]{Wurmbrand:2014aa})}
\xe

\noindent As Wurmbrand points out, the pregnancy time in the examples
of (\getref{pregnancy1}) may be understood as bound by
the believing/claiming time, a clear sign that the SOT rule has deleted
\textsc{past} in the embedded clause. This is of course not possible
if the infinitival middle clause is understood as containing \textsc{present},
since SOT deletes a lower tense under identity with the most
immediately superordinate tense. Wurmbrand concludes that it is the
\textsc{past} tense of the main clause that triggers deletion of the
\textsc{past} tense of the most embedded clause, and therefore the infinitival
middle clause must be viewed as tenseless.

Once again, however, the exfoliation/reverse engineering approach suggests an alternative. The contrasts in (\getref{eventive2}) show that a hearer cannot posit ancestral \textsc{past} for the infinitival complement of a verb like \textit{believe} or \textit{claim}, where \textsc{past} should survive until LF and receive its normal interpretation. If a hearer were to posit ancestral \textsc{past} in the middle clause of (\getref{pregnancy1}), it could be deleted by the SOT rule (since the tense of the higher clause is also \textsc{past}). When this happens, \textsc{past} in the middle clause will make no contribution of its own to LF interpretation, and will consequently count as a PURE-compatible choice for the hearer reverse-engineering the derivation of the middle clause. On this view it is \textsc{past} in the\textit{ middle} clause that triggers SOT deletion of \textsc{past} in the lowest clause (before it itself is deleted), not \textsc{past} in the highest clause. The logic is essentially the same as the logic behind our proposal for (\getfullref{wouldrestriction.b}).\footnote{Wurmbrand once again mentions the possibility that these infinitives might contain a “deleted \textsc{past}” (p. 432, fn. 25), but rejects this possibility as incapable of explaining “why the \textsc{past} must always delete, and how this is {[}im{]}possible {[}\textit{correcting a probable typo}{]} in non-SOT contexts (e.g. \textit{Julia claims to be pregnant} cannot mean `Julia claims that she was pregnant'). In the logical structure of the alternative suggested here, it is PURE that fills this explanatory gap. Undeleted (and unselected) \textsc{past} is not semantically minimal, and therefore cannot be posited by the (obligatorily unambitious) hearer in the process of reverse-engineering the derivation that produced an infinitival complement by exfoliation.} 

Wurmbrand notes a related contrast between infinitival complements to verbs like \textit{believe} and \textit{claim} amenable to the same alternative view. In examples like (\getref{pregnancy2}\getref{pregnancy2.a}--\getref{pregnancy2.b}), \textsc{present} embedded under \textsc{past} receives an obligatory \textit{double-access reading}, according to which Julia's pregnancy held at both the believing/claiming time (five years ago) and at utterance time (now), which is biologically impossible. The infinitival complements in (\getref{pregnancy2}\getref{pregnancy2.c}--\getref{pregnancy2.d}), by contrast, do not require a double-access interpretation, and permit the pregnancy time to be identified with the believing/claiming time. I thus cannot assume that these infinitival clauses are derived with any form of \textsc{present}:

\pex<pregnancy2>
\ptxt{Propositional infinitives that do not require double access reading (unlike \textsc{present})} 
\a<a>  \ljudge\#Five years ago, it was believed that Julia is pregnant.
\a<b>  \ljudge\#Five years ago, Julia claimed that she is pregnant.
\a<c>  Five years ago, Julia was believed to be pregnant.
\a<d>  Five years ago, Julia claimed to be pregnant.
\trailingcitation{(\citealt[432, ex. 58]{Wurmbrand:2014aa})}
\xe

\noindent As before, Wurmbrand concludes that these infinitives are deeply tenseless. Once again, however, the exfoliation/reverse engineering alternative permits these clauses to contain ancestral \textsc{past}. Since this instance of \textsc{past} is deleted by SOT, its presence may be posited in the reverse-engineering process without violating the strictures of PURE. Note that in the end, an infinitival clause that started its life with its tense specified as \textsc{past} ends up tenseless, just as in Wurmbrand's theory. The crucial difference is derivational. I am not proposing that infinitival clauses are intrinsically tenseless. Under the analysis suggested here, some are interpreted as containing \textsc{present}, since that tense is minimal, even though others do end  up truly tenseless, thanks to SOT deletion of \textsc{past}. 

\pex
\ptxt{Speaker and hearer summary: propositional infinitives under verbs like \textit{believe} and \textit{claim}}\medbreak
\textit{Speaker:} Speaker is free to posit any content whatsoever for T of the embedded clause (as before).\footnote{By ``any content whatsoever", I mean any content compatible with the rules that govern speaker's derivations.  Thus, for example, as Michelle Sheehan notes, the fact that a verb such as \textit{plan} requires a semantically future complement will impel the speaker to include a form of \textit{woll}. I should also note that some verbs impose post-exfoliation selectional requirements that reject derivations in which exfoliation has not created a nonfinite clause, as discussed in \citet{Pesetsky:2019aa}. Such requirements also restrict the speaker's derivation.}\\
\textit{Hearer (restricted by PURE):}
\a Hearer may not posit a modal because none is selected.
\a Hearer may posit \textsc{present} as the pre-exfoliation tense of the future modal because it is semantically minimal (as we saw in discussing \textsc{aux}-drop), yielding \textit{will} (as before).
\a Hearer may posit \textsc{past} as the pre-exfoliation tense of the future modal so long as it is semantically inert due to SOT (as is always the case with \textit{would}) (as before).
\xe

\subsection{Why do propositional infinitives show only one side of the factative
effect}

\label{subsec:An-important-unsolved} I have suggested that \textsc{aux}-drop and infinitival complementation tell a unified story about the effects of PURE. On one important point, however, the two discussions seem to point in different directions. In this section, I will suggest a way to reconcile them, though much remains to be worked out. 

The factative effect for \textsc{aux}-drop permits a silenced T to be understood by a hearer as \textsc{past} when the verb phrase it embeds is eventive, but not when it is a non-eventive. I suggested in Section \ref{subsec:Reverse-engineering-an-aux-drop} that \textsc{past} is available with an eventive verb phrase precisely because \textsc{present} is independently blocked with eventive predicates (unless they are understood as habitual or generic). For this reason, PURE permits the hearer to reverse-engineer a derivation in which the tense of the unpronounced auxiliary verb in C has the value \textsc{past}. This is the minimally semantically contentful choice compatible with the morphosyntactic environment. Why then do we not find a similar effect with propositional infinitives, where the same logic should permit a \textsc{past} interpretation for the embedded infinitival clauses of examples like (\getref{eventive2}\getref{eventive2.b}--\getref{eventive2.c})? 

\pex<puzzle>
\ptxt{\textsc{aux}-drop vs. propositional infinitive \textsc{past}-tense possibilities}
\a
\begingl
\gla You see John yesterday? \nogloss{\tabto{2.8in} (\textsc{aux}\textit{-drop})} //
\glft `Did you see John yesterday' //
\endgl 
\a 
\begingl 
\gla \ljudge*We believed Mary to see John yesterday. \nogloss{(\textit{propositional infinitive})} //
\glft \textit{intended}: `We believed that Mary saw John yesterday.' //
\endgl
\a
\begingl
\gla  \ljudge*Sue claimed to see John yesterday. //
\glft \textit{intended}: `Sue claimed that she saw John yesterday.' //
\endgl
\xe

An important clue may lie in a fact pointed out to me by Susi Wurmbrand (personal communication). In propositional infinitives like those under discussion here, simple \textsc{past} can actually be overtly expressed by the use of auxiliary verb \textit{have} plus the past participle, a combination that is obligatorily interpreted as perfect tense in clauses that remain finite:

\pex<eventive3>
\ptxt{Propositional infinitives in which \textit{have}+participle → \textsc{past}}
\a<a> They believe(d) Mary to have seen John yesterday at 8:00.
\a<b> They claim(ed) to have sung in the shower yesterday at 8:00.
\a<c> They believe(d) [Bill to have known German when they were young].
\a<d> They claim(ed) [to have known German well when they were young].
\xe

\noindent Independent of the puzzle of (\getref{puzzle}), the facts in (\getref{eventive3}) present an additional challenge to an exfoliation approach to non-finite clauses, since they display another unexpected difference in the semantics of finite clauses and their infinitival counterparts. I suggest that solving the puzzle of (\getref{eventive3}) may help solve the puzzle of (\getref{puzzle}) as well.

The nature of the English perfect tense is a hotly debated topic, but it appears that one of the several hypotheses still in the running (recently defended by \citealt{KlechaPerfect}) is the claim that auxiliary \textsc{have-}\textit{en} is a realization of \textsc{past}, yielding present perfect interpretation when the T that selects it is \textsc{present}, and pluperfect interpretation when that T is \textsc{past (}see \citealt[Section 3 (esp. 3.1)]{GrOnn:2021aa}, for discussion and summary). Suppose we accept this proposal. We must now ask why \textsc{have-}\textit{en} cannot be used as the sole tense-denoting element in a finite clause, which would incorrectly permit a sentence like \textit{They have seen John} to be understood as a simple \textsc{past}-tense utterance, rather than a perfect. Let us imagine that an English clause must obey the following surface filter on the featural content of T:

\pex<filter>
\ptxt{T-valuation filter}

* T unless specified for \textsc{past} or \textsc{present}.
\xe

\noindent  In a clause that contains T throughout the derivation, \textsc{have}-\textit{en} will never be able to serve as the sole bearer of tense. In any such clause, T must be \textsc{past} or \textsc{present} so as to not violate (\getref{filter}).\footnote{In constructions in which \textsc{have}-\textit{en} is embedded under an epistemic modal, its interpretation as \textsc{past} is extremely salient, e.g. \textit{Sue must have seen John} \textit{yesterday at 8:00}. \textsc{Have-}\textit{en} is not the sole bearer of tense here, however. Though \textit{must} does not show a morphologically overt \textsc{present\textasciitilde past} alternation like \textit{can\textasciitilde could}, we may presume that it is specified as \textsc{present}, and that (\getref{filter}) is therefore satisfied. I am grateful to Asia Pietraszko (personal communication) for raising this point. As Athulya Aravind (personal communication) notes, future perfect constructions make the same point even more clearly, e.g. \textit{Sue will have seen John yesterday at 8:00}. Here of course, we have independent evidence that \textit{will} includes a second instance of tense (\textit{woll}\,+\,\textsc{present}); cf. \textit{They claimed that Sue would have seen John by then} (with \textit{woll}\,+\,\textsc{past}).} The combination of T with \textsc{have-}\textit{en} will thus produce the semantics of pluperfect or perfect tense, depending on the value of T chosen.\footnote{The SOT rule does delete the \textsc{past} or \textsc{present} feature of T, and might be understood as producing a violation of the T-valuation filter as stated in (\getref{filter}), but the rule also binds the tense specification of the T that undergoes that rule to that of the T that triggered the rule. I will assume that for this reason a T that undergoes the SOT rule still counts as “specified” and does not violate (\getref{filter}).}

If, however, T is eliminated by exfoliation, then even if it was never valued \textsc{past} or \textsc{present}, it should not produce any sense of deviance: an instance of “salvation by deletion”. Such a derivation will produce no detectable violation of (\getref{filter}) precisely because the T that might have violated the filter is no longer present at the end of the derivation (after exfoliation). This is why \textsc{have}\textit{-en} may be the sole bearer of \textsc{past} in an infinitival clause, explaining the pure \textsc{past}-tense interpretation available to the embedded clauses of (\getref{eventive3}).\footnote{Perfect interpretation is also possible. For example, \textit{Mary lived here for many years} differs from \textit{Mary has lived here for many years} in implying that she no longer lives here, but \textit{I believe Mary to have lived here for many years} permits this reading.} 

Returning now to the puzzle in (\getref{puzzle}), we might explain the unavailability of a \textsc{past} interpretation for a propositional infinitive as the result of the hearer's strategy in (\getref{assumptions}):\largerpage

\pex<assumptions>
\ptxt{Constraint on hearer's ability to posit \textsc{past} in an infinitival clause}

Because \textsc{past} can be overtly expressed in an infinitival clause, the hearer will assume that speaker would have expressed it overtly (using \textsc{have}-\textit{en} if \textsc{past} interpretation had been intended, and will therefore never posit \textsc{past} as a value for T in the absence of \textsc{have}-\textit{en}.
\xe

\noindent This proposal conforms to the spirit of PURE, since it continues to enforce unambitiousness when the hearer considers positing unpronounced material, but does not directly follow from it as stated. I leave that as a problem for future work. Crucially, note that (\getref{assumptions}) concerns \textsc{past}-tense \textit{interpretation}, and therefore still does not prevent the hearer from positing a \textsc{past} specification for T that is deleted at LF by the SOT rule, as discussed in preceding sections.\footnote{Interestingly, I do not believe SOT applies to instances of \textsc{past} whose sole exponent is \textsc{have-}\textit{en}. \textit{Mary claimed to have been happy} lacks any reading in which happiness time overlaps claiming time. This too makes (\getref{assumptions}) irrelevant for cases in which I have proposed that the hearer may posit \textsc{past} as a pre-exfoliation value for T without violating PURE because it is deleted by SOT (and thus counts as semantically minimal). Why SOT fails to apply to \textsc{have}-\textit{en} in the first place, however, is unclear to me. Carlo Geraci suggests that SOT might be more generally constrained to apply only across a clause boundary. This would also explain why \textsc{past} T\,+\,\textit{\textsc{have-}}\textit{en}, e.g. \textit{Mary had written the letter already}, can only be understood as a pluperfect, and not a present perfect, as one might expect if the \textsc{past} semantics of \textsc{have-}\textit{en} could be deleted at LF by the SOT rule.}  

The most important question facing us now, however, concerns \textsc{aux}-drop. Why doesn't (\getref{assumptions})  prevent the hearer from posit \textsc{past} as an underlying value for T in an \textsc{aux}-drop clause with an eventive predicate, as it does in a propositional infinitive? The answer is in fact straightforward. In \textsc{aux}-drop as analyzed above (building on Fitzpatrick's proposals), T is never deleted. No exponent of T is heard by the hearer, true, but that is not because T has been deleted, but because T-to-C movement has applied and the contents of C were not interpreted by the phonology. In an \textsc{aux}-drop question, T is present throughout the derivation, so no end run around (\getref{filter}) occurs (no “salvation by deletion”). As a consequence, \textsc{have}-\textit{en} can never be the sole bearer of tense in an \textsc{aux}-drop clause, as illustrated by (\getref{perfectauxdrop}).

\pex<perfectauxdrop>
\ptxt{\textsc{have}-\textit{en} in \textsc{aux}-drop yields present perfect meaning only (not \textsc{past})}
\a \ljudge*Mary written that message yesterday at 8:00?\tabto{3in}{(\textit{attempt at} \textsc{past})}
\a Mary written that letter yet?\tabto{3in}{(\textit{present perfect})}
\xe

To summarize: though both entail non-pronunciation of an exponent of tense, \textsc{aux-}drop and infinitivizing instances of exfoliation are quite distinct processes. \textsc{Aux-}drop involves mere non-pronunciation of T in C, while infinitivizing exfoliation involves actual removal of T from the derivation. Their divergent behavior faced with an eventive predicate, seen in (\getref{puzzle}), follows from this difference. The T of clause that ends up non-finite may violate filter (\getref{filter}) without incurring any penalty. This in turn makes it possible for \textsc{have-}\textit{en} to produce a clause with simple \textsc{past} semantics, a possibility that prevents the hearer from positing \textsc{past} as the underlying specification for pre-exfoliation T in an infinitival clause without violating PURE, given (\getref{assumptions}). The T of an \textsc{aux-}drop clause is never deleted. Consequently filter (\getref{filter}) prevents \textsc{have-}\textit{en} from ever being the sole tense in the clause, (\getref{assumptions}) is never invoked, and \textsc{past} interpretation for an eventive VP is compatible with PURE. At the same time, though \textsc{aux-}drop and infinitivizing instances of exfoliation differ in this way, they impose a common burden on the hearer, who is faced in both cases with unpronounced instances of otherwise pronounced structure, hence their core similarity: the fact that \textsc{T} cannot be blithely posited as bearing the value \textsc{past} for a non-eventive predicate, unless later deleted by SOT, but can only be identified as \textsc{present} (or tenseless, when an end run around (\getref{assumptions}) is made possible by exfoliation).\footnote{Should tenselessness outcompete \textsc{present} as a value for T that may be assumed by a hearer reverse-engineering a propositional infinitive? If both count as maximally unambitious possibilities (total absence of value vs. value linked to always-available utterance time), the answer should be no, but some sharpening of the statement of PURE might be necessary.}

\subsection{Predicates imposing simultaneity}

Finally, we must take note of a third class of predicates discussed by Wurmbrand. These take infinitival complements, some of which have propositional semantics, but are fully compatible with eventive predicates and \textsc{past} interpretation of the complement, so long as the selecting predicate is itself \textsc{past} tense.

\pex<simul>
\ptxt{Predicates imposing their reference time on infinitival complement: \textsc{past}}
\a Yesterday, John tried/began . . . /managed . . . to sing (*tomorrow/*next week).
\a The bridge began/seemed to tremble (*tomorrow/*next week).
\trailingcitation{(\citealt[436, ex. 66]{Wurmbrand:2014aa})}
\xe

\noindent Substituting \textsc{present} tense for \textsc{past} eliminates
the possibilities seen in (\getref{simul}):

\pex<simul2>
\ptxt{Predicates imposing their reference time on infinitival complement: \\ \textsc{present}}
\a \ljudge*John seems to sing right now.
\a John seems to know German.
\trailingcitation{(cf. \citealt[437]{Wurmbrand:2014aa})}
\xe

Wurmbrand concludes that in the usage seen in (\getref{simul}), at least, these are “matrix predicates {[}that{]} impose their reference time as the reference time of the embedded infinitive” (p. 437). Once again, she proposes that these infinitival complements are deeply tenseless. Once again, the very fact that the matrix predicate imposes its reference time on the embedded infinitive can be understood as licensing the hearer to posit the corresponding tense specification as part of the pre-exfoliation derivation of the complement clause, as permitted by PURE.\footnote{Wurmbrand also discusses contexts in which predicates such as \textit{seem} behave more like \textit{believe}, which I will not summarize here. I believe the results of this discussion can be incorporated in the alternative advanced in this paper without change.} 

\section{Conclusions}

This paper has suggested an alternative to Wurmbrand's \citeyearpar{Wurmbrand:2014aa} analysis of English infinitives as inherently tenseless. This analysis is not merely compatible with the exfoliation approach to infinitivization that I proposed in \citet{Pesetsky:2019aa}, but also helps resolve a paradox lurking in the overall approach: the fact that infinitival clauses did not seem to present a derivational opacity argument for exfoliation from tense semantics parallel to the argument they offer from case morphology in examples like (\getref{predicatefronting}a--b) and (\getref{DATNOMR2}). While \textsc{nom} morphology survives the deletion of its finite T assigner, \textsc{past} tense and modal semantics in T does not.\footnote{There are a number of important elephants in the room, which I have ignored here (in keeping with normal usage of the elephant metaphor). To mention just two that demand immediate attention: 

Any instance of morphological syncretism, for example, raises issues for PURE.  Why is \textit{the sheep must leave} ambiguous between singular and plural \textit{sheep}, and likewise, why is \textit{They put up with it} ambiguous between \textsc{pres} and \textsc{past}? Material unpronounced as a consequence of a morphological paradigm must somehow be excluded from PURE calculations. 


Likewise, the phenomenon of ``surface anaphora'' (\citealt{Hankamer1976a}), whereby certain kinds of ellipsis demand an overt linguistic antecedent, is very much in the spirit of PURE (the antecedent licensing the otherwise non-minimal content posited as underlying the ellipsis), but recent work by \citet{Rudin:2019aa}, among others has called renewed attention to instances of ellipsis whose interpretation includes material unsupported by the overt antecedent, another challenge for PURE. I thank Peter Grishin, personal communication, for raising this issue.} 
If the proposal sketched here is correct, semantics does present a comparable derivational opacity argument in principle, but we are prevented from seeing it clearly by PURE, which prevents us as hearers from attributing non-minimal semantic content to a tense or modal that has been deleted by exfoliation. An additional argument for this approach came from the English \textsc{aux}-drop construction, where PURE resolves a key contradiction arising from Fitzpatrick's otherwise optimal account.

If this style of explanation is fruitful, we should ask whether there are other problems and paradoxes that might be resolved by permitting the class of producible derivations to misalign with the class of reverse-engineerable derivations, as I have proposed in this paper. I have suggested that certain problems might be resolved if certain derivations producible by the speaker may not be reproduced by the hearer. Perhaps other problems might be resolved in the opposite manner, if the reverse engineering process hosted by the hearer permits options that are in fact barred for the speaker. For example, imagine that when the hearer attempts to reproduce the syntactic derivation of the speaker, they are free to ignore EPP features, so that a raised nominal in the speaker's utterance might remain unraised in the reverse-engineered hearer's derivation. In this respect, the hearer's reverse engineering might show some ambition after all, in its reconstruction of the speaker's syntax, if not their semantics. This might be an approach to reconstruction phenomena worth exploring. Conversely, if one imagines that the hearer is free to assume EPP features not present in the speaker's derivation, one might be led to a new view of phenomena normally viewed as covert movement internal to the speaker's syntactic derivation. I will leave the question of whether these are (or are not) promising avenues of investigation open -- a topic, I hope, for future conversation and debate with the dedicatee of this volume.


\section*{Acknowledgments}
For useful discussion, I am grateful to the students in my Fall 2019 graduate seminar at MIT, where I first presented this material, and especially to my co-teacher and colleague Athulya Aravind.  I have also benefited from the comments of students and visitors at a January 2020 class at the University of Vienna and the 2021 Winter Virtual New York Institute; from questions following a presentation at the 13th Brussels Conference on Generative Linguistics; and from discussions with Jeroen van Craenenbroeck, Sabine Iatridou, Donca Steriade, Esther Torrego, and Susi Wurmbrand. I am particularly grateful to Carlo Geraci and Michelle Sheehan for valuable comments on an earlier draft which improved the discussion substantially. 

A December 2020 consultation on Facebook left me unclear whether ``reverse engineer" or ``reverse engineerer" is the better term.  Strong views were offered by friends and colleagues on both sides of the question.  I hope the dedicatee of this paper is content with my ultimate decision.

{\sloppy\printbibliography[heading=subbibliography,notkeyword=this]}

\end{document}
