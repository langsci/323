\documentclass[output=paper]{langscibook}
\author{Akihiko Arano\affiliation{University of Connecticut}}
\title[On the size of Spell-Out domains]
      {On the size of Spell-Out domains: Arguments for Spell-Out of intermediate projections}
\abstract{It is a widely held assumption in the Minimalist framework that Spell-Out domains are uniformly complements of phase heads. Contrary to this, the present paper proposes that a traditional intermediate or bar-level projection of phase heads constitutes Spell-Out domains if a phase head is in a spec-head agreement relation. I defend this proposal by examining cases of Spell-Out at CP-phase levels, discussing two types of phenomena which are sensitive to the size of Spell-Out domains. First, I discuss \posscitet{Richards:2010} Distinctness. It regulates the distribution of functional items within a Spell-Out domain. Case resistance effects observed by \citet{Stowell:1981} are investigated in terms of Distinctness and it is shown that the distribution of different types of clauses is correctly accounted for by the proposed analysis, but not by the standard account of Spell-Out. Second, I discuss ellipsis under the view that ellipsis sites correspond to Spell-Out domains. It is shown that this approach to ellipsis accounts for \posscitet{Merchant2001The-syntax-of-s} sluicing-COMP generalization and its exception when combined with the proposed analysis. }

\begin{document}
\SetupAffiliations{mark style=none}
\newcommand{\Xbar}[1]{#1$^\prime$}  
\newcommand{\sub}[1]{$_\mathrm{#1}$}
\newcommand{\wh}{\emph{wh}}
\newcommand{\Wh}{\emph{Wh}}
\newcommand{\shade}[1]{\colorbox[gray]{0.8}{#1}}
\maketitle


\section{Introduction}\label{aranosect1}
It has been widely assumed since \citet{Chomsky:2000} that the notion of phase plays a prominent role in the syntactic computation. One of its functions is to trigger the operation Spell-Out, which sends syntactic structures created by Merge in a bottom-up fashion to the sensorimotor interface. The application of Spell-Out makes its target inaccessible to syntactic operations at later stages (phase-impenetrability condition, \citealt{Chomsky:2000}) and, therefore, cyclicity effects are derived as a consequence of the multiple Spell-Out model adopted in the current Minimalist theorizing \citep{Uriagereka:1999a}. Moreover, the phase-impenetrability condition succeeds in reducing computational burden because search space can be limited by Spell-Out.

This paper makes a proposal on the size of Spell-Out, which is standardly assumed to be the complement of a phase head, as schematized in (\ref{arano1}):

\ea \label{arano1} \upshape 
\begin{forest}
[XP 
       [ZP] 
       [X$'$  [X]      [YP, name=yp     ] ] ]    
\node [left=0.25em of yp](p1){}; 
\node [above right=1em and 0.25em of yp] (p2) {};
\draw [overlay, thick] (p1) to[out=90, in=180] (p2);    
\node [right=-0.5em of p2] {$\rightarrow$ Spell-Out};
\end{forest}
\z

\noindent In (\ref{arano1}) X is a phase head and its complement, i.e. YP, constitutes a Spell-Out domain with the phase head and its specifier escaping being spelled out. The standard analysis assumes the size of Spell-Out domains not to change whether a phase head has a specifier or not. Instead, the paper proposes that a traditional intermediate projection undergoes Spell-Out if a phase head is in a spec-head agreement relationship, as shown in (\ref{arano2}):

\ea \label{arano2} \upshape 
\begin{forest}
[XP 
       [ZP, name=ZP] 
       [X$'$, name=yp  [X, name=X]      [YP     ] ] ]    
\node [left=0.25em of yp](p1){}; 
\node [above right=1em and 0.25em of yp] (p2) {};
\draw [overlay, thick] (p1) to[out=90, in=180] (p2);    
\node [right=-0.5em of p2] {$\rightarrow$ Spell-Out};
\draw[{Circle[length=2.5pt]}-{Circle[length=2.5pt]}, densely dotted, line width=0.8pt,shorten >= 1pt,shorten <= 1pt] (X.south)-- ++(south:1em)-| (ZP.south);
\node [below=1em of X] {agreement};
\end{forest}
\z


\noindent I empirically  motivate this proposal by showing that it accounts for cross-con\-structional and cross-linguistic patterns related to Spell-Out at CP-phase levels.

First, this paper discusses  \posscitet{Richards:2010} Distinctness condition. Roughly put, it prevents two nodes that are of the same functional type from being in a single Spell-Out domain. Therefore, the size of Spell-Out domains is crucial. Case resistance effects observed by \citet{Stowell:1981} are examined and the distributional properties of different types of clauses are shown to fall out from the present analysis. 


Second, I examine ellipsis under the view that it is a null form of Spell-Out, that is, ellipsis arises when Spell-Out domains receive no phonological realizations \citep{Gengel2006Phases-and-elli, Gengel2009Phases-and-elli, Craenenbroeck2010The-syntax-of-e, ,bovskovic2014now,Wurmbrand2017Stripping-and-t}.  It is shown that my proposal accounts for \posscitet{Merchant2001The-syntax-of-s} sluicing-COMP generalization and, potentially, its exception. 

This paper is organized as follows. Section \ref{aranosect2} presents my proposal and how it is feasible in the current theory of syntax. Section \ref{aranosect3} shows consequences my proposal brings to the Distinctness condition. Section \ref{aranosect4} aims to account for the sluicing-COMP generalization.  Section \ref{aranosect5} is a conclusion.


\section{Background and proposal}\label{aranosect2}
In the early Minimalist program \citep{Chomsky:1995}, the operation Merge is responsible for identifying labels and therefore labels are parts of the syntax. Thus, applied to two objects $\upalpha$ and $\upbeta$, Merge forms a new object K, of the form \{$\upgamma$, \{$\upalpha$, $\upbeta$\}\}, where $\upgamma$ is its label. This form of Merge is no longer available in \citet{Chomsky:2013, Chomsky:2015}, where Merge is defined in the simplest form: Merge ($\upalpha$, $\upbeta$) = \{$\upalpha$, $\upbeta$\} (see also \citealt{Collins:2002a}). Since we do not have labels in syntax, we cannot have syntactic notions that are defined in terms of labels, such as complement, specifier, or intermediate/maximal projection. In this context, it is impossible to state, for example, that maximal projections, but not intermediate projections, can be a target of syntactic operations. Since we do not have the distinction between maximal and intermediate projections due to the lack of labels, we cannot refer to only one of them. Thus, there are no principled reasons to prohibit the application of syntactic operations \emph{only} to traditional intermediate projections\footnote{I will keep using terms and labels of ``intermediate or bar-level projections'' and ``spec-head relations'' for expository purposes. Using these terms does not imply that they are syntactically definable.} and I propose that Spell-Out applies to an ``intermediate projection'' of a phase head if it undergoes ``spec-head agreement,'' arguing that selectional considerations make the proposed possibility of Spell-Out available.

(\ref{aranoQ}) shows the proposed derivation of embedded interrogative clauses which involve ``spec-head agreement'' (shading shows a Spell-Out domain):

\ea \label{aranoQ} {} \upshape
 [\sub{vP} wonder [ what$_\mathrm{i}$ \shade{C$_{\mathrm{Q}}$ [\sub{TP} you T$_\mathrm{\upphi}$ cook t$_\mathrm{i}$]}]]
\z 

\noindent In the interrogative clause, the C-head agrees with the \wh-phrase and it is moved to the edge of the CP, being in the ``spec-head'' configuration. Following \citet{Frampton:2000}, I assume Agree to be feature sharing and would like to suggest that ``the specifier'' shares features with a phase head as a consequence of ``spec-head agreement.'' I propose this shared feature on the specifier will do for selection from a higher head.\footnote{The idea that a shared feature plays a crucial role in selection is similar to \posscitet{Chomsky:2013} idea that the \{XP, YP\} structure can be labeled via Agree of their prominent features.}\footnote{\label{aranofn Ott}The proposed analysis shares the same spirit as \posscitet{Ott:2011} analysis of free relatives. He argues for Spell-Out of ``intermediate projections'' based on free relatives. (\ref{aranoFR}) shows his analysis of a free relative which occurs as a complement of verbs:

\ea \label{aranoFR} {} \upshape
 [\sub{vP} eat [ what$_\mathrm{i}$ \shade{C$_{\mathrm{FR}}$ [\sub{TP} you T$_\mathrm{\upphi}$ cook t$_\mathrm{i}$]}]]
\z 

\noindent Ott argues that the free relative is formed via the movement of a \wh-phrase triggered by the edge-feature of C$_{\mathrm{FR}}$ and the Spell-Out of \Xbar{C}. Ott motivates the Spell-Out of \Xbar{C} from the lack of interpretable features on C$_{\mathrm{FR}}$\@. He argues, since C-heads in free relatives lack interpretable features, they are spelled out with TP in \REF{aranoFR}, and the element in [Spec, CP] serves for selection and label determination. He does not allow for Spell-Out of ``intermediate projections'' for interrogative CP since interrogative C has an interpretable feature that serves for selection from a higher head.} In (\ref{aranoQ}) the verb \emph{wonder} selects an interrogative clause and I assume that, to satisfy this selectional requirement, the property of interrogative has to be syntactically present when the verb and the interrogative clause are merged. The property/feature of interrogative originally comes from the C-head but it also exists in the specifier of CP as a consequence of Agree\@. Since the feature on the specifier suffices for the selection, the phase C-head need not be accessible in the next cycle and, I propose, it is spelled out with its complement, as shown in (\ref{aranoQ}). Note that the possibility of Spell-Out of ``intermediate projections'' relies on the feature sharing. When there is no agreement relationship between a phase head and its ``specifier,'' the complement of the phase head constitutes a Spell-Out domain. Consider (\ref{aranodcl int a}), which shows the intermediate stage of the derivation of (\ref{aranodcl int b}):

\ea 
\ea 
\label{aranodcl int a} {} \textup{
 [\sub{vP} think [ what$_\mathrm{i}$ C \shade{[\sub{TP} you T$_\mathrm{\upphi}$ cook t$_\mathrm{i}$]}]]
  }  
\ex What does John think you cooked? \label{aranodcl int b}  
\z 
\z 

\noindent In (\ref{aranodcl int a}) no spec-head agreement takes place. If the phase head were spelled out with its complement, the verb \emph{think} would not see any feature of declarative when the verb and the clause are merged. Hence, the phasal complement, not `the intermediate projection,' has to undergo Spell-Out here. The same goes for (\ref{aranodcl a}), which shows the embedded clause with no specifier:

\ea 
\ea 
\label{aranodcl a} {} \textup{
 [\sub{vP} think [ C \shade{[\sub{TP} you T$_\mathrm{\upphi}$ cook]}]]
  }  
\ex Does John think you cook?
\z 
\z 

Summarizing, I have proposed that a traditional ``intermediate projection'' constitutes a Spell-Out domain if a phase head undergoes feature-sharing with its ``specifier.'' My proposal predicts that the size of Spell-Out domains changes depending on whether a phase head undergoes  ``spec-head agreement'' or not. In the following sections, I present two kinds of cross-linguistic and cross-construc\-tional evidence for my claim.


\section{Distinctness effects}\label{aranosect3}
\citet{Richards:2010} proposes Distinctness as a condition imposed on linearization of syntactic objects:

\ea\upshape
Distinctness\\
If a linearization statement 〈α, α〉 is generated, the derivation crashes.
\z 

\noindent It prohibits a linearization statement which instructs a  certain node has to precede itself because it is contradictory. Richards argues that under the certain assumptions regarding the organization of grammar, Distinctness leads to the consequences that there cannot be two functional elements of the same syntactic category in a single Spell-Out domain. 

Following \citet{Chomsky:1995,Chomsky:2000,Chomsky:2001a}, Richards assumes that trees created by syntax do not have information on linear order, and they are linearized via a version of Linear Correspondence Axiom \citep{Kayne:1994} at the point of Spell-Out. Moreover, he adopts the framework of Distributed Morphology \citep{Halle:1993b, Marantz:1996, Embick:2007}, where functional heads are associated with their phonological features via post-syntactic late insertion. Under this model of grammar, linearization of syntactic objects occurs prior to the assignment of phonological information to functional elements. It is then expected that different functional heads of the same type cannot be distinguished and may be regarded as the same syntactic object due to their scarcity of features that may be useful to differentiate them from each other. For concreteness consider the situation in which Spell-Out applies to the whole structure in \figref{fig:aranodist ex}, in which two instances of functional category $\upalpha$ are present.

\begin{figure}
\caption{Structure with a Distinctness violation\label{fig:aranodist ex}}
\begin{forest} for tree={minimum size=1.5em, inner sep=1pt} 
[XP, nice empty nodes [$\upalpha$, draw, circle] [  [\ldots]  [ [$\upalpha$, draw, circle]       [\ldots]    ]         ]    ]
\end{forest}
\end{figure}

Since the higher $\upalpha$ asymmetrically c-commands the lower one,  〈α, α〉 is generated. Crucially these $\upalpha$'s are not distinguished because of the lack of vocabulary insertion at the stage of Spell-Out and the derivation crashes. The Distinctness condition thus forbids the same kind of functional categories to be in the same Spell-Out domain. 


The Distinctness condition has implications for a wide range of linguistic phenomena. One of them is case resistance \citep{Stowell:1981}, which is illustrated by facts like (\ref{aranoCR1}):

\ea[*]  {\label{aranoCR1}
They're talking about {\ob}that they need to leave{\cb}.\\\upshape \citep[137]{Richards:2010}}
\z 


\noindent To account for the ungrammaticality of (\ref{aranoCR1}) in terms of Distinctness, Richards assumes the structure in \figref{fig:aranoCR2} and adopts two assumptions. First, P is not a phase head when taking CP-complements. Second, following \citet{Emonds:1985}, prepositions and complementizers are effectively of the same category, hence we cannot have P and C in a single Spell-Out domain. 

\begin{figure}
\caption{The PP of \REF{aranoCR1}\label{fig:aranoCR2}}
\begin{forest}
for tree={minimum size=1.5em, inner sep=1pt} 
[PP [P, draw, circle [about]]   [CP [C, draw, circle [that]]   [TP, name=yp  [they need to leave, roof]]     ]   ]
\node [left=0.25em of yp](p1){}; 
\node [above right=1em and 0.25em of yp] (p2) {};
\draw [overlay, thick] (p1) to[out=90, in=180] (p2);    
\node [right=-0.5em of p2] {$\rightarrow$ Spell-Out};
\end{forest} 
\end{figure}

Given these assumptions, (\ref{aranoCR1}) is ruled out because P and C are in the same Spell-Out domain. When the phase above CP triggers Spell-Out, P and C are linearized in the same Spell-Out domain. Since P and C belong to the same type, they cannot be linearized, causing a violation of Distinctness.

The case resistance principle does not apply to interrogative clauses, as \citet[139]{Richards:2010} notes:

\ea \label{aranoCR4}
They’re talking about {\ob}what they should buy{\cb}. \\ \upshape \citep[139]{Richards:2010}
\z 

\noindent This fact, however, cannot be accounted for in terms of Distinctness, if we assume the standard version of Spell-Out. It is incorrectly predicted that P and C induce a contravention of Distinctness in \figref{fig:aranoCR3}, as in \figref{fig:aranoCR2}.\footnote{Recall that linearization takes place before late insertion. Therefore, as far as linearization is concerned, phonologically overt and null functional items have the same status and both of them can cause a violation of Distinctness.}


\begin{figure} 
\caption{The PP of \REF{aranoCR4} in the standard analysis\label{fig:aranoCR3}}
\begin{forest}
for tree={minimum size=1.5em, inner sep=1pt} 
[PP [P, draw, circle [about]]   [CP [DP [what, roof]]  [\Xbar{C}   [C, draw, circle]    [TP, name=yp  [they should buy, roof]]     ]   ] ]
\node [left=0.25em of yp](p1){}; 
\node [above right=1em and 0.25em of yp] (p2) {};
\draw [overlay, thick] (p1) to[out=90, in=180] (p2);    
\node [right=-0.5em of p2] {$\rightarrow$ Spell-Out};
\end{forest} 
\end{figure} 

One might argue that case resistance effects are absent here because there is a DP-layer above CP and it triggers Spell-Out. This analysis predicts the absence of Distinctness effects between elements inside interrogative clauses and those outside them.\footnote{This is the analysis of the grammaticality of (\ref{aranoCR4}) by \citet[139, 215 fn. 67]{Richards:2010}. He motivates the presence of DP-layers by noting that interrogative clauses, like nominals, have to come with \emph{of} when they are complements of nominals:

\ea 
the question *(of) {\ob}what they should buy{\cb} \\ \upshape \citep[139]{Richards:2010}
\z 

The postulation of  DP-layers above interrogative clauses, however, leads to a problem when we look at (\ref{aranoCR5a}). \citet{Richards:2010} accounts for its ungrammaticality as a Distinctness effect with the structure in \figref{fig:aranoCR6}. Note that if there were a DP-layer above CP here, no Distinctness effects would arise because the D-head would trigger Spell-Out of CP\@. Thus, \citet{Richards:2010} needs to assume that interrogative clauses involve DP-layers when their specifier is DP, but not when their specifier is PP\@. In the following I develop an alternative analysis which avoids this complication. 

I also would like to mention that there are cases in which interrogative clauses do not need the insertion of \emph{of}, which may suggest that interrogative clauses need not be nominals at least in some cases:

\ea 
In many cases there is a question whether there is a code violation. \\
{\normalfont(\url{https://bellevuewa.gov/city-government/departments/community-development/conflict-assistance/types-of-conflicts})}
\z} There is a piece of evidence for the relevance of Distinctness here, however. Consider (\ref{aranoCR5}):

\ea \label{aranoCR5}
\ea[*]{They're talking about {\ob}with whom they should discuss this{\cb}.}\label{aranoCR5a} 
\ex[]{They don't know {\ob}with whom they should discuss this{\cb}. \\ \upshape \citep[139]{Richards:2010}}
\z
\z 

\noindent (\ref{aranoCR5}) shows interrogative clauses with a PP specifier. They can be complements of verbs, but not prepositions. This contrast suggests that the two prepositions in (\ref{aranoCR5a}) induce a violation of Distinctness, as shown in \figref{fig:aranoCR6}.


\begin{figure} 
\caption{The PP of \REF{aranoCR5a} in the standard analysis\label{fig:aranoCR6}}
\begin{forest} 
for tree={minimum size=1.5em, inner sep=1pt} 
[PP [P, draw, circle [about]]   [CP [PP [P, draw, circle [with]] [DP [whom, roof]]  ]  [\Xbar{C}   [C, draw, circle]    [TP, name=yp  [they should discuss this, roof]]     ]   ] ]
\node [left=0.25em of yp](p1){}; 
\node [above right=1em and 0.25em of yp] (p2) {};
\draw [overlay, thick] (p1) to[out=90, in=180] (p2);    
\node [right=-0.5em of p2] {$\rightarrow$ Spell-Out};
\end{forest} 
\end{figure}

Given the Distinctness-based account of (\ref{aranoCR5a}), the question arises why \figref{fig:aranoCR3} does not induce such a violation. The grammaticality of (\ref{aranoCR4}), on the one hand, suggests that the edge of the free relative is separated from the preposition by a Spell-Out boundary. The ungrammaticality of (\ref{aranoCR5a}), on the other hand, suggests that they belong to the same Spell-Out domain. This state of affairs is hard to reconcile under the standard analysis of Spell-Out since it defines the edge of phases as a phase head and its specifier uniformly. The proposed analysis, by contrast, gives us a correct characterization of Spell-Out domains to account for these cases. Consider the structure of these cases in terms of the present proposal given the structure of declarative and interrogative clauses.

First, declarative clauses take no specifier. Therefore, TP-complements of C are Spell-Out domains. Case resistance effects for declarative clauses then are expected given the structure in \figref{fig:aranoCR2}. Second, interrogative clauses involve ``spec-head agreement" with \wh-phrases. Thus, ``intermediate projections'' of C undergo Spell-Out. \figref{fig:aranoCR7} is the structure for (\ref{aranoCR4}).

\begin{figure} 
\caption{The PP of \REF{aranoCR4} in the proposed analysis\label{fig:aranoCR7}}
\begin{forest}
for tree={minimum size=1.5em, inner sep=1pt} 
[PP [P, draw, circle [about]]   [CP [DP [what, roof]]  [\Xbar{C}, name=yp   [C, draw, circle]    [TP  [they should buy, roof]]     ]   ] ]
\node [left=0.25em of yp](p1){}; 
\node [above right=1em and 0.25em of yp] (p2) {};
\draw [overlay, thick] (p1) to[out=90, in=180] (p2);    
\node [right=-0.5em of p2] {$\rightarrow$ Spell-Out};
\end{forest} 
\end{figure}

Crucially, the present analysis puts the phase head C into the Spell-Out domain with its complement. This separates the preposition and the complementizer into different Spell-Out domains, avoiding a violation of Distinctness. The absence of case resistance effects for interrogative clauses is thus also correctly predicted. Finally, consider (\ref{aranoCR5a}). The present analysis gives it the structure in \figref{fig:aranoCR8}.

\begin{figure} 
\caption{The PP of \REF{aranoCR5a} in the proposed analysis\label{fig:aranoCR8}}
\begin{forest} 
for tree={minimum size=1.5em, inner sep=1pt} 
[PP [P, draw, circle [about]]   [CP [PP [P, draw, circle [with]] [DP [whom, roof]]  ]  [\Xbar{C}, name=yp   [C, draw, circle]    [TP  [they should discuss this, roof]]     ]   ] ]
\node [left=0.25em of yp](p1){}; 
\node [above right=1em and 0.25em of yp] (p2) {};
\draw [overlay, thick] (p1) to[out=90, in=180] (p2);    
\node [right=-0.5em of p2] {$\rightarrow$ Spell-Out};
\end{forest} 
\end{figure}

Due to the ``spec-head agreement'' the phase head is spelled out with its complement. Still, the specifier of the phase head escapes Spell-Out. Therefore, the Distinctness effect is correctly predicted to be caused by the two prepositions.  The proposed analysis thus gives an account of the case resistance patterns of declarative and interrogative clauses.

It should be noted that my analysis allows an interrogative clause to occur as a complement of prepositions not because it is an interrogative clause but because ``spec-head agreement'' occurs within it. Similarly, it prevents declarative clauses from occurring as a complement of prepositions not because it is a declarative clause but because there is no Spell-Out of an ``intermediate projection.'' My proposal predicts that prepositions can take clauses as long as there is an application of Spell-Out of ``intermediate projection,'' irrespective of their semantic types. I show that this prediction is correct using \emph{whether}-clauses, \emph{if}-clauses, and \emph{how}-clauses.

\citet{Kayne:1991} discusses the status of interrogative \emph{whether} and \emph{if}. Though both of these can introduce embedded yes-no interrogative, they show certain syntactic differences. For example, consider (\ref{aranowhether if selection}):


\ea \label{aranowhether if selection}
\ea[] {I wonder whether I should go.} \label{aranowhether if selectiona}
\ex[] {I wonder whether to go.}
\ex[] {I wonder whom I should invite.}\label{aranowhether if selectionc}
\ex[] {I wonder whom to invite.}
\ex[] {I wonder where I should go.}
\ex[] {I wonder where to go.} \label{aranowhether if selectionf}
\ex[] {I wonder if I should go.}
\ex[*] {I wonder if to go.
\\ \upshape \citep[175--176]{Haegeman1999English-Grammar}} \label{aranowhether if selectionh}
\z 
\z 



\noindent (\ref{aranowhether if selectiona}--\ref{aranowhether if selectionf}) shows that \emph{whether}, like \wh-phrases, can introduce finite and non-finite clauses. This leads me to the treatment of \emph{whether} as a kind of \wh-phrase. (\ref{aranowhether if selectionc}--\ref{aranowhether if selectionh}) indicates that \emph{if} behaves differently from \wh-phrases with respect to the selection of clauses: it has to take finite clauses. To express the difference in question, I assume, following \citet{Kayne:1991}, that \emph{whether} is a \wh-phrase that occupies a specifier of C, while \emph{if} is a complementizer. More specifically, I assume the structures in \figref{fig:aranowhether-if str} for interrogative clauses introduced by these elements.

\begin{figure}
\begin{subfigure}[b]{.5\linewidth}\centering
\begin{forest}
[CP [whether] [\Xbar{C} [C]     [TP [\ldots, roof]]     ]]
\end{forest}
\caption{\textit{whether}}
\end{subfigure}\begin{subfigure}[b]{.5\linewidth}\centering
\begin{forest}
[CP [C [if]]     [TP [\ldots, roof]]     ]
\end{forest}
\caption{\textit{if}\label{fig:aranowhether-if strb}}
\end{subfigure}
\caption{Structures of \textit{whether}- and \textit{if}-clauses\label{fig:aranowhether-if str}}
\end{figure}

\emph{Whether} occupies a specifier of C which requires its specifier to be a \wh-phrase. This kind of C does not impose selectional restrictions on the finiteness of TP. \emph{If} is a complementizer and needs to take a finite clause as its complement as its selectional restrictions. What is the most important difference on these structures in the present discussion is that the \emph{whether}-clause involves ``spec-head agreement,'' whereas \emph{if}-clause does not. This difference leads to the prediction that  \emph{whether}-clauses, but not \emph{if}-clauses, can occur as complements of prepositions. Since there is ``spec-head agreement'' within \emph{whether}-clauses, C-heads are spelled out with their complements. Therefore, they are spelled out before prepositions which take them are, with no violations of Distinctness. \emph{If}-clauses, on the other hand, send TP to the interface given its structure. When P selects CP, then, C and P belong to the same Spell-Out domain and cause a violation of Distinctness. (\ref{aranoP-whether-if}) shows that the prediction is borne out:

\ea 
It depends on \textup{\{}whether\textup{|*}if\textup{\}} we have enough time left.
\\ \upshape \citep[974]{Huddleston2002The-Cambridge-G} \label{aranoP-whether-if}
\z 

\noindent The proposed analysis thus correctly predicts that clauses cannot be selected by prepositions with no spec-head agreement, even if they are interrogative.\footnote{The present analysis predicts the contrast in (\ref{aranoP-whether-if}) assuming the structural differences between \emph{whether}- and \emph{if}-clauses in \figref{fig:aranowhether-if str}. As a reviewer points out, some analyses of \emph{if}-clauses posit a null operator in its specifier (see \citealt{Larson1985On-the-Syntax-o, Han2004The-Syntax-of-W, Wu2020Why-if-or-not-b}). Under this analysis, the null operator would agree with \emph{if} and the present analysis does not predict the contrast between \emph{whether}- and \emph{if}-clauses in question. The reviewer points out that the variation in the structural analysis of \emph{if}-clauses may be related to speaker variation of judgment of data like (\ref{aranoP-whether-if}). S/he notes that ``\emph{[i]t depends on if} does not sound too bad in [his/her] dialect of English (maybe slightly worse than \emph{whether})'' and provides the following naturally occurring example of \emph{depends on if}:


\ea 
Carmelo Anthony's impact depends on if he finishes games. \\
{\normalfont(\url{https://www.youtube.com/watch?v=oa7aolbngrU})}
\z

\noindent Given the structural variation in the structure of \emph{if}-clauses, the present analysis predicts this variation among speakers. For speakers who reject \emph{depends on if}, they assume the structure in \figref{fig:aranowhether-if strb}, in which no spec-head agreement occurs, hence a violation of Distinctness is caused when \emph{if}-clauses occur as a complement of P\@. For speakers who accept it, \emph{if}-clauses have a null operator in its specifier and `spec-head agreement' triggers Spell-Out of  `intermediate projections,' as in \emph{whether-}clauses, and therefore they do not find the contrast between \emph{whether}- and \emph{if}-clauses in question. I would like to thank the reviewer for raising this point.

Another reviewer points out that \emph{if}-clauses can be used with prepositions in the combination of \emph{about if} and \emph{as if}. He or she also notes that in these usages the \emph{if}-clauses are not interrogative types, which I discussed in the main text. This may suggest that, contrary to interrogative \emph{if}-clauses, these \emph{if}-clauses involve structures with `spec-head agreement.' I would like to thank the reviewer for noting these constructions and to leave the investigation of these cases for future research.} 

The present analysis also predicts that declarative clauses can occur as complements of prepositions if they involve Spell-Out of `intermediate projections.' \citet{Legate:2010} discusses declarative clause introduced by \emph{how}:

\ea 
They told me how the tooth fairy doesn't really exist.\\ \upshape
`They told me that the tooth fairy doesn't really exist.'
\\  \citep[121]{Legate:2010}
\z 

\noindent She argues that \emph{how}-clauses are derived by base-generating \emph{how} in CP-specifiers. Interestingly, this type of declarative clauses can be complements of prepositions. 

\ea 
They told me about how the tooth fairy doesn't really exist.
\\ \upshape \citep[122]{Legate:2010}
\z 

\noindent Though Legate assumes null DP-layers above CP to account for their behaviors like definite DPs, this type of clause provides a potential case of declarative clauses with  ``spec-head agreement'' and they can be complements of P\@.\footnote{It is worth mentioning that Legate notes close resemblances between \emph{how}-clauses and free relatives involving \emph{how}, and \citet{Ott:2011} argues for Spell-Out of ``intermediate projections'' for the derivation of free relatives. See Footnote \ref{aranofn Ott} for his analysis of free relatives.}

To summarize, this section has discussed the distribution of various types of clauses. It has shown that the syntactic structure, but not the semantics, of clauses, is important. Given Distinctness, the proposed analysis has offered an account for it, correctly predicting that the presence or absence of `spec-head' agreement and the category of `specifier' play an important role.

\section{Sluicing}\label{aranosect4}
This section aims to derive \posscitet{Merchant2001The-syntax-of-s} sluicing-COMP generalization and give an account of its exceptions from the proposed mechanism of Spell-Out. In so doing, I assume that ellipsis has a direct connection with Spell-Out domains. Specifically, I assume that ellipsis arises as a consequence of not realizing a Spell-Out domain at PF \citep{Gengel2006Phases-and-elli, Gengel2009Phases-and-elli, Craenenbroeck2010The-syntax-of-e, ,bovskovic2014now,Wurmbrand2017Stripping-and-t}.

Based on a number of languages, \citet[62]{Merchant2001The-syntax-of-s} argues for the generalization (\ref{aranoMerchant gen}):


\ea \upshape \label{aranoMerchant gen}
In sluicing, no non-operator material may appear in COMP.
\z 

\noindent Let us first see the validity of this generalization. English, Dutch, German, and Danish all exhibit verb-second in matrix interrogatives:

\ea 
\settowidth\jamwidth{} 
\ea Who has Max invited? \jambox{\upshape[English]}
\ex Wen hat Max eingeladen? \jambox{\upshape[German]}
\ex Wie heeft  Max uitgenodigd?  \jambox{\upshape[Dutch]}
\ex Hvem har Max inviteret? \jambox{\upshape[Danish]}

 \upshape \citep[63]{Merchant2001The-syntax-of-s}
\z 
\z


\noindent When sluicing applies in theses sentences, the remnant cannot include the auxiliary:

\ea \label{aranosluice aux}
\settowidth\jamwidth{} 
\ea \textup{A:} Max has invited someone. 

\textup{B:} Really? Who \textup{(*}has\textup{)}? \jambox{\upshape[English]}
\ex \textup{A:} Max hat jemand eingeladen. 

\textup{B:} Echt? Wen \textup{(*}hat\textup{)}?\jambox{\upshape[German]}
\ex \textup{A:} Max heft iemand uitgenodigd.

\textup{B:} Ja? Wie \textup{(*}heeft\textup{)}? \jambox{\upshape[Dutch]}
\ex \textup{A:} Max har inviteret en eller anden.

\textup{B:} Ja? Hvem \textup{(*}har\textup{)}? \jambox{\upshape[Danish]}

 \upshape \citep[63]{Merchant2001The-syntax-of-s}
\z 
\z

\noindent Given the structure shown in \figref{aranostructure sluice} and the TP-Spell-Out/-ellipsis analysis of sluicing, the question arises as to why the auxiliaries must be elided in (\ref{aranosluice aux}).\footnote{One may account for the obligatory absence of auxiliaries in matrix sluicing by arguing that ellipsis of TP blocks T-to-C head-movement. \citet{Lasnik1999On-Feature-Stre} and \citet{Boeckx2001Head-ing-toward} develop such analyses. However, \citet{Merchant2001The-syntax-of-s} shows that the sluicing-COMP generalization holds even for material usually base-generated in C. For example, certain varieties of Dutch allow an overt complementizer to co-occur with a \wh-phrase in [Spec, CP]:

\ea \label{aranosou Dutch}
\gll Ik weet niet, wie \textup{(}of\textup{)} \textup{(}dat\textup{)} hij gezien heeft.   \\
 I know not who \phantom{(}if \phantom{(}that he seen has          \\  \jambox*{\upshape[(esp. Southern) Dutch]}
\glt `I don't know who he has seen.'

\citep[74]{Merchant2001The-syntax-of-s}  \\ 
\z
\il{Dutch} 

\noindent Importantly, a grammatical sluiced counterpart of (\ref{aranosou Dutch}) involves only \wh-phrase: 

\ea 
\gll Hij heeft iemand gezien, maar ik weet niet \textup{\{}wie \textup{|*}wie of \textup{|*}wie dat \textup{|*}wie of dat\textup{\}}.   \\
he has somone seen but I know not \phantom{\{}who \phantom{|*}who if \phantom{|*}who that \phantom{|*}who if that \\  
\glt `He saw someone, but I don’t know who.'

\citep[75]{Merchant2001The-syntax-of-s} \\ \jambox*{\upshape[Dutch]}
\z 

\noindent This shows that the absence of T-to-C movement in sluicing cannot be the whole story of the sluicing-COMP generalization.}


\begin{figure}
\begin{floatrow}
\captionsetup{margin=.05\linewidth}

\ffigbox{\begin{forest}
[CP [who$_{\mathrm{2}}$]  [\Xbar{C}    [C [has$_{\mathrm{1}}$]]     [TP, name=yp [Max t$_{\mathrm{1}}$ invited t$_{\mathrm{2}}$, roof]]]]
\node [left=0.25em of yp](p1){}; 
\node [above right=1em and 0.25em of yp] (p2) {};
\draw [overlay, thick] (p1) to[out=90, in=180] (p2);    
\node [right=-0.5em of p2] {$\rightarrow$ Spell-Out/$\varnothing$};
\end{forest}}
{\caption{Sluicing in the standard analysis\label{aranostructure sluice}}}
\ffigbox{\begin{forest}
[CP [who$_{\mathrm{2}}$]  [\Xbar{C}, name=yp    [C [has$_{\mathrm{1}}$]]     [TP [Max t$_{\mathrm{1}}$ invited t$_{\mathrm{2}}$, roof]]]]
\node [left=0.25em of yp](p1){}; 
\node [above right=1em and 0.25em of yp] (p2) {};
\draw [overlay, thick] (p1) to[out=90, in=180] (p2);    
\node [right=-0.5em of p2] {$\rightarrow$ Spell-Out/$\varnothing$};
\end{forest}}
        {\caption{Sluicing in the proposed analysis\label{aranomy structure sluice}}}
\end{floatrow}
\end{figure}

My proposal on Spell-Out domains accounts for the sluicing-COMP generalization straightforwardly. Consider \figref{aranomy structure sluice}, which shows the present analysis of sluicing. Since interrogative clauses involve ``spec-head agreement,'' Spell-Out/ellipsis targets \Xbar{C}. Since the C-head is a part of the Spell-Out domain, only the ``specifier,'' i.e. \wh-operator, can survive sluicing. 

It is tempting to try to account for counter-examples of the sluicing-COMP generalization in terms of the present analysis. \citet{Takita2012Genuine-Sluicin} provides such a counter-example from Japanese. He argues that a certain type of apparent sluicing in Japanese are ``genuine'' sluicing constructions in the sense that it is derived by movement of \wh-phrases followed by clausal ellipsis, as in sluicing, for example, in English.\footnote{Japanese has the construction that is  apparently sluicing but has a different structure from real sluicing. A notable characteristic of this construction is that it allows the copula \emph{da} to occur in the construction. 

\ea \label{aranononreal i}
\gll Taroo-wa \textup{[}Ziroo-ga nanika-o katta to\textup{]} itteita-ga, boku-wa \textup{[}nani-o \textup{(}da\textup{)} ka\textup{]} sir-anai. \\
Taroo-\textsc{top} \phantom{[}Ziroo-\textsc{nom} something-\textsc{acc} bought that said-but I-\textsc{top} \phantom{[}what-\textsc{acc} \phantom{(}\textsc{cop} \textsc{q} know-not   \\
\glt `Taroo said that Ziroo bought something, but I don’t know what.'
\z\il{Japanese} 

\noindent Importantly, this copula cannot occur in embedded questions: 

\ea \label{aranononreal ii}
\gll Taroo-wa \textup{[}Ziroo-ga nanika-o katta to\textup{]} itteita-ga, boku-wa \textup{[}kare-ga nani-o katta \textup{(}*da\textup{)} ka\textup{]} sir-anai. \\
Taroo-\textsc{top} \phantom{[}Ziroo-\textsc{nom} something-\textsc{acc} bought that said-but I-\textsc{top} \phantom{[}he-\textsc{nom} what-\textsc{acc} bought \phantom{(}\textsc{cop} \textsc{q} know-not   \\
\glt `Taroo said that Ziroo bought something, but I don’t know what he bought.'
\z

\noindent This contrast suggests that it is unlikely that (\ref{aranononreal i}) is derived from (\ref{aranononreal ii}). 

This kind of complication will not arise for ``genuine'' sluicing since it does not allow the copula to occur. Compare (\ref{aranogenuine cop}) and (\ref{aranoreal slucing a}):

\ea[*]{ \label{aranogenuine cop}
\gll Taroo-wa \textup{[}PRO dono zyaanaru-ni zibun-no ronbun-o das-oo ka\textup{]} kimeta-ga, Hanako-wa \textup{[}dono zyaanaru-ni da ka\textup{]} kimekaneteiru.    \\
Taroo-\textsc{top} {} which journal-to self-\textsc{gen} paper-\textsc{acc} submit-\textsc{inf} \textsc{q} decided-but Hanako-\textsc{top} \phantom{[}which journal-to \textsc{cop} \textsc{q} cannot.decide  \\
\glt `(intended) Though Taroo decided [to which journal [to submit his paper]], Hanako cannot decide [to which journal [to submit her paper]].'}
\z 
\il{Japanese} 

\noindent See \citet{Takita2012Genuine-Sluicin} for arguments for the real sluicing status of the construction in question.} 
He presents (\ref{aranoreal slucing a}) as a real sluicing example in Japanese. It involves control predicates which take interrogative non-finite clauses and the second sentence involves sluicing, whose structure is shown in (\ref{aranoreal slucing b}):

\ea 
\ea \label{aranoreal slucing a}
\gll Taroo-wa \textup{[}PRO dono zyaanaru-ni zibun-no ronbun-o das-oo ka\textup{]} kimeta-ga, Hanako-wa \textup{[}dono zyaanaru-ni ka\textup{]} kimekaneteiru.    \\
Taroo-\textsc{top} {} which journal-to self-\textsc{gen} paper-\textsc{acc} submit-\textsc{inf} \textsc{q} decided-but Hanako-\textsc{top} \phantom{[}which journal-to \textsc{q} cannot.decide  \\
\glt `(intended) Though Taroo decided [to which journal [to submit his paper]], Hanako cannot decide [to which journal [to submit her paper]].'
\ex\label{aranoreal slucing b}Hanako [$_{\mathrm{vP}}$\,[$_{\mathrm{CP}}$ \ConnectTail{to which journal$_{\mathrm{1}}$}[A] \shade{[$_{\mathrm{TP}}$ PRO \ldots \ConnectHead*[2ex]{t$_{\mathrm{1}}$}[A]]} C$_{\mathrm{Q}}$\,]\,cannot.decide\,]
\z
\z 
\il{Japanese} 

\noindent Note that sluicing in Japanese leaves the C-head as well as \wh-phrase intact, thus posing a counter-example to the sluicing-COMP generalization. Under the present analysis, that the C-head survives sluicing means that C-head does not undergo ``spec-head agreement'' and only the TP-complement is spelled out or elided. The absence of ``spec-head agreement'' in Japanese sluicing does not seem unreasonable given that Japanese is often characterized as an agreement-less language and lacks obligatory \wh-movement. Though the detail of the analysis needs to be worked out I believe that the present analysis tells us some insight as to why Japanese does not conform to the sluicing-COMP generalization.\footnote{As reviewers point out, there are other cases which are argued to be an instance of sluicing with an non-operator remnant, i.e., counter-examples to \posscitet{Merchant2001The-syntax-of-s} generalization (see \citet{Craenenbroeck2010The-syntax-of-e, Craenenbroeck2013What-sluicing-c, Marusic2015On-a-potential-, Marusic2018Surviving-sluic} a.o.). Generally speaking, these cases are analyzed within the cartographic approach, which posits the rich structure within CP-areas (\citet{Rizzi1997The-fine-struct} et seq.), and non-operator remnants are argued to be in the fine-grained CP-structures. The present paper assumes a parsimonious structure for CP\@. I hope to address in future research the question of how the present analysis deals with these cases and it can be implemented within the cartographic approach.}

To summarize this section has offered an account of \posscitet{Merchant2001The-syntax-of-s} generalization in terms of ellipsis as a null form of Spell-Out. That non-operator materials do not survive sluicing has been argued to be a consequence of `spec-head agreement' in sluicing, which makes traditional C-bar projections a Spell-Out/Ellipsis site.

\section{Conclusion}\label{aranosect5}
This paper has proposed that  ``intermediate projections'' undergo Spell-Out when phase heads enter a ``spec-head'' relationship. I have shown that the proposed analysis accounts for case resistance effects in terms of Distinctness and for the sluicing-COMP generalization under the view of ellipsis as null Spell-Out.

\section*{Acknowledgements}
I would like to thank two anonymous reviews for their insightful comments and suggestions. I also would like to thank Željko Bošković, Yoshiaki Kaneko, Mamoru Saito, Etsuro Shima, Susi Wurmbrand and audience at 31st meeting of the English Linguistics Society of Japan for their comments and discussion at various stages of the paper. 

\section*{Abbreviations}
\begin{tabularx}{.5\textwidth}{@{}lQ}
\textsc{acc} & accusative\\
\textsc{cop} & copula \\
\textsc{gen} & genitive \\
\textsc{inf} & infinitive\\
\end{tabularx}\begin{tabularx}{.5\textwidth}{lQ@{}}
\textsc{nom} & nominative\\
\textsc{q} & question particle/marker\\
\textsc{top} & topic\\
\end{tabularx}

{\sloppy\printbibliography[heading=subbibliography,notkeyword=this]}

\end{document}
