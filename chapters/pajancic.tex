\documentclass[output=paper]{langscibook}
\ChapterDOI{10.5281/zenodo.5524278}
\author{Caroline Pajančič\affiliation{University of Vienna}}
\title{Akan complements on the implicational complementation hierarchy}
\abstract{The implicational complementation hierarchy (ICH) formulated by \citet{wurmbrandlohninger2020} distinguishes three complement types: 
    Proposition, Situation and Event, which are ordered by independence, transparency, integration and complexity. 
    The ICH outlines the correlation between the semantic functions of the complement types, and the syntactic operations that run directionally along it. 
    The complements are in a coherent containment relation and have minimal requirements for the domain they project: 
    a theta domain for Events, a TMA domain for Situations, and an operator domain for Propositions. 
    If one type of complement can be finite, all complements to its left on the ICH can be too (finiteness universal, \citealp{wurmbrandetal2020}). 
    This chapter discusses the distribution of complements in Akan, a Kwa language spoken in Ghana, Ivory Coast and Benin, 
    which have traditionally been analysed as finite and requiring a mandatory complementiser. 
    However, new data indicates that the clause introducer \textit{sɛ} in Twi (\textit{dɛ} in Fante) can be dropped and non-finite complements are possible in Event structures. 
    I thus argue that Proposition, Situation and Event complements in Akan display the same properties predicted by the ICH and finiteness universal 
    and that finiteness in the language can occur in every domain.}

\begin{document}
\SetupAffiliations{mark style=none}
\newcommand{\compl}{\textsc{compl}}
\newcommand{\comp}{\textsc{comp}}
\newcommand{\pl}{\textsc{pl}}
\newcommand{\subj}{\textsc{subj}}
\newcommand{\fut}{\textsc{fut}}
\newcommand{\sg}{\textsc{sg}}
\newcommand{\tsc}{\textsc}
\newcommand{\prog}{\textsc{prog}}
\maketitle

\section{Theory}
\label{Pajsect:1}

\subsection{Introduction}
The implicational complementation hierarchy (ICH) formulated by \citet{wurmbrandlohninger2020} depicts the correlation between the semantic functions of complement constructions, and the syntactic operations that run directionally along it. The mapping is language-dependent and can vary, unless it violates the hierarchy. Built on \citegen{givon1980} binding hierarchy, \citet{wurmbrandlohninger2020} distinguish three complement classes for which they adapt \citegen{ramchandsvenonius2014} terminology as follows: Event (\ref{Paj1a}), Situation (\ref{Paj1b}) and Proposition (\ref{Paj1c}). Their order on the ICH is shown in Table \ref{Pajtab:1:frequencies}. 

\begin{exe}
\ex English
\begin{xlist}
\ex \label{Paj1a}
Lea tried to read a book (\#tomorrow).
\ex \label{Paj1b}
Lea decided to read a book (tomorrow). 
\ex \label{Paj1c}
Lea claims to be reading a book right now. 
\end{xlist}
\end{exe}


\begin{table}
\caption{Implicational complementation hierarchy (\citealt{wurmbrandlohninger2020})\label{Pajtab:1:frequencies}}
 \fittable{\begin{tabular}{l c l}
  \lsptoprule
 \tsc{most} \tsc{independent} &  \multirow{4}*{Proposition » Situation » Event}  & \tsc{least} \tsc{independent}\\
  \tsc{least} \tsc{transparent}&    & \tsc{most} \tsc{transparent}\\
    \tsc{least} \tsc{integrated}&    & \tsc{most} \tsc{integrated}\\
      \tsc{most} \tsc{complex} &    & \tsc{least} \tsc{complex} \\
  \lspbottomrule
 \end{tabular}}
\end{table}


The finiteness universal (\citealt{wurmbrandetal2020}) postulates a further implicational relation that if a language allows or requires finiteness in a type of complement, all complements to its left in the ICH do too. In this chapter, I will test to what extent the ICH and the finiteness universal (detailed further in \sectref{Paj1.2} and \ref{Paj1.3}) apply to the distribution of complements in Akan, a Kwa language spoken in Ghana, Ivory Coast and Benin.\footnote{There are several mutually intelligible dialects in Akan: Agona, Ahafo, Akuapem, Akwamu, Akyem, Asante, Assin, Bono, Denkyira, Fante, Kwahu, and Wassa. All but Fante belong to the Twi dialect (\citealt{osam2016}). Examples from my data are predominantly Twi.} Its complement clauses have traditionally been analysed as requiring a mandatory complementiser: \textit{sɛ} in Twi, \textit{dɛ} in Fante (\citealt{boadi1972}, \citealt{lord1976}, \citealt{osam1998}).

\begin{exe}
\ex Akan (Fante, \citealt{osam1998})

\begin{xlist}

\ex \label{Paj2a} 
\gll Kofi ka-a {d{ɛ}} {yɛ-ba-e}.\\
    Kofi say-{\compl} {\comp} 1{\pl}.{\subj}-come-{\compl}\\
\glt `Kofi said that we came.’\\

\ex \label{Paj2b}
\gll {Maame} {no} {hy{ɛ}-{ɛ} b{ɔ}} {d{ɛ}} {{ɔ}-b{ɔ}-k{ɔ}}.\\
    woman \textsc{def} promise.{\compl} {\comp} 3{\sg}.{\subj}-{\fut}-go\\
\glt `The woman promised to go.’\\

\ex \label{Paj2c}
\gll {Kofi} {b{ɔ}-ɔ} {mb{ɔ}dzen} {d{ɛ}} {ɔ-bɛ-yɛ} {edwum} {no}.	\\
    Kofi hit-{\compl} effort {\comp} 3{\sg}-{\fut}-do work \tsc{def}\\
\glt `Kofi tried to do the work.’\\

\end{xlist}
\end{exe}
 

In the Akan examples in (\ref{Paj3}) however we see that the complementiser \textit{sɛ} can be present in all of them, albeit optional in (\ref{Paj3d}). 

\begin{exe}
\ex \label{Paj3} Akan (Twi, personal communication)

\begin{xlist}

\ex \label{Paj3a}
\gll {Me-ka-a} {s{ɛ}} {me-kenkan-e} {nwoma} {no.}\\
    1{\sg}-say-{\compl} s{ɛ} 1{\sg}-read-{\compl} book \textsc{def} \\
\glt `I said that I read the book’\\

\ex \label{Paj3b}
\gll {Me-si-i} {gyinae{ɛ}} {s{ɛ}} {me-kenkan} {nwoma} {no.}\\
     1{\sg}-say-{\compl} decision s{ɛ} 1{\sg}-read book \textsc{def}\\
\glt `I decided to read the book’\\

\ex \label{Paj3c}
\gll {Me-b{ɔ}-{ɔ}} {mm{ɔ}den} {s{ɛ}} {me-kenkan-e} {nwoma} {no.}\\
     1{\sg}-hit-{\compl} effort s{ɛ} 1{\sg}-read-{\compl} book \textsc{def}\\
\glt `I tried to read the book’\\

\ex \label{Paj3d}
\gll {Me-b{ɔ}-{ɔ}} {me ho} {mm{ɔ}den} {kenkan-e} {nwoma} {no.}\\
     1{\sg}-hit-{\compl} myself effort read-{\compl} book \textsc{def}\\
\glt `I tried to read the book’\\

\end{xlist}
\end{exe}
 

My data from Twi speakers show that the clause introducer can be dropped in combination with certain matrix verbs which are recognised as restructuring verbs in Wurmbrand’s framework (\citeyear{wurmbrand2001}, \citeyear{todorovic2015}, \citeyear{wurmbrandetal2020}). Although the vast majority of complements still require a clause introducer to be grammatical, and the complement has to be finite, these findings provide a challenge to the assumption that a complementiser is compulsory in Akan complementation. 

In the remainder of \sectref{Pajsect:1}, I examine a possible theoretical framework to account for the variations in the three complement types. \sectref{Pajsect:2} gives a brief summary on relevant points of the verbal morphology in the language. \sectref{Pajsect:3} concerns the three complement types Proposition, Situation and Event in Akan. \sectref{Pajsect:4} outlines a preliminary conclusion that finiteness can occur in every domain in Akan, and the consequences of the findings in this chapter for the ICH and finiteness universal. 

\subsection{The implicational complementation hierarchy}\largerpage
\label{Paj1.2}

Across languages, complements can be divided into three types which are in an implicational hierarchy (see \tabref{Pajtab:1:frequencies}): Proposition, Situation and Event complements. The properties of the complement types are briefly summarised from \citet{wurmbrandlohninger2020} in Table \ref{Pajtab2}. 

%TABLE 2

\begin{table}
\caption{Properties of complement types (summarised from \citealt{wurmbrandlohninger2020})\label{Pajtab2}}
 \begin{tabularx}{\textwidth}{QQQ} 
  \lsptoprule
   Proposition  & Situation & Event\\
  \midrule
    Speech and epistemic contexts  &    Emotive and irrealis contexts      & Implicative and strong attempt contexts\\\midrule

    Embedded  reference time (attitude holder's \tsc{now}), no pre-specified tense value  &    No embedded reference time, pre-specified tense value      & Tenseless, simultaneous\\  \midrule

    May involve speaker-oriented parameters   &    No speaker- and utterance-oriented properties      & No speaker- and utterance-oriented properties\\  \midrule

    Anchored in an utterance or embedding context  &    With time and world parameters     & No time and world parameters\\  \midrule

    Partial control possible  &    Partial control possible      & Exhaustive control\\  \midrule

    Temporally independent  &    Future orientation     & Event time simultaneous with time of matrix event\\
  \lspbottomrule
 \end{tabularx}
\end{table}

The complements are in a containment relation from which a complexity hierarchy is derived, extending from the most clausal (Proposition) to the least clausal type (Event). Clausehood “[is] represented through criteria of independence, transparency, integration, and complexity, and the implicational nature of the hierarchy is observed (…) in that Class 3 can never be more independent, more complex, less transparent and less integrated than Class 2; and Class 2 can never be more independent, more complex, less transparent and less integrated than Class 1” (\citealt{wurmbrandlohninger2020}).\largerpage

The complement types are semantic sorts which express conceptual primitives. These are in a coherent containment relation: Propositions are elaborations of Situations, Situations are elaborations of Events (\citealt{ramchandsvenonius2014}: 18, 20). The containment relation involves an existentially closed Event, which is then related to a time; this creates a Situation. Combined with speaker-oriented parameters, a Proposition results. The complements have minimal requirements for their domains, resulting from their properties. Proposition complements have independent tense, thus require an operator domain (e.g. CP), since “(…) aspects of the meaning of an attitude configuration are situated in the operator domain of the complement clause. The operator domain also separates the matrix predicate and the embedded temporal domain (…)” (\citealt{wurmbrandlohninger2020}, following \citealt{kratzer2006} and \citealt{moultonnodatenat,moultonnodatecomp}). Situation complements (TP) have an independent temporal domain but the embedded clause needs future orientation from the matrix verb; they carry pre-specified tense value. Event complements (\emph{v}P) receive simultaneous interpretation to the matrix verb and do not have independent tense. See Table \ref{Pajtab3} for a summary.\largerpage

%TABLE 3 
\begin{table}
\caption{Complement composition (\citealt{wurmbrandlohninger2020})}
\label{Pajtab3}
 \begin{tabular}{l llll}
  \lsptoprule
            & Proposition & Situation  & Event \\
  \midrule
  Minimal requirements  &   Operator domain  &    TMA domain  &    Theta domain    \\
                         & TMA domain & Theta domain  &        \\
                         & Theta domain &   &        \\
  Complexity  &   Most complex &   Intermediate  &    Least complex    \\
  \lspbottomrule
 \end{tabular}
\end{table}

\begin{exe}
\ex \label{Paj4} \citet{wurmbrandlohninger2020} 
\begin{xlist}
\ex \label{Paj4a}
The ICH reflects increased syntactic and/or semantic complexity from the right to the left: a type of complement can never be obligatorily more complex than the type of complement to its left on ICH. 
\ex \label{Paj4b}
The implicational relations of the ICH arise through containment relations among clausal domains.
\end{xlist}
\end{exe}


ICH signature effects are even observed in languages like Greek (G) or Bulgarian (B) which exclusively have finite complement complements, through their choice of clause introducers for Proposition, Situation and Event complements: Proposition complements have \emph{če} (B) and \emph{oti} (G) as clause introducers, Event complements \emph{da} (B) and \emph{na} (G), and Situation complements can vary between the two but require overt future with \emph{če} (B)/\emph{oti} (G). The clause introducers align along the ICH, with Proposition and Event complements displaying the opposite values. It should also be noted although only finite complements are possible in Bulgarian and Greek, this does not violate the ICH: complements to the right of the hierarchy are never more independent or complex, or less transparent and integrated than the complements to their left (see \citet{wurmbrandlohninger2020}  and \citet{wurmbrandetal2020}  for a detailed analysis). The semantic classification of the complement types and the synthesis model proposed by \citet{wurmbrandlohninger2020} captures cross-linguistic differences, since it allows for flexibility and variation: complements have to match the semantic specifications of the matrix verb and are not syntactically selected, thus can have different forms; the morphosyntactic properties displayed in complements can differ from language to language, as long as the minimal requirements are met. The mapping between syntax and semantics “(…) allows mismatches in one direction: syntactic structure that has no consequence for interpretation is possible” (\citealt{wurmbrandlohninger2020}), which means larger structures are possible across languages. 

\subsection{A finiteness universal}
\label{Paj1.3}

Serbian displays another ICH signature effect. All types of complements can be finite:

\begin{exe}
\ex \label{Paj5} Serbian (\citealt{todorovickwurmbrand2020}: 2)
\begin{xlist}

\ex \label{Paj5a} 
\gll {Jovan} {je} {pokušao} {da} {čita} {knjigu.}\\
    Jovan \textsc{aux} tried \textsc{da} read.3.{\sg}.\textsc{pres}.\textsc{impfv} book\\
\glt `Jovan tried to read the book.’\\

\ex \label{Paj5b}
\gll {Jovan} {je} {odlučio} {da} {čita} {knjigu.} \\
    Jovan \textsc{aux} decided \textsc{da} read.3.{\sg}.\textsc{pres}.\textsc{impfv} book\\
\glt `Jovan decided to read the book.’\\

\ex \label{Paj5c}
\gll {Jovan} {je} {tvrdio} {da} {čita} {knjigu.}\\
    Jovan \textsc{aux} claimed \textsc{da} read.3.{\sg}.\textsc{pres}.\textsc{impfv} book\\
\glt `Jovan claimed to be reading the book.’\\

\end{xlist}
\end{exe}
 

Crucially, Event and Situation complements allow infinitives; Proposition complements do not. Again, Proposition and Event complements display opposing values while for Situation complements, both options are possible. 

\begin{exe}
\ex \label{Paj6} Serbian (\citealt{todorovickwurmbrand2020}: 2)
\begin{xlist}

\ex \label{Paj6a} 
\gll {Pokušala} {sam} \{{čitati/} {da} {čitam}\} {ovu} {knjigu.} \\
    tried.{\sg}.\tsc{fem} \tsc{aux}.1{\sg} \{read.\tsc{inf}.\tsc{impfv}/ \tsc{da} read.1{\sg}\} this book\\
\glt `I tried to read the book.’\\

\ex \label{Paj6b}
\gll {Odlučila} {sam} \{{čitati/} {da} {čitam}\} {ovu} {knjigu.} \\
    decide.{\sg}.\tsc{fem} \tsc{aux}.1{\sg} \{read.\tsc{inf}.\tsc{impfv}/ \tsc{da} read.1{\sg}\} this book\\
\glt `I decided to read the book.’\\

\ex \label{Paj6c}
\gll {Tvrdim} \{{*čitati/} {da} {čitam}\} {ovu} {knjigu.} \\
    claim.1{\sg} \{*read.\tsc{inf}.\tsc{impfv}/ \tsc{da} read.1{\sg}\} this book\\
\glt `I claim to be reading the book.’ [\citealt[305, (22a,b)]{vrzic1994}]\\

\end{xlist}
\end{exe}
 

Not all of the finite complements above involve a CP domain. Event complements do not allow an overt subject, Event and Situation complements in Serbian allow phenomena associated with size reduction such as clitic climbing (marginal in Situation complements), NPI/NC licensing by the matrix \tsc{neg}, free \emph{wh}-ordering etc. From their transparency it follows that they project less structure, TPs and \emph{v}Ps respectively. These operations are not possible in propositional complements, they are opaque and therefore project more structure, resulting in a CP. 

Assuming a TP for Situation and a \emph{v}P for Event complements in Serbian leaves the question of the presence of \emph{da}, traditionally analysed as a complementiser, to which I will return in \sectref{Pajsect:4}.
\citegen{todorovickwurmbrand2020} approach separates clause size from finiteness. By comparing finite and non-finite complements in the South Slavic languages in Table \ref{Pajtab4}, \citet{wurmbrandetal2020} further develop the approach into an implicational finiteness universal in \REF{Paj7} that operates along the ICH. Following \citet{adger2007} finiteness is assumed to be “(…) the spell-out of agreement features, which can occur on \emph{v}, T or C” (\citealt{wurmbrandetal2020}: 130).

%tables 4 
\begin{table}
\caption{Finiteness in South Slavic \citep[126]{wurmbrandetal2020}}
\label{Pajtab4}
 \begin{tabular}{l llll}
  \lsptoprule
            & Proposition & Situation  & Event \\
  \midrule
  Bulgarian, Macedonian  &  finite &   finite  &   finite    \\
  Serbian, Bosnian? & finite  & (non-)finite  & (non-)finite\\
  Slovenian, Bosnian & finite & (non-)finite  &    non-finite    \\
  Croation &   finite &   non-finite  &    non-finite     \\
  \lspbottomrule
 \end{tabular}
\end{table}


The implicational distribution of finiteness stems from the containment relations the complements are in: “Since clausal domains are in a containment configuration (…), it follows that settings in a lower domain affect all clauses that include that domain, i.e. also clauses with additional higher domains, since higher domains necessarily include the lower ones'' (\citealt{wurmbrandetal2020}: 133). 

\begin{exe}
\ex \label{Paj7} Finiteness universal (\citealt{wurmbrandetal2020}) \\ 
If a language {allows/requires} finiteness in a type of complement, all types of complements further to the \emph{left} on ICH also \{allow/require\} finiteness.
\end{exe}


Consequently, Proposition and Situation complements cannot be less finite than Event complements. However, finiteness is possible in all types of complements. It does not define clausehood; rather the syntactic structure aligns along the ICH.  

\section{Some aspects of verbal morphology in Akan}\label{Pajsect:2}

\subsection{Tense and aspect}\largerpage

\citet[92]{bhat1999} classifies languages as tense-prominent, aspect-prominent and mood-prominent. Languages choose tense, aspect or mood “(…) as the basic category and express distinctions connected with it in great detail; they represent the other two categories in lesser detail and further, they use peripheral systems like the use of auxiliaries, or other indirect means, for representing these latter categories” (\citealt[91]{bhat1999}). He further states that languages can select two or more equally prominent categories. According to \citet{osam2008}, Akan is an aspect-prominent language with four aspects (Completive, Perfect, Progressive, Habitual) and a future tense (expressed with the marker \emph{bɛ}). \citet{boadi2008} distinguishes between the Progressive, Habitual and Stative; he does not classify the Perfect as an aspect and states two tense markers, future and past.

\subsection{The discussion on past, perfect and completive}

\citet{osam2008} states that two aspects are “perfective”: completive and perfective, which are atemporal although there is a connection to a past tense: “There is a strong tendency for PFV [perfective] categories to be restricted to past time reference. I interpret this restriction as a secondary feature of PFV (…)” (\citealt{dahl1985}: 79). \citet{boadi2008} notes that “[b]oth the Past and Perfect depict the event described by a verb as having completed at, and as having occurred prior to, the time of speaking. In both \emph{ͻ à-dídí} ‘he has eaten’ and\emph{ ͻ dìdí-ì} ‘he ate’ the subject of the sentence is understood to have gone through an event prior to the time of speaking” (\citeyear[24]{boadi2008}).
The completive, realised by doubling the last vowel or consonant of the clause-medial verb stem, has been analysed as past tense and translated as such before but \citet[85]{osam2008} argues that “[d]espite the fact that past time is implied in the meaning of the completive suffix, my contention is that past time is a secondary meaning of the Akan morpheme. This is because the suffix cannot encode events that are located prior to the time of speech but which are imperfective. In the Akan aspectual forms (…), the Perfect, Progressive, and Habitual are all imperfective. When any past event is marked by any of these imperfective aspects, the coding does not involve in any way the use of the suffix I have called the completive”.

\begin{exe}
\ex \label{Paj8} Akan (Asante Twi, \citealt{osam2008}: 75) \\ 
\gll {Kofi} {hù-ù} {abofra} {no.}\\
    Kofi see-{\compl} child \tsc{def}\\
\glt `Kofi saw the child.’\\
\end{exe}


The completive refers to events or actions that have been completed before the utterance and does not occur with the other aspects. It cannot be used for imperfective events, which is one of the strongest arguments for it being an aspect, not tense marker. Imperfective is expressed via the temporal marker \emph{na} (\emph{nna} in Fante). 
The perfect aspect \emph{a-}, subject to vowel harmony with the verb stem, on the other hand signals that an event or action has occurred in the past but still has relevance to the present point in time.

\begin{exe}
\ex \label{Paj9} Akan (Fante, \citealt{osam2008}: 79) \\ 
\gll {Mà-á-t\`{ɔ}} {bi.}\\
    1{\sg}.{\subj}-\tsc{perf}-buy some\\
\glt `I have bought some.’\\
\end{exe}


Osam states that the completive corresponds to the traditional perfect aspect in other languages. Its connection to past time meaning is indisputable as even Osam himself (\citeyear[87]{osam2008}) assumes that the completive might be in the process of developing into a past tense form. No matter one’s stance on the aspect-tense debate, it must be acknowledged that the completive/past has an aspectual function and only refers to completed events. Boadi notes here: “The \tsc{past} Tense [completive aspect in \citet{osam2008}] affix \emph{-e} performs an aspectual function corresponding to that performed by the Perfective in the Slavic languages (…)” (\citeyear[29]{osam2008}). As the theoretical input for my analysis is originally based on data from the South Slavic languages, I adapt the completive as an aspect in this chapter. 

\subsection{The infinitive affix}

\citet{boadi2008} mentions a non-finite indicative affix \emph{a-} and disagrees with Osam’s (\citeyear{osam2008}) assessment of \emph{a-} as a consecutive marker: “The Infinitive is a one-member set represented by the prefix \emph{a-}. It differs from the other Indicative affixes in not expressing aspect and other temporal relations. Unlike the finite forms its verb does not occur as the only predicate in independent clauses.” (\citealt{boadi2008}: 12).\footnote{Both the perfect and infinitive affix are realised as \emph{a-}. They can be distinguished when the construction is negated: perfective \emph{a-da} becomes \emph{n-da-a}, while the negation of the infinite \emph{a-da} is \emph{a-n-da}.}

\begin{exe}
\ex \label{Paj10} Akan (\citealt{boadi2008}: 12) \\ 
\gll {\`{ɔ}} {rè-t\'{ɔ}}  {bí} {á-k\`{{ɔ}}.}\\
    he \textsc{prog}-buy some \tsc{inf}-go\\
\glt `He is buying some to take away.’\\
\end{exe}



The mention of a non-finite affix is especially interesting with regards to restructuring processes as “(…) [t]he close relationship between the TAM markers in Akan is evident in the fact that the non-finite mood affix \emph{-a} does not express aspect or time, while the finite affixes express aspect and tense” (\citealt{owusu2014}: 22). It will be shown in \sectref{Pajsect:3.3} that the infinite \emph{-a} occurs in Event complements. 

\section{Complement types}\label{Pajsect:3}

\subsection{Proposition complements}\label{Pajsect3.1}

Proposition complements in Akan are always finite. They do not have pre-spec\-i\-fied tense values and require the clause introducer \emph{sɛ/dɛ} in the examples below.

\begin{exe}
\ex \label{Paj11} Akan (Twi, personal communication)\footnote{%
‘Believe’ in Akan is formed with \emph{gye} ‘collect’ and \emph{di} ‘eat’. It is an integrated serial verb construction (see \citealt{osam2003}), meaning that “(…) the events encoded by the verbs are tightly integrated and thus cannot be separated into constituent parts” (\citealt{owusu2014}: 42). As seen below, the object of the complement clause can intervene between the two parts of the serial verb, and a pronoun has to be affixed to the verb in the complement clause. 

\begin{exe}
\ex \label{Paj1footnote} Akan (Twi, personal communication)
\begin{xlist}

\ex[*]{ \label{Paj1footnotea}
\gll {Akua} {gye-di} {s{ɛ}} {Kofi} {re-da} {seseyi.} \\
     Akua collect-eat s{ɛ} Kofi \tsc{prog}-sleep {right now}\\
\glt `Akua believes Kofi is sleeping right now.'}

\ex[*]{ \label{Paj1footnoteb}
\gll {Akua} {gye} {Kofi} {di} {s{ɛ}} {{ɔ}-re-da} {seseyi.} \\
     Akua collect Kofi eat s{ɛ} 3{\sg}-\tsc{prog}-sleep {right now}\\
\glt `Akua believes Kofi to be sleeping right now.'}

\ex[*]{ \label{Paj1footnotec}
\gll {{ɔ}-a-n-ka} {d{ɛ}} {{ɔ}-b{ɛ}-ba.} \\
     3{\sg}.{\subj}-{\compl}-\tsc{neg}-say {\comp} 3{\sg}.{\subj}-\tsc{fut}-come\\
\glt `S/he didn't say s/he will come.'}

\end{xlist}
\end{exe}}
\begin{xlist}
\ex[]{ \label{Paj11a}
\gll {Me-ka-a} {s{ɛ}} {me-kenkan-e} {nwoma} {no.} \\
    1{\sg}-say-{\compl} s{ɛ} 1{\sg}-read-{\compl} book \tsc{def}\\
\glt `I said that I read the book.’}

\ex[*]{ \label{Paj11b}
\gll {Me-ka-a} {kenkan-e} {nwoma} {no.} \\
    1{\sg}-say-{\compl} read-{\compl} book \tsc{def}\\
\glt `I said (claimed) to have read the book.’}

\ex[]{ \label{Paj11c}
\gll {Akua} {gye-di} {s{ɛ}} {Kofi} \{{b{ɛ}-da} {yiye}/ {re-da} {yiye}\}. \\
   Akua take-eat s{ɛ} Kofi \{\tsc{fut}-sleep well/ \tsc{prog}-sleep well\}\\
\glt `Akua believes that Kofi will sleep well/is sleeping well.'}

\ex[*]{ \label{Paj11d}
\gll {Akua} {gye-di} {Kofi} {da} {yiye.} \\
     Akua take-eat Kofi sleep well\\
\glt `Akua believes Kofi to sleep well.'}

\end{xlist}
\end{exe}

They require an overt subject which can differ from the subject of the matrix clause.

\begin{exe}
\ex \label{Paj12} Akan (Fante, \citealt{osam1998}: 29)
\begin{xlist}

\ex \label{Paj12a} 
\gll {Me-ka-a} {d{ɛ}} {o-hu-u} {maame} {no.} \\
    1{\sg}.{\subj}-say-{\compl} {\comp} 3{\sg}.{\subj}-see-{\compl} woman \tsc{def}\\
\glt `I said that s/he saw the woman.’\\

\ex \label{Paj12b}
\gll {Me-ka-a} {d{ɛ}} *{hu-u} {maame} {no.} \\
    1{\sg}.{\subj}-say-{\compl} {\comp} see-{\compl} woman \tsc{def}\\
\glt `I said that she saw the woman.’

\end{xlist}
\end{exe}
 

Lastly, clauses can be negated independently from each other. 

\begin{exe}
\ex \label{Paj13} Akan (Fante, \citealt{osam1998}: 37)
\begin{xlist}

\ex \label{Paj13a} 
\gll {{ɔ}-ka-a} {d{ɛ}} {{ɔ}-re-m-ba.} \\
    3{\sg}.{\subj}-say-{\compl} {\comp} 3{\sg}.{\subj}-{\prog}-\tsc{neg}-come\\
\glt `S/he said s/he will not come.’\\

\ex \label{Paj13b} 
\gll {{ɔ}-a-n-ka} {d{ɛ}} {{ɔ}-re-m-ba.} \\
    3{\sg}.{\subj}-{\compl}-\tsc{neg}-say {\comp} 3{\sg}.{\subj}-{\prog}-\tsc{neg}-come\\
\glt `S/he didn't say s/he will not come.’\\

\ex \label{Paj13c} 
\gll {ͻ-a-n-ka} dɛ ͻ-bɛ-ba \\
\tsc{3sg.subj}-\compl-\tsc{neg}-say {\comp} 3\tsc{sg.subj-fut}-come\\
\glt `S/he didn’t say s/he will come.'\\

\end{xlist}
\end{exe}
 

As expected, Proposition clauses are temporally independent (albeit connected to the matrix verb through the attitude holder’s NOW, see \citealt{wurmbrand2014a}) and do not show any subject restrictions. They also can be negated individually. Consequently, the construction consists of two clauses; an operator domain must be projected which means finiteness in Akan in the complement clause is mandatory. As in the South Slavic languages (\citealt{wurmbrandetal2020}: 131), I propose that the locus of finiteness in Proposition complements in Akan is in the CP. 

\subsection{Situation complements}

Situation complements in Akan require a future orientation, as expected. They are always finite. In Twi, the finite complement is possible without an overt future marker (\ref{Paj15}) while in Fante, overt future seems to be mandatory (\ref{Paj14}). Both dialects require the presence of \textit{sɛ/dɛ} in Situation complements. 

\begin{exe}
\ex \label{Paj14} Akan (Fante, \citealt{osam1998}: 31)
\begin{xlist}

\ex[]{ \label{Paj14a} 
\gll {Maame} {no} {hy{ɛ}-{ɛ} b{ɔ}} {d{ɛ}} {{ɔ}-b{ɔ}-k{ɔ}}. \\
   woman \tsc{def} promise-{\compl} {\comp} 3{\sg}.{\subj}-\tsc{fut}-go\\
\glt `The woman promised to go.’}

\ex[*]{ \label{Paj14b} 
\gll {Maame} {no} {hy{ɛ}-{ɛ} b{ɔ}} {d{ɛ}} {{ɔ}-k{ɔ}-e}. \\
    woman \tsc{def} promise-{\compl} {\comp} 3{\sg}-go-{\comp}\\
\glt `The woman promised that she went.’}

\end{xlist}

\ex \label{Paj15} Akan (Twi, personal communication)
\begin{xlist}

\ex []{\label{Paj15a} 
\gll {Me-si-i} {gyina{ɛ}e} {s{ɛ}} {me-kenkan} {nwoma} {no} {okyena}/ {*{ɛ}nnora.}\\
   1{\sg}-build-{\compl} decision s{ɛ} 1{\sg}-read book \tsc{def} tomorrow/ yesterday\\
\glt `I decided to read this book tomorrow/ yesterday.’ (not intended: I decided yesterday to read this book; `yesterday' refers to `read')}

\ex[*]{ \label{Paj15b} 
\gll {Me-si-i} {gyina{ɛ}e} {kenkan} {nwoma} {no}. \\
     1{\sg}-build-{\compl} decision read book \tsc{def}\\
\glt `I decided to read this book.’}

\end{xlist}
\end{exe}
 

It can be observed that the clauses possess a pre-specified tense value and are not temporally independent as Proposition complements. Furthermore, the complement does not have a simultaneous tense interpretation with the matrix verb, thus requires a TAM domain. Therefore, finite Situation complements in Akan project a \emph{v}P and a TP. According to \citet{adger2007}, agreement features expressing finiteness are not limited to a CP but can be on heads of a TP or \emph{v}P which explains the grammaticality of finite Situation complements as in (\ref{Paj15a}). 
	
A TMA domain is what is minimally required for Situation complements. However, the ICH and containment approach allow for larger projected structures than minimally required. This is the case for complement clauses with overt future such as in (\ref{Paj14a}). As shown in (\ref{Paj14}) and (\ref{Paj15a}), Situation complements demand an irrealis event. This interpretation is either reached via overt future or a covert future modal \tsc{woll} in the TMA domain (\citealt{todorovickwurmbrand2020}, \citealt{wurmbrand2014a}, \citealt{todorovic2015}, see also \citet{wurmbrandlohninger2020} for more evidence for \tsc{woll} from Greek). In (\ref{Paj15a}), \tsc{woll} is licensed via Merge with the matrix verb. In (\ref{Paj14a}), an operator domain is projected and prevents \tsc{woll} from merging with the Situation verb. \tsc{woll} is licensed by Tense, and the spell-out is an overt future marker. The projected operator domain is in line with the ICH since although the construction is more complex than minimally required, the semantics are unchanged.  

\subsection{Event complements}\label{Pajsect:3.3}

\subsubsection{\textit{tumi} `can, manage'}

In contrast to Indo-European languages, modality is not expressed via modal auxiliaries in the majority of Kwa languages but instead conveyed by different means such as affixes, periphrastic modal constructions and adverbs. \emph{Tumi} ‘can’ is analysed as a modal auxiliary, which can be dynamic, epistemic or deontic and requires a semantically full verb as complement, by \citet{owusu2014} who argues that it does not carry lexical meaning and only refers to ability. \citet{wurmbrandlohninger2020} include modals and the non-modal implicative ‘manage’ in the Event class, “(…) as they form the least clausal contexts in most languages (…)” although they maintain that “(…) modals may be functional heads in certain languages, which constitutes a different type of complementation (…). The generalizations regarding the ICH apply foremost to complements of lexical verbs.”

\begin{exe}
\ex []{\label{Paj16}Akan (\citealt{owusu2014} 104: 67) \\ 
\gll {Kofi} {tumi} {da}.\\
    Kofi be.able.to sleep\\
\glt `Kofi can sleep.’}

\ex[*]{\label{Paj17}Akan (Twi, personal communication) \\ 
\gll {Kofi} {tumi} {s{ɛ}} {da}.\\
    Kofi be.able.to s{ɛ} sleep\\}
\end{exe}


It never takes a clause introducer; its complements are non-finite. While the future tense/modal marker \emph{bɛ} is usually affixed to the verb in the complement clause in Akan (\ref{Paj20a}), it attaches to \emph{tumi}, the matrix verb of the construction (\ref{Paj20b}). The completive however is marked only in the complement (\ref{Paj20c}). \emph{Tumi} can both mean ‘can’ and ‘manage’.\largerpage[-1]

\begin{exe}
\ex\label{Paj18}Akan (Twi, personal communication)
\begin{xlist}
\ex \label{Paj18a} 
\gll {Ama} {kyer{ɛ}} {s{ɛ}} {{ɔ}-b{ɛ}-noa} {aduane.}\\
  Ama claim s{ɛ} 3{\sg}-\tsc{fut}-cook food\\
\glt `Ama claims that she will cook food.'

\ex \label{Paj18b} 
\gll {Ama} {b{ɛ}-tumi} {a-noa} {aduane.}\\
  Ama \tsc{fut}-can \tsc{inf}-cook food\\
\glt `Ama will be able to cook food.'

\ex \label{Paj18c} 
\gll {Ama} {tumi} {noa-a} {aduane} {{ɛ}nnora.}\\
  Ama can cook-{\comp} food yesterday\\
\glt `Ama managed to cook food yesterday.'
\end{xlist}
\end{exe}\largerpage
 

These complements have no temporal independence or pre-determined tense value, they can only be interpreted simultaneously to the matrix predicate. They are also subject to exhaustive control, which means that the complement cannot have a subject. 

\begin{exe}
\ex \label{Paj19} Akan (Twi, personal communication)
\begin{xlist}

\ex[*]{\label{Paj19a} 
\gll {Ama} {tumi} {noa-a} {aduane} {{ɔ}kyena}.\\
  Ama can cook-{\compl} food tomorrow\\
\glt `Ama managed to cook food tomorrow.'\\}

\ex[*]{\label{Paj19b} 
\gll {Ama} {tumi} {{ɔ}-noa}.\\
  Ama can 3{\sg}-cook\\}

\end{xlist}
\end{exe}
 

According to \citeauthor{haspelmath2016} (\citeyear{haspelmath2016}: 299, following \citealt{bohnemeyernodate}: 501), the only way to test for clause size that holds cross-linguistically is negation. If clauses cannot be negated independently, the structure is monoclausal. In constructions with \emph{tumi}, the verbs cannot be negated independently; both verbs have to carry the negative affix. 

\begin{exe}
\ex \label{Paj20} Akan (Twi, personal communication)
\begin{xlist}

\ex[]{\label{Paj20a} 
\gll {Akua} {n-tumi} {n-noa} {aduane}.\\
  Akua \tsc{neg}-can \tsc{neg}-cook food\\
\glt `Akua cannot cook.'}

\ex[*]{ \label{Paj20b} 
\gll {Akua} {n-tumi} {noa} {aduane}.\\
  Akua \tsc{neg}-can cook food\\
\glt `Akua cannot cook.'}

\ex[*]{ \label{Paj20c} 
\gll {Akua} {tumi} {n-noa} {aduane}.\\
  Akua can \tsc{neg}-cook food\\
\glt `Akua cannot cook.'}

\end{xlist}
\end{exe}

Since the complement can neither be negated individually, nor has its own temporal or aspectual domain, I conclude that the complement projects a \emph{v}P. 

\subsubsection{\textit{b{ɔ} mm{ɔ}den} `try'}

\emph{Bɔ mmɔden} ‘try’ is an interesting case as it can have either a finite complement with the clause introducer as in (\ref{Paj23a}), or a non-finite complement as in (\ref{Paj23b}). Crucially, neither complement can receive an interpretation with ‘tomorrow’, as Proposition and Situation complements do.

\begin{exe}
\ex \label{Paj21} Akan (Twi, personal communication)
\begin{xlist}

\ex \label{Paj21a} 
\gll {Me-bɔ-ɔ} {mmɔden} {sɛ} {mɛ-kenkan-e} {nwoma} {no} *{ɔkyena.}\\
    1{\sg}-hit-{\compl} effort s{ɛ} 1{\sg}-read-{\compl} book \tsc{def} *tomorrow\\
\glt `I tried to read the book *tomorrow.'\\

\ex \label{Paj21b} 
\gll {Me-bɔ-ɔ} {me ho} {mmɔden} {kenkan-e} {nwoma} {no} *{ɔkyena.}\\
    1{\sg}-hit-{\compl} myself effort read-{\compl} book \tsc{def} *tomorrow\\
\glt `I tried to read the book *tomorrow.'\\

\end{xlist}
\end{exe}

As expected for Event complements, the matrix verb and its complement show the tightest connection of all three complement types. The complement does not have a TMA domain and is dependent on the temporal value of the matrix verb, the verbs have to agree, and the non-finite complement cannot have a subject.  

\begin{exe}
\ex \label{Paj22} Akan (Twi, personal communication)\footnote{The speaker mentioned here that in written Akan, {Me-bɔ} in (\ref{Paj22a}) should be \emph{re-bɔ}, but in spoken Akan the progressive marker is omitted most of the time.}
\begin{xlist}

\ex[]{ \label{Paj22a} 
\gll {Me-bɔ} {me ho } {mmɔden} {a-kenkan} {nwoma} {no. }\\
   1{\textsc{sg}}-hit myself effort \tsc{inf}-read book \tsc{def}\\
\glt `I will try to read the book.'\\}

\ex[]{ \label{Paj22b} 
\gll {Me-bɔ-{ɔ}} {me ho } {mmɔden} {a-kenkan} {nwoma} {no.}\\
    1{\sg}-hit-{\compl} myself effort \tsc{inf}-read book \tsc{def} \\}

\ex[*]{ \label{Paj22c} 
\gll {Me-bɔ}  {me ho } {mmɔden} {kenkan-e} {nwoma} {no.}\\
    1{\sg}-hit myself effort read-{\compl} book \tsc{def} \\}

\ex[*]{ \label{Paj22d} 
\gll {Me-bɔ-{ɔ}}  {me ho } {mmɔden} {Kofi} {kenkan-e} {nwoma} {no.}\\
    1{\sg}-hit-{\compl} myself effort Kofi read-{\compl} book \tsc{def} \\
\glt `I tried that Kofi read the book.'\\}


\end{xlist}
\end{exe}

It should be emphasized that even finite complements with \textit{sɛ} cannot receive a temporal interpretation different from the matrix verb, and the subject has to co-refer with the subject of the matrix verb. 


\begin{exe}
\ex \label{Paj23} Akan (Twi, personal communication) 
\begin{xlist}

\ex[*]{ \label{Paj23a} 
\gll {Me-re-bɔ} {mmɔden} {s{ɛ}} {m{ɛ}-kenkan-e} {nwoma} {no. }\\
   1{\sg}-\tsc{prog}-hit effort s{ɛ} 1{\sg}-read-{\compl} book \tsc{def}\\
\glt `I'm trying to read the book.' (complement in the past) \\}

\ex[*]{\label{Paj23b} 
\gll {Me-re-bɔ} {mmɔden} {s{ɛ}} {Kofi} {kenkan-e} {nwoma} {no. }\\
   1{\sg}-\tsc{prog}-hit effort s{ɛ} Kofi read-{\compl} book \tsc{def}\\
\glt ‘I tried that Kofi read the book.’ \\}

\end{xlist}
\end{exe}

It has to be mentioned here that \citet[29]{osam1998} states that complements with \emph{bɔ mbɔdzen} `try’ always require overt future, and that aspectual/temporal agreement between the matrix verb and the complement is ungrammatical.

\begin{exe}
\ex \label{Paj24} Akan (Fante, \citealt[29]{osam1998}) 
\begin{xlist}

\ex[]{ \label{Paj24a} 
\gll {Kofi} {b{ɔ}-{ɔ}} {mb{ɔ}dzen} {d{ɛ}} {{ɔ}-b{ɛ}-y{ɛ}} {edwuma} {no.}\\
  Kofi hit-{\compl} effort {\compl} 3{\sg}-\tsc{fut}-do work \tsc{def}\\
\glt `Kofi tried to do the work.'}

\ex[*]{ \label{Paj24b} 
\gll {Kofi} {b{ɔ}-{ɔ}} {mb{ɔ}dzen} {d{ɛ}} {{ɔ}-y{ɛ}-{ɛ}} {edwuma} {no.}\\
  Kofi hit-{\compl} effort {\compl} 3{\sg}-do-{\compl} work \tsc{def}\\
\glt `Kofi tried to do the work.'}

\end{xlist}
\end{exe}

This is certainly interesting data; one could for example speculate if these constructions support \citet{owusu2014} analysis of \textit{bɛ} as a modal instead of future tense marker. \citet{wurmbrandlohninger2020} also mention that “(…) verbs like \emph{try} pose an interesting in-between case. While (…) a future interpretation is not possible, \emph{try} complements also involve an irrealis aspect since the embedded event cannot be realized (i.e. completed) yet in a trying situation. Since \emph{try} usually patterns with Event verbs, we have included it among this class, but we wish to note that it is a clear border-case (…) which may also show properties of the Situation class”. Since none of the speakers who worked with me have produced overt future in \emph{bɔ mmɔden}, I will leave this analysis for future work.

Lastly, non-finite complements with \emph{bɔ mmɔden} cannot be negated individually, both the matrix verb and the complement in (\ref{Paj25a}) have to carry the negation affix. Finite structures too cannot be negated independently, here in (\ref{Paj26}) only the matrix verb can have a negative affix which negates the entire construction. Thus both the non-finite and the finite Event structures project a \emph{v}P. 

\begin{exe}
\ex \label{Paj25} Akan (Twi, personal communication) 
\begin{xlist}

\ex[]{ \label{Paj25a} 
\gll {Me-m-mɔ} {mmɔden} {n-kenkan} {nwoma} {no. }\\
   1{\sg}-\tsc{neg}-hit effort \tsc{neg}-read book \tsc{def}\\
\glt `I'm not trying to read the book.'}

\ex[*]{ \label{Paj25b} 
\gll {Me-m-mɔ} {mmɔden} {kenkan} {nwoma} {no. }\\
   1{\sg}-\tsc{neg}-hit effort read book \tsc{def}\\
\glt `I'm not trying to read the book.'}

\ex[*]{ \label{Paj25c} 
\gll {Me-bɔ} {mmɔden} {n-kenkan} {nwoma} {no. }\\
   1{\sg}-hit effort \tsc{neg}-read book \tsc{def}\\
\glt `I'm trying not to read the book.'}

\end{xlist}

\ex \label{Paj26} Akan (Twi, personal communication) 
\begin{xlist}

\ex[]{ \label{Paj26a} 
\gll {Me-a-m-mɔ} {mmɔden} {s{ɛ}} {me-kenkan-e} {nwoma} {no. }\\
   1{\sg}-{\compl}-\tsc{neg}-hit effort s{ɛ} 1\tsc{sg}-read-{\compl} book \tsc{def}\\
\glt `I didn't try to read the book.'}

\ex[*]{ \label{Paj26b} 
\gll {Me-b{ɔ}-mɔ} {mmɔden} {s{ɛ}} {me-a-n-kenkan} {nwoma} {no. }\\
   1{\sg}-hit-{\compl} effort s{ɛ} 1\tsc{sg}-{\compl}-\tsc{neg}-read book \tsc{def}\\
\glt `I tried to not read the book.'}

\ex[*]{\label{Paj26c} 
\gll {Me-a-m-mɔ} {mmɔden} {s{ɛ}} {me-a-n-kenkan} {nwoma} {no. }\\
   1{\sg}-{\compl}-\tsc{neg}-hit effort s{ɛ} 1\tsc{sg}-{\compl}-\tsc{neg}-read book \tsc{def}\\
\glt `I didn't try to read the book.'}

\end{xlist}
\end{exe}

\section{Concluding remarks} \label{Pajsect:4}

In this chapter, I have examined Akan verbs with meanings similar to Proposition, Situation and Event verbs in other languages to find out whether they align along the ICH, and what domains their complements project. The findings confirm the hypotheses of the ICH (\citealt{wurmbrandlohninger2020}), and the finiteness universal (\citealt{wurmbrandetal2020}), both repeated below in Table \ref{Pajtab5} and \REF{Paj27}. 

%TABLE 1, REPEATED 
\begin{table}
\caption{Implicational complementation hierarchy (\citealt{wurmbrandlohninger2020})\label{Pajtab5}}
  \fittable{\begin{tabular}{l c l}
  \lsptoprule
 \tsc{most} \tsc{independent} &  \multirow{4}*{Proposition » Situation » Event}  & \tsc{least} \tsc{independent}\\
  \tsc{least} \tsc{transparent}&    & \tsc{most} \tsc{transparent}\\
    \tsc{least} \tsc{integrated}&    & \tsc{most} \tsc{integrated}\\
      \tsc{most} \tsc{complex} &    & \tsc{least} \tsc{complex} \\
  \lspbottomrule
 \end{tabular}}
\end{table}



\begin{exe}
\ex \label{Paj27} \citet{wurmbrandlohninger2020}
\begin{xlist}
\ex \label{Paj27a}
The ICH reflects increased syntactic and/or semantic complexity from the right to the left: a type of complement can 	never be obligatorily more complex than the type of	complement to its left on ICH. 
\ex \label{Paj27b}
The implicational relations of the ICH arise through containment relations among clausal domains. 
\end{xlist}
\end{exe}


Proposition and Event complements in Akan show opposite values on the ICH. Event complements are less complex than Situation complements, and Situation complements are less complex than Proposition complements. The complements also align hierarchically in terms of independence, transparency and integration. 

\begin{exe} 
\ex \label{Paj28} Finiteness universal (\citealt{wurmbrandetal2020}) \\
If a language \{allows/requires\} finiteness in a type of complement, all types of complements further to the \emph{left} on ICH also \{allow/require\} finiteness.
\end{exe}

The finiteness universal has also been confirmed in Akan. As seen in Table \ref{Pajtab6}, Akan shows ICH signature effects, with only Event complements allowing non-finite complements. As they themselves can be finite too, finiteness can be in every complement to its left, and every domain in Akan. 

%Table 6
\begin{table}
\caption{Finiteness in Akan complements}
\label{Pajtab6}
 \begin{tabular}{l cccc}
  \lsptoprule
            & Proposition & Situation  & Event \\
  \midrule
  finite  & ✔  & ✔    &   ✔    \\
  non-finite & *  & *  & ✔ \\ 
 
  \lspbottomrule
 \end{tabular}
\end{table}

The three complement types have minimal requirements for their domains, but larger structures are a possibility in this framework. Assuming a TP for Situation and a \emph{v}P for Event predicates leaves questions on the status of \emph{sɛ} which underwent a grammaticalisation process from a verb \emph{se} ‘say’ into a functional element (\citealt{osam1996}). 

\emph{Sɛ} (\emph{dɛ} in Fante) has traditionally been analysed as a complementiser (\citealt{lord1993}, \citealt{boadi1972} \citealt{osam1998} amongst others) but has various different lexical and grammatical functions, “(…) including a verb meaning `resemble'; a comparative particle; a factitive object marker; a that-complementizer; an adverbial subordinator introducing clauses of purpose, result, reason, and condition; and a component of miscellaneous adverbials meaning `until', `although', `unless', `or', and `how'” (\citealt{lord1993}: 151). \citet[127]{agyekum2002} lists another function of \emph{sɛ} as an interpretive marker `that'. I propose yet another function, as a finiteness visualiser.

\citet{todorovickwurmbrand2020} analyse \emph{da} in Serbian as a finiteness visualiser. As shown in \sectref{Paj1.3}, finiteness can occur in different domains in Serbian, and an analysis of \emph{da} as a complementiser in these structures is ruled out. Based on the positions of adverbs in complement constructions, they argue that \emph{da} is in T in Situation complements, and \emph{v} in Event complements. T and \emph{v} are “(…) not morphologically realized. If these heads are inserted with a [+\tsc{finite}] feature, \emph{da} can be seen as the morphological spell-out of this feature (…). We hypothesize that \emph{da} spells out [+\tsc{finite}] on a clausal head (C, T, \emph{v}), if no other feature of that head overtly expresses finiteness. For instance, if there is a semantic tense feature in T, the verb realizes that feature (either via lowering or V-movement) and [+\tsc{finite}] is made visible via the (true) tense feature and would not be spelled out in addition as \emph{da}” (\citealt{todorovickwurmbrand2020}).

As I have shown above, Event complements in Akan select a \emph{v}P, and Situation events either a TP or a CP. Finite \emph{sɛ} complements are possible in Event complements and obligatory in Situation complements, thus I preliminary conclude for now that \emph{sɛ}, as \emph{da} in Serbian, is a finiteness visualiser in these constructions that can be in different domains. 


\section*{Acknowledgments}
I would like to give a heartfelt thanks to Sabine Laszakovits and William Oduro for their help, patience and crucial input. Without them, I could have never completed this chapter.

{\sloppy\printbibliography[heading=subbibliography,notkeyword=this]}

\end{document}
