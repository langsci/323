\documentclass[output=paper]{langscibook}
\author{Jonathan David Bobaljik\affiliation{Harvard University} and Leo Bobaljik Wurmbrand\affiliation{Vienna, Austria}}
\title{Particle-verbs in an Austrian-American code-switching idiolect}
\abstract{We describe a productive construction in an Austrian-American code-switching idiolect, in which English particle verbs are integrated into a German frame with what appears to be a melange of grammatical properties from the two languages, including apparent doubling of inflectional morphology. We suggest that English particle verbs are too large to be simply borrowed as verbs, but are instead the right size to be pressed into the mold of a different complex verbal structure that occurs independently in German: verb clusters. This minimal reanalysis provides a structure that is similar enough to allow for congruent lexicalization in code mixing. Intuitions about the distribution of the construction suggest that it is systematic and not merely a case of doubling or copying of the suffixes, and our proposed analysis captures the main distinctions between possible and impossible contexts.}

\begin{document}
\maketitle

\section{Introduction}
In this paper, we describe a productive construction in an Austrian-American code switching (CSw) or code-mixing idiolect, illustrated in (\ref{setten-uppen}):

\ea\label{setten-uppen}
\ea{
\gll\label{setten-uppen-a}Ich geh es sett-en upp-en.\\
I go it set-\textsc{inf} up-\textsc{inf}\\
\glt`I'll go set it up.' (2017.12.26)
}
\ex{
\gll\label{hangen-outen-b}Wir werden hang-en out-en.\\
We will hang-\textsc{inf} out-\textsc{inf}\\
\glt `We're going to hang out.' (2017.06.02)
}\z
\z

The hallmark of this construction is the use of an English particle verb in a German sentence, with the English verb-particle order (otherwise impossible in German outside of verb-second contexts) but with German inflectional morphology on -- it appears -- both the verb and particle. The construction is freely available in contexts like (\ref{setten-uppen}), but is otherwise restricted, and for example is completely impossible in simple main clause configurations with no auxiliary or modal:

\ea\label{setten-final-up}
\ea[*]{
\gll Wir sett-en es heute Abend upp-en.\\
we set-\textsc{1pl} it today evening up-\textsc{1pl}\\
}
\ex[]{
\gll\label{setten-up}Wir sett-en es heute Abend up.\\
we set-\textsc{1pl} it today evening up\\
\glt `We'll set it up this evening.'
}
\z
\z 

Our first goal in this paper is to provide a description of the construction, and its distribution. Secondly, we offer some thoughts about why the construction has the peculiar distribution it has, focusing in particular on the contrast in (\ref{setten-uppen}) versus (\ref{setten-final-up}), relative to typologies of code-mixing. For example, \citet{muysken00} provides a typology of code-mixing with three major types, but this construction seems not to match up to any of them. Muysken's proposed types (p. 3) are as follows: 

\ea\label{muyskentypes}
\ea \textit{alternation} between structures from multiple languages
\ex \textit{insertion} of lexical items or entire constituents from one language into a structure from the other
\ex \textit{congruent lexicalization} of material from different lexical inventories into a shared grammatical structure
\z
\z

We assume that the matrix language frame for all examples considered here is German, and that these are not \textit{alternations}. The code-mixed contexts we describe here are perceived by the participants as being in German. Instead, we will focus on \textit{insertion} and \textit{congruence} -- the ways in which English material is integrated into a German syntactic frame (see also \citealp{myersscotton93}). We suggest that examples like (\ref{setten-up}) represent \textit{congruent lexicalization}: both German and English have particle-verb combinations that can be discontinuous, where the (inflected, finite) verb precedes the particle. That is, the sentence meets all the conditions of German syntax (including separation of the finite verb and the particle to meet the verb second [V2] requirement), and also satisfies the English syntax of the particle-verb construction: the verb precedes the particle, though may be separated from it by (leftward) verb movement.

The intuition we wish to pursue is that the construction in (\ref{setten-uppen}) represents a type of repair, when congruence would otherwise fail. The phrasal nature of particle verb combinations makes them ``too big'' to integrate via insertion into a V position. At the same time, there is no congruent parse which satisfies both the (German) verb-final requirement, and the English requirement that the particle follows the verb. Despite this, we suggest that (\ref{setten-uppen}) is nevertheless made congruent by coopting another independently occurring piece of German syntax, which results in the apparent morphological doubling. In a manner reminiscent of the notion of ``derivatively grammatical'' in \citet[242]{chomsky65}: the construction is not, as such, derived by the grammar, but it comes ``close-enough'' to a structure that is congruent to be usable.

 

\section{Setting up the puzzle}

\subsection{Background: The code-switcher}

The linguistic behaviour that we describe here is that of a balanced bilingual (Austrian German and American English). The speaker lived in New England (USA) until age 11, and has for the five-and-a-half years since then lived in Vienna (Austria), though over both periods spent time in the other location. In the US, his schooling, and thus peer-group, was exclusively English, but the language of the home was German (the native language of one parent, an  L2 of the other). In Austria, schooling is in German, though with some English instruction through a ``dual language program''; the peer group is primarily, though not exclusively, German-dominant, but all speak English as well. Data here is drawn from passive observations by the speaker's parents (sporadic language notes over many years), combined with the intuitions/acceptability judgments of the second author, who is a native speaker of the CSw idiolect. Use of English particle-verbs in German frames is attested in the parental notes since 2016, and the construction with doubled inflection occurs since 2017.\footnote{A reviewer asks about the extent of code-mixing in the data. The parental notes were not kept in a way that allows us to provide a quantitative answer. Our intuition is that some amount of mixing occurs daily, although most utterances are primarily in one language. We suspect that some conversational scenarios, such as discussion in German of events that took place in English (school, movies, computers, US news), favour more mixing. In response to the reviewer's query, the second author examined a sample of 100 text messages (of two words or more) between him and his L1-English parent. 93 of these were entirely in German, and 7 involved switching (none were entirely in English). Both authors share the intuition that this medium under-represents the frequency of switching in conversation. Most mixed SMSs involved an English word inserted into German, with German morphosyntax (\ref{renewed}), but some involved larger constituents (\ref{solemnly}): 

\ea\gll \label{renewed}Es renewed am 15.\\
It renew-{\textsc{3sg}} on 15th\\
\glt `It renews on the 15th.' (From context, it is clear that orthographic ``-ed" here represents German \textsc{3sg.pres} \textit{-t}. Autocorrect may have influenced the orthographic form.)
\ex\gll \label{solemnly}Mama musste solemnly swearen.\\
Mama must-{\textsc{past}} solemnly swear-\textsc{inf}\\
\glt `Mama had to solemnly swear.'
\z 
} 

As far as we have been able to determine, the construction is idiolectal. A small survey of seven members of the speaker's peer-group in Vienna, Austria, found no other users of the construction in (\ref{setten-uppen}), although none of the peer code switchers have the same linguistic background (other survey respondents have lived primarily in German-speaking social contexts for far longer, some exclusively). Nevertheless, the \textit{setten uppen} construction has been observed as a stable part of the speaker's grammar for a period spanning multiple years, was used un-self-consciously in spontaneous, running discourse, and the speaker has consistent acceptability judgments about the distribution of the construction. We assume, therefore, that the construction is grammatical in this idiolect. By all available evidence, it constitutes systematic, rule-governed linguistic behaviour, and thus it should be amenable to an account within a theory of code-switching. 

\subsection{Distribution of the construction}\largerpage

The \textit{setten uppen} construction is most natural, and most attested in our limited notes, in the configuration in (\ref{setten-uppen}), where it occurs in the clause-final position that corresponds to an infinitive verb in German:\footnote{The most frequent spontaneously occurring examples are those with no obvious parallel particle verb in German, such as \textit{hang-en out}, and \textit{sett-en up}. Examples like \textit{throw-en out}, and \textit{turn-en on}, for which there are German particle verb with the same meaning (\textit{weg-werfen} `away-throw'; \textit{ein-schalten} `on-switch') are judged less natural, but still possible.}

\ea\label{setten-uppen-2}
\ea{\gll Ich geh es sett-en upp-en. (CSw)\\
I go it set-\textsc{inf} up-\textsc{inf}\\
\glt `I'll go set it up.' (2017.12.26)}
\ex{\gll Wir werden hang-en out-en. (CSw)\\
We will hang-\textsc{inf} out-\textsc{inf}\\
\glt `We're going to hang out.' (2017.06.02)}
\ex{\gll Ich muss es noch turn-en on-en. (CSw)\\
I must it still turn-\textsc{inf} on-\textsc{inf}\\
\glt `I still have to turn it on.'}
\z
\z 

Doubling of any sort is categorically impossible in finite main clauses, that is, in the V2 construction where the verb is in second position, and the particle final (whether or not there is intervening material): 

\ea
\ea[*]{\gll Wir sett-en es heute am Abend upp-en. (CSw)\\
we set-\textsc{1pl} it today at evening up-\textsc{1pl}\\}
\ex[*]{\gll Wir sett-en up(p-en) es heute am Abend. (CSw)\\
we set-\textsc{1pl} up(\textsc{-1pl}) it today at evening\\}
\ex[*]{\gll Wir sett-en es up(p-en) heute am Abend. (CSw)\\
we set-\textsc{1pl} it up(\textsc{-1pl}) today at evening\\
\glt `We'll set it up this evening.'}
\z 
\z 

Instead, in simple main clauses, English particle verbs are readily integrated into German syntax: the verb is inflected as a finite verb and the particle occurs clause-finally, uninflected.\footnote{We assume that the particle in (\ref{heute-abend}--\ref{chop-up}) is English \textit{up} although phonetically, this is hard to distinguish from the German particle \textit{ab} {[}ap{]} `down.' We do feel that the vowel quality, in careful introspection, is reliably distinct: English {[}ʌp{]} rather than German {[}ap{]}. If it were reanalyzed as German \textit{ab}, we would not expect the behaviour in (\ref{setten-uppen-2}) etc., but instead patterning with (\ref{part.ger-verb}), below. Particle \textit{out} does not raise this issue, as there is no homphonous German particle.} 

\ea\label{v2-verb-particle}
\ea\label{heute-abend}{\gll Wir sett-en es heute am Abend up. (CSw)\\
we set-\textsc{1pl} it today at evening up\\
\glt `We'll set it up this evening.'}
\ex\label{chop-up}{\gll Und dann chopp-en sie die Oreos up. (CSw)\\
and then chop-\textsc{3pl} they the Oreos up\\
\glt `And then they chop the Oreos up.' (2017.08)}
\ex{\gll Sie figure-t es out. (CSw)\\
she figure-\textsc{3sg} it out\\
\glt `She'll figure it out.' (2017.06.02)}
\z\z 

Between these two poles, the data are somewhat less clear. Apparent doubling of inflectional morphology is possible in embedded finite clauses, where the finite verb is final, in particular with first and third person plural subjects (\ref{bevor-wir}). Thus we have minimal pairs between matrix and embedded clauses:\footnote{Since first person plural inflection is identical to infinitival morphology, the glosses in this example are to be taken with a grain of salt. We will suggest below that (\ref{bevor-wir}) does not actually involve doubling of the \textsc{1pl} finite inflection.}

\ea
\ea[]{\label{bevor-wir}\gll \ldots bevor wir hang-en out-en.\\
\ldots before we hang-\textsc{1pl} out-\textsc{1pl}\\
\glt `\ldots before we hang out.'} 
\ex[*]{\gll Wir hang-en out-en.\\
We hang-\textsc{1pl} out-\textsc{1pl}\\
\glt `We're hanging out.'}
\z\z 

\begin{sloppypar}
First and third person plural subject inflection has the property that it is homophonous with infinitival morphology. With other subject person-number combinations, judgments about finite embedded clauses are less clear. Over a small sample of introspective judgments, we find doubling sometimes accepted (\ref{bevor-du-settest}) but sometimes not (\ref{bevor-du-hangst}). 
\end{sloppypar}

\ea
\ea[*]{\label{bevor-du-hangst}\gll \ldots bevor du hangst outst.\\
\ldots before you hang-\textsc{2sg} out-\textsc{2sg}\\
\glt `\ldots before you hang out.'}
\ex[]{\label{bevor-du-settest}\gll \ldots bevor du es settest upst.\\
\ldots before you it set-\textsc{2sg} up-\textsc{2sg}\\
\glt `\ldots before you set it up.'}
\z\z

Note that there is an alternative with \textit{-en} on the verb and person inflection only on the particle:

\ea\label{bevor-du}
\ea{\gll \label{bevor-du-hangen}\ldots bevor du hangen out(e)st.\\
\ldots before you hang-\textsc{inf?} out-\textsc{2sg}\\
\glt `\ldots before you hang out.'} 
\ex{\gll \label{bevor-du-setten}\ldots bevor du es setten upst.\\
\ldots before you it set-\textsc{inf?} out-\textsc{2sg}\\
\glt `\ldots before you set it up.'}
\z\z 

Examples like this suggest that the construction is not to be modeled simply as copying or doubling of an inflectional affix. This seems to distinguish the construction from the colloquial English \textit{picker upper} nominalizations, to which we return below.   

Finally, we note that participial constructions are also verb-final in German. For particle verbs, whatever pattern there is does not appear to be  particularly stable. The following was attested, in spontaneous speech, as was \textit{gelined up} `lined up' (2017.02.05, noted without sentential context):

\ea\gll Ich hab-'s schon ge-sett-et up.\\
I have-it already \textsc{ptcp}-set-\textsc{ptcp} up\\
\glt `I already set it up.' (2016.08.15)
\z 

Other combinations also sometimes appear to be possible: 

\ea
\ea{\gll Ich hab(e) es ge-set-up-t.\\
I have it \textsc{ptcp}-set-up-\textsc{past}\\}
\ex{\gll Ich hab(e) es ge-sett-en up-t.\\
I have it \textsc{ptcp}-sett-\textsc{ptcp} up-\textsc{ptcp}\\}
\ex{\gll Ich hab(e) es ge-sett-et up-t.\\
I have it \textsc{ptcp}-set-\textsc{past} up-\textsc{ptcp}\\
\glt `I set it up.'}
\z\z 

Yet in the second author's judgment, for many participle verbs one might try to integrate, including five other combinations that freely enter into the \textit{setten uppen} configuration considered here, there is simply no acceptable outcome as a participle:

\ea
\ea[*]{\gll Ich hab(e) ge-hang-ed ge-out-ed.\\
I have \textsc{ptcp}-hang-\textsc{past} \textsc{ptcp}-out-\textsc{past}\\}
\ex[*]{\gll Ich hab(e) ge-hang-ed out.\\
I have \textsc{ptcp}-hang-\textsc{past} out\\}
\ex[*]{\gll Ich hab(e) ge-hang-out-ed.\\
I have \textsc{ptcp}-hang-out-\textsc{past}\\
\glt `I hung out.'}
\z\z 

\section{Analysis}

\subsection{Preliminaries}

We start by noting that the construction at issue differs from simple lexical borrowing since it involves an apparent blend of German and English grammar: German particle verbs (separable prefixes, see \citet{wurmbrand98} for an analysis) categorically show the order particle-verb when the verb is not in second position -- the order between verb and particle is decidedly English in (\ref{setten-uppen}) -- while the remainder of the sentence shows distinctly German grammar, for example, verb-finality when abstracting away from verb second -- note pre-verbal object in (\ref{throwen-outen}) and (\ref{setten-uppen}):

\ea\label{throw-out}
\ea{I have to throw out something. (Eng., also: \ldots throw something out)}
\ex{\gll Ich muss etwas weg-werf-en. (Ger.)\\
I must s.th. away-throw-\textsc{inf}\\
\glt `I must throw something away.'}
\ex{\gll \label{throwen-outen}Ich muss etwas throw-en out-en. (CSw)\\
I must s.th. throw-\textsc{inf} out-\textsc{inf}\\
\glt `I must throw something away.'}
\z\z

English particle verb combinations can be borrowed into German, and used fully within German syntax. Constructions where only the English verb is borrowed, and combined with a German particle, are well established: \textit{aus-flipp-en} `to flip out', \textit{aus-freak-en} `to freak out', etc. Spontaneous examples of this pattern are also attested in our corpus:\largerpage

\ea\label{part.ger-verb}
\ea{(CSw/Ger)\\\gll Ich glaub mein Lunch ist aus-ge-spill-ed.\\
I believe my Lunch is out-\textsc{ptcp}-spill-\textsc{past}\\
\glt `I think my lunch spilled (out).' (2016.04.21)}
\ex{(CSw/Ger)\\\gll \ldots und dann [es] runter-scrape-n, und es wird ein-ge-roll-t.\\
\ldots and then [it] down-scrape-\textsc{inf}, and it \textsc{aux} in-\textsc{ptcp}-roll-\textsc{past}\\
\glt `and then [they] scrape [it] off, and it is rolled in.' (2017.09.01)}
\z\z 

In one corpus study \citep{willeke06} this is the only form in which English particle verbs are attested as borrowings into German: the particle is always German and only the verb is borrowed.\footnote{Willeke phrases the observation differently, since he treats verbs like \textit{download, upgrade, update} as particle verbs. For English, we reserve the term \textit{particle verb} for those constructions in which the verb and particle do not form a single word (sometimes also called ``phrasal verbs''), so our category includes \textit{set up, flip out,} etc., but not \textit{download, upgrade} since \textit{load down} and \textit{grade up} are not possible forms of these verbs. Willeke's classification focuses on participial forms such as \textit{down-ge-loaded} and \textit{up-ge-graded}, in which the position of the participial prefix \textit{ge} suggests separability. It is worth noting that with the exception of a single occurrence of \textit{load \ldots down} (from \textit{Die Presse}), these forms do not occur with the prefix separated in Willeke's corpus. Such separation is felt to be wholly unnatural to the second author of the present study:

\ea
\ea[]{\gll Ich upgrade bald meinen Computer. {[}ʌpgreɪd-ə{]}\\
I upgrade soon my computer {[}upgrade-\textsc{1sg}{]}\\}
\ex[*]{\gll Ich grade bald meinen Computer up.\\
I grade soon my computer up\\
\glt `I'll upgrade my computer soon.'}
\z\z

} These may of course also occur in separated form:

\ea(Ger)\\\gll Steffi flipp-te nach ihrem sechsten Streich in Wimbledon nicht aus, \ldots\\
Steffi flip-\textsc{past} after her sixth coup in Wimbledon not out\\
\glt `Steffi didn't flip out after her sixth coup at Wimbledon, \ldots.' (\textit{Mannheimer Morgen} cited in \citealp[67]{willeke06})
\z 

Some (though not all) members of the code-switching peer-group also permit borrowing of both English verb and particle, but with the particle showing fully German syntax, at least for some examples. Five of seven speakers reported (\ref{out-throw-2}) as acceptable and 4 allowed (\ref{out-hand}):\largerpage

\ea\label{part.eng-verb}\judgewidth{\%}
\ea[\%]{\gll \label{out-throw-2}Ich muss etwas out-throw-en. (CSw-peers)\\
I must s.th. out-throw-\textsc{inf}\\
\glt `I must throw something away.'}
\ex[\%]{\gll \label{out-hand}Ich muss es noch out-hand-en. (CSw-peers)\\
I must it still out-hand-\textsc{inf}\\
\glt `I still have to hand it out.'}
\z\z 

Finally, we note that a sporadically attested pattern of borrowing treats the English verb-particle combination as an unanalyzed whole: Two speakers accepted (\ref{hangouten}) and one of them also accepted (\ref{turnonen}), suggesting that those particular particle verbs have been reanalyzed as stems, but these were not widely accepted in the group:

\ea\judgewidth{\%}
\ea[\%]{\gll \label{hangouten}Wir hang-out-en am Abend. (CSw-peers)\\
we hang-out-\textsc{1pl} on evening\\
\glt `We're hanging out this evening.'}
\ex[\%]{\gll \label{turnonen}Wir turn-on-en das Gerät. (CSw-peers)\\
we turn-on-\textsc{1pl} the machine\\
\glt `We turn on the machine.'}
\z\z 

The \textit{setten uppen} construction differs from all of these in that it combines elements of both German and English syntax, rather than embedding English morphemes in a completely German syntactic frame. At the same time, it differs from the three canonical code-switching constructions in the typology proposed by \citet{muysken00}, which recognizes (i) insertion of single constituents into the other language, (ii) a mid-sentence switch from one code (language) to the other, and (iii) congruent lexicalization where the gross syntax of the two languages coincides. This construction is instead a blend of the two, conforming directly to neither, and this is what we think makes it interesting.

Before proceeding further, we note also that in all of the code-switched examples in the idiolect described here, English phonology is clearly retained in the English morphemes even when they combine with German inflectional morphology. Thus e.g., {[}θɹoʊ\nobreakdash-n̩{]} is pronounced with phonemes that are impossible in German. This is not merely a question of possibly borrowed non-native phonemes, but also application of English phonology (e.g., flapping in \REF{gevisited}) or violations of German phonological constraints, such as the failure of final devoicing in examples such as the following:\footnote{The preservation of English final voiced consonants in \textit{ge-} prefixed participles among German emigrant code-switching is also noted in \citet{gross00}, as cited in \citet[159--160]{myersscotton02}. Examples are also attested in our corpus with English phonology on roots combined with adjectival morphology (note English diphthong {[}eɪ{]}, and interdental {[}θ{]}):

\ea\gll Ich glaub, er hat was {alien-es. {[}ˈeɪliən-əs{]}}\\
I believe he has something alien-\textsc{neut.sg}\\
\glt `I think he has something alien.' (2016.04.25)
\ex\gll Für jeden Holiday machen sie etwas {ge-theme-t-es. {[}ge-θim-t-əs{]}}\\
For every holiday make they something \textsc{ptcp}-theme-\textsc{past}-\textsc{neut.sg}\\
\glt `For every holiday, they do something themed.' (2015.09)
\z
}

\ea\gll Die war die Einzige, die nicht {ge-begg-ed ({[}ge-bɛg-d{]})} hat.\\
she was the.\textsc{fem} only.one who.\textsc{fem} not \textsc{ptcp}-beg-\textsc{past} has\\
\glt `She was the only one who didn't beg.' (2014.06.01)\footnote{On some interesting properties of the \textit{Einzige} construction, in particular as regards semantic versus grammatical gender, see \citet{wurmbrand17a}.}
\ex\gll \label{gevisited}Der Mann, der {ge-visit-ed ({[}gə-vɪzɪɾ-əd{]})} hat, hat gesagt, dass jemand hat {ge-sledd-ed  ({[}gə-slɛd-əd{]})}, und ist waist-deep ins Wasser gekommen. \\
The man who.\textsc{masc} \textsc{ptcp}-visit-\textsc{past} has has said that someone has \textsc{ptcp}-sled-\textsc{past} and is waist-deep into water come\\
\glt `The man who visited said that someone went sledding and went waist-deep into the water.' (2016.02.15)
\ex\gll Es war von jemandem, der ein Server-owner ist, und er hat jemanden ge-bann-ed ({[}ge-b{\ae}n-d{]}).\\
it was from someone who a Server-owner is and he has someone \textsc{ptcp-}ban-\textsc{past}\\
\glt `It [a video] was about someone who owns a server, and he had banned someone.' (2016.05.01)
\z

This is in apparent violation of the Free Morpheme Constraint \citep{poplack80} and the related claim in \citet[45]{macswan99} that phonological systems cannot be mixed. Our observations thus align with those of \citet[159--160]{myersscotton02} and others cited there, where code-switching within the word, with morphemes retaining the phonology of their source language, is both possible and routine.

\subsection{Integrating English particle verbs}

We return now to some thoughts on the analysis of the \textit{setten-uppen} construction. Above, we have noted that it is more of a blend than a switch between two codes: the construction preserves features of both languages. More specifically, the context of use of all of these utterances is perceived by the participants to be German, although it is clearly recognized that these are English elements. We suggest, then, that one way to think about all of these examples is that they involve integration of an English particle verb into an otherwise German sentential frame. We say ``integration'' (i.e., Muysken's \textit{congruent lexicalization}) rather than ``borrowing'' for the reasons noted in the previous subsection: the construction involves preservation of the English syntactic order: \textit{verb {\ldots} particle}. 

Seen this way, we can explain why in the verb second configuration, nothing special happens: In German main clauses in simple tenses (with no modal or auxiliary), the verb moves to second position, yielding a surface string that can be superficially similar to English (as in \ref{gebe-auf}). Even when a non-subject topic precedes the verb, the construction still shares with English the property that the finite, inflected verb precedes the uninflected particle (as in \ref{gebe-ich-auf}):

\ea
\ea{I gave it up.}
\ex{\gll \label{gebe-auf}Ich gebe es auf. (Ger)\\
I give it up\\
\glt `I'll give it up.'}
\ex{\gll \label{gebe-ich-auf}Seine Spitzenposition gibt er nicht so leicht auf. (Ger)\\
His Lead.position give.\textsc{3sg} he not so easily up\\
\glt `He won't give his lead up so easily.'}
\z\z 

English particle-verbs can be fully integrated (other than phonology), and there is no motivation for inflection doubling or any other accommodation:

\ea\label{v2-verb-particlea}
\ea{\gll \label{heute-abend-2}Wir sett-en es heute am Abend up. (CSw)\\
we set-\textsc{1pl} it today at evening up\\
\glt `We'll set it up this evening.'}
\ex{\gll Und dann chopp-en sie die Oreos up. (CSw)\\
and then chop-\textsc{3pl} they the Oreos up\\
\glt `And then they chop the Oreos up.' (2017.08)}
\ex{\gll Sie figure-t es out. (CSw)\\
she figure-\textsc{3sg} it out\\
\glt `She'll figure it out.' (2017.06.02)}
\z\z 

Even examples where the particle is not strictly final can be seen as being fully integrated into German syntax, since German allows extraposition of PPs, as in (\ref{fangt-an}) (with German particle-verb \textit{an-fangen} `on-catch'\,=\,`begin, start'):\largerpage

\ea\label{kickt-off}
\ea{\gll Die 2. Staffel kick-t off mit einem Cliffhanger. (CSw)\\
the 2nd season kick-\textsc{3sg} off with a Cliffhanger\\
\glt `The second season kicks off with an event.' (2020.05.26, adapted)}
\ex{\gll \label{fangt-an}Der 2. Staffel fängt \textit{t}$_{\text{PP}}$ an [ mit einer wichtigen Szene ]. (Ger)\\
the 2nd season catch-\textsc{3sg} {} on {} with an important scene\\
\glt `The second season starts with an important scene.'}
\z\z 

Outside of simple declaratives (more accurately: apart from verb-second contexts with no auxiliaries), the languages diverge. For example, in the presence of a modal or auxiliary, or in an embedded finite clause, English preserves the coarse syntax: \textit{verb {\ldots} particle}, but in German, the order of particle and verb are inverted and the particle occurs as a (separable) prefix on the verb (see \citealp{wurmbrand98} for an analysis):

\ea
\ea{\gll \label{aufgeben}Ich werde es \{auf-geben\} / \{*geben auf\}. (Ger)\\
I will  it up-give / give up\\
\glt `I will give it up.'}
\ex{\gll \label{aufgegeben}Ich habe es \{auf-ge-geben\} / \{*ge-geben auf\}. (Ger)\\
I have it up-\textsc{ptcp}-given / \textsc{ptcp-}given up\\
\glt `I have given it up.'}
\ex{\gll \label{bevor-aufgeben}\ldots bevor wir es \{auf-geben\} / \{*geben auf\}. (Ger)\\
\ldots before we it on-give / give up\\
\glt `\ldots before we give it up.'}
\z\z 

Thus, the idea of integration as congruent lexicalization, meeting grammatical conditions of both languages, allows us not only to explain why the particle verbs integrate unchanged into matrix contexts, but why they fail to do so in clause-final position. In clause-final position, there is no way to simultaneously preserve the demands of English and German syntax, since these impose conflicting linearization constraints on the particle and the verb. 

So why, then, is doubling a solution?\largerpage

We suggest that part of the answer, though not the whole answer, lies in the kind of ambivalent structure of English particle verbs that leads to doubling, at least colloquially, in agent-nominalizations like \textit{picker upper, hanger outer, setter upper} and so on. As many authors have observed, (see, for example, \citealp{sproat85}, \citealp[160]{ackneel04}), such forms seem to be the result of a tension between trying to add the suffix \textit{-er} to the genuine verb, and on the other, to the right edge of the phrasal verb as a lexical unit. Doubling is then the (or an) optimal repair. This suggests a representation along the lines in (\ref{pv-bracketed}), where the phrasal verb is larger than a single word, yet not a maximal phrase, leaving some ambiguity about the projection level of the topmost node: 

\ea\label{pv-bracketed}
\begin{forest}
[V$^{n}$ 
    [V [set] ] 
    [ up ] ]  
\end{forest}
= {[} {[} set {]}$_{\text{V}}$ up {]}$_{\text{V}^n}$
\z

But this can't be the whole answer for the code-switching construction, since forms such as \textit{setten upst} in (\ref{bevor-du-setten}) suggest there is more than just copying or doubling involved. 

We suggest, tentatively, that the answer might lie in another construction made available by German syntax, namely, verb clusters (see \citealp{wurmbrand17vcl}), in particular, constructions involving more than one verb in the clause-final position, either as infinitive forms or where the structurally highest (and often, but not always rightmost) form is inflected: 

\ea
\ea{\gll \label{kaufen-koennen}Sie wird das Buch kauf-en könn-en. (Ger)\\
she will the book buy-\textsc{inf} can-\textsc{inf}\\
\glt `She will be able to buy the book.'}
\ex{\gll \ldots bevor sie es kauf-en könn-en. (Ger)\\
\ldots before they it buy-\textsc{inf} can-\textsc{3pl}\\
\glt `\ldots before they can buy it.'}
\ex{\gll \ldots bevor du es kauf-en kann-st. (Ger)\\
\ldots before you it buy-\textsc{inf} can-\textsc{2sg}\\
\glt `\ldots before you can buy it.'}
\z\z 

Superficially, examples like (\ref{kaufen-koennen}) have a sequence of \textit{-en} morphemes on the two final elements of the clause, just like the \textit{setten uppen} construction in (\ref{setten-uppen}). 

\ea 
\ea{\gll \label{kaufen-koennen-b}Sie wird es kauf-en könn-en. (Ger)\\
she will it buy-\textsc{inf} can-\textsc{inf}\\
\glt `She will be able to buy the book.'}
\ex{\gll \label{setten-uppen-b}Sie wird es sett-en upp-en.(CSw)\\
she will it set-\textsc{inf} up-\textsc{inf}\\
\glt `She will set it up.'}
\z\z 

Of course, \textit{up} is not a verb, and the ``doubling'' in (\ref{kaufen-koennen-b}) is simply an effect of the future modal \textit{wird}  selecting for an infinitival complement \textit{können} which in turn selects an infinitival complement. But there are two ways in which these constructions are more similar than they may appear. 

First, in phrase-final position, German verb clusters have properties of compounds, such as compound stress \citep{wurmbrand98}. In addition, they have special grammatical properties including re-ordering effects that show variation among speakers, languages, and dialects. In a head-final language like German, verbs selecting clausal (or verbal) complements are expected to line-up in the mirror-order of their English counterparts, as in (\ref{3-2-1}). While this is possible in German, numerous other possibilities exist. Austrian varieties also allow, for the combination of the future auxiliary, a modal, and a main verb, the orders shown in (\ref{clusters}) \citep{wurmbrand17vcl}:\footnote{We follow \citet{wurmbrand17vcl} in using subscripts to note the hierarchical order of the verbs -- in all of the orders in (\ref{clusters}), the future modal \textit{wird} `will' is finite, and selects the modal complement headed by \textit{können}, which in turn selects the phrase \textit{es kaufen} 'to buy it'.}

\ea\label{clusters}
\ea{\gll \label{3-2-1}\ldots weil er es kaufen$_{3}$ können$_{2}$ wird$_{1}$. (Ger)\\
\ldots since he it buy can will\\}
\ex{\gll \ldots weil er es kaufen$_{3}$ wird$_{1}$ können$_{2}$ . (Ger)\\
\ldots since he it buy will can\\}
\ex{\gll \ldots weil er es wird$_{1}$ kaufen$_{3}$ können$_{2}$ . (Ger)\\
\ldots since he it will buy can\\
\glt all: `since he will be able to buy it.' \citep{wurmbrand17vcl}}
\z\z 

The important observation for us here is that these cluster effects in many varieties (but not all) implicate a kind of compound-like structure consisting of all (and only) the verbs, perhaps derived via head-movement from phrasal complementation structures:

\ea\label{cluster-bracketed} 
\begin{forest}
[V$^{n}$
    [V$^{n}$ 
        [V [kaufen] ] 
        [können] ] 
    [ wird ]
    ]  
\end{forest}
= {[} {[} {[} kaufen {]}$_{\text{V}}$ können {]}$_{\text{V}}$ wird {]}$_{\text{V}}$
\z

Special (re-)ordering rules then apply within this derived complex verb (see \citealp{wurmbrand04} for a survey of the empirical landscape of West Germanic verb clusters). 

Our suggestion is that it is this type of compound-like verbal structure in clause-final position that provides the linguistic scaffolding for the \textit{setten uppen} construction. The verb cluster structure in (\ref{cluster-bracketed}) on the one hand, and the peculiar English phrasal-verb structure in (\ref{pv-bracketed}) that leads to suffix-doubling in nominalizations on the other, conspire together to provide a point for congruent lexicalization. In clause-final position, complex verbs are allowed, with inflection (including infinitival marking) on multiple elements. In German, this happens of course uniquely where all of the inflected elements are themselves verbal, but since particle verbs are a type of compound verb (\ref{pv-bracketed}), they can, informally ``sneak in'' to the cluster structure, permitting inflection on the final element because the whole complex constituent is, in some sense, verbal. 

Since clusters only arise in final position, the inflectional doubling will not be supported in simple verb-second clauses: those allow integration without needing to borrow the cluster structure, as in (\ref{v2-verb-particlea}). And since the basic cluster structure involves infinitival complementation, we can draw an analogy to the more acceptable examples of finite inflection of borrowed particle-verbs in final position considered above. In examples like (\ref{bevor-du}), repeated here: we see the actual verb inflected as an infinitive, with the finite morphology expressed at the end of (on the rightmost element of) the verbal complex, comparable to a cluster (\ref{kaufen-kannst}).

\ea
\ea{\gll \label{bevor-du-hangen2}\ldots bevor du hangen out(e)st. (CSw)\\
\ldots before you hang-\textsc{inf?} out-\textsc{2sg}\\
\glt `\ldots before you hang out.'} 
\ex{\gll \label{bevor-du-setten2}\ldots bevor du es setten upst. (CSw)\\
\ldots before you it set-\textsc{inf?} out-\textsc{2sg}\\
\glt `\ldots before you set it up.'}
\z\z 
\ea\gll \label{kaufen-kannst}\ldots bevor du es kauf-en kann-st. (Ger)\\
\ldots before you it buy-\textsc{inf} can-\textsc{2sg}\\
\glt `\ldots before you can buy it.'
\z 

Finally, we note that the cluster hypothesis does not provide a transparent model for the participial construction. Only modal-verb complements of \textit{haben} `have' enter into cluster formation, and then only in the \textit{infinitivus pro participio} ‘infinitive for participle’ construction (IPP), in which the modal complement to the auxiliary surfaces in infinitival form (\textit{könn-en}), where a participle (\textit{gekonnt}) would have been syntactically expected:

\ea{\gll \label{ipp}Der Kommissar hat den Fall nicht lös-en könn-en. (Ger)\\
The detective has the case not solve-\textsc{inf} can-\textsc{inf}\\
\glt `The detective couldn't solve the case.' (after \citealp{wurmbrand17vcl})}
\z 

Since there is no modal in \textit{set up} or \textit{hang out}, the IPP provides a poor basis for analogy, and since the participle in German involves a combination of prefix and suffix, there is no easy way to be ambivalent about whether the morphology is inflecting the combination as a whole, or just the peripheral element, which was the key to our analysis of the particle as part of a verbal cluster in the examples where the \textit{setten uppen} construction succeeds. We suggest that this may be why judgments about the participle construction are far less robust than with those configurations that do match up nicely to widely attested verb clusters. 

\section{Discussion and conclusion}

To wrap up, we come back to one point that we started with -- it seems clear that the \textit{setten uppen} construction is an admixture of two grammars. In matrix clauses, code-shifting takes the form of a \textit{verb {\ldots} particle} order, which is fully integrated into German syntax, and at the same time meets the demands of English syntax that the verb precede the particle. The \textit{setten uppen} construction comes about only when simple integration, in the form of borrowing of the individual pieces, is not possible, and the demands of the two languages conflict. In clause-final position, particles precede their verbs in German but follow them in all clause types in English. We suggested that the outcome is nevertheless grammatical, with the restrictions discussed above, in the mental grammar of the specific code switching idiolect we are documenting. It is clear from the evidence that the construction is rule-governed, and involves integration of an English constraint into a German grammatical frame. But ultimately we must recognize that the construction deviates from some aspect of German grammar somewhere along the line. Whatever causes German particles to surface as (separable) prefixes, for example, is not respected. Also, we have treated the construction as saying that it adheres to the linear order imposed by English verb-particle syntax, but have not said what that syntax is. It seems entirely reasonable to think that the \textit{verb {\ldots} particle} order in English is not a special property of particle verbs, but rather a general property of the head-initial nature of English syntax. This general property does not carry over to the CSw idiolect: the object of the particle verb follows regular German syntax, freely preceding the English-sourced verb. Intuitively, we think that the singling out of the linear relation between the verb and the particle is because it is these elements together that are listed as a unit with special meaning, and thus the evaluation of congruence in the CSw grammar ``cares'' only about the syntactic properties relating these two elements. But we are a long way from being able to formalize that in any useful way. Perhaps for these reasons, we might reconsider the notion of ``derivatively grammatical'' from \citet[242]{chomsky65}: that the grammar itself does not in fact \textit{directly} generate this construction at all, but instead, the on-the-fly demands of code switching provide for a type of ``close-enough'' acceptability. Code-switching may be seamless when congruence is achieved, and simply switching out lexical items (especially content words) yields strings that locally respect morphosyntactic demands on the lexical items. But, we suggest, it may also be nearly seamless when congruence as such is impossible to achieve, but at least some derivation exists that provides for a close analogy to support a given string within the matrix language. In the case at hand, German allows for verb clusters at the right periphery of the clause, and English allows for idiosyncratic interpretations of particle-verb combinations that have a tight structure as a kind of complex verb. These two properties are not normally coextensive, but provide just enough structure to allow for a successful integration of an English verb-particle combination into German as a complex, but separable, lexical unit, including its hallmark English syntactic order, but bearing inflection on each head. 

\section*{Acknowledgements}
We are very grateful to Susi for so much more than can be acknowledged here. This paper literally could not have been written without her. Vielen lieben Dank! We would like to thank the second author's classmates in Vienna for sharing their intuitions about code-switching, and two anonymous reviewers for their suggestions and questions.

{\sloppy\printbibliography[heading=subbibliography,notkeyword=this]}

\end{document}
