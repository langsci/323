\documentclass[output=paper]{langscibook}
\author{Troy Messick\orcid{0000-0002-1453-5212}\affiliation{Rutgers University} and Deepak Alok\affiliation{Panligua Language Processing}}
\title{Stripping in Hindi: Does clause size matter?}
\abstract{\citet{wurmbrand17} shows that \emph{that}-less complements can embed the ellipsis construction known as stripping in English. In Hindi, it is possible to embed stripping even in the presence of the complementizer like element \emph{ki}. We argue that the crucial difference between English and Hindi is the position in the structure the complementizer resides. The analysis of Hindi stripping also sheds light on negative stripping and alternative questions.}

\begin{document}
\maketitle

\section{Introduction}
\citet{wurmbrand17} explores the elliptical operation known as stripping. In the original formulation of the stripping transformation, the structural condition for the transformation was specified as conjunction (see e.g.,  \citealt{hankamer79}). This was done to ensure that ellipsis could not occur in embedded environments. As the sentences in \REF{maex1} suggest, stripping is possible in conjoined structures, but not embedded under speech and attitude verbs (elided material appears in \sout{strikeout}).

\ea \label{maex1}
    \ea[]{\label{maex1:a}Jane loves to study rocks and geography \sout{she likes to study t} too.}
    \ex[*]{\label{maex1:b}Jane loves to study rocks and John says that geography \sout{she loves to study t} too.}
    \z 
\z 
Following in this tradition, \citet{merchant03} also specifies that stripping can only occur in coordinations by having the ellipsis licensing E-feature come with a \emph{u}Conj feature that must be checked in an agree relation with a conjunction head. Wurmbrand notes that such theories cannot account for cases where stripping is possible in embedded environments when there is no overt complementizer as shown in \REF{maex2}.

\ea \label{maex2}
    Jane loves to study rocks and John says geography \sout{she loves to study t} too.
\z
Based on the distinction between sentences like \REF{maex1:b} and \REF{maex2}, Wurmbrand proposes the generalization in \REF{maex3}.

\ea \label{maex3}
    \emph{Embedded stripping generalization}\\
    Embedded stripping is only possible when the embedded clause lacks a CP. 
\z
She goes on to propose a novel analysis of stripping that accounts for this generalization. Near the end of her paper, she considers languages where it is not clear that \REF{maex3} holds. Consider the example in \REF{maex4} from Hungarian \citep{craenenbroeck06,craenenbroeck08,craenenbroeck13}. In \REF{maex4}, it appears that embedded stripping is possible even though the complementizer head \emph{hogy} is present.

\ea \label{maex4}
    \gll J\'{a}nos megh\'{i}vott valakit \'{e}s azt jiszem hogy B\'{e}l\'{a}t.\\
    J\'{a}nos invited someone.\textsc{acc} and that.\textsc{acc} think that B\'{e}l\'{a}.\textsc{acc}\\
    \glt `J\'{a}nos invited someone and I think that it was Bela.'
\z
One language not discussed by Wurmbrand is Hindi. In this paper we demonstrate that Hindi \emph{ki} can also be present in embedded stripping, but does not conform to the generalization that languages that allow for a structure like \REF{maex4} have \textit{wh}-movement to the specifier of FocP. We instead put forth an analysis based on the respective height of the complementizer like elements. This analysis can provide a satisfactory answer to this puzzle and also has implications for the nature of Hindi complementation, the structure of negative stripping and also the derivation of alternative questions.

The paper is outlined as follows, in Section \ref{mamasect2}, we explore the Hindi data and the distribution and nature of \emph{ki} in Hindi. In Section \ref{mamasect3}, we present our analysis and some extensions to different elliptical constructions. In Section \ref{mamasect4}, we conclude.

\section{Initial Hindi data}\label{mamasect2}
Like English, Hindi allows for stripping in coordinations with \emph{lekin} `but' and \emph{aur} `and'. This is demonstrated in \REF{maex5} with \emph{lekin}. In \REF{maex5}, the second conjunct has undergone stripping only leaving negation and \emph{Mohan-ko} behind as the remnant.

\ea \label{maex5}
    \gll Sita-ne Ram-ko tohafaa diyaa, lekin Mohan-ko nahi.\\
    Sita-\textsc{erg} Ram-\textsc{dat} gift give.\textsc{prf} but Mohan-\textsc{dat} \textsc{neg}\\
    \glt `Sita gave Ram a gift, but not Mohan.'
\z 
Note that \emph{Mohan-ko} must bear the dative case, if it appears in the unmarked absolutive as in \REF{maex6}, the example becomes ungrammatical. This follows from the ellipsis analysis of \REF{maex5}, as case connectivity is a hallmark characteristic of clausal ellipsis \citep{merchant01}.

\ea[*]{\label{maex6}
    \gll \dots lekin Mohan nahi.\\
    \dots but Mohan.\textsc{abs} \textsc{neg}\\
    \glt `\dots but not Mohan.'
}
\z 
In addition to case-connectivity, Hindi stripping also conforms to the P-stranding generalization. Example \REF{maex7} shows that Hindi postpositions cannot be stranded under movement and also obligatorily appear in the Hindi sluicing like construction.

\ea \label{maex7}
    \ea[*]{ 
        \gll Kis aap ke saath kaam kar-te haiN.\\
        who 2\textsc{pl} \textsc{gen} with work do-\textsc{hab} \textsc{aux}\\
        \glt Intended: `Who do you work with?'
    }
    \ex[]{ 
        \gll Sita khaana pakaa rahii hai, par Ali-ko nahiiN pa-taa kis-ke liye/*kis/kuan.\\
        Sita food cook \textsc{prog} \textsc{aux.prs}, but Ali-\textsc{dat} \textsc{neg} know-\textsc{hab.m} who-\textsc{gen} for/*who.\textsc{obl}/*who.\textsc{nom}\\
        \glt `Sita is cooking, but Ali doesn't know for whom.'\\ \hfill \citep[643]{gribanova16}
    }
    \z 
\z 
Just as in the sluicing like construction, stripping also obligatorily requires the postposition, as shown in \REF{maex8}.

\ea \label{maex8}
    \gll Ham-ne Ravi ke liye khaanaa banaayaa, aur Mohan ke *(liye) bhii.\\
    \textsc{1pl-erg} Ravi \textsc{gen} for food make.\textsc{prf.3} and Mohan \textsc{gen} *(for) also\\
    \glt `We made food for Ravi and, for Mohan too.'
\z 
Now let us turn to the stripping in embedded clauses. These judgments are less clear cut than others presented here. \citet{gribanova16} assign similar examples ``?/*''. It is unclear whether this indicates that there is inter speaker variation. The Hindi speakers consulted for this paper (including the second author) allow for embedded striping, and as shown in \REF{maex9}, it is possible with or without the complementizer like element \emph{ki}.\footnote{\citet{Bhattacharya12} note that \emph{ki} occurs in Hindi sluicing like constructions as well, as shown below.

\ea \label{maex9a}
    \gll Raam-ne kuch ciiz cori-kii-thii, par muhje nahe maluum *(ki) kyaa.\\
    Raam some thing stealing-\textsc{do-pst} but I \textsc{neg} know \textsc{c} what\\
    \glt `Ram is stealing something but I don't know what.' \hfill \citep[199]{Bhattacharya12}
\z 
Our informants also allow for \emph{ki} to occur in sluicing like constructions, but like the stripping examples its presence is not obligatorily, but is slightly preferred.}

\ea \label{maex9}
    \gll Sita-ne daawaa kiyaa ki Ram use bahar ghumaane le jaa saktaa hai, lekin vah nahii sochtii (ki) Mohan bhii.\\
    Sita-\textsc{erg} claim do.\textsc{prf} \textsc{ki} Ram her out visit.\textsc{inf} take go can.\textsc{imprf} be.\textsc{pres}, but she \textsc{neg} think.\textsc{imprf} (\textsc{ki}) Mohan also\\
    \glt `Sita claimed that Ram would ask her out, but she didn't think Mohan too.
\z 
\citet{kush16} also reports similar variation in such structures, which he refers to as single remnant gapping. For a subset of his consultants, the examples in \REF{maex11} and \REF{maex12} are acceptable.

\ea \label{maex11}
    \gll Akhbaar-me likhaa thaa ki Manu-ne Sita-ko dehk-aa, lekin magazin-me likhaa thaa ki Rina-ko \sout{dekh-aa}.\\
    newspaper-in written aux.\textsc{past.m.3sg} \textsc{c} Manu-\textsc{erg} Sita-\textsc{obj} see-\textsc{pfv.m.sg} but magazine-in written aux.\textsc{past.m.3sg} \textsc{c} Rina-\textsc{obj} \sout{see-\textsc{pfv.m.sg}}\\
    \glt `It was written in a newspaper that Manu saw Sita, but it was written a magazine that (Manu saw) Rita.'
    
\ex \label{maex12}
    \gll Akhbaar-me likhaa thaa ki Manu-ne Sita-ko dehk-aa, lekin magazin-me likhaa thaa ki Rina-ne \sout{Sita-ko dekh-aa}.\\
    newspaper-in written aux.\textsc{past.m.3sg} \textsc{c} Manu-\textsc{erg} Sita-\textsc{obj} see-\textsc{pfv.m.sg} but magazine-in written aux.\textsc{past.m.3sg} \textsc{c} Rina-\textsc{erg} \sout{Sita-obj see-\textsc{pfv.m.sg}}\\
    \glt `It was written in a newspaper that Manu saw Sita, but it was written in a magazine that Rina (saw Sita).' \hfill \citep[70 \& 71]{kush16}
\z 
This variation also appears to tied to availability of embedded gapping. For all of Kush's consultants that found \REF{maex11} and \REF{maex12} acceptable, they also allowed for embedded gapping, as shown in \REF{maex13} (see also \citealt{farudi2013} for similar observations and for further discussion).

\ea \label{maex13}
    \gll Manu-ne Sita-ko dekh-aa aur [ Rina-ne soch-aa/ Rina-ko lag-aa ] ki Tanu-ne Mira-ko \sout{dekh-aa}.\\
    Manu-\textsc{erg} Sita-\textsc{obj} see-\textsc{pfv.m.sg} and [ Rina-\textsc{erg} think-\textsc{pfv.m.sg}/ Rina-\textsc{dat} strike-\textsc{pfv.m} ] \textsc{c} Tanu-\textsc{erg} Mira-\textsc{obj} \sout{see-\textsc{pfv.m.sg}}\\
    \glt `Manu saw Sita and Rina thought/ it seemed to Rina that Tanu saw Mira.' \hfill \citep[53]{kush16}
\z 
This correlation is suggestive of analyses that treats gapping as a subspecies of stripping, but with multiple remnants (see \citealt{johnson18} for extensive discussion of the relation between the two constructions).\footnote{As Johnson notes, gapping examples, originally from \citet{weir14}, parallel to Wurmbrand's stripping examples are also acceptable, as shown below.

\ea 
    \ea 
        John ate oysters and I suspect Mary swordfish.
    \ex 
        John ate oysters and I imagine Mary swordfish. \hfill \citep[333]{weir14}
    \z 
\z 
Just as in Wurmbrand's examples, the complementizer \emph{that} must be absent in such examples.}

While the interspeaker variation found in Hindi is interesting and deserves further attention, for our purposes, we will focus on the subset of Hindi speakers that do allow for embedded stripping and gapping. For such speakers, both embedded stripping and gapping are allowed in the presence of the complementizer like element \emph{ki}.

Note again that we find the case connectivity effects that we saw in the more classic cases of stripping \REF{maex15:a}, and as shown in \REF{maex15:b} we once again see obligatory postposition pied piping.

\ea 
    \ea \label{maex15:a}
        \gll Sita-ne Ram-ko tohafaa diyaa aur mujhe lagtaa hai Mohan-*(ko) bhii.\\
        Sita-\textsc{erg} Ram-\textsc{dat} gift give.\textsc{prf} and \textsc{1sg.dat} feel be.\textsc{pres} Mohan-*(\textsc{dat}) also\\
        \glt `Sita gave Ram a gift and I think Mohan too.'
    \ex \label{maex15:b}
        \gll Ham-ne Ravi ke liye khaanaa banaayaa aur mujhe lagtaa hai Mohan ke liye bhii.\\
        \textsc{1pl-erg} Ravi \textsc{gen} for food make.\textsc{prf.3} and \textsc{1sg.dat} feel be.\textsc{pres} Mohan \textsc{gen} for also\\
        \glt `We made food for Ravi and I think for Mohan too.'
    \z 
\z 
This once again suggests that clausal ellipsis is also at work in such examples.

We find another type of clausal ellipsis reminiscent of stripping sometimes refered to as  alternate negation clauses in \citet{Sinha05}. As far as we know, this construction has received less attention in the generative literature.  Interestingly for our purposes, the negative element that proceeds the remnant in such constructions is morphologically complex, consisting of a negative morpheme \emph{naa} and \emph{ki}, the complementizer like element.\footnote{\emph{na(a)} is just one of the three negative morphemes found in Hindi (\emph{mat} and \emph{nahii} being the other two). It occurs with most non-indicative verb forms and also in \emph{neither \dots\ nor} constructions. See \citealt{bhatia95} for extensive discussion of negation in Hindi.} Just as in the previous examples, case-matching is enforced, as shown in \REF{maex16}.\footnote{It has also been claimed that \emph{kyuNki} `because' can be decomposed in to \emph{kyuuN} `why' + \emph{ki} and \emph{jabki} `whereas' can be decomposed into \emph{jab} `when' + \emph{ki}.}

\ea \label{maex16}
    \gll Ham-ne aap-ko bulaayaa thaa naaki un-*(ko).\\
    \textsc{1pl-erg} \textsc{2sg-dom} called be.\textsc{pst} \textsc{neg.ki} \textsc{3pl-dom}\\
    \glt `We called you, not them.'
\z 
As with the other examples, postposition omission is not allowed, as shown in \REF{maex17}.

\ea \label{maex17}
    \gll Ham-ne Ravi ke liye khaanaa banaayaa naaki Mohan ke *(liye).\\
    \textsc{1pl-erg} Ravi \textsc{gen} for food make.\textsc{prf.3} \textsc{neg.ki} Mohan \textsc{gen} *(for)\\
    \glt `We made food for Ravi, not for Mohan.'
\z 
The above data also rule out the possibility that the ellipsis site contains a cleft or copula structure. In the examples below we see that continuations with a copula are ungrammatical.

\ea 
    \ea[*]{ 
        \gll Ham-ne Ravi ke liye khanna banaayaa thaa aur mujhe lagtaa hai ki Mohan ke liye bhii thaa.\\
        \textsc{1pl-erg} Ravi \textsc{gen} for food make.\textsc{prf.3} be.\textsc{pst} and \textsc{1sg.dat} feel be.\textsc{pres} \textsc{ki} Mohan \textsc{gen} for also be.\textsc{pst}\\
        \glt Intended: `We made food for Ravi and I think for Mohan too.'
    } 
    \ex[*]{
        \gll Ham-ne aap-ko bulaayaa thaa naaki un-ko thaa.\\
        \textsc{1pl-erg} \textsc{2sg-dom} call be.\textsc{pst} \textsc{neg.ki} \textsc{3pl-dom} be.\textsc{past}\\
        \glt Intended: `We called you, not them.'
    }
    \z 
\z 
So it appears that the complimentizer like element \emph{ki} can occur in stripping like constructions in Hindi. Both in embedded environments (for some speakers) and in the alternate negation clauses.

\subsection{Does Hindi have \emph{wh} focus movement?}
As we have shown Hindi does have a stripping like operation even in the presence of the complementizer like element \emph{ki}. In this section, we consider whether Hindi conforms to the generalization that languages that allow for stripping with complementizers have obligatory focus driven \textit{wh}-movement \citep{craenenbroeck13}. 

Hindi \textit{wh}-questions have been extensively studied (see \citealt{dayal17} for a recent discussion), and it has been suggested that Hindi does have focus driven movement, but to the specifier of \emph{v}P, not a position in the clausal periphery. This explains the fact that \emph{wh}-elements occur immediately before the verb, as shown in \REF{maex19}.

\ea \label{maex19}
    \ea 
        \gll Anu-ne kyaa khariidaa?\\
        Anu-\textsc{erg} what bought\\
        \glt `What did Anu buy?'
    \ex 
        \gll Yeh kavitaa kis-ne likhii?\\
        this poem who-\textsc{erg} wrote\\
        \glt `Who wrote this poem?'
    \ex 
        \gll Tum-ne paisaa kis-ko diyaa?\\
        you-\textsc{erg} money who-\textsc{dat} gave\\
        \glt `Who did you give money to?'
    \z 
\z 
It is unclear whether such movement is obligatory, however. As we see in \REF{maex20}, the \textit{wh}-elements can also remain in-situ without issue, and in some cases, sound more natural than their counterparts in \REF{maex19}.

\ea \label{maex20}
    \ea 
        \gll Kis-ne yeh kavitaa likhii?\\
        who-\textsc{erg} this poem write\\
        \glt `Who wrote this poem?'
    \ex 
        \gll Tum-ne kis-ko paisaa diyaa?\\
        you-\textsc{erg} who-\textsc{dat} money gave\\
        \glt `Who did you give money to?'
    \z 
\z 
So it is quite tenuous to claim that Hindi has obligatory focus movement. Even if we were to accept this claim, Hindi may still pose an issue for \citet{craenenbroeck13} as the claim in that work is that the head that attracts the \emph{wh}-element is the head the hosts the E-feature (i.e., the head whose complement undergoes ellipsis). Under this theory, we are lead to predict that Hindi sluicing/stipping targets VP. \citet{gribanova16} show that this cannot be case, as the auxilary verb \emph{ho}, typically thought to be a realization of a T head, is elided in sluicing.

\ea 
    \gll Ali koi kitaab caah-taa hai. Ham-eN nahiiN pa-taa kaunsii \sout{Ali caah-taa hai}.\\
    Ali some book want-\textsc{hab.m} \textsc{aux}. We-\textsc{dat} \textsc{neg} know-\textsc{hab.m} which.\textsc{f} {Ali want-\textsc{hab.m} \textsc{aux}}\\
    \glt `Ali wants some book, but we don't know which.' \hfill \citep[643]{gribanova16}
\z 
A similar test can be used to show the stripping also targets something larger than VP. Below the auxiliary \emph{hai} is part of the elided material suggesting that ellipsis must be larger than VP. 

\ea 
    \gll Ali kitaab caah-taa hai aur muhje lagtaa hai ki kalam bhii.\\
    Ali book want-\textsc{hab.m} \textsc{aux} and \textsc{1sg.dat} feel \textsc{aux} \textsc{ki} pen also\\
    \glt `Ali wants a book. I think (he wants) a pen too.'
\z 
\subsection{What is \emph{ki}?}
\begin{sloppypar}
The element \emph{ki} is subject of debate in the literature. Some researchers have claimed that it is similar to a coordination marker, others have claimed that it is a complementizer similar to English \emph{that}, we show that neither view fully captures the behavior of \emph{ki}. 
\end{sloppypar}

\citet{dwivedi94} suggests that \emph{ki} is in fact a conjunction marker that has a selection restriction such that it may only conjoin two CPs. Since this proposal, there have been several arguments against it. Take negative sensitive items licensing as an example. As shown in \REF{maex21}, negation in the first conjunct of a true coordination cannot license a negative sensitive item in the second conjunct. Example \REF{maex21:a} involves negation in the first clause and the negative sensitive element in the second clause and the result is ungrammatical. If both the negation and negative sensitive element are within the same clause, then the sentence is grammatical, as seen in \REF{maex21:b}.

\ea \label{maex21}
    \ea[*]{ \label{maex21:a}
        \gll MaiN-ne bahut logoN-ko nahi bulaaya thaa lekin koii bhi  aayaa.\\
        I-\textsc{erg} very people-\textsc{dom} \textsc{neg} invite-\textsc{prf} be-\textsc{pst} but someone even come-\textsc{prf}\\
        \glt Intended: `I did not invite many people, but nobody came.'
    } 
    \ex[]{\label{maex21:b}
        \gll MaiN-ne bahut logoN-ko bulaaya thaa lekin koii bhii    nahi aayaa.\\
        I-\textsc{erg} very people-\textsc{dom} invite-\textsc{prf} be-\textsc{pst} but someone even \textsc{neg} come-\textsc{prf}\\
        \glt `I invited many people, but nobody came.'
    }
    \z 
\z 
If \emph{ki}, conjoined two clauses, we would predict that negation in the first clause could not license a negative sensitive item in the second clause. This prediction is not correct as shown in \REF{maex22}. The negation in the first clause can license the use of the negative sensitive item in the second clause.  

\ea \label{maex22}
    \gll Sarita-ne nahii kahaa ki koii bhii aayaa.\\
    Sarita-\textsc{erg} \textsc{neg} say \textsc{ki} someone even came\\
    \glt `Sarita did not say that anyone came.'
\z 
The fact that negation can license the negative sensitive item in the second clause suggests that the second clause is subordinate to the first clause. This allows for the matrix negation to c-command/scope over the negative sensitive item and properly license it.

This suggests that the second clause introduced by \emph{ki} is in fact embedded within the first clause suggesting it is complementizer like English \emph{that}, but, as shown in \REF{maex23}, \emph{ki} does not have the same selection restrictions as \emph{that}. It can introduce both declarative \REF{maex23:a} and interrogative \REF{maex23:b} complement clauses.

\ea \label{maex23}
    \ea \label{maex23:a}
        \gll Us-ne kahaa ki maiN sach boluNgaa.\\
        \textsc{3sg.erg} said \textsc{ki} \textsc{1sg} truth speak.\textsc{fut}\\
        \glt `He said that I speak the truth.'
    \ex \label{maex23:b}
        \gll Sudha-ne puchaa ki maiN kab jaauNgii.\\
        Sudha-\textsc{erg} asked \textsc{ki} \textsc{1sg} when go-\textsc{go}\\
        \glt `Sudha asked whether I will leave.'
    \z 
\z 
This suggests that Hindi \emph{ki} does not correspond directly to English \emph{that}. Following previous works, we suggest that \emph{ki} is a general subordination marker and does not contribute information about clause type. 

\section{Towards an analysis: Height matters}\label{mamasect3}
In this section, we explore the idea that the variation in the height of heads in the left periphery affects their ability to coincide with ellipsis. We propose that \emph{ki} resides higher in the clausal periphery than English \emph{that} and this height difference explains the difference in behavior in stripping as well.

We have seen that \emph{ki}, unlike English complementizers, appears agnostic to clause type. It shows up in both declarative and interrogative complements. This leads us to postulate that \emph{ki} is in fact just a marker of subordination and does not encode clause type information (see \citealt{bhatt91} for a similar proposal and also \citealt{davison03} who argues that \emph{ki} resides high in a Force projection). This is supported by examples like \REF{maex24}. In \REF{maex24} we see both \emph{ki} and the polar question marker \emph{kyaa} in the embedded clause. Note that the order of the two elements is fixed: \emph{ki} must precede \emph{kyaa}. The other order would result in the utterance becoming ungrammatical.

\ea \label{maex24}
    \ea[]{
        \gll Ram-ne puchhaa ki kyaa Sita aayegii.\\
        Ram-\textsc{erg} asked \textsc{ki} what Sita come.\textsc{fut}\\
        \glt `Ram asked whether Sita will come.'}
    \ex[*]{ 
        \gll Ram-ne puchhaa kyaa ki  Sita aayegii.\\
        Ram-\textsc{erg} asked  what \textsc{ki} Sita come.\textsc{fut}\\
        \glt `Ram asked whether Sita will come.'
    }
    \z 
\z 
This data suggests that \emph{ki} occupies a higher position than the head that contributes clause type information. We will assume a expanded CP in line with \citet{rizzi97}. We suggest that \emph{ki} simply marks subordination between two clauses and resides in a subordination phrase (SubP) and that the height of the complementizer that allows it to survive stripping. We assume the representation in \figref{maex25} for the embedded stripping cases. \emph{Ki} heads the subordination phrase that is the topmost projection in the clause and takes a Focus projection as its complement. The remnant of stripping moves to the specifier of the Focus projection followed by ellipsis of the complement of FinP.\footnote{An anonymous reviewer asks what drives the movement of the remnant to the Focus position. We assume, following \citet{hartman09}, that focused phrases dominated by e-given phrases are given an interpreted focus feature, it is this feature that ensures that the remnant moves to the focus projection and avoids ellipsis.}

\begin{figure}
    \caption{\color{red}Please provide a caption\label{maex25}}
    \Tree [.SubP ki [.FocP Mohan$_i$ [.Foc' Foc \qroof{\dots \emph{t}$_i$ \dots}.\sout{FinP} ] ] ]
\end{figure}

This analysis correctly predicts that other materially such as markers of Force can occur in stripping in Hindi. In \REF{maex26}, \emph{kyaa} marks the clause as interogative and can survive stripping.\footnote{Hindi also has a construction similar to \emph{why}-stripping where a focused constituent and \emph{kyuN} (`why') survive ellipsis as shown below.

\ea \label{maex26a}
    \gll Ram-ne roTii khaaii, lekin mujhe nahii maaluum roTii hii kyuN.\\
    Ram-\textsc{erg} bread eat.\textsc{pst} but \textsc{1sg.dat} not know bread \textsc{emp} why\\
    \glt `Ram ate bread, but I don't know why only bread.'
\z 
We leave further investigation of this construction as a matter of future research.}

\ea \label{maex26}
    \gll Sitaa-ne Ravi ke liye khaanaa banaayaa lekin mai jaanaa chaahataa huN ki kyaa Mohan ke liye bhii.\\
    Sita-\textsc{erg} Ravi \textsc{gen} for food make.\textsc{prf.3} but I to.know want be.\textsc{pres} \textsc{ki} what Mohan \textsc{gen} for also\\
    \glt `Sita made food for Ravi but I want to know whether (she made food) for Mohan also.' 
\z 
So our analysis of Hindi stripping allows for heads higher in the left periphery to survive ellipsis. Interestingly, the idea that height of the complementizer like element plays a role in its ability to survive stripping has recently been proposed by \citet{yoshida18}. They are analyzing stripping like constructions under \emph{if} in English, as shown in \REF{maex27}.

\ea \label{maex27}
    John likes to drink whiskey. If scotch, I will pour him an Islay. \hfill \citep[1]{yoshida18}
\z 
Note that like stipping in coordinations the remnant can occur with negation as shown in \REF{maex28}.

\ea \label{maex28}
    John likes to drink scotch, if not scotch, then bourbon.
\z 
\citet{yoshida18} argue that if \emph{if} is a type of complementizer, then such examples may also constitute a counterexample to the embedded stripping generalization. They argue that \emph{if} is a Force head that sits atop the focus projection that hosts the remnant of stripping in its specifier. Since it resides high in the clause, it is able to appear in stripping parallel to our treatment of \emph{ki} in Hindi. English \emph{that} on the other hand is low in the structure in Fin (e.g., \citealt{baltin10}) and cannot survive ellipsis.

%Note that while ellipsis appears possible in \LLast and \Last. 

%It is much less clear when \emph{if} is introducing an embedded polar question that ellipsis is still possible, as shown in \Next.

%\ex. John is flirting with someone. I wonder if *(it is) Rose.
\subsection{Stripping with negation}
Let us now turn to stripping like constructions that involve negation. These included stipping in a coordination \REF{maex29} but also the alternate negation clause \REF{maex30}.

\ea \label{maex29}
    \gll Sita-ne Ram-ko tohafaa diyaa, lekin Mohan-ko nahi.\\
    Sita-\textsc{erg} Ram-\textsc{dat} gift give.\textsc{prf} but Mohan-\textsc{dat} \textsc{neg}\\
    \glt `Sita gave Ram a gift, but not Mohan.'
\ex \label{maex30}
    \gll Ham-ne aap-ko bulaayaa thaa naaki un-*(ko).\\
    \textsc{1pl-erg} \textsc{2sg-dom} called be.\textsc{pst} \textsc{neg.ki} \textsc{3pl-dom}\\
    \glt `We called you, not them.'
\z 
In the literature on negative stripping, there has been two proposals about the structure of negation \citep{merchant03,wurmbrand17,dendikken18}. Under one view, it is argued that negation in negative stripping is the result of a high sentential negation \figref{maex31}. The other view argues instead that such structures involve constituent negation \figref{maex32}.

\begin{figure}
\begin{floatrow}
\captionsetup{margin=.05\linewidth}
\ffigbox{\Tree [.NegP Not [.Neg' Neg [.FocP Remnant [.Foc' Foc \qroof{\dots}.TP ] ] ] ]}
        {\caption{\color{red}Please provide a caption\label{maex31}}}
\ffigbox{\Tree [.FocP [.NP Not Remnant ] [.Foc' Foc \qroof{\dots}.TP ] ]}
        {\caption{\color{red}Please provide a caption\label{maex32}}}
\end{floatrow}
\end{figure}
    
The Hindi data, especially the alternate-negation, seem to favor the sentential approach, as it appears that negation does not form a constituent with the remnant, but rather forms a morphological word with the subordination marker \emph{ki}. To account for this structure we assume that high sentential negation takes the subordination phrase as it complement, the remnant moves to the focus projection followed by FinP ellipsis. \emph{ki} undergoes head movement to the negation head. At PF, negation in the specifier of NegP and \emph{ki} form a word via m-merger. The syntax we assume is shown in \figref{maex33}.\footnote{It is important to note that headedness is not harmonic in Hindi with some heads following their complements and some heads proceeding them. We present the left periphery as uniformly head initial, but this is an idealization as we can be seen from comparison of \REF{maex29} and \REF{maex30}, what appears to be the negation head can either proceed or follow the remnant. We leave an analysis of the word order variation for future research.}

\begin{figure}
\caption{\color{red}Please provide a caption\label{maex33}}
    \Tree 
    [.NegP 
        Naa 
        [.Neg' 
            { \node{neg}Neg+ki_i } 
            [.SubP 
                {\node{sub}{\emph{t}}_i} 
                [.FocP 
                    Mohan-ko$_j$ 
                    [.Foc' 
                        Foc 
                        \qroof{\ldots{} \emph{t}_j \ldots}.{\sout{FinP}} 
                    ] 
                ] 
            ] 
        ] 
    ]
    \anodecurve[bl]{sub}[b]{neg}{0.3in}
\end{figure}

By treating \emph{ki} as a high subordination marker, we can account for its appearance in stripping like constructions in Hindi. We argued that height of the complementizer mattered for its ability to appear in stripping, both with and without negation. This approach mirrors a similar proposal of \emph{if}-stripping in English made by \citet{yoshida18}. 

\subsection{Extension to alternative questions}
We have argued that \emph{ki} is a subordinator. A potential issue for this analysis is that \emph{ki} can behave as a disjunction marker as shown in \REF{maex34}.

\ea \label{maex34}
    \gll (Kyaa) tum-ne Ravi ke liye khaanaa banaayaa yaa/ki Mohan ke liye?\\
    (what) \textsc{2pl-erg} Ravi \textsc{gen} for food make.\textsc{prf.3} or/\textsc{ki} Mohan \textsc{gen} for\\
    \glt `Did you make food for Ravi or for Mohan?'
\z 
For many Hindi speakers, it is also possible that \emph{yaa} and \emph{ki} co-occur, again making a morphologically complex word \emph{yaaki} in such examples. This may appear on the surface to be an issue for our analysis as it appears that \emph{ki} in \REF{maex34} can take a PP as a complement instead of a clause level projection. There is reason to believe that such examples actually also involve a clausal complement, but with another ellipsis operation. First note that such questions in English are ambiguous between a polar reading which requires a Yes/No answer and alternative reading which is answered with one of the two PPs.

\ea 
    Did you make food for Ravi or for Mohan?
    \ea 
        Yes/No (\emph{Polar})
    \ex 
        For Ravi/For Mohan (\emph{Alternate})
    \z 
\z 
The examples with \emph{ki} in Hindi, however, only allow for the alternative reading. This is important, as it has been argued that the alternative reading involves clausal ellipsis \citep{han04,gracanin16,podobryaev17}. Additional evidence for an ellipsis analysis comes from P-omission. \citet{podobryaev17} shows that in alternative questions in Russian, the second disjunct can only omit a preposition if that preposition can be stranded under movement, i.e., it conforms to the p-stranding generalization \citep{merchant01}. In light of this, compare the examples in \REF{maex35}. In the English example \REF{maex35:a}, it is possible to omit the preposition in the second disjunct, as it is possible to strand prepositions in English. In the Hindi example in \REF{maex35:b} omission of the postpostion in second disjunct leads to ungrammaticality. This follows from the ellipsis analysis as we have already seen that Hindi does not tolerate postposition stranding under movement or P-omission under sluicing.

\ea \label{maex35}
    \ea[]{ \label{maex35:a}
        Did you make food for Ravi or Mohan? 
    }
    \ex[*]{ \label{maex35:b}
        \gll Kyaa tum-ne Ravi ke liye khaanaa banaayaa ki Mohan ke?\\
        what \textsc{2pl-erg} Ravi \textsc{gen} for food make.\textsc{prf.3} \textsc{ki} Mohan \textsc{gen}\\
        \glt `Did you make food for Ravi or for Mohan?'
    }
    \z 
\z 
We assume the structure below in \figref{maex36} for the second disjunct in alternative questions. Once again, \emph{ki} will act as a subordination marker, there is movement of the remnant to a focus projection followed by clausal ellipsis. This analysis hence allows us to keep a uniform syntax for \emph{ki} (i.e., it always takes a clause complement) and also accounts for the lack of P-omission in Hindi.

\begin{figure}
\caption{\color{red}Please provide a caption\label{maex36}}
    \Tree 
    [.Disj' 
        {\node{neg}yaa+ki$_i$} 
        [.SubP 
            {\node{sub}{\emph{t}}$_i$} 
            [.FocP 
                \qroof{Mohan ke liye}.PP$_j$ 
                [.Foc' 
                    Foc 
                    \qroof{\dots \emph{t}$_j$ \dots}.{\sout{FinP}} 
                ] 
            ] 
        ] 
    ] 
    \anodecurve[bl]{sub}[b]{neg}{0.3in}
\end{figure}


\section{Conclusion}\label{mamasect4}
By discovering that stripping can occur in embedded environments in English as long as there was no complementizer, \citet{wurmbrand17} argued that clause size mattered for the availability of stripping. In this paper we attempted to show that height in the clause also mattered for the availability of certain complementizer-like heads to survive ellipsis.


\section*{Acknowledgements}
In addition to the second author, we also consulted Girija Nandan Sharma, Digvijay Narayan and Anand Abhishek for the Hindi data presented here. Thanks to them for sharing their intuitions with us. We also thank two anonymous reviewers for comments that improved the paper. We also thank Susi, whose work inspired the exploration presented here. All errors are ours.

{\sloppy\printbibliography[heading=subbibliography,notkeyword=this]}
\end{document}
