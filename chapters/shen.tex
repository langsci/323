\documentclass[output=paper]{langscibook}
\author{Zheng Shen\affiliation{National University of Singapore}}
\title{Some notes on MaxShare}
\abstract{MaxShare, a constraint on size of multi-dominated elements, was first proposed in \citet{Citko:2006} and later supported with independent evidence in \citet{Shen:2018a}. 
This paper discussses three aspects of MaxShare: 1. the specific formulations of MaxShare; 2. the restrictions on MaxShare; and 3. the alternatives to MaxShare.}

\begin{document}
\SetupAffiliations{mark style=none}
\maketitle


\section{MaxShare: A size constraint on sharing}
\label{shensect:intro}
Multi-dominance, or structural sharing,  has been proposed to account for a variety of constructions including across-the-board movement (ATB), right node raising (RNR), gapping, and  parasitic gaps among others.
On the other hand, how to restrict such an operation is much less discussed in the literature (but see \citealt{Gracanin-Yuksek:2007}). 

This paper discusses a constraint on the \textit{size} of multi-dominated\slash shared elements, MaxShare, which was first proposed in \citet{Citko:2006} and later supported by independent evidence in \citet{Shen:2018a}.  
This section summarizes MaxShare and its motivating evidence.\footnote{It is well beyond the scope of this paper to discuss whether multi-dominance is the right analysis for all the phenomena that it has been claimed to account for. 
I will largely restrict the discussion to NP RNR and left branch extraction\,+\,ATB and leave the potential wider implication of MaxShare aside.}  
Section \ref{shensect:formulations} compares  two different formulations of MaxShare. 
Section \ref{shensect:restrict} discusses how to restrict MaxShare. 
Section \ref{shensect:alternatives} discusses potential alternative analyses to MaxShare. 
Section \ref{shensect:summary} summarizes the discussion and directions for future research.\largerpage[-1]

\subsection{MaxShare in across-the-board left branch extraction}
\label{shensect:atb}

\citet{Citko:2006} observes that while Slavic languages like Polish allow left branch extraction (LBE) of the nominal modifiers in \REF{shenatb1:a} and across-the-board movement (ATB) in \REF{shenatb1:b} independently, the combination of the two movements in \REF{shenbad} is banned. I will label this movement ATB LBE.

\ea Polish
	\label{shenatb1}
	\ea[]{ 
		\label{shenatb1:a}
		\gll Którą$_{\textnormal{i}}$ Jan przeczytał [t$_{\textnormal{i}}$ ksią\.z{}kę]?\\
		which Jan read {~} book\\
		\glt `Which book did Jan read?' (\citealt[ex. 5a]{Citko:2006})
	}
	\ex[]{ 
		\label{shenatb1:b}
		\gll Ktora ksią\.z{}kę$_{\textnormal{i}}$ [Maria poleciła t$_{\textnormal{i}}$] a [Jan przeczytał t$_i$]? \\
		which book Maria recommended {~} and Jan read {~} \\
		\glt `Which book did Maria recommend and Jan read?'  (\citealt[ex. 6a]{Citko:2006})
	}
	\ex[*]{
		\label{shenbad}
		\gll Którą$_{\textnormal{i}}$ Maria poleciła [t$_{\textnormal{i}}$ ksią\.z{}kę] a Jan przeczytał [t$_{\textnormal{i}}$ ksią\.z{}kę]?\\
		which Maria recommended {~} book and Jan read {~} book\\
		\glt `Which book did Mary recommend and John read?' (\citealt[ex. 7a]{Citko:2006})
	}
	\z 
\z 

Note that in \REF{shenbad}, the head nouns in both objects are identical (\textit{book}). Curiously, when the head nouns in the objects are distinct, ATB LBE is allowed, as is shown in \REF{shenatb2}.

\ea
	\label{shenatb2}
	\gll Ile$_{\textnormal{i}}$ Maria napisała t$_{\textnormal{i}}$ książek a Jan przeczytał t$_{\textnormal{i}}$ artykułów? \\
	how-many Maria wrote {~} books and Jan read {~} articles\\
	\glt `How many books did Maria write and how many articles did Jan read?'  (Polish, \citealt[ex. 10a]{Citko:2006})
\z 
\citet{Citko:2006} assumes that ATB moved elements are necessarily base-generated using multi-dominance. As is shown in \figref{shenhowmany1} for \REF{shenbad} and \figref{shenex4} for \REF{shenatb2}, the DP modifiers, \textit{how many} and \textit{which}, are simultaneously merged with both object nouns, and then moved to the Spec,CP position.  

\begin{figure}\small
\captionsetup{margin=.05\linewidth}
\begin{floatrow}
\ffigbox{\begin{forest}
qtree edges
	[CP
		[Which$_1$, name=land]
		[C'
			[C]
			[\&P
				[TP
					[Maria]
					[T'
						[T]
						[vP
							[v]
							[VP
								[recommended]
								[DP,   calign=last
									[t$_1$, name=t, l*=2.2, s=-30]	
									[\textbf{book}]
								]
							]
						]
					]
				]
				[\&'
					[\&]
					[TP
						[Jan]
						[T'
							[T]
							[vP
								[v]
								[VP
									[read]
									[DP, name=dp
										[\textbf{book}]
									]
								]
							]
						]
					]
				]
			]
		]
	]
	\draw (dp.south) -- (t.north);
%	\draw[->] (t.west) to [bend left=45] (land.south);
\end{forest}}
{\caption{\label{shenhowmany1}Structure of \REF{shenbad}: *}}
\ffigbox{\hspace*{-5.98677pt}\begin{forest}
qtree edges
	[CP
		[How many$_1$, name=land]
		[C'
			[C]
			[\&P
				[TP
					[Maria]
					[T'
						[T]
							[VP
								[wrote]
								[DP,   calign=last
									[t$_1$, name=t, l*=2.2, s=-30]	
									[\textbf{books}]
								]
							]
					]
				]
				[\&'
					[\&]
					[TP
						[Jan]
						[T'
							[T]
								[VP
									[read]
									[DP, name=dp,  calign=last
										[\textbf{articles}]
									]
							]
						]
					]
				]
			]
		]
	]
	\draw (dp.south) -- (t.north);
%		\draw[->] (t.south) to [bend left=80] (land.south);
\end{forest}}
        {\caption{Structure of \REF{shenatb2}: OK\label{shenex4}}}
\end{floatrow}
\end{figure}

The contrast between the two derivations above needs to be accounted for: \REF{shenbad} is not accepted while \REF{shenatb2} is OK. The relevant difference is whether the nouns in the object DPs are identical or distinct. When they are distinct in \REF{shenatb2}, the sentence is OK; when they are identical in \REF{shenbad}, the sentence is out. To rule out the structure in~\figref{shenhowmany1} and retain the structure in~\figref{shenex4}, \citet{Citko:2006} proposes a constraint on multi-dominance structures where the shared material must be maximized.{\largerpage} The structure with less shared material is ruled out if an alternative structure with more shared material is available. Regarding the pattern at hand, the structure in \figref{shenhowmany1} where only \textit{which} is shared is compared with the alternative in \REF{shenhowmany2} where the entire object \textit{how many books} is shared with \textit{how many} moving away. Given \REF{shenhowmany2}, the structure with less shared material in \figref{shenhowmany1} is ruled out. As predicted by this constraint, the sentence in \REF{shenhowmany2} is indeed better than \REF{shenbad}.\footnote{\citet{Citko:2006} assumes that the sentence in \REF{shenhowmany2} involves the movement of \textit{books} to a higher position. This is to keep in line with the linearization constraint of sharing which states that all shared elements must be moved to a non-shared position to be linearized (see also \citealt{Gracanin-Yuksek:2007}). I do not follow this assumption that the noun \textit{books} moves in \REF{shenhowmany2} as many other linearization algorithms (e.g. \citealt{Wilder:2008, deVries:2009, Gracanin-Yuksek:2013}) can linearize shared materials in situ.} Note that the structure in \figref{shenex4} is not ruled out since the head nouns are distinct thus the whole DP cannot be shared. 

\ea[?]{
	\label{shenhowmany2}
	\gll Ile$_{\textnormal{i}}$ Maria  poleciła a Jan przeczytał t$_{\textnormal{i}}$ ksią\.z{}ek?\\
	How-many Maria recommended and Jan read ~ books?\\
	\glt `How many books did Maria recommended and how many books did Jan read?' (modified from \citealt[p. 238, 26b]{Citko:2006})}
\z

\begin{figure}\small
\caption{Share element maximized: How many books}
\begin{forest}
qtree edges
	[CP
		[How many$_1$]
		[C'
			[C]
			[\&P
				[TP
					[Maria]
					[T'
						[T]
						[vP
							[v]
							[VP, name=vp
								[recommended]
							]
						]
					]
				]
				[\&'
					[and]
					[TP
						[Jan]
						[T'
							[T]
							[vP
								[v]
								[VP
									[read]
									[DP, name=dp
										[t$_1$]
										[books]
									]
								]
							]
						]
					]
				]
			]
		]
	]
	\draw (dp.north) -- (vp.south);
\end{forest}
\end{figure}

I will label this constraint of maximizing shared materials MaxShare. Before moving on to the formulation of MaxShare, the next section discusses an independent piece of evidence for this constraint.

\subsection{MaxShare in NP right node raising}
\label{shensect:rnr}

In addition to ATB LBE, another case of MaxShare is independently observed in NP RNR.
\citet{Shen:2018a} discusses number marking on the head noun which is shared by two singular DPs as is shown in \REF{shennrnr}. 
For (\ref{shennrnr:a}--\ref{shennrnr:c}), the head noun must be singular despite the subject refers to two individuals. 
For \REF{shennrnr:d}, on the other hand, the singular head noun is not available. 
This contrast is the focus of Chapter 2 of \citet{Shen:2018a}.

\ea 
	\label{shennrnr}
	\ea[]{ 
		\label{shennrnr:a}	
		This and that student are a couple.
	}
	\ex[]{ 
		\label{shennrnr:b}
		This tall and that short student are a couple.
	}
	\ex[]{ 
		\label{shennrnr:c}
		John's tall and Mary's short student are a couple.
	}
	\ex[*]{
		\label{shennrnr:d}
		John's and Mary's student are a couple.\footnotemark{}
	}
	\z 
\z 
\footnotetext{Note that the only relevant reading here is the one with two students. This is accomplished by the use of the predicate \textit{are a couple}. As a reviewer correctly noted, the singular head noun under possessive DPs is OK when referring to one single student: \textit{John's and Mary's student is tall}.}
Shen proposes that the singular noun in (\ref{shennrnr:a}--\ref{shennrnr:c}) results from a multi-dominance structure in \figref{shenex7}. The number feature within a DP is assumed to originate on the \textsc{num} head and get copied onto other elements including nouns and determiners. When the noun is shared by two singular DPs, it gets two [\textsc{sg}] values, which, in languages like English, is spelled out as singular.\footnote{Other languages of this type include German, Dutch, Icelandic, Slovenian, Polish, Bosnia-Serbia-Croatian and so on. Bulgarian and Russian are different in this aspect. See \citet{Shen:2019} for discussion regarding this variation. I will focus on the English type of languages here.} 


%The plural shared noun in \Last, on the other hand, results from a coordinated specifier structure in \NNext where only one plural Num head is involved, from which the noun gets that [\textsc{pl]}. Complex remnants like \textit{John's tall and Mary's short} and \textit{this tall and that short} are incompatible with the structure in \NNext since they would involve non-constituent coordination. The agreeing remnants like \textit{this and that} are also incompatible because the [\textsc{sg}] features on \textit{this} and \textit{that} cannot be valued by the [\textsc{pl}] Num head.
%
%\begin{multicols}{2}

%
%\ex. John's and Mary's students\\
%\small
%\begin{forest}
%qtree edges
%	[PossP
%		[ConjP
%			[John's]
%			[Conj'
%				[and]
%				[Mary's]
%			]
%		]
%		[Poss'
%			[Poss]
%			[NumP
%				[Num]
%				[NP
%					[students]
%				]
%			]
%		]
%	]
%\end{forest}
%
%\end{multicols}

On the other hand, the fact that \textit{John's and Mary's} does not allow the shared noun to be singular in \REF{shennrnr:d} indicates that the multi-dominance structure is not available under possessive DPs. 
Figures~\ref{shennp1} and~\ref{shennp2} illustrate what the structure would look like under the intended dual student reading. 
In \figref{shennp1}, the noun is shared and the \textsc{num}P is shared in \figref{shennp2}. 

\begin{figure}\small
\caption{Multi-dominance structure for NP RNR\label{shenex7}} 
\begin{forest}
for tree={ fit=band, }
	[ConjP
		[DP1
			[this]
			[\textsc{num}P
				[\textsc{num}\\{[\textsc{sg}]}, base=bottom, align=center]
				[NP, name=np
					[tall]
				]
			]
		]
		[Conj'
			[and]
			[DP2
				[that]
				[\textsc{num}P
					[\textsc{num}\\{[\textsc{sg}]}, base=bottom, align=center]
					[NP
						[short]
						[student, name=n]
					]
				]
			]
		]
	]
	\draw (np.south) -- (n.north);
\end{forest}\\
This tall and that short student
\end{figure}

\begin{figure}\small
\caption{Candidate structure 1: MaxShare not satisfied\label{shennp1}}
\begin{forest}
qtree edges
	[ConjP
		[PossP1
			[John's]
			[Poss'
				[Poss]
				[NumP, name=nump
					[\textsc{Num}\\{[\textsc{sg}]}, base=bottom, align=center]
				]
			]
		]
		[Conj'
			[and]
			[PossP2
				[Mary's]
				[Poss'
					[Poss]
					[NumP
						[\textsc{Num}\\{[\textsc{sg}]}, base=bottom, align=center]
						[NP, name=n, l*=0.8, s=2
							[student]
						]
					]
				]
			]
		]
	]
	\draw (nump.south) -- (n.north);
\end{forest}\\
*John's and Mary's student
\end{figure}

\begin{figure}\small
\caption{Candidate structure 2: MaxShare not satisfied\label{shennp2}}
\begin{forest}
qtree edges
	[ConjP
		[PossP1
			[John's]
			[Poss', name=possp
				[Poss]
			]
		]
		[Conj'
			[and]
			[PossP2
				[Mary's]
				[Poss'
					[Poss]
					[NumP, name=nump
						[\textsc{Num}\\{[\textsc{sg}]}, base=bottom, align=center]
						[NP
							[student]
						]
					]
				]
			]
		]
	]
	\draw (possp.south) -- (nump.north);
\end{forest}\\
*John's and Mary's student
\end{figure}

\begin{figure}\small
\caption{\label{shenmax}Candidate structure 3: Agree constraint violated}
\begin{forest}
qtree edges
	[ConjP
		[PossP1, name=possp
			[\textit{John's}]
		]
		[Conj'
			[\textit{and}]
			[PossP2
				[\textit{Mary's}]
				[Poss', name=shared
					[Poss]
					[NumP
						[Num\\{[\textsc{sg}]}, base=bottom, align=center]
						[NP, name=n
							[\textit{student}]
						]
					]
				]
			]
		]
	]
	\draw (possp.south) -- (shared.north);
\end{forest}\\
*John's and Mary's student
\end{figure}

Similar to Citko, Shen proposes a MaxShare constraint on the size of the shared element to rule out the structures in Figures~\ref{shennp1} and~\ref{shennp2}. The constraint is as seen in \REF{shenform1} where \textit{sharable} is defined as non-distinct.\largerpage[2]
 
\eanoraggedright
	\label{shenform1}
	MaxShare: XP can be shared only if there is no YP such that YP dominates XP and YP is shareable, if the XP sharing structure and the YP sharing structure have identical interpretations. 
\z
Shen claims that according to  \REF{shenform1}, the potential alternative structure in \figref{shenmax} where the Poss' is shared rules out the structures in Figures~\ref{shennp1} and~\ref{shennp2}. 
As one can see, the shared constituent in \figref{shenmax}, Poss', properly contains the ones shared in Figures~\ref{shennp1} and~\ref{shennp2}.{\interfootnotelinepenalty=10000\footnote{The readers will notice that \REF{shenftn5exi} is OK where the head noun is plural under possessive DPs. 
\citet{Shen:2018a} argues that \REF{shenftn5exi} involves a structure with a single DP with the conjoined possessors in its specifier position with a plural \textsc{num} head. 
See Section \ref{shensect:ban} for discussion and \citealt{Shen:2018a} for more details.

\ea 
	\label{shenftn5exi}
	John's and Mary's students are a couple.
\z
}}

Note that although the structure in \figref{shenmax} does not violate MaxShare, it must be ruled out as well since the string \emph{John's and Mary's student} is not acceptable under the dual student reading. \citet{Shen:2018a} proposes that the structure in \figref{shenmax} is ruled out by an independent requirement on sharing: the Agree constraint, which requires the head of the shared element (Poss in \figref{shenmax}) agrees with the remnants (\textit{John's} and \textit{Mary's}). Since there is no agreement between the possessors and the Poss' or the Poss head, the structure in \figref{shenmax} is ruled out. I will follow this analysis here (see brief discussion in Section \ref{shensect:ban}).
%
%The full structure of two coordinated DPs is shown in \Next. To derive the string `John's and Mary's student', three constituents can be potentially shared: Poss', NumP, and NP (in bold below). Structures that only share NumP and NP are blocked in violation of \Last, since a larger constituent, Poss', can be shared. 
%
%\ex. John's and Mary's student\\
%\small
%\begin{forest}
%qtree edges
%	[ConjP
%		[PossP1
%			[John's]
%			[\textbf{Poss'}
%				[Poss]
%				[\textbf{NumP}, name=nump
%					[Num]
%					[\textbf{NP}
%						[student]
%					]
%				]
%			]
%		]
%		[Conj'
%			[and]
%			[PossP2
%				[Mary's]
%				[\textbf{Poss'}
%					[Poss]
%					[\textbf{NumP}
%						[Num]
%						[\textbf{NP}, name=n
%							[student]
%						]
%					]
%				]
%			]
%		]
%	]
%\end{forest}
In sum, NP RNR in English and ATB LBE in Polish among other languages show supporting evidence of MaxShare, a constraint limiting sharing based on the size of the shared elements. 

\section{A note on the formulations}
\label{shensect:formulations}

Having established the empirical motivations for a MaxShare constraint, this section discusses its different formulations. 

The notion of \textit{size} in the formulation proposed by \citet{Shen:2018a} in \REF{shenform1} is defined in terms of domination. A derivation with a shared XP is compared with derivations where XP's mother or daughter nodes are shared. This formulation can account for both patterns discussed above: for ATB\,+\,LBE movement in Polish, derivations with a shared \textit{how many} and its mother node \textit{how many books} are compared; for NP RNR, derivations that share a NP, its mother node \textsc{num}P, and the mother node of \textsc{num}P, Poss', are compared. I will label this formulation as the \textit{dominance} MaxShare.

\citet{Citko:2006}, on the other hand, offers a more derivational conception of MaxShare. According to her, MaxShare follows from a general economy principle. The derivations being compared are restricted by their numerations: given two numerations with the same set of lexical items, the numeration where a given lexical item is selected fewer times is more economical. For example, \REF{shenshenex12} illustrates the numerations involved in \figref{shenhowmany1} and \REF{shenhowmany2} with English translation. Each numeration include the set of items that are used in the derivation and the indexes indicate the number of times that each item is selected. The only difference between them is that \textit{books} is selected twice in \REF{shenshenex12:a} and only once in \REF{shenshenex12:b}. \REF{shenshenex12:b} is more economical and \REF{shenshenex12:a} is blocked as a result. 
The pattern in NP RNR can also be accounted for in this manner, see \citet[104]{Shen:2018a}. I will refer to this formulation as the \textit{numeration} MaxShare.\largerpage

\ea 
	\label{shenshenex12}
	Competing numerations
	\ea 
		\label{shenshenex12:a}
		Numeration for \figref{shenhowmany1} = \{how-many$_1$, Maria$_1$, Jan$_1$, recommended$_1$, read$_1$, \textbf{books$_2$}, and, T$_2$, v$_2$, C\}
	\ex 
		\label{shenshenex12:b}
		Numeration for \REF{shenhowmany2} = \{how-many$_1$, Maria$_1$, Jan$_1$, recommended$_1$, read$_1$, \textbf{books$_1$}, and, T$_2$, v$_2$, C\} 
	\z 
\z 
Both the dominance and the numeration formulation can account for the data presented so far. 
But the two formulations make different predictions regarding \textit{bulk} and \textit{non-bulk} sharing. 
Specifically, the dominance MaxShare is only applicable to bulk sharing while the numeration MaxShare is compatible with both bulk and non-bulk sharing. 

\citet{Gracanin-Yuksek:2007} introduces the distinction between bulk and non-bulk sharing.\footnote{\citet{Gracanin-Yuksek:2007} proposes constraints on non-bulk sharing as well as linearization of sharing structure which I will leave aside here.} 
Bulk sharing refers to structures where one constituent (including its daughter nodes and so on) is shared. 
All examples of ATB LBE and NP RNR we have seen so far involve bulk sharing. 
In ATB LBE, it is the object DP or the modifier of the DP that is shared whereas in NP RNR, it is the NP, \textsc{num}P or the Poss' node that is shared. 
On the other hand, non-bulk sharing refers to structures where multiple constituents that are not in dominance relation are shared. 
Take \REF{shenex13} for example (modified from \citealt[(14)]{Gracanin-Yuksek:2007}). 
In this structure, two constituents (W and M) are independently shared. 
Neither dominates the other. 

The case relevant to MaxShare is the comparison between \REF{shenex13}, and \REF{shenex14} which involves non-bulk sharing of Y, W, and M. 
As one can see, \REF{shenex14} shares one more node (namely, Y) than \REF{shenex13} does. 
According to the numeration MaxShare, \REF{shenex13} should be ruled out given \REF{shenex14}. 
However, since the share nodes do not dominate each other, the dominance MaxShare does not make predictions regarding \REF{shenex14}. 


\begin{multicols}{2}

\ea 
\label{shenex13}
Y$_1$ W M Q Y$_2$ H\\
\footnotesize
\begin{forest}
qtree edges
%for tree={
%	fit=band,
%}
	[XP
		[YP$_1$
			[Y$_1$]
			[WP$_1$
				[W, name=w]
				[MP$_1$
					[M, name=m]
					[Q]
				]
			]
		]
		[YP$_2$
			[Y$_2$]
			[WP$_2$, name=wp2
				[MP$_2$, name=mp2
					[H]
				]
			]		
		]
	]
		\draw (w.north) -- (wp2.south);
			\draw (m.north) -- (mp2.south);
\end{forest}
\z


\begin{samepage}
\ea 
\label{shenex14}
Y W M Q H\\
\footnotesize
\begin{forest}
qtree edges
%for tree={
%	fit=band,
%}
	[XP
		[YP
			[Y, name=y]
			[WP$_1$
				[W, name=w]
				[MP$_1$
					[M, name=m]
					[Q]
				]
			]
		]
		[ZP, name=zp
			[WP$_2$, name=wp2
				[MP$_2$, name=mp2
					[H]
				]
			]		
		]
	]
				\draw (y.north) -- (zp.south);
		\draw (w.north) -- (wp2.south);
			\draw (m.north) -- (mp2.south);
\end{forest}
\z
\end{samepage}

\end{multicols}


%
%\citet{Gracanin-Yuksek:2007} makes the distinction between bulk and non-bulk sharing. See a minimal pair in \Next. Both derivations involve the string \textit{X$_1$ and X$_2$ Y ZP}, however, in \Next[a], the YP is shared as a whole whereas in \Next[b] it is the daughter nodes of YP, namely Y and ZP, that are shared respectively. 
%
%\begin{multicols}{2}
%
%\ex.\label{shenbnnb1}
%\a. X$_1$ and X$_2$ Y ZP ~\\
%\begin{forest}
%qtree edges
%[\&P
%		[XP$_1$, name=xp1
%			[X$_1$]		
%		]
%		[\&'
%			[and]
%				[XP$_2$
%					[X$_2$]
%					[YP, name=yp2
%						[Y]
%						[ZP]
%					]
%				]
%		]
%]
%\draw (xp1.south) -- (yp2.north);
%\end{forest}
%\b. X$_1$ and X$_2$ Y ZP\\
%\begin{forest}
%qtree edges
%[\&P
%			[XP$_1$
%				[X$_1$]
%				[YP$_1$,name=yp1
%				]
%			]
%			[\&'
%				[and]
%					[XP$_2$
%						[X$_2$]
%						[YP$_2$, name=yp2
%											[Y, name=y1]
%							[ZP, name=zp2]
%						]
%					]
%			]
%	]
%	\draw (yp1.south) -- (y1.north);
%	\draw (zp2.north) -- (yp1.south);
%\end{forest}
%
%\end{multicols}
%
%Note that although the string in \Last can be generated through either bulk or non-bulk sharing, the set of strings that non-bulk sharing can generate properly contains the set of strings generated by bulk sharing. The derivation in \Next where the left most head X and the right most phrase ZP are both shared requires non-bulk sharing. Another example is illustrated in \Next[b] where both X and Y are shared. Since X and Y do not form a constituent on their own, the structure in \Next[b] can only be generated via non-bulk sharing. 
%
%\begin{multicols}{2}
%\ex.\label{shenbnnb2}
%\a.  X Y$_1$ and Y$_2$ ZP\\
%\begin{forest}
%qtree edges
%[\&P
%			[XP$_1$, name=xp1
%				[X, name=x1]
%				[YP$_1$,name=yp1
%					[Y$_1$, name=y1]
%				]
%			]
%			[\&'
%				[and]
%					[XP$_2$, name=xp2
%						[YP$_2$, name=yp2
%							[Y$_2$, name=y2]
%							[ZP, name=zp2]
%						]
%					]
%			]
%	]
%	\draw (zp2.north) -- (yp1.south);
%	\draw (x1.north) -- (xp2.south);
%\end{forest}
%\b. X Y ZP$_1$ and ZP$_2$\\
%\begin{forest}
%qtree edges
%	[\&P
%		[XP$_1$
%			[X, name=x1]
%			[YP$_1$, name=yp1
%				[Y, name=y1]
%				[ZP$_1$, name=zp1]
%			]
%		]
%		[\&'
%			[and]
%			[XP$_2$, name=xp2
%				[YP$_2$, name=yp2
%					[ZP$_2$, name=zp2]
%				]
%			]
%		]
%	]
%	\draw (x1.north) -- (xp2.south);
%	\draw (y1.north) -- (yp2.south);
%\end{forest}
%
%
%\end{multicols}

%Going back to MaxShare, since the numeration MaxShare does not rely on dominance, it can apply to non-bulk sharing. 
%On the other hand, the dominance MaxShare cannot compare two derivations where one shares more nodes than the other. 

In other words, the numeration MaxShare predicts that once one constituent is shared, all other shareable constituents must be shared as well, even these shareable constituents do not dominate each other. Here I discuss examples with ATB LBE and gapping.
As is discussed above, ATB LBE has been argued to involve sharing. 
Similarly, the sole verb in gapping has been argued to be structurally shared by \citet{Citko:2011}, 
%Citko2006a,
but see \citet{Citko:2018}.

 
The crucial contrast is shown in \REF{shengapAtb}.  
The sentence in \REF{shengapAtb:a} involves ATB LBE of \textit{which} as well as gapping: the two conjuncts share the single verb \textit{ordered}. 
The sentence in \REF{shengapAtb2}, on the other hand, only involves ATB LBE of \textit{which}. The verb is seen in both conjuncts.
As we can see, \REF{shengapAtb:a} is acceptable and \REF{shengapAtb2} is not. 

\ea 
	\label{shengapAtb}
	\ea[]{ 
		\label{shengapAtb:a}
		\gll Jaką$_{\textnormal{i}}$ Maria zamówiła t$_{\textnormal{i}}$ kawę a Jan t$_{\textnormal{i}}$ herbatę?\\
		which Maria ordered {} coffee and Jan {} tea\\
		\glt `What kind of coffee did Maria order and what kind of tea did Jan order?'
	}
	\ex[*]{
		\label{shengapAtb2}
		\gll Jaką$_{\textnormal{i}}$ Maria zamówiła t$_{\textnormal{i}}$  kawę, a Jan zamówił t$_{\textnormal{i}}$ herbatę?\\
		which Maria ordered {} coffee, and Jan ordered {} tea\\
		\glt `What kind of coffee did Maria order and what kind of tea did Jan order?' (see \citealt[(28)]{Citko:2006} for another example)
	}
	\z 
\z 
%\bg.Ile$_1$ Maria przeczytala \textit{t}$_1$ ksiazek a Jan przeczyal t$_1$ recenzji?\\
%how-many Maria read.\textsc{fem} {} books and Jan read.\textsc{masc} {} reviews?\\
%`How many books did Maria read and how many books did Jan read?'

The structures are illustrated in \figref{shenex16} with English translation.\footnote{The shared verb is assumed to move to a higher node in \citet{Citko:2011}'s proposal. 
%Citko2006a,
Here I kept the verb at the shared position to better illustrate the fact that the verb is shared.}
In the structure for \REF{shengapAtb:a} in \figref{shenex16:a}, both the pre-nominal modifier and the verb are shared, the sentence is accepted. 
In \figref{shenex16:b} for \REF{shengapAtb2}, only one of the two shareable elements is shared, i.e.  the pre-nominal modifier \textit{which}, whereas the verb \textit{ordered} which could be shared, is not.

\begin{figure}
\begin{subfigure}[b]{.5\linewidth}\centering\footnotesize
\begin{forest}
qtree edges
[CP
	[which$_j$]
	[C'
		[C]
	[TP
		[Maria$_i$]
		[T'
			[T]
			[\&P
				[vP1
					[t$_i$]
					[v'
						[\textbf{ordered}, name=v1, l*=1.2, s=-30]
						[DP
							[t$_j$, name=tj, l*=2, s=-30]
							[coffee]
						]
					]
				]
				[\&'
					[and]
					[vP2
						[Jan]
						[v', name=vp2
							[DP, name=dp2
								[tea]
							]
						]
					]
				]
			]
		]
	]
]
]	
	\draw (vp2.south) -- (v1.north);
	\draw (dp2.south) -- (tj.north);
\end{forest}
\caption{\label{shenex16:a}Structure for \REF{shengapAtb:a}}
\end{subfigure}\begin{subfigure}[b]{.5\linewidth}\centering\footnotesize
\begin{forest}
qtree edges
[CP
	[which$_j$]
	[C'
		[C]
	[TP
		[Maria$_i$]
		[T'
			[T]
			[\&P
				[vP1
					[t$_i$]
					[v'
						[\textbf{ordered}, name=v1]
						[DP
							[t$_j$, name=tj, l*=2, s=-30]
							[coffee]
						]
					]
				]
				[\&'
					[and]
					[vP2
						[Jan]
						[v', name=vp2, ignore edge
							[\textbf{ordered}, name=v2]
							[DP, name=dp2
								[tea]
							]
						]
					]
				]
			]
		]
	]
]
]	
	\draw (dp2.south) -- (tj.north);
\end{forest}
\caption{\label{shenex16:b}Structure for \REF{shengapAtb2}}
\end{subfigure} 
\caption{Structures for \REF{shengapAtb:a} and \REF{shengapAtb2}\label{shenex16}}
\end{figure}

The numeration MaxShare correctly rules out the derivation in \figref{shenex16:b}. The numerations of Figures~\ref{shenex16:a} and \ref{shenex16:b} are shown in \REF{shennum}. They contain the same items but the verb \textit{ordered} is selected once in \REF{shennum:a} but twice in \REF{shennum:b}. Thus the numeration in \REF{shennum:b} is ruled out given the more economical numeration in \REF{shennum:a}.
On the other hand, the dominance MaxShare does not predict the contrast in \REF{shengapAtb}, since \figref{shenex16:a} does not involves sharing of a constituent that dominates the shared constituent in \figref{shenex16:b}.

\ea 
	\label{shennum}
	\ea 
		\label{shennum:a}
		Numeration for \figref{shenex16:a} = \{which$_1$, Jan$_1$, Maria$_1$, \textbf{ordered$_1$}, coffee$_1$, tea$_1$, and$_1$, T$_2$, v$_2$, C$_1$\} 
	\ex 
		\label{shennum:b}
		Numeration for \figref{shenex16:b} = \{which$_1$, Jan$_1$, Maria$_1$, \textbf{ordered$_2$}, coffee$_1$, tea$_1$, and$_1$, T$_2$, v$_2$, C$_1$\} 
	\z 
\z 
Since sharing a phrase as a whole can be derived from sharing the terminal nodes within the phrase but not vice versa, bulk sharing can only generate a subset of the structure generated by non-bulk sharing. In turn, the derivations that can be ruled out by the dominance MaxShare are a proper subset of those ruled out by the numeration MaxShare. The contrast in \REF{shengapAtb} indicates that the numeration MaxShare is more descriptively adequate than the dominance MaxShare. 

%(\textcolor{red}{ZS: we want other cases of N Max - D Max})

So far I have been implicitly assuming that only one of the two formulations exists. There is preliminary evidence pointing to the possibility that both the numeration and the dominance MaxShare exist. This evidence comes from the distinct effects of violating the two types of MaxShare. Judgments seem to vary regarding the acceptability of the sentences in \REF{shengapAtb}. One of Polish speaking informants commented that the sentence without gapping is not outright bad but ``somewhat awkward because of the unnecessary repetition of the verb''. Similarly, both of the English sentences in \REF{shenex18} are accepted by my English speaking informants. \REF{shenex18:a} only involves ATB (sharing of \textit{to whom}) while both ATB and gapping are present \REF{shenex18:b} (sharing of \textit{to whom} and \textit{serve}). According to the numeration MaxShare, \REF{shenex18:a} should be ruled out by \REF{shenex18:b}. 

\ea \judgewidth{?}
	\label{shenex18}
	\ea[?]{\label{shenex18:a}To whom did some serve mussels and others serve swordfish? (ATB only)
	}
	\ex[]{
		\label{shenex18:b}
		To whom did some serve mussels and others swordfish? (ATB\,+\,gapping)
	}
	\z 
\z 
One possible explanation for the degraded but accepted status of \REF{shenex18:a} and \REF{shengapAtb2}, both of which violate the numeration MaxShare, is that the numeration MaxShare is a violable constraint and does not immediately cause a derivation to crash. On the other hand, the unacceptability in NP RNR in \REF{shennrnr} (repeated here as \REF{shenex19}) is quite strong. \citet{Shen:2018a} reports experimental results using judgments on a 7 point Likert scale from 45 participants and show that \REF{shenex19} which involves a MaxShare violation has a mean rating of 2.33 out of 7. Note that \REF{shenex19} violates the dominance MaxShare (and by entailment, also the numeration MaxShare). 

\ea[*]{
	\label{shenex19}
	John's and Mary's student came from the U.S. (2.33/7)
}
\z
%\a. *John’s tall and Mary’s tall student

The difference between \REF{shenex18} and \REF{shenex19} points to an option where both the dominance MaxShare and the numeration MaxShare exist as independent constraints. Since both constraints are violated in \REF{shenex19} while only the numeration MaxShare is violated in \REF{shenex18:a}, the stronger penalty observed in \REF{shenex19} is expected. A another possibility would be that violating the dominance MaxShare invokes a stronger penalty than violating the numeration MaxShare. However distinguishing their effects is tricky since the former entails the latter.

So far we have seen that the numeration MaxShare is more powerful in terms of coverage than the dominance MaxShare, however, the former might be a weaker constraint in terms of its effects on acceptability. A further step along this line is to look at more cases which can be ruled out by the numeration MaxShare but not by the dominance MaxShare, in addition to the ATB\,+\,gapping case, and check whether the penalty on acceptability is weaker than the violations of the dominance MaxShare. I will leave this for future research.

\section{A note on restricting MaxShare}
\label{shensect:restrict}

Regardless of the two formulations I have been discussing, one issue that needs to be addressed is how to not block sentences with no sharing at all. We have seen that MaxShare allows sentences in \REF{shenex20} where the shared elements are maximized, and we have seen that MaxShare blocks sentences where some shareable elements are not shared. 

\ea
	\label{shenex20}
	\ea 
		John's tall and Mary's short student are a couple.
	\ex 
		\gll Ile$_{\textnormal{i}}$ Maria napisała t$_{\textnormal{i}}$ ksią\.z{}ek a Jan przeczytał t$_{\textnormal{i}}$ artykułow?\\
		how-many Maria wrote {~} books and Jan read {} articles\\
		\glt `How many books did Maria write and how many articles did Jan read?'
	\z 
\z 
Following this pattern, one might expect sentences that share \textit{no} potentially shareable element to be ruled out. For example, in \REF{shenex21:a}, nouns are present inside both conjoined DPs and in \REF{shenex21:b}, both wh-elements are present in the conjoined questions. This prediction is not borne out. Both of these sentences are perfectly acceptable, thus MaxShare must be restricted so that it does not block sentences of the form in \REF{shenex21}. 
 
\ea 
	\label{shenex21}
	\ea 
		\label{shenex21:a}
		John's tall student and Mary's short student are a couple.
	\ex 
		\label{shenex21:b}
		\gll Ile$_{\textnormal{i}}$ Maria napisała t$_{\textnormal{i}}$ ksią\.z{}ek a ile$_{\textnormal{i}}$  Jan przeczytał t$_i$ artykułow? \\
		how-many Maria wrote {~} books and how-many Jan read {~} articles\\
		\glt `How many books did Maria write and how many articles did Jan read?'
	\z 
\z 
This restriction is difficult to derive from the dominance MaxShare. One would have to stipulate that the structures being compared are restricted to ones that share at least one element. However, there is a way for the numeration MaxShare to account for this restriction. 

In the implementation of the numeration MaxShare presented so far, the entire utterance including the conjunction phrase is assumed to share one numeration, as is illustrated above in \REF{shennum}. In order to account for \REF{shenex21}, we need to further break down the derivation. In the multiple spell-out model proposed by \citet{Uriagereka:1999a, Chomsky:2000}, numeration, derivation, and spell-out occur in phases. \citet{Oda:2017} proposes that the \&P and its conjuncts are phases to account for the cross-linguistic patterns of the coordinate structure constraint. 
As a result, each conjunct corresponds to a numeration (or a sub-array) and the comparison of numerations is restricted within phases. The combination of these assumptions correctly rules in sentences in \REF{shenex21} while maintaining the effect of MaxShare. 

Take NP RNR for an example. In \REF{shenex22} where there is no sharing between the two conjuncts, each conjunct, being a phase, corresponds to a numeration. The \&P also corresponds to a numeration which includes DP1, DP2, and the conjunction head \textit{and}.  In \REF{shenex23}, on the other hand, since some elements are shared by the two conjuncts, the whole conjunction phrase has one numeration. Given that they contain the same set of lexical items, Numeration1 and Numeration2 in \REF{shenex23} are compared and the second numeration is less economical since it involves the Poss head being extracted twice. Thus Numeration2 is ruled out. The crucial point here is that Numeration1 and Numeration2 are compared with each other and not with the numerations in \REF{shenex22}, because none of the numerations in \REF{shenex22} contains the same set of lexical items as the ones in \REF{shenex23}. 

\ea 
\label{shenex22}
[\textsubscript{\&P} [\textsubscript{DP1} John's student] and [\textsubscript{DP2} Mary's student]] are a couple.\\
Numeration\textsubscript{DP1}: [John's$_1$, Poss$_1$, Num$_1$, student$_1$]\\
Numeration\textsubscript{DP2}: [Mary's$_1$, Poss$_1$, Num$_1$, student$_1$]\\
Numeration\textsubscript{\&P}: [and$_1$, DP1, DP2]

\ex
\label{shenex23}
[\textsubscript{\&P} John's and Mary's student] are a couple.
\ea 
	Numeration1\textsubscript{\&P}: [John's$_1$, Poss$_\textbf{1}$, Num$_1$, student$_1$, Mary's$_1$, and$_1$]
\ex 
	Numeration2\textsubscript{\&P}: [John's$_1$, Poss$_\textbf{2}$, Num$_1$, student$_1$, Mary's$_1$, and$_1$]
\z 
\z 

The claim here is that only the set of numerations that meet certain conditions are compared in terms of economy. One such condition is that these numerations must contain the same set of unique lexical items. They can, however, differ in the number of ``copies'' of the lexical items.\footnote{This restriction on MaxShare is by no means the only restriction. \citet[Section 2.6.2]{Shen:2018a} briefly discusses an interpretative restriction on MaxShare:  the structures being compared must be of the same interpretation. The evidence is shown in \REF{shenftn8exi}. The sentence in \REF{shenftn8exi} is ambiguous between \REF{shenftn8exi:a} where \textit{tall} is not shared, and \REF{shenftn8exi:b} where \textit{tall} is shared. If MaxShare does not care about interpretations, the interpretation in \REF{shenftn8exi:a} should not be available since it involves a structure where less material is shared than in the structure that generates \REF{shenftn8exi:b}.

\ea 
	\label{shenftn8exi}
	The old and the young tall student are a couple.
	\ea 
		\label{shenftn8exi:a}
		`The old student and the young tall student are a couple.'
	\ex 
		\label{shenftn8exi:b}
		`The old tall student and the young tall student are a couple.'
	\z 
\z 
}


\section{A note on an alternative to MaxShare}
\label{shensect:alternatives}

\subsection{Ban on string vacuous multi-dominance in NP right node raising}
\label{shensect:ban}

This section explores an alternative to MaxShare to account for patterns of NP RNR. Evidence for MaxShare from NP RNR comes from the unacceptability of \figref{shensvs}. \citet{Shen:2018a} argues that the singular marking on the shared noun requires multi-dominance and the unacceptability of \figref{shensvs} indicates that a multidominance structure is ruled out. As mentioned above, the account proposed in \citet{Shen:2018a} involves two constraints on multi-dominance: one is MaxShare, and the other is an Agree requirement where the shared element and the sharing elements must agree. We have seen how MaxShare rules out structures that would generate \figref{shensvs} in the discussion above. The Agree requirement rules out the structure that does not violate MaxShare. In the structure in \figref{shensvs}, the largest shareable constituent \textsc{poss'} is shared in accordance with MaxShare, however, this structure is ruled out because there is no agreement relation between the possessors \textit{John's} and \textit{Mary's} and the shared \textsc{poss} head.

\begin{figure}\small
\captionsetup{margin=.05\linewidth}
\begin{floatrow}
\ffigbox
{\begin{forest}
qtree edges
	[ConjP
		[PossP1, name=possp
			[\textit{John's}]
		]
		[Conj'
			[\textit{and}]
			[PossP2
				[\textit{Mary's}]
				[Poss', name=shared
					[Poss]
					[NumP
						[Num]
						[NP, name=n
							[\textit{student}]
						]
					]
				]
			]
		]
	]
	\draw (possp.south) -- (shared.north);
\end{forest}\\
*John's and Mary's student are a couple.}
{\caption{Candidate structure: Agree constraint violated, MaxShare satisfied\label{shensvs}}}

\ffigbox
{\begin{forest}
qtree edges
	[DP
		[\&P
			[John's]
			[\&'
				[\&]
				[Mary's]
			]
		]
		[D'
			[Poss]
			[NumP
				[Num]
				[NP
					[students]
				]
			]
		]
	]
\end{forest}\\
John's and Mary's students are a couple.}
{\caption{\label{shenex25}Coordinated possessor structure}}
\end{floatrow}
\end{figure}

\figref{shensvs} contrasts with the sentence in \figref{shenex25} where the head noun is plural. \citet{Shen:2018a} argues that the plural noun indicates a different structure, illustrated below: \textit{John's} and \textit{Mary's} are conjoined in the Spec,DP position. No sharing\slash multidominance is involved. 

The motivation behind the Agree requirement and MaxShare is to rule out \figref{shensvs} independently from \figref{shenex25}. However, based on the contrast between the two sentences, one can imagine an alternative where it is the availability of the structure in \figref{shenex25} that blocked the multidominance structure in \figref{shensvs}. I formulate the constraint in \REF{shenban} and refer to it as the \textsc{ban}.

\eanoraggedright
	\label{shenban}
	Ban on string vacuous sharing: A string cannot be parsed as multidominance if an alternative non-sharing parse is available. 
\z 
The idea behind \REF{shenban} is that the option of sharing can only be entertained if the string cannot be generated otherwise. From this perspective, sharing is used as a last resort operation. Let's see how the \textsc{ban} in \REF{shenban} can rule out \figref{shensvs}. In the string \textit{John's and Mary's X}, there are at least two possible parses shown in \figref{shenex27}. \figref{shenex27:a} is a parse with a shared X while \figref{shenex27:b} involves conjoined specifiers and no sharing. The \textsc{ban} in \REF{shenban} states that the \figref{shenex27:a} is ruled out since \figref{shenex27:b} is available. Thus this constraint alone can replace both MaxShare and the Agree requirement.

\begin{figure}
\begin{subfigure}[b]{.5\linewidth}\centering
\begin{forest} 
qtree edges
	[\&P
		[DP1, name=dp1
			[John's]
		]
		[\&'
			[and]
			[DP2, name=dp2
				[Mary's]
				[D', name=d', l*=1.5, s=50
					[Poss]
					[X]
				]
			]
		]
	]
	\draw (dp1.south) -- (d'.north);
\end{forest}
\caption{Sharing X\label{shenex27:a}}\end{subfigure}%
\begin{subfigure}[b]{.5\linewidth}\centering 
\begin{forest} 
qtree edges
	[DP
		[\&P
			[John's]
			[\&'
				[and]
				[Mary's]
			]
		]
		[D'
			[Poss]
			[X]
		]
	]
\end{forest}
\caption{Conjoined specifier, no sharing\label{shenex27:b}}
\end{subfigure}
\caption{\label{shenex27}John's and Mary's X}
\end{figure}

The \textsc{ban} predicts that sharing becomes available once the string cannot be generated otherwise. This prediction is supported by the phrase in \figref{shenex28}. The singular noun indicates that the head noun \textit{student} is shared. This is expected since \textit{John's tall} and \textit{Mary's short} cannot be conjoined as is shown in \figref{shenex28:a} because they do not form constituents (assuming that only constituents can be conjoined). In other words, the string cannot be generated without invoking sharing, thus sharing is available as is shown in \figref{shenex28:b}.

\begin{figure}\small
\begin{subfigure}[b]{.5\linewidth}\centering
\begin{forest} 
qtree edges
	[DP
		[\&P
			[DP1
				[John's]
				[tall]
			]
			[\&'
				[and]
				[DP2
					[Mary's]
					[short]
				]
			]
		]	
		[student]
	]
\end{forest}
\caption{No sharing structure ruled out\label{shenex28:a}}
\end{subfigure}\begin{subfigure}[b]{.5\linewidth}\centering
\begin{forest} 
qtree edges
	[\&P
		[DP1
			[John's]
			[D'
				[D]
				[NP, name=np1
					[tall]
				]
			]
		]
		[\&'
			[and]
			[DP2
				[Mary's]
				[D'
					[D]
					[NP, name=np2
						[short]
						[student, name=n, l*=1.2, s=10]
					]
				]
			]	
		]
	]
	\draw (np1.south) -- (n.north);
\end{forest}
\caption{Sharing structured ruled in\label{shenex28:b}}
\end{subfigure}
\caption{\label{shenex28}John's tall and Mary's short student are a couple.}
\end{figure}

A brief discussion of the alternative non-sharing structures is in order. The two structures being compared above include one sharing structure with coordinated DPs and the non-sharing structure with coordinated Spec,DPs. As it turns out, all the non-sharing structures to be considered in this section will involve conjunction of two smaller constituents than in the sharing structure. This is expected since in the sharing structure, the shared node is inside the conjuncts whereas in the non-sharing structure, it is outside the conjuncts. In the paper, I will restrict the broad term \emph{alternative non-sharing parse} to this type of structure with conjunction of smaller constituents. Whether other non-sharing structures should/can be covered by the \textsc{ban} in \REF{shenban} is left for future research.

Based on its formulation, the effect of the \textsc{ban} should be observed when two conditions are met: 1. a string that can be generated via sharing and a non-sharing structure; 2. a telltale indication of which structure is being used. In the case discussed above, the string of \emph{John's and Mary's N} can be generated via sharing of the N or the conjoined possessor analysis. The telltale sign is the number marking on the noun: when the phrase refers to two individuals, sharing requires the noun to be singular and the conjoined possessor analysis requires the noun to be plural. As we saw in the case of NP RNR, the availability of the conjoined possessor structure (indicated by the plural noun) blocked sharing (as indicated by the unavailability of the singular noun).

Another case where these conditions are met is in \figref{shenex29}, which also has two potential structures. \figref{shenex29:a} illustrates one where the T' is shared and \figref{shenex29:b} illustrates one that does not involve sharing but the conjunction of the subjects. The telltale sign to differentiate the two structure is the number marking on the verb. According to \citet{Kluck:2009, Grosz:2015}, and \citet{Shen:2019}, the structure in \figref{shenex29:a} is compatible with both the singular and the plural auxiliary whereas the conjoined subject in \figref{shenex29:b} requires the auxiliary to be plural. As is shown in \figref{shenex29}, only the plural auxiliary is available, which indicates that the sharing structure \figref{shenex29:a} is ruled out while the non-sharing structure \figref{shenex29:b} is ruled in. This is expected from the \textsc{ban}.

\begin{figure}\small
\begin{subfigure}[b]{.5\linewidth}\centering
\begin{forest}
qtree edges
	[\&P
		[TP1, name=tp1
			[John]
		]
		[\&'
			[and]
			[TP2, name=tp2
				[Mary]
				[T', name=t
					[T
						[has]
					]
						[VP\\$\dots$
						]
				]
			]
		]
	]
	\draw (tp1.south) -- (t.north);
\end{forest}\\
*John and Mary has eggs for $\dots$
\caption{\label{shenex29:a}Sharing T', violating the \textsc{ban}}
\end{subfigure}\begin{subfigure}[b]{.5\linewidth}\centering
\begin{forest}
qtree edges
	[TP
		[\&P
			[John]
			[\&'
				[and]
				[Mary]
			]
		]
		[T'
			[T
				[have]
			]
			[VP\\$\dots$
			]
		]
	]
\end{forest}\\
John and Mary have eggs for $\dots$
\caption{\label{shenex29:b}Conjoined specifier structure}
\end{subfigure}
\caption{\label{shenex29}John and Mary \textsc{have} eggs for breakfast.}
\end{figure}

Like in NP RNR, once we modify the string so it cannot be generated by the non-sharing structure, the sharing structure becomes available. 
In \REF{shenex30}, neither \textit{John always} nor \textit{Mary sometimes} form a constituent, thus conjunction structure in \figref{shenex29:b} is impossible. 
Since the string can only be generated by sharing, it is predicted that the singular auxiliary becomes available, as is confirmed in \REF{shenex30}.\footnote{What is curious is that the plural auxiliary in \REF{shenftn9exi} is not acceptable. Although surprising under the sharing analysis, this does not immediately rule out this analysis. It is possible that the plural auxiliary under sharing is further restricted. This type of restrictions are discussed in \citealt{Yatabe:2003, Grosz:2015, Belk:2018}.

\ea[*]{ 
	\label{shenftn9exi}
	John always, and Mary never, have eggs for breakfast.
}
\z 
} 

\ea[?]{ 
	\label{shenex30}
	John always, and Mary never, has eggs for breakfast.
}
\z 


The next example of the \textsc{ban} I will present here also involves NP RNR but with a different telltale sign: interpretation. The string in~\figref{shenex31} can potentially be generated via a structure where \textit{dress} is shared by \textit{blue} and \textit{black} shown in~\figref{shenex31:a} or via a structure where \textit{blue} and \textit{black} are conjoined as shown in~\figref{shenex31:b}. The non-sharing structure with the singular noun must refer to a single dress that's both blue and black, whereas the sharing structure, also with the singular noun, must refer to two different dresses, one being blue and the other black. The absence of the two-dress reading indicates that sharing is ruled out while the non-sharing structure is available (indicated by the one-dress reading). 

\begin{figure}\small
\begin{subfigure}[b]{.5\linewidth}\centering
\begin{forest}
qtree edges
	[DP
		[the]
		[\&P
			[NP1, name=np1
				[blue]
			]
			[\&'
				[and]
				[NP2
					[black]
					[dress, name=dress]
				]
			]
		]
	]
	\draw (np1.south) -- (dress.north);
\end{forest}\\
\# two dresses each with one color
\caption{\label{shenex31:a}Sharing dress, violating the \textsc{ban}}
\end{subfigure}\begin{subfigure}[b]{.5\linewidth}\centering
\begin{forest}
qtree edges
	[DP
		[the]
		[NP
			[\&P
				[blue]
				[\&'
					[and]
					[black]
				]
			]
			[dress]
		]
	]
\end{forest}\\
one dress with two colors
\caption{\label{shenex31:b}Conjoined adjectives}
\end{subfigure}
\caption{\label{shenex31}The blue and black dress}
\end{figure}

The blocking nature of the \textsc{ban} predicts there to be no overlapping distribution of the two structures: when the non-sharing structure is available, the sharing structure is blocked; only when the non-sharing structure is not available does the sharing structure emerge. This predicts complementary distribution of the two structures, i.e. one string cannot optionally show telltale signs for both structures. This is borne out for the cases we have seen in English. In a given string of NP RNR in \REF{shenex32}, either the singular or the plural shared noun is allowed, but not both. The complementary distribution in NP RNR is also observed in all the languages reported in \citet{Shen:2018a} including Brazilian Portuguese, Cypriot Greek, Dutch, English, German, Icelandic, Italian, Polish, Serbo-Croatian, and Slovenian, Spanish. 

\ea 
	\label{shenex32}
	\ea 
		John's and Mary's students/*student are a couple.
	\ex 
		John's tall and Mary's short student/*students are a couple.
	\z 
\z 

\subsection{Can the ban on string vacuous sharing replace MaxShare}
\label{shensect:replace}

I have shown that the \textsc{ban} can replace MaxShare and the Agree requirement in NP RNR. Now we look at whether it can replace MaxShare in the ATB LBE and gapping paradigm discussed in \citet{Citko:2006} and earlier in this paper.

First, let's look at the ATB LBE data in \REF{shenex33} (repeated from \REF{shenbad} and \REF{shenatb2}). 
Both sentences involve two conjoined TPs and one adjective shared by two NPs (indicated by the traces). 
There is no conceivable alternative structure that involves no sharing and a smaller conjunction site as discussed above.  
Thus, the \textsc{ban} correctly does not rule out the sharing structure which made ATB LBE possible in \REF{shenatb22}. 
However, this means that the \textsc{ban} can not rule out the less acceptable \REF{shenbad2}. 
An additional constraint like MaxShare is still needed.

\ea 
	\label{shenex33}
	\ea[]{ 
		\label{shenatb22}
		\gll Ile$_{\textnormal{i}}$ [$_{\textnormal{TP}}$ Maria napisała t$_{\textnormal{i}}$ ksią\.z{}ek] a [$_{\textnormal{TP}}$ Jan przeczytał t$_{\textnormal{i}}$ artykułow]? \\
		how-many {~} Maria wrote {~} books and {~} Jan read {~} articles\\
		\glt `How many books did Maria write and how many articles did Jan read?'
	}
	\ex[*]{
		\label{shenbad2}
		\gll Ktorą$_{\textnormal{i}}$ [$_{\textnormal{TP}}$ Maria poleciła t$_{\textnormal{i}}$ ksią\.z{}kę] a [$_{\textnormal{TP}}$ Jan przeczytał t$_i$ ksią\.z{}kę]? \\
		which {~} Maria recommended {~} book and {~} Jan read {~}  book\\
		\glt `Which book did Mary recommend and John read?'
	}
	\z 
\z 
Second, ATB LBE\,+\,gapping discussed in \REF{shengapAtb} with English glosses repeated in \REF{shenex34}. \REF{shenex34a} involves both ATB LBE of \textit{which} and gapping whereas \REF{shenex34b} only involves ATB LBE.
The sharing analysis of gapping involves conjunction of vPs and sharing of the verb (\textit{ordered} in \ref{shenex34}). Again, there is no alternative non-sharing structure with a smaller conjunction site. Similar to \REF{shenex33},  the \textsc{ban} correctly does not rule out \REF{shenex34a} but the less acceptable \REF{shenex34b} is not ruled out either.

\ea\judgewidth{*?}
	\label{shenex34}
	\ea[]{ 
		\label{shenex34a}
		Which$_1$ Maria \textbf{ordered} t$_1$ coffee and Jan t$_1$ tea?
	}
	\ex[*?]{
		\label{shenex34b}
		Which$_1$ Maria \textbf{ordered} t$_1$ coffee and Jan \textbf{ordered} t$_1$ tea? \\
		(English glosses for Polish sentences in \REF{shengapAtb})
	}
	\z 
\z 
The positive note is that the \textsc{ban} is compatible with the paradigm above in that it does not rule out the acceptable sentences; however, it also does not help accounting for \REF{shenbad2} and \REF{shenex34b}. MaxShare as proposed by \citet{Citko:2006} is still needed and cannot be replaced by the \textsc{ban}. This result is not surprising, since the \textsc{ban} only rules out a sharing structure in the face of a non-sharing one. What MaxShare accomplishes is choosing between two sharing structures, with one sharing more materials than the other. So far, I have shown that MaxShare can account for all the data in ATB LBE noted so far and part of the NP RNR paradigm, while the \textsc{ban} can account for all the data in NP RNR but not ATB LBE. Considering both ATB LBE and NP RNR, MaxShare is needed plus either the Agree requirement or the \textsc{ban}. The answer to ``can the \textsc{ban} replace MaxShare'' is \emph{no}. It turns out that it's not MaxShare that the \textsc{ban} can potentially replace but the Agree requirement. With MaxShare independently motivated, now the question becomes whether to retain the Agree requirement as in \citet{Shen:2018a} or to replace it with the ban on string vacuous sharing. I will leave this question for future research.

\subsection{Another alternative} 

The \textsc{ban} is by no means the only potential alternative to MaxShare. Another possible alternative that I do not have space to discuss here beyond several sentences is related to the contrast conditions on ellipsis proposed in \citet{Hartmann:2000, Hartmann:2003, Fery:2005a}. Although the original proposals are meant for ellipsis, the phenomena the proposed conditions cover include RNR and gapping, largely overlapping with MaxShare. \citet{Hartmann:2000, Hartmann:2003} argues that RNR is derived from phonetic deletion rather than multi-dominance, and that phonetic deletion requires the preceding materials to be contrastive. In addition, a \textit{maximal contrast principle} in \REF{shenMCP} is proposed for gapping, which is very similar in essence to MaxShare. I will group the various conditions proposed in these works and label them as \textit{contrastive conditions}.

\ea 
	\label{shenMCP}
	The maximal contrast principle\\
	In a Gapping construction maximize the number of contrasting remnant-correspondent pairs. (\citealt[p. 165, 43]{Hartmann:2000})
	\z 


Assuming that ATB movements are subject to contrast conditions of the same nature, the contrast is required not only on the material preceding the shared element but also the materials following it to account for the ATB LBE data in \REF{shenatb1} and \REF{shenatb2} from \citet{Citko:2006}. The interaction of ATB movement and gapping shown in \REF{shengapAtb} where one requires the other can be accounted for as well by applying \REF{shenMCP} to ATB and gapping. 

Regarding NP RNR, requiring the materials preceding the shared noun to be contrastive can correctly rule out Figures~\ref{shennp1} and~\ref{shennp2}. However, something like the Agree requirement or the \textsc{ban} is still needed in addition to rule out the structure in \figref{shenmax}. Since \textit{John's} and \textit{Mary's} are contrastive, but as we have learned, sharing of the noun phrase following these contrasting elements is disallowed. 

Further research is needed to thoroughly evaluate whether the contrastive conditions can replace MaxShare in general. For example, \citet{Hartmann:2000} proposes that the domain of application of the condition in \REF{shenMCP} is the phonological phrase. It remains to be seen whether such restrictions are the same when the condition is applied to ATB and NP RNR.  The full paradigm including ATB, gapping, and NP RNR can be accounted for by different combinations of the conditions/constraints discussed in this paper: MaxShare, the Agree requirement on sharing, and the two alternatives presented in this section. Pros and cons of each combination require careful investigation that goes beyond this paper.

\section{Summary}
\label{shensect:summary}

This paper discussed three aspects of MaxShare: its formulation, its restrictions, and possible alternatives. We have seen that the numeration formulation of MaxShare is more empirically powerful in ruling out sentences and less stipulative regarding the motivation of such a constraint on sharing. At the same time, the effects of the numeration MaxShare seems less robust within or across languages than that of the dominance MaxShare. I have also shown that the effects of MaxShare need to be restricted within structures that involves sharing in the first place. Lastly, the ban on string vacuous sharing, a potential alternative to MaxShare, turns out to be successful for NP RNR but not for other cases of sharing. 

In this paper, I was only able to scratch the surface of these issues, which all deserve more detailed, cross-linguistic research. One promising direction is on the locality of MaxShare, i.e. the domain within which MaxShare is enforced. MaxShare states that the shared materials within a domain must be maximized, in other words, if one element is shared in this domain, all other shareable elements must be shared as well.  The locality question is how far the two shared elements can be for one to trigger the sharing of the other. The cases we have been looking at are limited in this aspect. In NP RNR, the domain of MaxShare is within two conjoined DPs: the sharing of the head noun triggered the sharing of the Poss head and the \textsc{num} head within the DP. In ATB LBE cases, the domain is within two conjoined matrix clauses: sharing of the adjective of the objects triggers sharing of the verb via gapping. We have not seen long distance triggering where, for example, the sharing of the \textit{embedded} object forces gapping of the \textit{matrix} verb. We also have not seen triggering across islands or other boundaries proposed in the literature.\footnote{An assumption made in Section \ref{shensect:ban} is that numerations are evaluated phase by phase. It follows then that the MaxShare effects should be confined within phases. However, this is not compatible with the interaction of ATB of adjectives of the objects and the gapping of the verb discussed in \citet{Citko:2006}.} To address this question, one interesting project would be to look at the interaction of MaxShare, a constraint on size of shared constituents, and clausal complements of different sizes. 

\section*{Acknowledgements}

\begin{sloppypar}
I thank Barbara Citko, Michael Yoshitaka Erlewine, and an anonymous reviewer for their comments and suggestions, Natalia Banasik-Jemielniak and Paulina Lyskawa for their Polish judgments and Michael Yoshitaka Erlewine and Lyn Tieu for the English ones. The research reported in this paper started as part of my dissertation for which Susi was the chair of the committee of. 
\end{sloppypar}


{\sloppy\printbibliography[heading=subbibliography,notkeyword=this]}

\end{document}
