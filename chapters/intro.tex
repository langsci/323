\documentclass[output=paper]{langscibook}
\author{Zheng Shen\affiliation{National University of Singapore} and Sabine Laszakovits\affiliation{Austrian Academy of Sciences; University of Connecticut}}


\title{Introduction: The size of things}
\abstract{}

\begin{document}
\maketitle

\noindent\textit{Size} in grammar, broadly construed, is the focus of this two-volume collection, \textit{The size of things.}
Under the umbrella term \textit{size} fall the size of syntactic projections, the size of feature content, and the size of  reference sets. 
Size and structure building is the shared focus of papers in Volume~I, while Volume~II presents papers looking into size effects in movement, agreement, and interpretation. 
Integrating a variety of research projects under this common theme, we hope this collection will inspire new connections and ideas in generative syntax and related fields. 

The most productive research program in syntax where size plays a central role revolves around clausal complements. 
Part~1 of Volume~I contributes to this program with papers arguing for particular structures of clausal complements as well as papers employing sizes of clausal complements to account for other phenomena. 
The ten contributions cover a variety of languages, many of which are understudied. 
Hanink discusses the availability of restructuring with thematic nominalizations in Washo. 
Kelepir investigates the size of the verbal domain under the nominalizing head in Turkman, Noghay and Turkish. 
Radkevich looks into aspectual verbs in Lak, and Alexiadou \& Anagnostopoulou into aspectual verbs in Greek.
Pajancic explores sizes of clausal complements in Akan in the context of the Implicational Complementation Hierarchy and the Finiteness Universal. 
Pesetsky offers an alternative account for non-finite clauses in English to the one in \cite{Wurmbrand:2014}.
Shimamura also contributes to the Implicational Complementation Hierarchy with a novel analysis of sentential complementation of \textit{yoo} in Japanese. 
Takahashi uses scope properties of nominative objects in Japanese to support the phrasal complementation approach to restructuring. 
Saito attributes the different behaviors of the Japanese particles \textit{teki} and \textit{ppoi} as well as \textit{mitai} and \textit{yoo} to the sizes of the clausal complement they take.
Todorović uses different sizes of clausal complements in Gitksan to account for the distribution of future interpretation. 

The papers in Part~2 of this volume explore the interaction between size and structure building beyond clausal complements. 
There are six papers in this part covering different domains in sentence structure.
Within the CP domain, Arano explores the debate over the size of the Spell-out domain in the CP and argues the CP phase to be the Spell-out domain. 
Messick and Alok use restrictions on stripping in Hindi to argue that the size of an embedded clause with the complementizer \textit{ki} in Hindi is different from an embedded clause with the complementizer \textit{that} in English. 
Inside the vP domain, Kuo argues for different positions of the applicative \textit{gei} in Mandarin Chinese, and  
Lacerda looks into object shift and middle-field topicalization. 
Bobaljik and L. B. Wurmbrand discuss a productive Austrian-American code-switching pattern involving English particle verbs and German verb clusters. 
Regarding the NP domain, 
the contribution by Pereltsvaig surveys the sizes of noun phrases in articleless languages and illustrates different behaviors of DPs and small nominals. 
Lastly, Shen discusses several aspects of the MaxShare constraint on multi-dominance, which maximizes the size of the shared elements. 

All the papers in these two volumes are influenced in various ways by the work of Susi Wurmbrand, who not only pioneers the investigation into clausal complements across languages from the lenses of 
binding, finiteness, movement, restructuring, tense, and verb clusters, 
but has also deepened our understanding of 
Agreement, Case, features, and quantifier raising. 
Furthermore, Susi has had a direct personal impact on the work of all contributors and editors, and so we dedicate this book to her not only in recognition of her achievements, but also in gratitude of her generosity to us. 


{\sloppy\printbibliography[heading=subbibliography,notkeyword=this]}

\end{document}
