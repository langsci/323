\documentclass[output=paper]{langscibook}
\author{Renato Lacerda\affiliation{University of Connecticut}}
\title{The middle field of Brazilian Portuguese and the size of the verbal domain}
\abstract{This paper investigates word-order permutations in the (sentence-internal) postverbal area (i.e., ``middle field'') of Brazilian Portuguese, 
    in order to determine the precise make-up and size of the verbal domain in the language. 
    Two operations that independently place elements in postverbal vP-external positions are analyzed, 
    namely object shift and topicalization, and lead to the proposal of an independent vP-external functional projection XP, 
    whose A-specifier hosts shifted objects and to which middle-field topics adjoin. 
    The relationship between middle-field topics and shifted objects is shown to provide evidence for the phasehood of XP, 
    which thus delimits the extended verbal domain of Brazilian Portuguese as a phasal domain. 
    Additionally, a brief comparison between Brazilian Portuguese middle-field topics and German \textit{Mittelfeld} topics is entertained, 
    which shows the position of sentence-internal topics relative to sentential adverbs to be a safe diagnostic for the availability 
    of aboutness topic interpretation.
}

\begin{document}
\SetupAffiliations{mark style=none}
\maketitle

\section{Introduction}
In Brazilian Portuguese, word-order permutations in the (sentence-internal) post-verbal area are fairly common.\footnote{A lengthier discussion of the issues presented in this paper appears in my Ph.D. dissertation (\citealt{Lacerda2020b}).}
This is illustrated in the paradigm in (\ref{lacerda1}), which manipulates the relative order of the direct object, the indirect object, and a subject-related floating quantifier. With the subject and the verb traditionally assumed to move to TP, as represented in (\ref{lacerda2}), the question arises of what the structural make-up of the area between the traditional TP and vP is in the language.

\begin{exe}
\ex \label{lacerda1}
\begin{xlist}
\ex \label{lacerda1a}
\gll Os professores deram dois livros cada um pros alunos.\\
the teachers gave two books each one to-the students\\

\ex \label{lacerda1b}
\gll Os professores deram cada um dois livros pros alunos.\\
the teachers gave each one two books to-the students\\

\ex \label{lacerda1c}
\gll Os professores deram, 	pros alunos, dois livros cada um.\\
the teachers gave to-the students two books each one\\

\ex \label{lacerda1d}
\gll Os	professores deram, 	pros alunos, cada um dois livros.\\
the teachers gave to-the students each one two books\\
\glt‘The teachers gave the students two books each.’	
\end{xlist}
\end{exe}

\begin{exe}
\ex \label{lacerda2}
[$_{\textnormal{TP}}$ \emph{subject} \emph{verb} [\hspace{0.5mm}??? [$_{\textnormal{vP}}$ [$_{\textnormal{VP}}$\hspace{0.5mm}]\hspace{0.5mm}]\hspace{0.5mm}]\hspace{0.5mm}]
\end{exe}

In this paper, I will analyze how two distinct syntactic operations can be resorted to in order to derive different orders of postverbal elements, such as the ones illustrated in (\ref{lacerda1}) above. Both of these operations will be shown to place elements in postverbal vP-external positions, in an area of the clause that I will descriptively refer to as the “middle field”. The analysis of these operations will allow us to determine the “size” of the verbal domain in Brazilian Portuguese, that is, to propose a characterization of the structural make-up of the extended verbal domain of the language (which will be argued to be a phase).

First, I will discuss an operation that I will refer to as “object shift”, which will be shown to place the (single) highest internal argument of the verb in the A-specifier of an independent functional projection immediately above vP. Next, I will discuss an operation that I will refer to as “middle-field topicalization”, which places elements interpreted as topics in a position immediately above shifted objects. Finally, based on a comparison with the German \emph{Mittelfeld}, I will argue that the (low) structural height of the Brazilian Portuguese middle field is responsible for ruling out aboutness interpretation for topics realized in that area of the clause.

\largerpage[2]

\section{Object shift}
To start the discussion of the structural make-up of the middle field of Brazilian Portuguese, I will address an operation that I will refer to as “object shift”, which places the (single) highest internal argument of the verb in a postverbal vP-external position.\footnote{I use the term “object shift” here for mere ease of exposition, since the operation in question usually (but not always) targets direct objects. This operation is in fact similar to what \citet{LasnikSaito1991} and \citet{Boskovic1997}, among others, argue is object shift in English, which is different from the operation discussed by e.g. \citet{Holmberg1986} and \citet{Diesing1996} for other Germanic languages.}  I will argue that this operation is an instance of A-movement that is not semantically or informationally motivated and can be likened to subject movement.
Assuming as a point of departure that the basic structure of the verbal domain in Brazilian Portuguese includes at least two projections (VP and vP) and that manner adverbs are adjoined to the structure at the vP layer, as in (\ref{lacerda3}), we can see in the paradigm in (\ref{lacerda4}) that in a neutral, broad-focus sentence such as (\ref{lacerda4}B1) the direct object can either precede or follow the vP-adverb \emph{direito} ‘properly’. When the indirect object undergoes such a movement in an informationally-neutral context, on the other hand, the sentence is ruled out, as in (\ref{lacerda4}B2). This contrast thus suggests that only the highest internal argument of the verb can occupy the object shift position (the base order being DO--IO). In other words, object shift in Brazilian Portuguese may rearrange the order between arguments and adjuncts, but not between arguments.

\begin{exe}
\ex \label{lacerda3}
[$_{\textnormal{vP}}$ \emph{manner adverb} [$_{\textnormal{vP}}$ \emph{agent} [$_{\textnormal{v'}}$ v$^{0}$ [$_{\textnormal{VP}}$ \emph{theme} [$_{\textnormal{V'}}$ V$^{0}$ \emph{goal}]\hspace{0.5mm}]\hspace{0.5mm}]\hspace{0.5mm}]\hspace{0.5mm}]
\end{exe}

\begin{exe}
\ex \label{lacerda4}
 \begin{xlist}
\exi{A:} \label{lacerda4A}
O que aconteceu?\\
`What happened?'

\exi{B1:} \label{lacerda4B1}
\gll O João não explicou \{uma história\} direito \{uma	história\} pra Maria.\\
the John not explained \{a story\} right \{a story\} to-the Mary\\

\exi{B2:} \label{lacerda4B2}
\gll \# O João não explicou pra Maria direito uma história.\\
{} the	John not explained to-the Mary right a story\\
\glt ‘John didn’t explain a story to Mary properly.’

    \end{xlist}
\end{exe}

Further evidence that the direct object may leave the vP in (di)transitive constructions comes from its relative positioning with respect to subject-related floating quantifier \emph{cada um} ‘each one’. \citet{Lacerda2012,Lacerda2016a} argues that \emph{cada um}, when following the direct object in what resembles the so-called “binominal each” construction in English (see \citealt{SafirStowell1988}, \citealt{Stowell2013}), also marks the edge of vP. More precisely, given that \emph{cada um} in sentences like those in (\ref{lacerda5}) is related to the subject (and is thus base-generated in the external argument position), the lowest position where it can be stranded is Spec,vP. The fact that \emph{dois presentes} precedes \emph{cada um} in (\ref{lacerda5a}) therefore shows that the direct object has moved to a vP-external position (with (\ref{lacerda5b}) being equally well formed). We can then assume the structure of the extended verbal domain represented in (\ref{lacerda6}), which includes an additional projection XP, whose specifier can host shifted objects.\footnote{The category of XP is immaterial for the purposes of this paper.} 

\begin{exe}
\ex \label{lacerda5}
\begin{xlist}
\ex \label{lacerda5a}
\gll Os alunos deram dois presentes cada um pro professor.\\
the students gave two gifts	each	one	to-the teacher\\

\ex \label{lacerda5b}
\gll Os alunos deram cada um dois presentes pro professor.\\
the students	gave each one	two	gifts to-the teacher\\
\glt‘The students gave two gifts each to the teacher.’
\end{xlist}

\ex \label{lacerda6} \emph{subject verb} [$_{\textnormal{XP}}$ \{\emph{DO}\} [$_{\textnormal{vP}}$ (\emph{manner adverb}) [$_{vP}$ (\emph{each}) [$_{\textnormal{VP}}$ \{\emph{DO}\} \emph{IO}] ] ] ]
\end{exe}

It is important to note that object shift in Brazilian Portuguese is not semantically or informationally motivated. In the relevant examples above, the direct objects were illustrated by indefinite DPs (\emph{uma história} ‘a story’ in (\ref{lacerda4}), and \emph{dois presentes} ‘two gifts’ in (\ref{lacerda5})), but as can be seen  in (\ref{lacerda7}), definiteness also does not play a role in triggering or preventing movement of the definite direct object \emph{o livro} ‘the book’. Both (\ref{lacerda7a}) and (\ref{lacerda7b}) are acceptable answers to a “what happened?” question. In the same fashion, the quantificational status of the direct object is also immaterial to object shift, as in (\ref{lacerda8}). 

\begin{exe}
\ex \label{lacerda7}
\begin{xlist}
\ex \label{lacerda7a}
\gll O	professor	não	explicou 	o 	livro 	direito 	pros 	alunos.\\
    the	teacher	not	explained	the	book	right	to-the	students\\

\ex \label{lacerda7b}
\gll O	professor 	não	explicou 	direito 	o	livro	pros	alunos.\\
    the	teacher	not	explained	right	the	book	to-the	students\\
\glt‘The teacher didn’t explain the book to the students properly.’\\

\end{xlist}

\ex \label{lacerda8}
\begin{xlist}
\ex \label{lacerda8a}
\gll O	professor	não	explicou 	nenhum	livro	direito 	pros 	alunos.\\
the	teacher	not	explained	no	book	right	to-the	students\\

\ex \label{lacerda8b}
\gll O	professor 	não	explicou 	direito 	nenhum	livro	pros	alunos.\\
    the	teacher	not	explained	right	no	book	to-the	students\\
\glt‘The teacher didn’t explain any book to the students properly.’

\end{xlist}
\end{exe}

Like I argued above is the case with object shift, subject movement in Brazilian Portuguese is also standardly assumed not to be semantically motivated. Rather, it is assumed to be an instance of formal A-movement (see e.g. \citealt{Nunes2010} for arguments to that effect). To further argue that object shift should be likened to subject movement in the language, I will show that shifted objects and subjects pattern alike with respect to the possibility of reconstruction in two independent semantic domains, namely variable binding (as seen in pronoun binding) and quantifier scope (as seen in distributivity).

First note in (\ref{lacerda9a}) that a subject quantifier can bind a pronoun in the direct object, but the converse in (\ref{lacerda9b}) is ruled out, which shows that the subject in (\ref{lacerda9b}) cannot reconstruct to its base position for pronoun binding purposes (recall from (\ref{lacerda6}) above that Spec,vP is lower than Spec,XP, which is a possible position for the direct object).\footnote{Note also that in spoken Brazilian Portuguese, the pronoun \emph{seu} usually can only refer to third person when bound; otherwise, it refers to second person.}



\begin{exe}
\ex \label{lacerda9}
\begin{xlist}
\ex[]{ \label{lacerda9a}
\gll Cada autor$_{\textnormal{i}}$	publicou	seu$_{\textnormal{i}}$ melhor	livro.\\
each	author	published	his	best	book\\
\glt ‘Each author$_{\textnormal{i}}$ published their$_{\textnormal{i}}$ best book.’}

\ex[*]{ \label{lacerda9b}
\gll [Seu$_{\textnormal{i}}$	pior	livro]$_{\textnormal{k}}$	envergonhou	cada$_{\textnormal{i}}$	autor 	t$_{\textnormal{k}}$.\\
his	worst	book	shamed	each	author\\
\glt‘Their$_{\textnormal{i}}$ worst book shamed each author$_{\textnormal{i}}$.’}
\end{xlist}

\end{exe}

Now let us look at (\ref{lacerda10}). Similarly to the subject case, in (\ref{lacerda10a}) the quantified direct object can bind the pronoun in the adjunct PP, whereas the reverse relation is not possible in (\ref{lacerda10b}).\footnote{In fact, structures like (\ref{lacerda10a}) were used by \citet{LasnikSaito1991} to argue for object shift in English.}  This state of affairs shows that the vP-external direct object cannot reconstruct to its base position for pronoun binding purposes, for in that position the pronoun should be able to be bound by the quantifier in the adjunct PP. That this is the case is shown by the grammaticality of (\ref{lacerda11}), where the direct object with the pronoun is realized lower than the adjunct PP.

\begin{exe}
\ex \label{lacerda10}
\begin{xlist}
\ex[]{ \label{lacerda10a}
\gll Eu 	comprei 	cada 	livro$_{\textnormal{i}}$ 	no 	seu$_{\textnormal{i}}$ 	lançamento.	\\
I 	bought 	each 	book 	on-the 	its 	launch\\
\glt‘I bought each book$_{\textnormal{i}}$ on its$_{\textnormal{i}}$ launch.’}

\ex[*]{ \label{lacerda10b}
\gll Eu 	encontrei 	[seu$_{\textnormal{i}}$ 	índice]$_{\textnormal{k}}$ 	em 	cada 	livro$_{\textnormal{i}}$	t$_{\textnormal{k}}$.\\
I 	found 	its 	index 	in 	each 	book\\
\glt‘I found its$_{i}$ index in each book$_{\textnormal{i}}$.’\\}
\end{xlist}

\ex \label{lacerda11}
\gll Eu 	identifiquei 	em	cada 	artigo$_{\textnormal{i}}$ 	seu$_{\textnormal{i}}$	melhor 	argumento.\\
I 	identified	in	each 	article 	its 	best	argument\\
\glt‘I identified in each article$_{\textnormal{i}}$ its$_{\textnormal{i}}$ best argument.’
\end{exe}

Now recall from the discussion of (\ref{lacerda5}) above that the quantifier \emph{cada um} ‘each one’ can float in a position as low as Spec,vP and that when the direct object precedes \emph{cada um}, it has undergone object shift to a vP-external position. If this movement is akin to subject movement and thus cannot reconstruct for pronoun binding, the prediction is that a pronoun in the direct object can be bound by the floating quantifier in the FQ--DO order but not in the DO--FQ order. This prediction is borne out, as the contrast in (\ref{lacerda12}) shows. 

\begin{exe}
\ex \label{lacerda12}
\begin{xlist}
\ex[]{\label{lacerda12a}
\gll Os	autores	publicaram	cada	um$_{\textnormal{i}}$	seu$_{\textnormal{i}}$	melhor	livro.\\
the	authors	published	each	one	his	best	book\\}

\ex[*]{ \label{lacerda12b}
\gll Os	autores	publicaram	[seu$_{\textnormal{i}}$	melhor	livro]$_{\textnormal{k}}$	cada	um$_{\textnormal{i}}$	t$_{\textnormal{k}}$.\\
the	authors	published	his	best	book	each	one\\
\glt‘The authors each$_{\textnormal{i}}$ published their$_{\textnormal{i}}$ best book.’}
\end{xlist}
\end{exe}

The ungrammaticality of (\ref{lacerda12b}) may seem surprising given the acceptability of (\ref{lacerda5a}) above, where the floating quantifier can distribute over the direct object even in the DO--FQ order. This contrast simply shows that pronoun binding and distributivity are computed in different ways (a matter I will put aside here); regardless of how it is to be accounted for, what is relevant here is that this contrast shows that object shift again patterns with subject movement in the relevant respect: As can be seen in (\ref{lacerda13}), a cardinal in subject position can also be distributed over by a quantifier realized in a lower position (in a sharp contrast with (\ref{lacerda9b}), where pronoun binding is at stake).

\begin{exe}
\ex \label{lacerda13}
\gll Dois 	alunos 	leram 	cada 	livro.\\
two	students	read	each	book\\
\glt‘Two students read each book.’
\end{exe}

Unsurprisingly, A’-movement on the other hand produces opposite results from what we have just seen above. In (\ref{lacerda14}), the direct object is topicalized in the left periphery of the sentence, and despite preceding the quantified subject, it allows for the binding of the pronoun. Interestingly, the quantifier cannot fulfill its strong distributivity requirement just by binding the pronoun (i.e., the topic in question does not reconstruct for distributivity), which in turn forces the presence of another expression over which \emph{cada um} can distribute, such as \emph{num ano diferente} ‘in a different year’.

\begin{exe}
\ex \label{lacerda14}
\gll [Seu$_{\textnormal{i}}$	pior	livro]$_{\textnormal{k}}$,	cada	autor$_{\textnormal{i}}$	publicou	t$_{\textnormal{k}}$ 	*(num	ano	diferente).\\
his	worst	book	each	author	published	{}	*(in-a	year	different)\\
\glt‘Their$_{\textnormal{i}}$ worst book, each author$_{\textnormal{i}}$ published in a different year.’
\end{exe}

In sum, we saw above that in Brazilian Portuguese A-movement may reconstruct for distributivity, but not for pronoun binding, whereas A’-movement may reconstruct for pronoun binding, but not for distributivity. The contrasts between distributivity and pronoun binding therefore provide additional evidence that object shift is best analyzed as A-movement, in that it patterns with subject movement in relevant respects. I take the fact that object shift in Brazilian Portuguese targets an A-position as evidence that it involves a separate projection (such as XP in \ref{lacerda6}), rather than vP-adjunction, given that one could reasonably expect that any purported vP position higher than both the base position of the agent and vP adverbs would be an A’-position. Moreover, recall that the object shift position is “picky”, in that it can only host the (single) highest internal argument of the verb; the presence of superiority effects thus suggests that some kind of probing is involved, which more likely creates a Spec-head configuration than an adjunction configuration.

Finally, I will point out one more similarity between XP and TP: just like TP attracts the highest argument of the verb, XP attracts the highest \emph{internal} argument of the verb. In the absence of an external argument (as in passive and unaccusative constructions), Spec,TP can host an internal argument of the verb; likewise, in the absence of a direct object, Spec,XP can host an oblique argument. As is shown in (\ref{lacerda15}) and (\ref{lacerda16}), the complement PP can either precede (vP-externally) or follow (vP-internally) the manner adverb and the floating quantifier \emph{cada um} ‘each one’.\footnote{Unlike Spec,TP, Spec,XP does not involve morphological agreement, whence the ability to host PPs. In fact, as was argued by \citet{Avelar2009}, Spec,TP can host PPs when the verb shows default (third-person singular) agreement, as in locative inversion constructions (see \citealt{Lacerda2016a,Lacerda2020b} for additional evidence).} Interestingly, the object shift position may also host the (postverbal) subject of a passive construction, which in this case is the highest internal argument of the verb, as is shown in (\ref{lacerda17}).

\begin{exe}
\ex \label{lacerda15}
\begin{xlist}
\ex \label{lacerda15a}
\gll O 	RH 	se 	mudou 	pro 	quarto	andar 	completamente	(no 	ano 	passado).\\
the 	HR 	self	moved	to-the	fourth	floor	completely 	(in-the	year	past)\\


\ex \label{lacerda15b}
\gll O 	RH 	se 	mudou 	completamente 	pro 	quarto	andar 	(no 	ano 	passado).\\
the 	HR 	self	moved	completely	to-the	fourth	floor	(in-the	year	past)\\
\glt‘Human Resources completely moved to the fourth floor (last year).’\\
\end{xlist}

\ex \label{lacerda16}
\begin{xlist}
\ex \label{lacerda16a}
\gll Os 	participantes 	apostaram 	em 	dois 	cavalos 	(até 	agora) 	cada 	um.\\
the	participants	bet	in	two	horses	(until	now)	each	one\\

\ex \label{lacerda16b}
\gll Os 	participantes 	apostaram 	(até 	agora) 	cada 	um	em 	dois 	cavalos.\\
the	participants	bet	(until	now)	each	one	in	two	horses\\
\glt‘The participants bet on two horses each (so far).’\\
\end{xlist}

\ex \label{lacerda17}
\gll Foram 	devolvidos [$_{\textnormal{XP}}$ os 	livrosk [$_{\textnormal{VP}}$ (ontem) 	[$_{\textnormal{VP}}$ cada 	um$_{\textnormal{i}}$ t$_{\textnormal{k}}$	pro 	seu$_{\textnormal{i}}$ 	autor] ] ].\\
were returned 	{}	the 	books 	{}	(yesterday) {}	each one {}	to-the	its 	author\\
\glt‘Each of the books$_{\textnormal{i}}$ was returned to its$_{\textnormal{i}}$ author (yesterday).’
\end{exe}

In conclusion, I argued in this section that the middle field of Brazilian Portuguese includes a vP-external “object shift” position, which can host the (single) highest internal argument of the verb. Like subject movement, object shift was shown to be the result of A-movement and not to be triggered by interpretive requirements. I argued that object shift involves a separate functional projection above vP, which extends the verbal domain. In the next section, I will continue to probe into the structural make-up of the middle field from the perspective of topicalization.

\section{Middle-field topicalization}

In this section, I will show that another operation can displace elements to a vP-external position in the middle field of Brazilian Portuguese, namely, topicalization. I will show that middle-field topics target a position immediately above the object shift position and must be associated with a focus in the object shift position, as a consequence of a phase-based locality constraint. For concreteness, I will assume here that middle-field topics are adjoined to XP, as represented in (\ref{lacerda18}) (see \citealt{Lacerda2019,Lacerda2020a,Lacerda2020b} for arguments against a cartographic analysis of topicalization in Brazilian Portuguese).

\begin{exe}
\ex \label{lacerda18}
[$_{\textnormal{TP}}$ \emph{subject verb} [$_{\textnormal{XP}}$ \emph{topic} [$_{\textnormal{XP}}$ \{DO\} [$_{\textnormal{vP}}$ [$_{\textnormal{VP}}$ \{DO\} IO] ] ] ] ]
\end{exe}

Unlike object shift, middle-field topicalization can be used to rearrange the order between arguments, as in (\ref{lacerda19}B), where the topicalized indirect object \emph{pros alunos} ‘to the students’ precedes the direct object \emph{dois livros} ‘two books’ (the base order being DO--IO). Sentence (\ref{lacerda20}B) shows that non-argumental constituents, such as an adnominal PP, may also be topicalized in the middle field.

\begin{exe}
\ex \label{lacerda19}
\begin{xlist}
\exi{A:} \label{lacerda19a}
O que os professores deram pros alunos?\\
‘What did the teachers give to the students?’

\exi{B:} \label{lacerda19B}
\gll Eles 	deram, 	\emph{pros} 	\emph{alunos$_{\textnormal{TOP}}$}, 	dois 	livros$_{\textnormal{F}}$ 	cada 	um 	(até 	agora).\\
they 	gave 	to-the 	students 	two 	books 	each 	one 	(until 	now)\\
\glt‘They gave \emph{the students$_{\textnormal{TOP}}$} two books$_{\textnormal{F}}$ each (so far).’

\end{xlist}

\ex \label{lacerda20}
\begin{xlist}
\exi{A:} \label{lacerda20A}
Quantos livros do Chomsky os alunos leram?\\
‘How many books by Chomsky did the students read?’

\exi{B:} \label{lacerda20B}
\gll Eles 	leram, 	\emph{do} 	\emph{Chomsky}$_{\textnormal{TOP}}$, 	dois 	livros$_{\textnormal{F}}$ 	cada 	um 	(até 	agora).\\
they 	read 	of-the 	Chomsky 	two 	books 	each 	one 	(until 	now)\\
\glt‘They read two books$_{\textnormal{F}}$ \emph{by Chomsky$_{\textnormal{TOP}}$} each (so far).’
\end{xlist}

\end{exe}

While topics are allowed in the middle field and may even reiterate, as in (\ref{lacerda21}B1--B2), focalized elements may not move to that area of the clause, as is shown by the unacceptability of (\ref{lacerda22}B1) and (\ref{lacerda23}B1). Foci are better off in situ, as in (\ref{lacerda22}B2) and (\ref{lacerda23}B2).

\begin{exe}
\ex \label{lacerda21}
\begin{xlist}
\exi{A:} \label{lacerda21A}
Quantos livros do Chomsky a Maria doou pro departamento?\\
‘How many books by Chomsky did Mary donate to the department?’

\exi{B1:} \label{lacerda21B1}
\gll Ela 	doou, 	\emph{do} 	\emph{Chomsky}$_{\textnormal{TOP}}$, 	\emph{pro} 	\emph{departamento}$_{\textnormal{TOP}}$,	dez 	livros$_{\textnormal{F}}$ 	(até 	agora).\\
she 	donated	of-the	Chomsky	to-the 	department	ten 	books	(until	now)\\

\exi{B2:} \label{lacerda21B2}
\gll Ela 	doou, 	\emph{pro} 	\emph{departamento}$_{\textnormal{TOP}}$,	\emph{do} 	\emph{Chomsky}$_{\textnormal{TOP}}$, 	dez 	livros$_{\textnormal{F}}$	(até 	agora).\\
she 	donated	to-the 	department 	of-the	Chomsky 	ten 	books	(until	now)\\
\glt‘She donated ten books$_{\textnormal{F}}$ \emph{by Chomsky$_{\textnormal{TOP}}$} \emph{to the department$_{\textnormal{TOP}}$} (so far).’\\


\end{xlist}

\ex \label{lacerda22}
\begin{xlist}
\exi{A:} \label{lacerda22A}
Pra quem os professores deram dois livros cada um?\\
‘To whom did the teachers give two books each?’

\exi{B1:} \label{lacerda22B1}
\gll ??Eles 	deram 	\emph{só} 	\emph{pra} 	\emph{Maria$_{\textnormal{F}}$} 	dois 	livros 	cada 	um 	(até 	agora).\\
they 	gave 	only 	to-the 	Mary 	two 	books 	each 	one (until 	now)\\

\exi{B2:} \label{lacerda22B2}
\gll Eles 	deram 	dois 	livros 	cada 	um 	\emph{só} 	\emph{pra} 	\emph{Maria}$_{\textnormal{F}}$ 	(até 	agora).\\
they 	gave 	two 	books 	each 	one 	only 	to-the 	Mary 	(until 	now)\\
\glt‘They gave two books each \emph{only to Mary$_{\textnormal{F}}$} (so far).’	\\

\end{xlist}

\ex \label{lacerda23}
\begin{xlist}
\exi{A:} \label{lacerda23A}
De que autor os alunos leram cada um dois livros?\\
‘The students each read two books by which author?’

\exi{B1:} \label{lacerda23B1}
\gll *Eles 	leram 	\emph{do} 	\emph{Chomsky}$_{\textnormal{F}}$ 	cada 	um 	dois 	livros 	(até 	agora).\\
they 	read 	of-the 	Chomsky 	each 	one 	two 	books 	(until 	now)\\

\exi{B2:} \label{lacerda23B2}
\gll Eles 	leram 	cada 	um 	dois 	livros 	\emph{do} 	\emph{Chomsky}$_{\textnormal{F}}$ 	(até 	agora).\\
they 	read 	each 	one 	two 	books 	of-the 	Chomsky 	(until 	now)\\
\glt‘They each read two books \emph{by Chomsky$_{\textnormal{F}}$} (so far).’
\end{xlist}

\end{exe}

That the topics in question are indeed in a sentence-medial position is corroborated by the licensing of negative concord and vP-ellipsis. First observe (\ref{lacerda24}) (adapted from \citealt[258]{Lacerda2016b}). The direct object is a negative concord item that is properly licensed by the preverbal negation, which shows that \emph{não} ‘not’ c-commands \emph{nenhuma pessoa} ‘no person’. As a consequence, the topic \emph{dos Democratas} ‘of the Democrats’ must be somewhere in-between the TP area (where the negation is) and the (extended) verbal domain (where the direct object is). In (\ref{lacerda25}B), the ellipsis of vP (containing the indirect object \emph{pra ela} ‘to her’ and the vP-adjoined adverbial PP \emph{no Natal} ‘on Christmas’) spares both the middle-field topic \emph{do Chomsky} ‘by Chomsky’ and the direct object \emph{cinco livros} ‘five books’, which provides further evidence for the vP-external position of these elements. 

\begin{exe}
\ex \label{lacerda24}
\gll O 	FBI 	não 	investigou, 	dos 	Democratas$_{\textnormal{TOP}}$, 	\emph{nenhuma} 	\emph{pessoa}	na 	eleição 	passada.\\
the 	FBI 	not 	investigated 	of-the 	Democrats 	no 	person	in-the	election	past\\
\glt‘The FBI didn’t investigate any person of the Democrats in the previous elections.’\\

\ex \label{lacerda25}
\begin{xlist}
\exi{A:} \label{lacerda25A}
A Maria adora ganhar livros de linguística.\\
O João deu dois livros do Pinker e três livros do Chomsky pra ela no Natal.\\
`Mary loves receiving linguistics books.\\
John gave two books by Pinker and three books by Chomsky to her on Christmas.'
\exi{B:} \label{lacerda25B}
\gll E 	eu 	dei, 	\emph{do} 	\emph{Chomsky}$_{\textnormal{TOP}}$, 	\emph{cinco} 	\emph{livros}$_{\textnormal{F}}$ 	<pra 	ela 	no 	Natal>.\\
and	I 	gave 	of-the 	Chomsky 	five 	books 	<to 	her 	in-the 	Christmas>\\
\glt‘And I gave \emph{five books}$_{\textnormal{F}}$ \emph{by Chomsky}$_{\textnormal{TOP}}$ <to her on Christmas>.'

\end{xlist}
\end{exe}

Having shown the existence of middle-field topics in Brazilian Portuguese, I will now argue that their positioning at the top of the extended verbal domain, right above the object shift position, grants them a very close relationship with shifted objects. In particular, I will argue that only elements that can independently reach Spec,XP (that is, the highest internal argument of the verb) can be focalized in the presence of a middle-field topic (especially in cases of contrastive topicalization, where the topic is associated with a focus; see \citealt{Buring2003}, \citealt{Wagner2012}).

Let us observe the paradigm in (\ref{lacerda26}). In the answers in (\ref{lacerda26}B1--B4), the topic \emph{do Chomsky} ‘by Chomsky’ is contrastively topicalized as an alternative to \emph{do Pinker} ‘by Pinker’ in the question in (\ref{lacerda26}A) (leaving the question about Pinker unresolved and proposing a new alternative question about Chomsky, which is in turn resolved). When the topic is realized in the left periphery and the focus (namely, the indirect object \emph{pra Ana} ‘to Anna’) is realized in situ, the sentence is grammatical and felicitous, as (\ref{lacerda26}B1) shows. Considering that the PP \emph{do Chomsky} is otherwise an acceptable middle-field topic (see e.g. \ref{lacerda25}B above) and that indirect objects can independently be focalized in situ, the unacceptability of (\ref{lacerda26}B2) is rather surprising.

\begin{exe}
\ex \label{lacerda26}
\begin{xlist}
\exi{A:} \label{lacerda26A}
Pra quem você recomendou livros do Pinker ontem?\\
‘Who did you recommend books by Pinker to yesterday?’
\exi{B1:} \label{lacerda26B1}
\gll /\emph{Do} 	\emph{Chomsky}$_{\textnormal{CT}}$/, 	\symbol{92}eu 	recomendei 	livros\symbol{92} 	/\emph{pra} 	\emph{Ana}$_{\textnormal{F}}$/	(ontem).\\
of-the 	Chomsky 	I 	recommended 	books 	to-the 	Anna	(yesterday)\\

\exi{B2:} \label{lacerda26B2}
\gll ?? Eu 	recomendei, 	/\emph{do} 	\emph{Chomsky}$_{\textnormal{CT}}$/, 	\symbol{92}livros\symbol{92} 	/\emph{pra} 	\emph{Ana}$_{\textnormal{F}}$/ 	(ontem).\\
{} I 	recommended 	of-the 	Chomsky 	books 	to-the 	Anna	(yesterday)\\

\exi{B3:} \label{lacerda26B3}
\gll * Eu 	recomendei, 	/\emph{do} 	\emph{Chomsky}$_{\textnormal{CT}}$/, 	/\emph{pra} \emph{Ana}$_{\textnormal{F}}$/	\symbol{92}livros\symbol{92}	(ontem).\\
{} I 	recommended 	of-the 	Chomsky 	to-the 	Anna 	books	(yesterday)\\


\exi{B4:} \label{lacerda26B4}
\gll ??/\emph{Do} 	\emph{Chomsky}$_{\textnormal{CT}}$/, 	\symbol{92}eu 	recomendei\symbol{92} 	/\emph{pra} \emph{Ana}$_{\textnormal{F}}$/ 	\symbol{92}livros\symbol{92} 	(ontem).\\
of-the 	Chomsky 	I 	recommended 	to-the 	Anna 	books 	(yesterday)\\
\glt‘I recommended books \emph{by Chomsky$_{\textnormal{CT}}$ to Anna$_{\textnormal{F}}$} (yesterday).’
\end{xlist}
\end{exe}

The well-formedness of (\ref{lacerda27}B) below, where the direct object is focalized instead, suggests that the focus must be close enough to the middle-field contrastive topic (in a way to be defined below). However, attempting to bring the focalized indirect object closer to the topic in (\ref{lacerda26}B3) leads to utter ungrammaticality. This result is in fact expected if we consider two observations made above: First, that the indirect object cannot undergo object shift past the direct object (cf.~\ref{lacerda4}B2), and second, that there is no focus-driven movement to the middle field (cf.~\ref{lacerda22}B1, \ref{lacerda23}B1) -- note that moving the focalized indirect object to the middle field is enough to ruin even the otherwise acceptable (\ref{lacerda26}B1), as in (\ref{lacerda26}B4).

\begin{exe}
\ex \label{lacerda27}
\begin{xlist}
\exi{A:} \label{lacerda27A}
Você recomendou quantos livros do Pinker pra Ana ontem?\\
‘How many books by Pinker did you recommend to Anna yesterday?’	

\exi{B:} \label{lacerda27B}
\gll \symbol{92}Eu 	recomendei\symbol{92}, 	/\emph{do}	 \emph{Chomsky}$_{\textnormal{CT}}$/,	/\emph{dois} 	\emph{livros}$_{\textnormal{F}}$/	\symbol{92}pra 	Ana\symbol{92}	(ontem).\\
I 	recommended	of-the 	Chomsky 	two	books 	to-the 	Anna	(yesterday)\\
\glt‘I recommended \emph{two books}$_{\textnormal{F}}$ \emph{by} \emph{Chomsky}$_{\textnormal{CT}}$ to Anna (yesterday).’

\end{xlist}
\end{exe}

The contrasts we just saw in (\ref{lacerda26}) and (\ref{lacerda27}) above therefore lead us to the conclusion that only an element that can independently reach Spec,XP is an accessible focus for a middle-field topic adjoined to XP. In order to account for that restriction, I will assume a contextual approach to phasehood, in particular \citet{boskovic2014} system, where the highest projection in the extended domain of a lexical category is a phase. With the object shift projection XP extending and closing off the verbal domain, as I argue here, XP can be taken to be a phase under that approach to phasehood. Independent evidence for this claim comes from ellipsis.

As \possciteauthor{boskovic2014} \citeyear{boskovic2014} argues, only phases and complements of phases can undergo ellipsis. Assuming the structure in (\ref{lacerda18}) above, repeated below in (\ref{lacerda28}), the phasehood of XP thus predicts that both the phase XP and its phasal complement vP can be (independently) elided. We already saw in (\ref{lacerda25}B) above that vP can be elided (while sparing the vP-external shifted object in Spec,XP). Ellipsis of the phase XP itself can be seen in cases of V-stranding VP-ellipsis, as in (\ref{lacerda29}B). Note that the direct object \emph{cada livro} must be outside the vP (i.e., in Spec,XP) in order to bind into the adjunct, which shows that the entire XP is elided in (\ref{lacerda29}B) (not just the vP).

\begin{exe}
\ex \label{lacerda28}
[$_{\textnormal{TP}}$ \emph{subject verb} [$_{\textnormal{XP}}$ \emph{topic} [$_{\textnormal{XP}}$ \{DO\} [$_{\textnormal{vP}}$ [$_{\textnormal{VP}}$ \{DO\} IO] ] ] ] ] = (\ref{lacerda18})

\end{exe}

\begin{exe}
\ex \label{lacerda29}
\begin{xlist}
\exi{A:} \label{lacerda29A}
\gll O 	João 	comprou 	cada 	livro$_{\textnormal{i}}$ 	no 	seu$_{\textnormal{i}}$ 	lançamento.\\
the 	John 	bought 	each 	book 	in-the 	its 	launch\\
\glt‘John bought each book on its launch.’

\exi{B:} \label{lacerda29B}
\gll Eu 	também 	comprei 	<cada 	livro$_{\textnormal{i}}$ 	no 	seu$_{\textnormal{i}}$ 	lançamento>.\\
I 	also	bought	<each	book 	in-the	its	launch>\\
\glt‘I did too.’

\end{xlist}
\end{exe}

Having independently motivated the phasehood of the object shift projection XP, we can then return to the restriction observed in (\ref{lacerda26}--\ref{lacerda27}) above, namely that only the shifted object can be focalized in the presence of a middle-field topic. I propose that this restriction follows from the phase-based locality constraint in (\ref{lacerda30}).


\begin{exe}
\ex \label{lacerda30}
\emph{Middle-field topic-focus association}\\
A topic adjoined to XP must be associated with a focus in the same spell-out domain.
\end{exe}

With XP being a phase, X$^{0}$ triggers the spell-out of vP, as in (\ref{lacerda31}). Later in the derivation, when the higher spell-out domain including the topic is sent to the interfaces, only the shifted object in Spec,XP is still accessible as a focus, given the constraint in (\ref{lacerda30}) above.

\begin{exe}
\ex \label{lacerda31}
[$_{\textnormal{XP}}$ \emph{topic} [$_{\textnormal{XP}}$ \emph{focus-DO} [$_{\textnormal{X'}}$ X$^{0}$ \sout{[$_{\textnormal{vP}}$ [$_{\textnormal{VP}}$ IO] ]} ] ] ]
\end{exe}

I will leave the precise deduction of the constraint in (\ref{lacerda30}) open for the time being (but see \citealt{Lacerda2020b} for a proposal and relevant discussion). However, it is important to point out now that the ungrammaticality of the relevant example (\ref{lacerda26}B2) above cannot be reduced to a mere adjacency constraint. This is corroborated by (\ref{lacerda32}). First recall from (\ref{lacerda21}) above that middle-field topics may reiterate in Brazilian Portuguese. Now note in (\ref{lacerda32}B) that the discourse-given topic \emph{pro departamento} ‘to the department’ may intervene between the contrastive topic \emph{do Chomsky} ‘by Chomsky’ and its associated focus \emph{dez livros} ‘ten books’. Crucially, as is shown in (\ref{lacerda33}), the focalized direct object is still accessible to the contrastive topic. By being able to undergo object shift to Spec,XP, \emph{só dez livros} can be realized in the same spell-out domain as \emph{do Chomsky}, in the manner discussed above, and the sentence is therefore acceptable.

\begin{exe}
\ex \label{lacerda32}
\begin{xlist}
\exi{A:} \label{lacerda32A}
Quantos livros do Pinker a Maria doou pro departamento?\\
‘How many books by Pinker did Mary donate to the department?’

\exi{B:} \label{lacerda32B}
\gll \symbol{92}Ela 	doou\symbol{92}, 	/\emph{do} 	\emph{Chomsky}$_{\textnormal{CT}}$/, 	\symbol{92}pro 	departamento$_{\textnormal{GT}}$\symbol{92},	/\emph{só}	\emph{dez} 	\emph{livros}$_{\textnormal{F}}$/ 	(até 	agora).\\
she 	donated	of-the	Chomsky	to-the 	department	only 	ten 	books	(until	now)\\
\glt‘She donated only \emph{ten books}$_{\textnormal{F}}$ \emph{by Chomsky}$_{\textnormal{CT}}$ to the department$_{\textnormal{GT}}$ (so far).’
\end{xlist}

\ex \label{lacerda33}
[$_{\textnormal{XP}}$ \emph{do Chomsky$_{\textnormal{CT}}$} [$_{\textnormal{XP}}$ pro departamento$_{\textnormal{GT}}$ [$_{\textnormal{XP}}$ \emph{só dez livros$_{\textnormal{F}}$} [$_{\textnormal{X'}}$\,[$_{\textnormal{vP}}$\,]\,]\,]\,]\,]
\end{exe}

Finally, recall from examples (\ref{lacerda15}) and (\ref{lacerda16}) above that in the absence of a direct object, an oblique argument can reach the object shift position (that is, object shift exhibits superiority effects). Unsurprisingly under the current analysis, in the absence of a direct object an oblique argument can be the focus associated with a middle-field topic, as in (\ref{lacerda34}B), which sharply contrasts with sentences (\ref{lacerda26}B2--B3) above, where an indirect object cannot be focalized in the presence of a direct object.

\begin{exe}
\ex \label{lacerda34}
\begin{xlist}
\exi{A:} \label{lacerda34A}
Em quantos alvos os atletas atiraram no campeonato de tiro?\\
‘How many targets did the athletes shoot at in the shooting championship?’	
\exi{B:} \label{lacerda34B}
\gll Bem,	os 	atletas 	atiraram, 	\emph{na} 	\emph{prova} 	\emph{final}$_{\textnormal{CT}}$, 	\emph{só} 	\emph{em} 	\emph{dois} 	\emph{alvos}$_{\textnormal{F}}$ 	cada 	um.\\
well 	the 	athletes 	shot 	in-the 	round 	final 	only 	in 	two 	targets	each	one\\
\glt ‘Well, the athletes each shot \emph{at only two targets$_{\textnormal{F}}$ in the final round}$_{\textnormal{CT}}$.’
\end{xlist}

\end{exe}

In conclusion, the discussion above regarding the constraints on middle-field topicalization provides further evidence for the structural make-up of the extended verbal domain of Brazilian Portuguese proposed in this paper. Assuming that object shift targets an independent projection XP above vP and that XP is a phase, it follows that only shifted objects can escape the spell-out of vP, which in turn grants shifted objects the ability to become an accessible focus for middle-field topics adjoined to XP, considering the requirement that middle-field topics and their associated foci must be in the same spell-out domain (as stated in (\ref{lacerda30}) above).

In the next section, I will discuss how the structural height of the middle field of Brazilian Portuguese constrains the availability of aboutness topics in that area of the clause.

\section{The height of aboutness topics}
To conclude this paper, I will argue that the structural height of the middle field of Brazilian Portuguese is responsible for preventing topics in that area of the clause from having an aboutness interpretation (in the sense of \citealt{Reinhart1981}). The middle-field topics that appeared in the relevant examples in the previous section were restricted to contrastive and discourse-given interpretation. I will now briefly compare Brazilian Portuguese middle-field topics and German \emph{Mittelfeld} topics and I will point out what is responsible for allowing an aboutness interpretation for the latter but not for the former.

That aboutness topic interpretation is ruled out in the middle field of Brazilian Portuguese is shown in (\ref{lacerda35}). While the traditional “tell me about X” test, which follows \citet{Reinhart1981}'s notion of aboutness, is felicitous with the left-peripheral topic in (\ref{lacerda35}B1), it leads to an infelicitous result with the middle-field topic in (\ref{lacerda35}B2).

\begin{exe}
\ex \label{lacerda35}
\begin{xlist}
\exi{A:} \label{lacerda35A}
Me conta alguma coisa sobre a feira renascentista que você foi ontem!\\
‘Tell me something about the renaissance fair you went to yesterday!’

\exi{B1:} \label{lacerda35B1}
\gll \emph{(N)a} 	\emph{feira} 	\emph{renascentista}$_{\textnormal{AT}}$, 	eu 	comi 	várias 	comidas 	típicas 	(lá).\\
(in-)the 	fair 	renaissance 	I 	ate 	several 	foods 	typical	(there)\\

\exi{B2:} \label{lacerda35B2}
\gll \# Eu 	comi, 	\emph{na} 	\emph{feira} 	\emph{renascentista}$_{AT}$, 	várias 	comidas 	típicas	(lá).\\
{} I 	ate	in-the 	fair 	renaissance 	several	foods 	typical	(there)\\
\glt‘\emph{At the renaissance fair$_{\textnormal{AT}}$}, I ate several traditional dishes.’
\end{xlist}
\end{exe}

Additionally, note that the use of an aboutness-shifting strategy (in the sense of \citealt{BianchiFrascarelli2010}), as overtly indicated by the topic-shifting particle \emph{já} in (\ref{lacerda36}), is only allowed with a left-peripheral topic, as in (\ref{lacerda36}B1), with the middle-field counterpart in (\ref{lacerda36}B2) being ruled out as ungrammatical.

\begin{exe}
\ex \label{lacerda36}
\begin{xlist}
\exi{A:} \label{lacerda36A}
O Pedro leu dez livros do Chomsky pra esse curso.\\
‘Peter read ten books by Chomsky for this course.’
\exi{B1:} \label{lacerda36B1}
\gll \emph{Já} 	\emph{do} 	\emph{Pinker}$_{\textnormal{AT}}$, 	ele 	não 	leu 	nenhum 	livro.\\
\textsc{já} 	of-the 	Pinker 	he 	not 	read 	no 	book\\

\exi{B2:} \label{lacerda36B2}
\gll * Ele 	não 	leu, 	\emph{já} 	\emph{do} 	\emph{Pinker}$_{\textnormal{AT}}$, 	nenhum 	livro.\\
 {} he 	not 	read 	\textsc{já} 	of-the 	Pinker 	no 	book\\
\glt‘Now \emph{by Pinker}$_{\textnormal{AT}}$, he didn’t read any book.’
\end{xlist}
\end{exe}

\begin{sloppypar}
We cannot ascribe the unavailability of aboutness topic interpretation in (\ref{lacerda35}B2) and (\ref{lacerda36}B2) above to the mere fact that the relevant topics are sentence-internal (i.e., not left-peripheral), in light of the fact that elements in the so-called German \emph{Mittelfeld} can be interpreted as aboutness topics. As \citet{Frey2004} notes, elements that precede sentential adverbs, such as \emph{wahrscheinlich} ‘probably’ in (\ref{lacerda37}), are felicitous in the context of the “tell me about X” test mentioned above, as in (\ref{lacerda37}B1), while elements that follow sentential adverbs are not, as in (\ref{lacerda37}B2).
\end{sloppypar}

\begin{exe}
\ex \label{lacerda37}
\begin{xlist}
\exi{A:} \label{lacerda37A}
\gll Ich 	erzähle 	dir 	etwas 	über 	Maria.\\
I 	tell 	you 	something 	about 	Mary\\
\exi{B1:} \label{lacerda37B1}
\gll Nächstes 	Jahr 	wird 	\emph{Maria} 	wahrscheinlich 	nach 	London 	gehen.\\
next 	year 	will 	Mary 	probably 	to 	London 	go\\

\exi{B2:} \label{lacerda37B2}
\gll \# Nächstes 	Jahr 	wird 	wahrscheinlich 	\emph{Maria} 	nach 	London 	gehen.\\
{} next 	year 	will 	probably 	Mary 	to 	London 	go\\
\glt‘Next year Mary will probably go to London.’\\
\citep[158]{Frey2004}
\end{xlist}
\end{exe}

\citet{Frey2004} describes sentence (\ref{lacerda37}B1) above as representing a topic-comment structure about Maria, in that “[t]he given context demands that the information of the following sentence should be stored under the entry Maria” \citep[158]{Frey2004}. Crucially, this informational import is possible when the topic in question precedes sentential adverbs, but not when it follows them. The possibility of \emph{Mittelfeld} topics having an aboutness interpretation can therefore be argued to follow from their privileged position at the edge of TP, where they can take scope over a full proposition (which is necessary for a felicitous topic-comment configuration to obtain; see e.g. \citealt{Reinhart1981}, \citealt{BianchiFrascarelli2010}). As \citet{Frey2003, Frey2004} notes, sentential adverbs delimit propositional content (i.e., like aboutness topics, sentential adverbs must also take scope over a full proposition). As such, German \emph{Mittelfeld} topics allow for the presence of aboutness topic-related particles, such as \emph{jedenfalls} ‘at any rate’ in (\ref{lacerda38}), which is licensed when the topic precedes the sentential adverb \emph{zum Glück} ‘luckily’ in (\ref{lacerda38}a), but not when the topic follows the adverb in (\ref{lacerda38}b).

\begin{exe}
\ex \label{lacerda38}
\begin{xlist}
\ex[ ]{\label{lacerda38a}
\gll weil 	[Peter 	jedenfalls] 	zum-Glück 	morgen 	mithelfen 	wird.\\
since 	Peter 	at-any-rate 	luckily 	tomorrow 	help 	will\\}

\ex[*]{ \label{lacerda38b}
\gll weil 	zum-Glück 	[Peter 	jedenfalls] morgen 	mithelfen 	wird.\\
since 	luckily 	Peter 	at-any-rate	tomorrow 	help 	will\\
\glt $\lbrack$‘Since Peter at any rate will luckily help tomorrow.’$\rbrack$\\}
\citep[162]{Frey2004}
\end{xlist}
\end{exe}

Brazilian Portuguese middle-field topics, on the other hand, cannot precede sentential adverbs, as is shown in (\ref{lacerda39}B1--B2). In fact, as the contrast between (\ref{lacerda40}a) and (\ref{lacerda40}b) shows, middle-field topics must be as low as following the lexical verb, which is assumed to move to a  position in the (low) TP area (see e.g. \citealt{TescariNeto2013}) -- attempting to place a topic in any position of the auxiliary system leads to ungrammaticality, as is shown in \REF{lacerda41}.\footnote{With middle-field topics in Brazilian Portuguese being located in a vP-external position closing off the verbal domain, as I argued, the data in (\ref{lacerda39}--\ref{lacerda41}) provide further evidence that all verbs in the language, inflected or not, must move to the TP area of the clause.}

\begin{exe}
\ex \label{lacerda39}
\begin{xlist}
\exi{A:} \label{lacerda39A}
Quantos livros do Chomsky o João leu pro curso de sintaxe?\\
‘How many books by Chomsky did John read for the syntax course?’
\exi{B1:} \label{lacerda39B1}
\gll ?* Ele 	sem dúvida, 	\emph{do} 	\emph{Chomsky}$_{\textnormal{TOP}}$, 	infelizmente 	não 	leu 	\emph{nenhum} 	\emph{livro}$_{\textnormal{F}}$.\\
{} he	w/o 	doubt 	of-the	Chomsky 	unfortunately	not 	read 	no 	book\\
\exi{B2:} \label{lacerda39B2}
\gll ?* Ele, 	\emph{do} 	\emph{Chomsky}$_{\textnormal{TOP}}$, 	sem 	dúvida 	infelizmente 	não 	leu 	\emph{nenhum} 	\emph{livro}$_{\textnormal{F}}$.\\
{} he	of-the 	Chomsky 	w/o 	doubt 	unfortunately not 	read 	no 	book\\
\glt ‘He undoubtfully unfortunately did not read any \emph{book$_{\textnormal{F}}$ by Chomsky$_{\textnormal{TOP}}$.}’

\end{xlist}

\ex \label{lacerda40}
\begin{xlist}
\ex[]{ \label{lacerda40a}
\gll O 	João 	não 	[$_{\textnormal{TP}}$ 	leu, [$_{\textnormal{XP}}$ 	\emph{do} 	\emph{Chomsky}$_{\textnormal{TOP}}$, [$_{\textnormal{XP}}$ 	só 	dois \hspace{1em}	livros$_{\textnormal{F}}$ 	t$_{\textnormal{TOP}}$ ] ] ].\\
the 	John 	not	{}	read {}		of-the	Chomsky	{}	only	two {} books\\}
\ex[*]{ \label{lacerda40b}
\gll O 	João 	não, \emph{do} \emph{Chomsky}$_{\textnormal{TOP}}$, 	[$_{\textnormal{TP}}$ 	leu 	[$_{\textnormal{XP}}$ só 	dois 	\hspace{2em} livros$_{\textnormal{F}}$ 	t$_{\textnormal{TOP}}$ ] ].\\
the 	John	not	of-the	Chomsky	{}	read	{}	only	two 	{} books\\
\glt `John didn’t read only two books$_{\textnormal{F}}$ \emph{by Chomsky$_{\textnormal{TOP}}$}.’}

\end{xlist}

\ex \label{lacerda41}
\begin{xlist}\sloppy
\exi{A:} \label{lacerda41A}
Quantos livros o João vai estar lendo pro curso de linguística?\\
‘How many books is John going to be reading for the linguistics course?’
\exi{B1:} \label{lacerda41B1}
\gll O 	João 	vai 	estar 	lendo, 	\emph{do} 	\emph{Chomsky}$_{\textnormal{TOP}}$, 	só 	dois 	livros$_{\textnormal{F}}$.\\
the 	John 	will 	be 	reading 	of-the 	Chomsky 	only 	two 	books\\
\exi{B2:} \label{lacerda41B2}
\gll * O 	João 	vai 	estar, 	\emph{do} 	\emph{Chomsky}$_{\textnormal{TOP}}$, 	lendo 	só 	dois 	livros$_{\textnormal{F}}$.\\
{} the 	John 	will 	be 	of-the 	Chomsky 	reading 	only 	two 	books\\
\exi{B3:} \label{lacerda41B3}
\gll * O 	João 	vai, 	\emph{do} 	\emph{Chomsky}$_{\textnormal{TOP}}$, 	estar 	lendo 	só 	dois 	livros$_{\textnormal{F}}$.\\
{} the 	John 	will 	of-the 	Chomsky 	be 	reading 	only 	two 	books\\
\glt‘John will be reading only two books$_{\textnormal{F}}$ \emph{by Chomsky}.’
\end{xlist}


\end{exe}

In conclusion, Brazilian Portuguese middle-field topics are in too low a position to have an aboutness interpretation. Located at the edge of the verbal domain, as I argued, middle-field topics in this language do not c-command a full proposition and thus cannot create the topic-comment articulation of the clause that traditional aboutness topics must conform to, with contrastive and discourse-given interpretation, which can be argued not to depend on a topic-comment articulation (see \citealt{Lacerda2020b} for relevant discussion), being in principle available.

Furthermore, the contrasts between German \emph{Mittelfeld} topics and Brazilian Portuguese middle-field topics discussed in this section can be taken to provide evidence for the view that the availability of different topic types is a matter of structural height.

\section{Final remarks}
In this paper, I analyzed two distinct operations in Brazilian Portuguese, namely “object shift” and “middle-field topicalization”, which place elements in postverbal vP-external positions. The analysis of these two operations allowed us to probe into the “size” of the extended verbal domain in the language, which was argued to include an independent vP-external functional projection XP, whose A-specifier hosts shifted objects and to which middle-field topics adjoin. The close-knit relationship between middle-field topics and shifted objects (as far as information-structural relations are concerned) allowed us to additionally determine the phasehood of XP, thus delimiting the extended verbal domain of Brazilian Portuguese as a phasal domain.

Compared with the German \emph{Mittelfeld}, I argued that the middle field of Brazilian Portuguese is structurally too low to allow for aboutness topics -- with sentential adverbs delimiting propositional content, the position of topics with respect to sentential adverbs was thus shown to be a safe diagnostic for the availability of aboutness interpretation. Following all elements of the TP area, middle-field topics and shifted objects were shown to be part of the extended verbal domain of Brazilian Portuguese.

\begin{comment}
\newpage
\section*{Abbreviations}
\begin{tabularx}{.45\textwidth}{lQ}
... & \\
... & \\
\end{tabularx}
\begin{tabularx}{.45\textwidth}{lQ}
... & \\
... & \\
\end{tabularx}
\end{comment}

\section*{Acknowledgements}
I thank Željko Bošković, Jairo Nunes, and Susi Wurmbrand for their continuous support and guidance. 
I would also like to thank the editors and the anonymous reviewers for their contributions to this chapter.

\printbibliography[heading=subbibliography,notkeyword=this]

\end{document}
