\documentclass[output=paper]{langscibook}
\author{Neda Todorović\affiliation{University of British Columbia}}

\title{Future interpretation in Gitksan and reduced clausal complements}

\abstract{This paper explores temporal interpretations in clausal complements in Gitksan, a language without temporal morphology. Bare predicates in Gitksan can receive present or past reading. \citet{johannsdottirmatthewson2007a} capture these readings with a covert non-future tense. For future reading, bare predicates must combine with a marker \emph{dim}; in syntax, \emph{dim} combines with the non-future tense. In this paper, I focus on the connection between the syntactic make-up of Gitksan complements and the availability of future-oriented reading. Assuming the non-future tense in Gitksan, I show that the attested readings can only be captured if some of the complements project TPs, while the others do not. I propose that the observed patterns follow straightforwardly from \possciteauthor{wurmbrand2001a} (\citeyear{wurmbrand2001a} et seq.) idea that clausal complements are of different sizes: some complements are CPs, but some can project as little as vPs. Gitksan provides support for this approach through the syntax-semantics interaction in the embedded temporal-modal domain.}

\begin{document}
\SetupAffiliations{mark style=none}
\newcommand{\tsc}{\textsc}
\newcommand{\CP}{\textsc{cp}}
\newcommand{\tp}{\textsc{tp}}
\newcommand{\aspp}{\textsc{aspp}}
\newcommand{\vp}{\textsc{vp}}
\newcommand{\MOD}{\textsc{mod}}
\newcommand{\woll}{\textsc{woll}}
\newcommand{\dn}{\textsc{dn}}
\newcommand{\cn}{\textsc{cn}}
\newcommand{\pn}{\textsc{pn}}
\newcommand{\seriesI}{\textsc{i}}
\newcommand{\seriesII}{\textsc{ii}}
\newcommand{\seriesIII}{\textsc{iii}}
\newcommand{\irr}{\textsc{irr}}
\newcommand{\glossNeg}{\textsc{neg}}
\newcommand{\foc}{\textsc{foc}}
\newcommand{\prog}{\textsc{prog}}
\newcommand{\sg}{\textsc{sg}}
\newcommand{\tr}{\textsc{tr}}
\newcommand{\sx}{\textsc{sx}}
\newcommand{\ax}{\textsc{ax}}
\newcommand{\pl}{\textsc{pl}}
\newcommand{\indp}{\textsc{indp}}
\newcommand{\comp}{\textsc{comp}}
\newcommand{\pass}{\textsc{pass}}
\maketitle

\section{Introduction}\label{Todoro:sect1}\largerpage[-1]
One of the ways to classify clausal complements is according to the temporal relation between the embedded and the matrix predicate. 
In English, for example, \citet{wurmbrand2014a} argues for a three-way division of  infinitival complements, based on the available temporal readings and the aspect of the embedded eventives. In complements of, e.g. \emph{believe}, the event time (ET) of the embedded predicate is simultaneous with the ET of the matrix predicate, as in (\ref{Todoro1a}); these are propositional complements. The embedded predicate can also receive backward-shifted interpretation.\footnote{Infinitives cannot receive future orientation, but their finite counterparts can: \ea Leo claims that he will be eating.\z} In complements of, e.g. \emph{want}, the ET of the embedded predicate is necessarily future-oriented with the respect to the ET of the matrix predicate, as in (\ref{Todoro1b}); these are future-irrealis complements. In complements of e.g. \emph{try}, the ET of the embedded predicate is necessarily simultaneous with the ET of the matrix predicate, as in (\ref{Todoro1c}); given the temporal dependency of the embedded predicate, these are tenseless complements. Propositional and tenseless complements are thus on the opposite side of the temporal dependency spectrum. 

\begin{exe}
\ex \label{Todoro1}
\begin{xlist}
\ex \label{Todoro1a}
Propositional simultaneous\\
    Leo \textit{claimed/believed} [to be eating (*tomorrow)].\\
    Leo \textit{claimed} [to have eaten (yesterday)].
\ex \label{Todoro1b}
Future-irrealis\\
Leo \textit{decided/planned/planned/wanted} [to eat	(tomorrow)].
\ex \label{Todoro1c}
Tenseless\\
    Leo \textit{tried/began/managed/forgot} [to eat (*tomorrow)].	\\
    \hfill \citep{wurmbrand2014a}
\end{xlist}
\end{exe}\largerpage[2]

\begin{sloppypar}
The semantic division also receives syntactic support. \citeauthor{wurmbrand2001a} (\citeyear{wurmbrand2001a} et seq.) observes that, cross-linguistically, tenseless complements are most transparent for cross-clausal syntactic operations (e.g. clitic climbing, long object movement, NPI-licensing). Future-irrealis complements, under restructuring, are somewhat less transparent, and propositional complements are most opaque. \citeauthor{wurmbrand2001a} argues that the observed syntactic and semantic differences can systematically be captured if these complements are of different sizes, i.e. not necessarily CPs. I refer to this proposal as \emph{Different Complement Sizes Hypothesis} (DCSH). Propositional simultaneous complements are CPs, future-irrealis complements, under restructuring, are TPs or ModPs, and tenseless infinitives (typically restructuring) are vPs or even VPs, as in (\ref{Todoro2}). The ``porous'' structure of future-irrealis and tenseless complements captures both their transparency and dependency on the matrix domain.
\end{sloppypar}

\begin{exe}
\ex \label{Todoro2}
\begin{xlist}
\ex \label{Todoro2a}
Propositional simultaneous: CPs \\
    V \emph{claim} [{\CP}[{\tp}[{\aspp}[{\textit{v}P}...]]]]
\ex \label{Todoro2b}
Future-irrealis: TP/ModP \\
V \emph{want} [\tp/\textsc{modp} \woll [\aspp[\textit{v}P...]]]
\ex \label{Todoro2c}
Tenseless \textit{v}P/VP \\
V \emph{try} [\textit{v}P/\vp]
\end{xlist}
\end{exe}

Importantly, DCSH is not confined to non-finite embedded domains. \citet{todoroviwurmbrand2020b} show that, in Serbian, clausal complements of verbs in (\ref{Todoro1}) are finite, yet they are strikingly syntactically and semantically similar to their non-finite counterparts in English (e.g. temporal interpretation), Czech (e.g. clitic climbing) and German (e.g. long object movement). They argue that the observed phenomena are captured if syntax of these complements is similar to those in English (\ref{Todoro2}); that is, if finite complements do not necessarily project a CP (\citealt{adger2007a}, \citealt{kornfilt2007a}; cf.  \citealt{rizzi1997a}).

Rather than finiteness, \citealt{wurmbrand2019a} argue for the syntax-semantics connection along the lines of implicational complementation hierarchy -- event complements (including tenseless) are the most transparent, least clausal (and thus the smallest), and most dependent on the matrix verb; they are on one end of the scale.  Situation complements (including future-irrealis) can only be less transparent, more clausal (thus bigger) and less dependent on the matrix verb, while the propositional complements, on the opposite end of the scale, are least transparent, most clausal (project most structure), and least dependent on the matrix verb.\footnote{The terminology is from \citet{ramchandsvenonius2014a}.} This is confirmed for Slavic (see also \citealt{wurmbrandetal2020}), Romance, Cypriot Greek, Scandinavian (see also \citealt{wurmbradnchristos2020}), Buruyat (see also \citealt{bondarenko2018a}), Akan, Japanese and Austronesian languages (\citealt{wurmbrand2017a}). 

In this paper, I test the implicational complementation hierarchy with respect to the DCSH and its effects on temporal interpretation in Gitksan. I focus on future-oriented reading. While Gitksan lacks present or past morphology, it conveniently marks future in all future-oriented contexts. Building on work by \citet{matthewson2018a}, I show that the availability of future-oriented reading follows from the DCSH: propositional complements are CPs, future-irrealis complements are ModPs, and tenseless complements are vPs (complements of \emph{si’ix} ‘try’) or ModPs (complements of \emph{bak̲} ‘try’).Gitksan data provide preliminary support for the implicational complementation hierarchy.

The paper is organized as follows: \sectref{Todoro:sect2} provides an overview of Gitksan morphosyntax and its temporal system. \sectref{Todoro:sect3} discusses future reading in embedded domains. \sectref{Todoro:sect4} discusses the connection between syntax of propositional and future-irrealis complements and their attested readings. \sectref{Todoro:sect5} discusses two types of ‘try’s, \sectref{Todoro:sect6} concludes the paper. 

\section{Gitksan}\label{Todoro:sect2}

Gitksan is a Tsimshianic language spoken in Northwest Interior of British Columbia, Canada. It is a member of the Interior branch of the family and it forms a dialect continuum. With 531 fluent speakers, it is ``threatened'' (\citealt{dunlop2018a}).\footnote{\url{https://www.ethnologue.com/language/git}}

All the uncited data in this paper were collected during fieldwork. I conducted elicitations of Gitksan utterances with Vincent Gogag (from Gitanyaaw (Kitwancool)), Barbara Sennott (from Ansba’yaxw (Kispiox)), Hector Hill, Ray Jones and Barry Sampere (from Gijigyukwhla (Gitsegukla)). The data were collected following the methodology for semantics fieldwork in \citet{matthewson2004a} and \citet{burton2015a}: the elicitations were conducted in controlled discourse contexts, via translation tasks, acceptability judgment tasks, and storyboard tasks.

\subsection{A brief overview of Gitksan morphosyntax}

Gitksan is a predicate-initial language, with a VSO order. This is shown in (\ref{Todoro3}) for root and in (\ref{Todoro4}) for embedded clauses.

\begin{exe}
\ex \label{Todoro3}
\begin{xlist}

\ex \label{Todoro3a} 
\gll {Bax̱=hl} {hanak̲'.}\\
    run={\cn} woman\\
\glt `The woman ran.’ \citep[159]{davis2015a}\\

\ex \label{Todoro3b}
\gll {Gup-i-t=s} {Mary=hl} {hun.}\\
    eat-{\tr}-3.{\seriesIII}={\dn} Mary={\cn} fish\\
\glt `Mary ate the fish.’ \citep[8]{forbes2019a} \\

\end{xlist}

\ex \label{Todoro4}
\gll {Ha-'nii-goot=s} {James} [{ji=t} {gup=s} {Tyler=hl} {anaax.}] \\
{\tsc{ins}}-on-heart-3,{\seriesIII}={\dn} James {\irr}=3.{\seriesI} ear-3.{\seriesII}={\dn} Tyler={\cn} bread \\
\glt `James thinks that Tyler ate the bread.' \\ (Lit. `James' on-heart is that Tyler ate the bread.') (\citealt[57]{davis2011a})
\end{exe}

Agreement in Gitksan is quite complex (see \citealt{davis2011a}, \citealt{davis2015a}, \citealt{forbes2017a}, \citeyear{forbes2018a}, \citeyear{forbes2019a}) and it can be divided into three series (\citealt{rigsby1986a}), as in (\ref{Todoro5}). Gitksan follows the ergative/absolutive split.\largerpage


\begin{exe}
\ex \label{Todoro5}
\begin{xlist}
\ex \label{Todoro5a}
series {\seriesI}: a pre-predicative clitic 
\ex \label{Todoro5b}
series {\seriesII}: post-predicative affix 
\ex \label{Todoro5c}
series {\seriesIII}: post-predicative independent word \\ 
(\citealt{davis2015a}: 157)
\end{xlist}
\end{exe}

The distribution of agreement indicates if the clause is dependent or independent (\citealt{rigsby1986a}). Dependent clauses are introduced by complementizers, subordinating verbs, clausal coordinator \textit{ii}, negation, irrealis, imperatives, and aspectual markers, as in (\ref{Todoro6}). \citet{forbes2019a} observes that in dependent clauses, absolutive argument occurs with a suffix (series {\seriesII} agreement), -‘\emph{y} in (\ref{Todoro7a}) and (\ref{Todoro7b}), while the transitive subject occurs with a pre-predicative clitic (series {\seriesI} agreement),  \emph{t} in (\ref{Todoro7b}). 


\begin{exe}
\ex \label{Todoro6}
\gll {Yukw=hl} {bax̱=s} {Cheyenne.} \\
{\prog}={\cn} run={\pn} Cheyenne \\
\glt `Cheyenne is running.' (\citealt{schwan2019a}: 8)

\ex \label{Todoro7}
\begin{xlist}

\ex \label{Todoro7a} 
\gll {Nee=dii} {bas-\textit{'y}.}\\
    {\glossNeg}={\foc} run-\textit{1{\sg}.{\seriesII}}\\
\glt `I didn't run.’  \citep[65]{forbes2019a}\\

\ex \label{Todoro7b}
\gll {Nee=dii=\textit{t}} hilen-\textit{'y}.\\
    {\glossNeg}={\foc}=\textit{3.{\seriesI}} chase-\textit{1{\sg}.{\seriesII}} \\
\glt `She didn't chase me.’ (\citealt{forbes2019a}: 65) \\

\end{xlist}
\end{exe}

In independent clauses, absolutive argument occurs with a pronoun (series {\seriesIII}), \emph{nii’y} in (\ref{Todoro8a}) and (\ref{Todoro8b}), and transitive subject occurs with a suffix (series {\seriesII}), \emph{-t} in (\ref{Todoro8b}).

\begin{exe}
\ex \label{Todoro8}
\begin{xlist}

\ex \label{Todoro8a} 
\gll {Bax̱} {\textit{'nii'y}}.\\
   run \textit{1{\sg}.{\seriesIII}}\\
\glt `I ran.’  \citep[65]{forbes2019a}\\

\ex \label{Todoro8b}
\gll {Hilen-i-\textit{t}} {\textit{'nii'y}}.\\
    chase-{\tr}-\textit{3.{\seriesII}}  \textit{1{\sg}.{\seriesIII}} \\
\glt `She chased me.’ (\citealt{forbes2019a}: 65) \\

\end{xlist}
\end{exe}

\citet{forbes2019a} also observes that agreement morphology changes in A’-ex\-trac\-tion. Subject and object extraction create a post-predicative affix, -\textit{it} in (\ref{Todoro9b}) and \emph{-yi} in (\ref{Todoro10b}). There is a common noun determiner \emph{-hl} on the wh-phrase. Object extraction also creates a suffixal ergative agreement, \emph{-n} in (\ref{Todoro10b}), which occurs in independent clauses (series {\seriesII}). Transitive subject extraction surfaces with a pre-predicative \textit{an} (\ref{Todoro11b}) and there is no determiner. There is also an ergative clitic agreement \emph{t}, which occurs in dependent clauses (series {\seriesI}). 

\begin{exe}
\ex \label{Todoro9}
\begin{xlist}

\ex \label{Todoro9a}
\gll {Limx} {'nit.} \\
    sing 3.{\seriesIII} \\
\glt `He's singing.' (\citealt{forbes2017a}: 2)

\ex \label{Todoro9b}
\gll {Naa=\textit{hl}} {lim=\textit{it}} \uline{~~~}? \\
    who={\cn} sing={\sx} \\
\glt `Who sang?' (\citealt{rigsby1986a}: 303)

\end{xlist}

\ex \label{Todoro10}
\begin{xlist}

\ex \label{Todoro10a}
\gll {Hilmoo-\textit{yi}-'y=t} Mary. \\
    help-{\tr}-1.{\sg}.{\seriesII}={\pn} Mary \\
\glt `I helped Mary.'  
\ex \label{Todoro10b}
\gll {Naa=\textit{hl}} {hlimoo-\textit{yi-n}} \uline{~~~}?\\
    who={\cn} help-{\tr}-2{\sg}.{\seriesII} \\
\glt `Who did you help?' (\citealt{rigsby1986a}: 303) 

\end{xlist}

\ex \label{Todoro11}
\begin{xlist}

\ex \label{Todoro11a}
\gll {Gub-i=s} {Jeremy=hl} {hon-n.} \\
    eat-{\tr}={\cn} Jeremy={\cn} fish-2{\sg}.{\seriesII}\\
\glt `Jeremy ate your fish.' (\citealt{forbes2017a}: 3)

\ex \label{Todoro11b}
\gll {Naa} {\textit{an=t}} {gup(\uline{~~~})=hl} {susiit}? \\
    who {\ax}=3.{\seriesI} eat(\uline{~~~})={\cn} potatoes\\
\glt `Who ate the potatoes?' (\citealt{davis2011a}: 50) 

\end{xlist}
\end{exe}

Finally, note that predicates can also be preceded by one or more ``preverbals'' that often convey adverbial notions, as in (\ref{Todoro12a}), or other pre-predicative operators, one of which is the future marker \emph{dim}, as in (\ref{Todoro12b}).

\begin{exe}
\ex \label{Todoro12}
\begin{xlist}

\ex \label{Todoro12a}
\gll {\textit{Luu}} {\textit{sga}} {het-xw} {`nii'y.}\\
    in blocking stand-{\pass} 1{\sg}.{\seriesIII}\\
\glt `I stook in, blocking the way.' (\citealt{rigsby1986a}) 

\ex \label{Todoro12b}
\gll {\textit{Dim}} amksiwaa-max-da. \\
    \textsc{\textit{fut}} white.person-language-3{\pl}.\textsc{indp}\\
\glt `They'll speak English.' (\citealt{rigsby1986a}) 

\end{xlist}
\end{exe}

\subsection{Temporal system of Gitksan}

Gitksan typologically patterns with a number of languages in the Northwestern North America in lacking temporal morphology. First analysis of the Gitksan temporal system was offered by \citeauthor{johannsdottirmatthewson2007a} (\citeyear{johannsdottirmatthewson2007a}; J\&M; see also \citealt{matthewson2013a}). They show that a bare predicate in root clauses is ambiguous between present and past reading, as in (\ref{Todoro13}). Temporal adverbials can disambiguate between them, but not license them. Crucially, a bare predicate cannot receive future reading even with a future adverbial, as in (\ref{Todoro14a}), but it requires a futurity marker \emph{dim}, as in (\ref{Todoro14b}). 

\begin{exe}
\ex \label{Todoro13}
\gll {Siipxw=t} {James} {(k'yoots).} \\
    sick={\pn} James (yesterday)\\
\glt `James is sick. / James was sick (yesterday). / *James will be sick.' \\ (\citealt{matthewson2013a}: 363)

\ex \label{Todoro14}
\begin{xlist}

\ex[*]{ \label{Todoro14a}
\gll {Yookw=t} {James} {ji} {taahlakxw}. \\
    eat={\cn} James \tsc{prep} tomorrow\\
\glt `James will eat tomorrow.' (\citealt{johannsdottirmatthewson2007a})}

\ex[]{ \label{Todoro14b}
\gll {\textit{Dim}} {yookw-t} {James} ({ji} {taahlakxw}). \\
    \tsc{prosp} eat-{\cn} James \tsc{prep} tomorrow\\
\glt `James will eat (tomorrow).' (\citealt{johannsdottirmatthewson2007a}) }

\end{xlist}
\end{exe}


J\&M primarily focus on temporal readings in root clauses. They posit a covert pronominal non-future tense in (\ref{Todoro15}) to capture the present and past reading of predicates; (\ref{Todoro15}) presupposes that the reference time (RT) is not after the UT. The UT is taken as the default RT in root clauses. 

\begin{exe}
\ex \label{Todoro15}
\textlbrackdbl\tsc{non-future}\textrbrackdbl$^{g,C}$ = {$\lambda$}t : t {\leq} t$_{C}$ . t
\end{exe}

J\&M analyze a pre-predicative marker \emph{dim} (\citealt{rigsby1986a}:304), as a prospective aspect in (\ref{Todoro16}). The non-future tense combines with \textit{dim} to derive future reading.

\begin{exe}
\ex \label{Todoro16}\textlbrackdbl\emph{dim}\textrbrackdbl$^{g,C}$ = {$\lambda$}P$_{\langle i,st\rangle}$. {$\lambda$}t.{$\lambda$}w. {$\exists$}t' [t<t’ \& P(t’)(w)]
\end{exe}

\emph{Dim} is analyzed as a prospective aspect and not as a modal because of its nature when it co-occurs with other modals. \citet{matthewson2013a} shows that with deontic modals, which are obligatorily future-oriented (\citealt{abusch2012a}, \citealt{thomas2014a}, \citealt{klecha2011a}, \citealt{chen2017a}, i.a.),  \emph{dim} is obligatory, as in (\ref{Todoro17}). But with epistemic modals, \emph{dim} occurs only if it contributes future-orientation, as in (\ref{Todoro18}). Crucially, \emph{dim} makes no modal contribution, but it only brings future orientation.

\begin{exe}
\ex \label{Todoro17}
\gll {Sgi} \#({dim}) ({ap}) {ha'w=s} {Lisa} \\
    \textsc{circ.necess} \#(\textsc{prosp}) (\tsc{}verum) go.home={\pn} Lisa\\
\glt `Lisa should go home.' (adapted from \citealt{matthewson2013a}: 380) 

\ex \label{Todoro18}
\begin{xlist}

\ex \label{Todoro18a} [{You can hear people hollering, so the Canucks might be winning.}] \\
\gll {Yugw=imaa=hl}	{xsdaa-diit}. \\
    \tsc{ipfv}=\tsc{epis}={\cn} win-3{\pl}.{\seriesII} \\
\glt `They might be winning.' 

\ex \label{Todoro18b} [{You are watching in the Canucks. They might win.}]\\
\gll {Yugw=imaa=hl} {dim} {xsdaa-diit}. \\
    \tsc{ipfv}=\tsc{epis}={\cn} \tsc{fut} win-3{\pl}.{\seriesII} \\
\glt `They might be winning.' (\citealt{matthewson2013a}: 374)

\end{xlist}
\end{exe}

\begin{sloppypar}
\citet{todorovietal2020a} propose that, despite not being a modal itself, the prospective \textit{dim} comes with a covert modal in root clauses (possibly only when there is no overt modal). This is motivated by the modal flavors it gets (e.g. in offers, warnings, see \citealt{copley2009a} et seq.; \citealt{klechaetal2008a}, \citealt{klecha2011a}). In this paper, I will treat \emph{dim} as a prospective aspect with a null modal, but nothing in the analysis hinges on it: what matters is its future-oriented contribution. 
\end{sloppypar}

This paper extends J\&M’s analysis to embedded clauses, by exploring the connection between future-oriented reading and the syntax of those complements. It also expands on the relations between the RT and the ET. J\&M show that the RT for the embedded event can be in the past. I further show that the embedded ET can be interpreted as ‘present’, i.e. simultaneous with this RT, or as ‘past’, i.e. back-shifted from it. In other words, I argue that the non-future tense in Gitksan is relative. Finally, this paper builds on \citet{matthewson2018a}’s discussion of future readings in clausal complements; it extends the empirical coverage and shows that, to capture all the readings, TP crucially must be absent from future-irrealis and tenseless complements. In \sectref{Todoro:sect3}, I start by discussing the distribution of \emph{dim} in embedded clauses in Gitksan. 

\section{Future in embedded clauses}\label{Todoro:sect3}
\begin{sloppypar}
One peculiarity of Gitksan is that futurity is overtly marked by \emph{dim} in all future-oriented contexts.\footnote{The data in this section are from \citet{matthewson2018a}.} \citealt{matthewson2018a} (M\&T) explore its distribution in complement clauses. They note that, if the DCSH is assumed, it makes clear predictions about the distribution of futurity marker in Gitksan: (a) if propositional simultaneous complements are CPs, they have space in syntax for \emph{dim}; these complements allow for simultaneous, back-shifted and forward-shifted readings of the embedded eventuality (see \sectref{Todoro:sect4.1}), so \emph{dim} should occur only if there is future-oriented interpretation. This is confirmed in (\ref{Todoro19}); (b) if future-irrealis complements are ModP or TP, they also have space in syntax for \emph{dim}, and, due to their obligatory future-orientation, \emph{dim} will always surface, as is the case in (\ref{Todoro20}); (c) if tenseless complements are VPs or vPs, they have no room for \emph{dim -- dim} never occurs in them, as confirmed in (\ref{Todoro21a}); the absence of temporal-modal domain in these complements explains their obligatorily simultaneous interpretation, as in (\ref{Todoro21b}).\footnote{At this point, I remain agnostic with respect to presence/absence of \tsc{aspp} in tenseless complements.}
\end{sloppypar}

\begin{exe}
\ex \label{Todoro19} [I’m looking for Colin. I ask you ``Where is Colin?'' You reply:] \\
\gll {Ha'niigood-i'y} 	[({dim})	{yukw=hl}	{bax̱-t}].  \\
    believe-1{\sg}.{\seriesII}	[(\tsc{prosp})	\tsc{prog}={\cn}	run-3.{\seriesII}]\\
\glt 	i. 	Without {\emph{dim}}:	‘I think he is running (now).’ \\
		ii. 	With {\emph{dim}}:	‘I think he will run.’  \\
		Consultant’s comment on (ii): ``He’s just about to/going to start''

\ex \label{Todoro20} [There’s a charity run next week. Will Colin run?] \\
\gll {Hasak̲-t}	[\#({dim})	{bax̱-t}].	 \\
    want-3.{\seriesII}	 \#(\tsc{prosp})	run-3.{\seriesII}\\
\glt `He wants to run.' 

\ex \label{Todoro21} [We are watching the race and I spot injured Colin trying to run, limping along. I tell you:] 
\begin{xlist}

\ex[]{ \label{Todoro21a}
\gll {Yukw}[={hl}]	{si'ix} 	[(\#{dim}) 	{bax̱-t}].	  \\
    \tsc{prog}[={\cn}]	try	 (\#\tsc{prosp})	run-3.{\seriesII}\\
\glt ‘He’s trying to run.’	\\             
		Consultant’s comment: “\emph{Si'ix} and {\emph{dim}} don’t go together.”}


\ex[*]{ \label{Todoro21b}
\gll {Gyoo’n}  {sik’ihl}    [{gup-d-i=hl} {hun}  {t’aahlakw}]. \\
    now       try        eat-\tsc{t-tr=cn}       salmon	tomorrow\\
\glt `He tried today to eat the salmon tomorrow.' }

\end{xlist}
\end{exe}

A note is in order regarding the future-oriented and tenseless complements. Given that \emph{dim} always occurs pre-predicatively in complements of ‘want’, M\&T argue that \emph{dim} is located in the embedded clause. This is different from future-oriented complements in languages like English, in which there is no overt futurity marker. M\&T take the overtness of \emph{dim} in these complements as an additional argument for the futurity stemming from the complements of verbs like `want', rather than the verb itself (in line with \citealt{abusch2004a}, \citealt{wurmbrand2014a}, \citealt{todoroviwurmbrand2020b}, pace \citealt{ogihara1996a}, \citealt{abusch1997a}, \citealt{pearson2017a}, i.a).	

Regarding ‘try’ in Gitksan, it is realized either as a pre-verbal element \emph{si’ix}, with which \emph{dim} never occurs, as in \REF{Todoro21}, or as a verb \emph{bak̲}, which obligatorily takes \emph{dim}. I return to this difference in \sectref{Todoro:sect5}. 
	
\begin{exe}
\ex \label{Todoro22} [We are watching the race and I spot injured Colin trying to run, limping along. I tell you:] \\
\gll {Bag-a-t}	[\#(\textit{{dim}}) 	{bax̱-t}].  \\
    {try}-\tsc{tr-3.ii} [\#(\tsc{prosp}) run-3.{\seriesII}]\\
\glt `He's trying to run.'
\end{exe}

Future reading in Gitksan complements provides preliminary support for the DCSH. To sketch a more precise picture of the syntax of these complements, in \sectref{Todoro:sect4}, I discuss all the possible interpretations of propositional and future-irrealis complements in Gitksan. 

\section{Non-future tense: Simultaneous or backward-shifted reading}\label{Todoro:sect4}

In root clauses in Gitksan, J\&M’s non-future tense accounts for the availability of the readings simultaneous (present) with or back-shifted (past) from the UT (\ref{Todoro13}). In this section, I show that non-future tense in Gitksan is relative, on the example of embedded clauses. Relative tense is predicted to make the embedded ET simultaneous or back-shifted from the RT established by the matrix predicate. This gives us the four settings in Table \ref{Todorotab1}. 

\begin{table}
\caption{Four readings\label{Todorotab1}}
 \begin{tabular}{ll}
  \lsptoprule
 matrix present &	embedded present (simultaneous) \\
matrix present &	embedded past (back-shifted) \\
matrix past	& embedded ‘present’ (simultaneous)\\
matrix past	& embedded ‘past’ (back-shifted)\\
  \lspbottomrule
 \end{tabular}
 \end{table}
 
 These readings are all attested in propositional complements, as shown in \sectref{Todoro:sect4.1}. In other words, the available interpretations support the presence of TP
 %Tense
  in these complements. And given that these clauses can be introduced with a complementizer \emph{wil}, I propose that they are CPs. Conversely, obligatory future-oriented reading of future-irrealis complements is only accounted for if there is no TP, as shown in \sectref{Todoro:sect4.2}.\largerpage[-1]
 
 \subsection{Propositional complements}\label{Todoro:sect4.1}
 
 Let us first consider bare predicates. As shown in (\ref{Todoro19}) and repeated in (\ref{Todoro23a}), propositional complements allow for the reading where the believing time and the running time coincide, both happening at the UT. This reading can be derived as in (\ref{Todoro23b}). Both tenses are interpreted as present: the matrix tense locates the ET at the UT, the lower tense introduces the RT for the embedded ET simultaneous with the believing time. 

\begin{exe}
\ex \label{Todoro23}
\begin{xlist}

\ex \label{Todoro23a} [I’m looking for Colin. I ask you ``Where is Colin?'' You reply:]
\gll {Ha'niigood-i'y} 	[{yukw=hl}		{bax̱-t}]. \\
    believe-1{\sg}.{\seriesII}	[\tsc{prog=cn}		run-3.{\seriesII}] \\
\glt ‘I believe he is running (now).’ 

\ex \label{Todoro23b}
\tsc{[tp \textit{pres} [aspp [vp [cp [tp \textit{pres} [aspp prog [vp]]]]]]]}

\end{xlist}
\end{exe}\largerpage[-1]

Another setting is this: matrix present -- embedded past, as in (\ref{Todoro24a}). Matrix tense locates the ET at the UT, and the embedded tense locates the embedded ET prior to the believing time, i.e. prior to the UT, as in (\ref{Todoro24b}).

\begin{exe}
\ex \label{Todoro24}
\begin{xlist}

\ex \label{Todoro24a} [There was a race yesterday. You saw Colin preparing for it in       
     front of the start line. But you left before the race began. I ask you   
    today: ``Did Colin run?'']\\
\gll {Ha'niigood-i'y} 	[{bax̱-t} {k'yoots}]. \\
    believe-1{\sg}.{\seriesII}	[run-3.{\seriesII} yesterday] \\
\glt ‘I believe he ran yesterday.’ 

\ex \label{Todoro24b}
\tsc{[tp \textit{pres} [aspp [vp [cp [tp \textit{past} [aspp [vp]]]]]]]}

\end{xlist}
\end{exe}

The next option is: matrix past -- embedded ‘present’, i.e. simultaneous interpretation, as in (\ref{Todoro25a}). Matrix past locates the saying ET in the past, while the embedded present sets the saying time as the RT for the embedded ET, as in  (\ref{Todoro25b}). 

\begin{exe}
\ex \label{Todoro25}
\begin{xlist}

\ex \label{Todoro25a} [I called Mary yesterday. I asked her about Susan’s health. Mary    
    told me: ``Susan’s feeling tired.'' Today, I called Susan’s sister and   
     told her:]\\
\gll {Mehl-d-i=s} 	  {Mary}  {loo-’y}  {ky'oots}        [{win} {hlebiksxw=s} 	     {Susan}]. \\
    tell-\tsc{t-tr=pn} Mary  \tsc{obl-1sg.ii} yesterday  {\tsc{comp}} tired=\tsc{pn}	     Susan \\
\glt ‘Mary said yesterday that Susan was tired.’ 

\ex \label{Todoro25b}
\tsc{[tp \textit{past} [aspp [vp [cp [tp \textit{pres} [aspp [vp]]]]]]]}

\end{xlist}
\end{exe}

The last option is: matrix past -- embedded ‘past’; the latter shifts the RT for the embedded ET back from the saying time, as in (\ref{Todoro26}). 

\begin{exe}
\ex \label{Todoro26}
\begin{xlist}

\ex \label{Todoro26a} [I called Mary yesterday. I asked her about Susan’s health. She   
     said: ``Susan was feeling tired on Sunday.'' Today I call Susan’s 
     sister and tell her:]\\
\gll {Mehl-d-i=s} {Mary}  {loo-’y} {ky'oots} [{win} {hlebiksxw=s} {Susan} {ha’niisgwaa’ytxwsa}].  \\
    tell\tsc{-t-tr=pn} Mary \tsc{obl-1sg.ii} yesterday \tsc{comp} tired=\tsc{pn} Susan Sunday \\
\glt ‘Mary said yesterday that Susan was feeling tired on Sunday.’ 

\ex \label{Todoro26b}
\tsc{[tp \textit{past} [aspp [vp [cp [tp \textit{past} [aspp [vp]]]]]]]}

\end{xlist}
\end{exe}

Examples in (\ref{Todoro25}) and (\ref{Todoro26}) show that the embedded predicate in past contexts can get either back-shifted or simultaneous interpretation in Gitksan (for aspectual restrictions, see \citealt{todorovic2020}). This resembles the SOT effects in English. If the SOT effects in English are derived from the interaction between matrix and embedded tense (\citealt{ogihara1995a}, \citealt{gronnstechow2010a}, \citealt{zeijlstra2012a}, i.a.; cf. \citealt{altshuler2012a}), the corresponding interpretations in Gitksan can be captured by positing TP
%Tense 
in these complements, as shown above. Conversely, similarities between English and Gitksan are puzzling if there is no TP  %Tense
in these complements in Gitksan.  

Consider now what happens when the embedded relative non-future tense combines with \emph{dim}. The following four combinations are predicted:

\begin{enumerate} 
\item matrix present -- embedded present\,+\,\emph{dim}
\item matrix present -- embedded past\,+\,\emph{dim}
\item matrix past -- embedded ‘present’\,+\,\emph{dim}
\item matrix past -- embedded ‘past’\,+\,\emph{dim}
\end{enumerate}
Each but last interpretation is attested. The first option is (\ref{Todoro19}), repeated in (\ref{Todoro27}). Matrix present locates the believing time at the UT. The embedded present introduces the time interval simultaneous with the believing time, i.e. the UT. \emph{Dim} extends from the UT and locates the embedded ET in the future. 

\begin{exe}
\ex \label{Todoro27}
\begin{xlist}

\ex \label{Todoro27a} [I’m looking for Colin. I ask you ``Where is Colin?'' You reply:]\\
\gll {Ha'niigood-i'y} 	[{dim}	{yukw=hl}	{bax̱-t}].  \\
    believe-1\tsc{sg.ii}	[\tsc{prosp}	\tsc{prog=cn}	run-3.{\seriesII}] \\
\glt ‘I believe he will run.’ 

\ex \label{Todoro27b}
\tsc{[tp \textit{pres} [aspp [vp [cp [tp \textit{pres} [modp $\emptyset$ [aspp \textbf{\textit{dim}} [vp]]]]]]]]}

\end{xlist}
\end{exe}

Second option is in (\ref{Todoro28}). Matrix present locates the time of believing at the UT. The embedded past introduces an interval before the UT, at the time when I saw Colin yesterday. \emph{Dim} extends forward from this past interval and locates the embedded ET after the time when I saw Colin.

\begin{exe}
\ex \label{Todoro28}
\begin{xlist}

\ex \label{Todoro28a} [You saw Colin yesterday and it looked like he was getting ready 
     to go for a run. I ask you: ``What was Colin doing when you saw him?'' You say:]\\
\gll {Ha'niigood-i'y} 	[{dim}	{bax̱-t}].  \\
    believe-1\tsc{sg.ii}	[\tsc{prosp}	run-3.{\seriesII}] \\
\glt ‘I think he was going to run.’ 

\ex \label{Todoro28b}
\tsc{[tp \textit{pres} [aspp [vp [cp [tp \textit{past} [modp $\emptyset$ [aspp \textbf{\textit{dim}} [vp]]]]]]]]}

\end{xlist}
\end{exe}

In the third option, Diana’s statement in (\ref{Todoro29}) was 2 weeks ago (adapted from \citealt{johannsdottirmatthewson2007a}). The embedded ‘present’ (simultaneous) sets the time of Diane’s statement as the RT for \emph{dim}. \emph{Dim} extends in the future from that point, so the embedded ET is located during the last week.

\begin{exe}
\ex \label{Todoro29}
\begin{xlist}

\ex \label{Todoro29a} [It is December 14 today. I met Diana 2 weeks ago, on 
November 30. I asked her about her plans. She said that her sister had a birthday party in Winipeg on December 7 and that she would go to that party.]\\
\gll {Gilbil-hl} {anuutxw=hl}	{nda}          {mahl-i=s} {Diana}    {dim}      {wil} {yee-t}	{g̱o’o=hl} {Winnipeg} {am} {k’i’y=hl}	{g̱anuutxw}. \\
    two-\tsc{connn} week={\cn} 	when 	     tell-\tsc{t=pn} Diana	\tsc{prosp}      \tsc{comp} go-3.{\seriesII} \tsc{loc=cn}         Winnipeg	only one=\tsc{cn}	week\\
\glt ‘Diana said two weeks ago that she would go to Winnipeg after 	 one week.’ 

\ex \label{Todoro29b}
\tsc{[tp \textit{past} [aspp [vp [cp [tp \textit{pres} [modp $\emptyset$ [aspp \textbf{\textit{dim}} [vp]]]]]]]]}

\end{xlist}
\end{exe}

Consider the last option (matrix past -- embedded ‘past’\,+\,\emph{dim}). In (\ref{Todoro30}), John’s statement is located in the past. The embedded tense back-shifts from the matrix past, as in (\ref{Todoro30}). \emph{Dim} would then need to extend from that time, i.e. before John’s statement. This is in principle possible -- the time of Mary’s arrival could be before the time of John’s statement. This reading is not attested, which is puzzling.\largerpage[-2]

\begin{exe}\judgewidth{\#}
\ex \label{Todoro30}
\begin{xlist}

\ex{\label{Todoro30a} [You saw John yesterday. He thought Mary was in town and he was looking for her. He told you that, according to what he knew, she would have arrived to town last Sunday.]\\
\gll \#{He=s} {John} {ky'oots} {dim} {'witxw=g̱at=t} {Mary} {jihlaa} {ha'niigwaa'ytxw} \\
    say={\pn} John yesterday \tsc{prosp} arrive=\tsc{report=pn} Mary when Sunday\\
\glt ‘John said that Mary would have arrived on Sunday.’ }

\ex[]{\label{Todoro30b}
\tsc{[tp \textit{past} [aspp [vp [cp [tp \textit{past} [modp $\emptyset$ [aspp \textbf{\textit{dim}} [vp]]]]]]]}}

\end{xlist}
\end{exe}

One possible explanation for (\ref{Todoro30}) comes from English. In the English example in (\ref{Todoro31a}), \textit{would} is necessarily future-oriented with respect to the time of finding out. The syntax is as in (\ref{Todoro31b}): \textit{would} is standardly assumed to be composed of past tense and the modal {\sc woll} (\citealt{abusch1985a}, \citeyear{abusch1988a}). Given that the embedded past is c-commanded by the matrix past, this creates the SOT environment, i.e. the embedded past can be deleted and be interpreted as simultaneous. \citet{kusumoto1999a} argues that with \textit{would} in embedded contexts in English, past tense undergoes obligatory deletion. This explains why the {\sc woll} component in these contexts is always future-oriented with respect to the matrix ET and not with respect to the embedded past time (see also \citealt{wurmbrand2014a}). 

\begin{exe}
\ex \label{Todoro31}
\begin{xlist}

\ex \label{Todoro31a}
We found out a month ago that the trial would be last week. 

\ex \label{Todoro31b}
\tsc{[tp \textit{past} [aspp [vp [cp [tp \sout{\textit{past}} [modp \textit{woll} [aspp [vp]]]]]]]]}

\end{xlist}
\end{exe}

If the same mechanism applies in Gitksan propositional complements, then the embedded past, when (a) combined with \emph{dim}, and (b) c-commanded by matrix past, should undergo obligatory deletion. In these clauses, the embedded past will always be interpreted as simultaneous; \emph{dim} can then only extend from matrix ET. This is exactly the only licit interpretation of this sentence, as in (\ref{Todoro32}).  

\begin{exe}
\ex \label{Todoro32}
\begin{xlist}

\ex \label{Todoro32a} [You saw John last Wednesday. John was expecting for Mary to arrive to town soon. He told you, that according to what he knew, Mary would arrive this past Sunday.]\\
\gll {He=s}    {John} {ky'oots}  {dim} {'witxw=g̱at=t} {Mary} {jihlaa} {ha'niis-gwaa'ytxw}. \\
    say=\tsc{pn} John yesterday \tsc{prosp} arrive=\tsc{report=pn} Mary when Sunday\\
\glt ‘John said that Mary would arrive on Sunday.’ 

\ex \label{Todoro32b}
\tsc{[tp \textit{past} [aspp [vp [cp [tp \textit{\sout{past}} [modp $\emptyset$ [aspp \textbf{\emph{dim}} [vp]]]]]]]]}

\end{xlist}
\end{exe}

\subsection{Future-irrealis complements}\label{Todoro:sect4.2}

Future-irrealis complements are necessarily future-oriented and have obligatory \emph{dim}. If %Tense
TP is projected in these complements, there are four predicted readings: (1) present -- present\,+\,\emph{dim}, (2) past -- ‘present’\,+\,\emph{dim}, (3) present -- past\,+\,\emph{dim}, (4)~past -- ‘past’\,+\,\emph{dim}. Crucially, only the first two readings are attested. I argue that the distribution is accounted for only if there is no TP 
%Tense 
in these complements.

With the first option, we predict the reading in (\ref{Todoro20}), repeated in (\ref{Todoro33}). Matrix present locates the ET at the UT, embedded present introduces an interval simultaneous with the matrix ET. \emph{Dim} locates the embedded ET in the future.

\begin{exe}
\ex \label{Todoro33}
\begin{xlist}

\ex \label{Todoro33a} [There’s a charity run next week. Will Colin run?]\\
\gll {Hasak̲-t}	[\#({dim})	{bax̱-t}].	 \\
    want-3.{\seriesII}	[\#({prosp})	run-3.{\seriesII}]	\\
\glt ‘He wants to run.’ 

\ex \label{Todoro33b}
\tsc{[tp \textit{pres} [aspp [vp [tp \textit{pres} [modp $\emptyset$ [aspp \textbf{\emph{dim}} [vp]]]]]]]}

\end{xlist}
\end{exe}

The second option is in (\ref{Todoro34}), i.e. future-in-the-past reading. Matrix tense locates the wanting time in the past and the embedded ‘present’ introduces the RT for the embedded ET simultaneous with the wanting time. \emph{Dim} then locates the movie watching in the future from the wanting time.

\begin{exe}
\ex \label{Todoro34}
\begin{xlist}

\ex \label{Todoro34a} [You wanted to see Tenet yesterday, they were showing it in the cinema. But you were really busy the entire day and you didn’t make it in time to the cinema, so you didn’t see it. And they are not showing it anymore.]\\
\gll {Sim} {hasag-a’y}	{dim}	{algal--i’y}		{a=hl}		{Tenet}.	 \\
    really 	want-1{\sg}	\tsc{prosp}	watch-1\tsc{sg.ii}	 \tsc{prep=cn} 	Tenet	\\
\glt ‘I wanted to watch the film (but it is not being shown anymore).’ 

\ex \label{Todoro34b}
\tsc{[tp \textit{past} [aspp [vp [tp \textit{pres} [modp $\emptyset$ [aspp \textbf{\emph{dim}} [vp]]]]]]]}

\end{xlist}
\end{exe}

Crucially, the remaining two combinations are unattested. Consider first matrix past -- embedded ‘past’, as in (\ref{Todoro35a}). Matrix tense locates the wanting time in the past. The embedded past moves the RT for the embedded ET before the wanting time. \emph{Dim} should then extend from that point and in principle allow for the reading where the running occurs before wanting. But this is not the case -- the only attested interpretation of this sentence is in (\ref{Todoro36}) -- the wanting occurs before running. How do we account for this? One option is -- keeping the embedded TP and saying that past embedded under another past, when combined with \emph{dim}, undergoes obligatory deletion. The \emph{dim} is then correctly predicted to extend in the future from the wanting time, as in \REF{Todoro36}.

\begin{exe}
\ex \label{Todoro35}
\begin{xlist}

\ex \label{Todoro35a} [There was a 5k race on Sunday in your town, the only one this    
     year. Your friends ran, but you didn’t feel like it. Yesterday, you    
    finally felt like running that race, but it was too late, the race was    
    over.]\\
\gll \# {Hasag̱-a'}y {dim} 	{bax̱-a'y} 	{e=hl} 	{g̱olt}. \\
    {} want-1{\sg}  \tsc{prosp}  run-1{\sg}  	\tsc{prep=cn} 	race	\\
\glt ‘I wanted to have run the race.’

\ex \label{Todoro35b}
\tsc{[tp \textit{past} [aspp [vp [tp \textit{past} [modp $\emptyset$ [aspp \textbf{\emph{dim}} [vp]]]]]]]}

\end{xlist}

\ex \label{Todoro36}

[There was a 5k race on Sunday in your town, the only one this   
         year. I know you wanted to run, but you sprained you ankle. You   
         say to me:]\\
\gll {Hasag̱-a'}y {dim} 	{bax̱-a'y} 	{e=hl} 	{g̱olt}. \\
    want-1{\sg}  \tsc{prosp}  run-1{\sg}  	\tsc{prep=cn} 	race	\\
\glt ‘I wanted to run the race.’ 

\end{exe}

However, even if keep the TP analysis, it cannot derive the remaining reading: matrix present -- embedded past, as in (\ref{Todoro37}). Matrix present locates the wanting time at the UT. The embedded past shifts the RT back from the UT, i.e. to yesterday. \emph{Dim} moves it forward; it is thus predicted that eating the salmon can happen before wanting it. This reading is unattested. As (\ref{Todoro37b}) shows, this time we cannot resort to any kind of deletion of the embedded past, since this is not the licensing environment (the matrix tense is not past). Thus, by positing the embedded TP, we incorrectly rule in this reading. 

\begin{exe}
\ex \label{Todoro37}
\begin{xlist}

\ex \label{Todoro37a} [There was a party yesterday and there was a lot of food. There   
     was also smoked salmon, but you didn’t eat it. Today, you are thinking how you should’ve tried that salmon, it looked delicious.]\\
\gll \# {Hasag-a’y}  [{ni}	{dim} 	{gup=hl} 	{hun}].\\
    {} want-1.{\seriesII}  	1.\tsc{i}	\tsc{prosp}	eat={\cn}	salmon	\\
\glt Intended meaning: ‘I want to have eaten the salmon’

\ex \label{Todoro37b}
\tsc{[tp \textit{pres} [aspp [vp [tp \textit{past} [modp $\emptyset$ [aspp \emph{\textbf{dim}} [vp]]]]]]]}

\end{xlist}
\end{exe}

Crucially, this problem does not arise if there is no TP in the embedded clause: the RT for \emph{dim} is the wanting time, as in (\ref{Todoro38}). \emph{Dim} then extends in the future from it, regardless of whether the wanting is in the present (\ref{Todoro38a}) or in the past (\ref{Todoro38b}). This correctly allows (\ref{Todoro33}), (\ref{Todoro34}) and (\ref{Todoro36}) and excludes (\ref{Todoro35}) and (\ref{Todoro37}) -- the embedded ET is always after the wanting time.

\begin{exe}
\ex \label{Todoro38}
\begin{xlist}
\ex \label{Todoro38a} 
\tsc{[tp \textit{pres} [aspp [vp want [modp $\emptyset$ [aspp \textit{dim} [vp]]]]]]}
\ex \label{Todoro38b}
\tsc{[tp \textit{past} [aspp [vp want [modp $\emptyset$ [aspp \textit{dim} [vp]]]]]]}
\end{xlist}
\end{exe}

In sum, if tense is simultaneous with or back-shifted from the RT in Gitksan, then the temporal interpretations in propositional complements are captured with TP in them. Conversely, the readings in future-irrealis complements can be captured only without TP.

\section{A remaining question: Two ‘try’s}\label{Todoro:sect5}

In \sectref{Todoro:sect3}, I have shown that \emph{si’ix} ‘try’ does not allow \emph{dim}, but that \emph{bak̲} requires it, as repeated in (\ref{Todoro39}). 

\begin{exe}
\ex \label{Todoro39} [Colin injured himself before the run. He is stubborn and decides 
    to try anyway. We are watching the race and I spot him trying to     
   run, limping along. I tell you:] \\
\begin{xlist}

\ex \label{Todoro39a}
\gll {Yukw[=hl]}	{si'ix} 	(\#{\textit{dim}}) 	{bax̱-t}.	  \\
  \tsc{prog[=cn]}	try	(\#\tsc{prosp})	run-3.{\seriesII} \\
\glt `He’s trying to run.' 

\ex \label{Todoro39b}
\gll {Bag-a-t}	[\#({\textit{dim}}) 	{bax̱-t}]. \\
    try-\tsc{tr-3.ii}	[\#(\tsc{prosp})	run-3.{\seriesII}]\\
\glt `He’s trying to run.' 
\end{xlist}
\end{exe}

While the two can both be used in majority of contexts, M\&T show that in non-agentive contexts, only \emph{si’ix} is fine. 

\begin{exe}\judgewidth{\#}
\ex \label{Todoro40} [How was the weather yesterday? (\emph{Guuhl wihl lax ha k'yoots?})]
\begin{xlist}

\ex[]{\label{Todoro40a}
\gll {Si'ix} 	{wis} 	{ky’oots} 	({gi}). \\
   try	rain	yesterday	\tsc{prior.evid} \\
\glt `It tried to rain yesterday.' }

\ex[\#]{\label{Todoro40b}
\gll {Bag-a-t} 	[{dim} 	{wis} 	{ky’oots}]. \\
   try-\tsc{tr-3.ii} 	\tsc{prosp}	rain	yesterday\\
\glt `It tried to rain yesterday.' \\ {Consultant’s comment:} “No. An individual can’t make it rain. Not unless you’re the rain dancer.”}
\end{xlist}
\end{exe}

M\&T propose that the distribution of \emph{dim} with these verbs is due to their semantics -- \emph{si’ix} is more like English `try' and \emph{bak̲} is more like English `want/decide/plan’ (modulo the agentivity requirement). An argument for \emph{si’ix} -- ‘try’ correspondence builds on \citet{sharvit2003a} observation that ‘try’ has both an intensional and an extensional component. The extensional component asserts that there is an event in the real world. And if that is the case, then the event like ``cutting a tomato'' in (\ref{Todoro41}), requires there to be a tomato. ‘Want’ lacks the requirement of the object existing in the actual world. The examples in Gitksan in (\ref{Todoro42}) show that only \emph{si’ix} has a requirement that there are tomatoes, while \emph{bak̲} does not. In other words, only \emph{si’ix} behaves like English ‘try’.

\begin{exe}\judgewidth{\#}
\ex \label{Todoro41}
John \textit{wanted/\#tried} to cut a tomato, but there were no tomatoes to cut.
	\\(\citealt{sharvit2003a}:404-405)

\ex \label{Todoro42} 
[John is coming into a room, and he’s got his knife handy and is 	planning to cut tomatoes and then he notices that there is nothing there.]
\begin{xlist}

\ex[\#]{\label{Todoro42a}
\gll {Si'ix} {ḵ’ots-d-i=s}	{John=hl} {tomato,} {ii}  {ap}  {nee}  {dii}  {dox=hl}  {tomatoes}. \\
    try    cut-\tsc{t-tr=pn} 	John=\tsc{cn} tomato  \tsc{ccnj}	\tsc{verum} \tsc{neg} \tsc{foc} be.on.\tsc{pl=cn} 	tomatoes\\
\glt `John tried to cut a tomato, but there were no tomatoes.’\\
{Consultant’s comment}: “\textit{Si'ix} means he tried. But he didn’t try yet because there were no tomatoes.”}
 

\ex[]{\label{Todoro42b}
\gll {Bag-a=s}   {John} {dim=t}        {ḵ’ots=hl}  {	tomato, } {ii}  {ap}  {nee}  {dii}  {dox=hl}  {tomatoes}. \\
    try-\tsc{tr=pn}    John \tsc{fut}=3{\seriesI} cut=\tsc{cn} tomato  \tsc{ccnj}	\tsc{verum} \tsc{neg} \tsc{foc} be.on.\tsc{pl=cn} 	tomatoes\\
\glt {Consultant’s volunteered scenario:} ``John is coming into a room,   
     and he’s got his knife handy and his companion is right there and 
     then they notice that there are no tomatoes.''}

\end{xlist}
\end{exe}

Regarding \emph{bak̲}, M\&T argue that it is similar to \citeauthor{grano2011a}’s (\citeyear{grano2011a}, \citeyear{grano2017a}) ‘try’ in which: (a) agent is presupposed, (b) volitional events have an initial stage that corresponds to a mental action, (c) ‘try’ picks out this initial stage of the event, i.e. it asserts that the event is realized to a degree above zero; (d) it is associated with an ordering source based on the agent’s intentions. 

M\&T argue that the initial stage of volitional action and ‘try’ referring to agent’s intentions capture \emph{bak̲} -- a mental stage of preparing to cut tomatoes counts as the initial stage of trying. This mental preparatory stage makes \emph{bak̲} similar to ‘want/decide/plan’ (the difference is that \emph{bak̲} can only refer to the events in the immediate future). 

Structurally, \emph{bak̲} patterns with ‘want/decide/plan’ in having \emph{dim} in the complement, and \emph{si’ix} patterns with ‘try’ in not having it. And a preliminary investigation shows that there is more syntactic parallelism.
First, \emph{si’ix} is a pre-verbal element and it does not allow for the subject to intervene between it and the verb, while \emph{bak̲} is an independent lexical verb and it embeds a complement containing both subject and the verb, as in (\ref{Todoro43}). The contrast is shown in (\ref{Todoro44}). Note that \emph{hasak̲} ‘want’ has the same configuration as \emph{bak̲}, as in (\ref{Todoro45}). 

\begin{exe}\judgewidth{\#}
\ex \label{Todoro43}
\begin{xlist}

\ex \label{Todoroa}
[\emph{si'ix} 	 V+inflection	(DP-subject)]
\ex \label{Todorob}
[\emph{bak̲}+inflection (DP-subject) [V+inflection]]
\end{xlist}

\ex \label{Todoro44}
\begin{xlist}

\ex[]{\label{Todoro44a}
\gll {Siki’hl} {gub-i=s} 		{John} 	{hun}. \\
   try 	eat-\tsc{tr=pn}     	John 	salmon \\
\glt `John tried to eat salmon.' }

\ex[\#]{\label{Todoro44b}
\gll {Sik’ihl} {John}    {gup}    {hun}. \\
    try     	John 	  eat    salmon \\}

\ex[]{\label{Todoro44c}
\gll {Bag̱-a=s} 	{John} [{dim=t} 	    { gup-hl} 		{hun}]. \\
    try-\tsc{tr=pn}    John \tsc{prosp=3.I} eat-\tsc{cn} 		salmon\\
\glt `John tried to eat salmon.' }
\end{xlist}

\ex \label{Todoro45}
\gll {Hasak̲=s} 	{John} [{dim=t} 		{gup=hl}        {hun}]. \\
    Want=\tsc{pn}        John \tsc{prosp=dm}	eat=\tsc{cn}     salmon\\
\glt `John wanted to eat salmon.' 
\end{exe}

Second indicator is a behavior  under negation (Clarrisa Forbes, p.c.). \emph{Bak̲}  behaves  like \emph{hasak̲} ‘want’ with respect to the word order in the embedded domain and the agreement marking on the prospective aspect (series \tsc{i}), as in (\ref{Todoro46}). \emph{Si’ix} patterns with a desiderative verb \emph{‘nim} (another way to express desire) in having  a predicate-initial word order in the embedded domain, and with Colin carrying the common noun determiner, as in (\ref{Todoro47}). 

\begin{exe}
\ex \label{Todoro46}
\begin{xlist}

\ex \label{Todoro46a}
\gll {Nee} {dii-t} 	   {bak̲=}s	{Colin}	[{dim=t}	        {gup=hl} 	{hun}]. \\
    \tsc{neg}	\tsc{foc}-3.{\seriesI}  try={\pn} 	Colin 	 \tsc{prosp}=3.{\seriesI}   eat=\tsc{cn}	 fish\\
\glt `Colin didn’t try to eat fish.' 

\ex \label{Todoro46b}
\gll {Nee} 	{dii} 	   {hasak̲=s} 	{Colin}	[{dim=t}	        {gup=hl} 	{hun}]. \\
    \tsc{neg}	\tsc{foc} 	   want={\pn}	Colin 	 \tsc{prosp}=3.{\seriesI}   eat=\tsc{cn}	 fish\\
\glt ‘I didn’t want to eat fish.' 

\end{xlist}

\ex \label{Todoro47}
\begin{xlist}

\ex \label{Todoro47a}
\gll {Nee} 	{dii=t} 		{si’ix} 	[{gup=s} 	{Colin=hl}	{hun}]. \\
    \tsc{neg} 	\tsc{foc}=3.{\seriesI} 	try 	eat={\pn} 	Colin=\tsc{cn} 	fish\\
\glt ‘Collin didn’t try to eat fish.’ (Clarisa Forbes, p.c.) 

\ex \label{Todoro47b}
\gll {Nee} 	{dii=t} 	   {‘nim} 	[{gup=s} 	{Colin=hl}	{hun}].\\
    \tsc{neg} 	\tsc{foc}=1.{\seriesI}  \tsc{desider}	eat={\pn}	Colin=\tsc{cn} 	fish\\
\glt `Collin didn’t want to eat fish.’               (Clarisa Forbes, p.c.) 

\end{xlist}
\end{exe}

Moreover, \emph{‘nim}, like \emph{si’ix}, cannot be followed by \emph{dim}:\footnote{I would like to thank the reviewer for drawing my attention to this.}  

\begin{exe}
\ex \label{Todoro48}
[There’s a charity run next week. Will Colin run?] \\
\gll {‘Nim}	  (\#{\textit{dim}})	{bax̱-t}          {Colin}.\\ 
want	   \tsc{prosp}	run-3.{\seriesII} 	Colin\\
\glt ‘He wants to run.’	
\end{exe}


Thus, \emph{bak̲} and \emph{si’ix} are not the only two verbs that seem to belong to different classes. One possibility is to say that tenseless complements are either vP (with \emph{si’ix}) or ModP (with \emph{bak̲}) and that future-irrealis complements are either vP (with \emph{‘nim}) or ModP (with \emph{hasak̲}). Another option is to follow \possciteauthor{wurmbrand2019a} idea that a lexical verb can belong to one class (e.g. have a smaller complement) in one language and another class (e.g. have a larger complement) in another language. In other words, no class contains exactly the same  set of verbs in every language. Rather, how much structure is projected within a complement of a verb is determined by the transparency of the embedded domain and its dependence on the matrix domain. The natural next step is to determine further syntactic and semantic properties of (pre-)verbs that seemingly belong to the same class in Gitksan and to differentiate between the two approaches.

\section{Conclusion and outlook}\label{Todoro:sect6}

In this paper, I argued that the future-oriented readings in Gitksan complements provide evidence for structural differences between these complements: propositional complements are CPs, future-irrealis complements are ModP and tenseless complements are either vPs or ModP. The absence of TP in future-irrealis and tenseless complements (and of ModP in complements of \emph{si’ix}), systematically limits the availability of temporal readings in them, while its presence in propositional complements expectedly enables most temporal interpretations. The conveniently marked futurity in these complements makes the differences between them easier to spot. The findings from Gitksan provide preliminary support for the implicational complementation hierarchy. 
One potential avenue for further research would be the left periphery. The proposed structural analysis makes the following prediction about the distribution of a complementizer \emph{wil}: it should be able to occur in propositional complements, but not in future-irrealis and tenseless complements. This prediction is borne out, as shown in (\ref{Todoro49}). 

\begin{exe}
\ex \label{Todoro49}
\begin{xlist}

\ex \label{Todoro49a}
\gll {Gilbil=hl} {g̱anuutxw=hl}  {dat} 	  { mahl-i=s}	{ Diana}  [(\#{\textit{wil}}) {\textit{dim}}   {\textit{wil}}     {yee-t}   {g̱o’o=h}l { Winnipeg} {am} {k’i’y=hl}	{g̱anuutxw}]. \\
     two=\tsc{cn} weeks=\tsc{cn}    	when 	   tell-\tsc{t=pn} Diana	\tsc{comp} \tsc{prosp} \tsc{comp} go-3.{\seriesII}  \tsc{loc}=\tsc{cn}  Winnipeg  only one=\tsc{cn}    week\\
\glt `Diana said two weeks ago that she would go to Winnipeg.' 

\ex \label{Todoro49b}
\gll {Sim} 	   {hasag̱-a’y}     [(\#{\textit{wil}}) \textit{{dim}} (\#\textit{{wil}})	{alg̱al-i’y} 		{a=hl}		{Tenet}		{ky’oots}]. \\
	Really want-{\sg}.{\seriesII}    \tsc{comp} \tsc{prosp} \tsc{comp} 	watch-{\sg}.{\seriesII} 	\tsc{prep}=\tsc{cn} 	Tenet 		yesterday\\
\glt `I wanted to watch Tenet yesterday (but it is not being shown anymore).' 

\ex \label{Todoro49c}
\gll [{Yukw=hl}]	{si'ix} 	\textit{(\#{wil})} 	{bax̱-t}.\\
		\tsc{prog}[=\tsc{cn}]	try	(\#\tsc{comp})	run-3.{\seriesII} \\
\glt ‘He’s trying to run.' 

\ex \label{Todoro49d}
\gll {Bag-a-t}	 [\textit{(\#{wil})}  \textit{{dim}}		\textit{(\#{wil})} {bax̱-t}]. \\
	try-\tsc{tr}-3.{\seriesII}	 [(\#\tsc{comp})   \tsc{prosp} 	(\#\tsc{comp})	run-3.{\seriesII}]\\
\glt `He’s trying to run.' 

\end{xlist}
\end{exe}

What is puzzling is the order of \emph{dim} and \emph{wil}: \emph{dim} standardly precedes \emph{wil} in both Gitksan (\citealt{rigsby1986a}) and neighboring Nisga’a (\citealt{tarpent1987a}). Syntactically, this is problematic, since \emph{wil} is supposedly a complementizer and \emph{dim} is a prospective aspect. While I do not have a straight-forward solution at this point, note that \emph{dim} also proceeds conjunction \emph{ii} ‘and then’ in Nisg’a \citep[434]{tarpent1987a}, but can be preceded by complementizer \emph{ji} ‘whether’ \citep[430]{tarpent1987a}. It is also in an unexpected place when combined with a progressive marker \textit{yukw}, as in (\ref{Todoro50}). (\ref{Todoro50}) is about a future event, so \emph{dim} should be taking a scope over \emph{yukw}, which is not reflected on the surface. One option is that there is some kind of phonological requirement that determines the surface order of \emph{dim}. Finally, \emph{dim} is obligatory in purpose clauses. But it can occur with \emph{wil} in either order, resulting in two different interpretations, as in (\ref{Todoro51}). I leave this puzzle for further research.\largerpage[1]

\begin{exe}
\ex \label{Todoro50}
\gll \textit{{Yukw}} 	\textit{{dim}} 	{wis}.  \\
 \tsc{prog} \tsc{fut} rain \\
\glt `It is going to rain.' 

\ex \label{Todoro51} [{Why did Rosemary come to UBC today?}] 
\begin{xlist}

\ex \label{Todoro51a}
\gll {Witxw}	{‘nit}	  \textit{{dim}}	\textit{{wil}}	{hahla’ls-t}. \\
    arrive	3\tsc{sg.ii}   \tsc{prosp}	\tsc{comp}	work-3.{\seriesII}\\
\glt `She arrived to work.' 

\ex \label{Todoro51b}
\gll {’Witxw}	{‘nit}	     \textit{{wil}}	\textit{{dim}} 	{hahla’ls-t}. \\
    arrive	3\tsc{sg.ii}     \tsc{comp}	\tsc{prosp}	work-3.{\seriesII}\\
\glt `She came because she works there.' 

\end{xlist}
\end{exe}


\section*{Acknowledgments}
Ha’miiyaa to my dear consultants Vince Gogag, Barbara Sennott, Hector Hill, Ray Jones and Barry Sampere for teaching me their language. This work would be impossible without your help! I would also like to thank the GitLab and the TAP Lab at UBC for their ongoing support and feedback. Special thanks go to Yurika Aonuki, Henry Davis, Clarissa Forbes, Marianne Huijsmans, John Lyon, Lisa Matthewson, Hotze Rullmann, Michael Schwan and Anne-Michelle Tessier for their useful feedback, as well as to Zheng Shen and Sabine Laszakovits for their patience and constant help. This work was made possible by the Jacobs Research Fund awarded to the GitLab, and by the Social Sciences and Humanities Research Council of Canada grant (\#435-2016-0381) awarded to the TAP Lab. Last but not least, I would like to thank Susi for unselfishly sharing with me her knowledge and support over the years, and for inspiring me to continue working on this topic. Vielen Dank, Susi!


\section*{Abbreviations}
\begin{tabularx}{.5\textwidth}{@{}lQ}
1/2/3 & first/second/third person\\
\textsc{i/ii/iii} & series \textsc{i/ii/iii} pronoun\\
\tsc{ax} & agent extraction\\
\tsc{circ.necess} & circumstantial necessity\\
\comp & complementizer\\
\tsc{ccjn} & conjunction\\ 
\tsc{cn} & common noun determiner\\ 
\dn & determinate noun determiner\\
\tsc{epis} & epistemic\\
\foc & focus\\
\tsc{ins} & instrumental\\
\end{tabularx}\begin{tabularx}{.5\textwidth}{lQ@{}}
\tsc{ipfv} & imperfective\\
\tsc{irr} & irrealis\\
\tsc{neg} & negation\\
\tsc{pass} & passive\\
\pl & plural\\
\pn & proper noun determiner\\
\tsc{prep} & preposition\\
\tsc{prior.evid} & prior evidence\\
\tsc{prog} & progressive\\
\tsc{prosp} & prospective\\
\tsc{report} & reportative\\
\sx & intransitive subject extraction\\
\tr & transitive\\
\tsc{verum} & verum\\

\end{tabularx}

{\sloppy\printbibliography[heading=subbibliography,notkeyword=this]}

\end{document}
